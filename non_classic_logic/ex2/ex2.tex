\documentclass[11pt,a4paper]{article}
\usepackage{amsmath}
\usepackage{amssymb}
\usepackage{enumitem}
\usepackage{amsthm}
\usepackage{MnSymbol}
\setlength{\parindent}{0pt}
\usepackage[utf8]{inputenc}
\usepackage{listings} [python]
\usepackage{url}
\usepackage{bussproofs}

\newtheorem{theorem}{Theorem}[section]
\newtheorem{corollary}{Corollary}[theorem]
\newtheorem{lemma}[theorem]{Lemma}
\newtheorem{mydef}{Definition}

%opening


\newcommand{\lto}{\supset}
\newcommand{\some}{\Diamond}
\newcommand{\all}{\Box}

\newcommand{\sand}{\; and \;}
\newcommand{\sor}{ \; or \;}
\newcommand{\sneg}{not \;}
\newcommand{\sto}{\Rightarrow}
\newcommand{\negmodels}{\nvDash}

\begin{document}

%\maketitle

\section*{Note}
\textbf{Missing Exercises:}
\begin{itemize}
\item \textbf{Exercise 12}
\item \textbf{Exercise 14}
\item \textbf{Exercise 15}
\item \textbf{Exercise 16} Equation $\all (A \lto B) \lto (\all A \lto \all B)$
\item \textbf{Exercise 20}
\end{itemize}

\section*{Exercise 9}
\begin{quote}
Derive the above in detail from $\some = \neg \all \neg $ and from the semantics for $\all F$ and $\neg F$ classical (first order) reasoning (at the meta-level), indicating at each step what is used.
\end{quote}

Let $\varphi$ be a formula, $\mathcal{M}$ be an arbitrary Kripke model. and $w$ an arbitrary world. Given the syntactic definition of $\some$, the formula $\some \varphi$ can be rewritten as such
\begin{equation*}
\begin{split}
v_{\mathcal{M}}(\some \varphi, w) \stackrel{syntax}{=} v_{\mathcal{M}}(\neg \all \neg \varphi,w)
\end{split}
\end{equation*}

One has to switch onto a semantic level.

\begin{equation*}
\begin{split}
v_{\mathcal{M}}(\neg \all \neg \varphi, w)=
\begin{cases}
1 & \quad \mathit{if}  \; \mathit{not} \; (v_{\mathcal{M}}(\all \neg \varphi, w)=1) \\
0 & \quad otw.
\end{cases}
\end{split}
\end{equation*}
iff
\begin{equation*}
\begin{split}
v_{\mathcal{M}}(\neg \all \neg \varphi, w)=
\begin{cases}
1 & \quad \mathit{if}  \; \mathit{not} \; (\forall u ( wRu  \Rightarrow (v_{\mathcal{M}}(\neg \varphi, u)=1))) \\
0 & \quad otw.
\end{cases}
\end{split}
\end{equation*}
iff
\begin{equation*}
\begin{split}
v_{\mathcal{M}}(\neg \all \neg \varphi, w)=
\begin{cases}
1 & \quad \mathit{if}  \; \mathit{not} \; (\forall u ( wRu  \Rightarrow (not \; (v_{\mathcal{M}}(\varphi, u)=1)))) \\
0 & \quad otw.
\end{cases}
\end{split}
\end{equation*}
If something does not hold for all elements, then there exists an element for which its negation holds. Therefore, 
\begin{equation*}
\begin{split}
v_{\mathcal{M}}(\neg \all \neg \varphi, w)=
\begin{cases}
1 & \quad \mathit{if}  \;  \exists u (not \; ( wRu  \Rightarrow (not \; (v_{\mathcal{M}}(\varphi, u)=1)))) \\
0 & \quad otw.
\end{cases}
\end{split}
\end{equation*}
Since reasoning on the meta level is classic, one can rewrite $A \Rightarrow B$ as $(not \; A) \; or \; B$. That is, an implication is always satisfied, except if $A$ holds and $B$ does not. Hence,
\begin{equation*}
\begin{split}
v_{\mathcal{M}}(\neg \all \neg \varphi, w)=
\begin{cases}
1 & \quad \mathit{if}  \;  \exists u (not \; ( (not \; wRu) \; or \; (not \; (v_{\mathcal{M}}(\varphi, u)=1)))) \\
0 & \quad otw.
\end{cases}
\end{split}
\end{equation*}
Again, reasoning on the meta-level is done with respect to classical logic, allowing the appeal to the semantic form of the DeMorgan laws. That is, if a disjunction $A \; or \; B$ does not hold. Then $(not \; A) \; and \; (not \; B)$ must be the case, and vice versa. 
\begin{equation*}
\begin{split}
v_{\mathcal{M}}(\neg \all \neg \varphi, w)=
\begin{cases}
1 & \quad \mathit{if}  \;  \exists u  ( (not \; (not \; wRu)) \; and \; (not \;(not \; (v_{\mathcal{M}}(\varphi, u)=1)))) \\
0 & \quad otw.
\end{cases}
\end{split}
\end{equation*}
Now consider the fact that (in classical logic) by negating an already negated assertion one obtains the original assertion.
\begin{equation*}
\begin{split}
v_{\mathcal{M}}(\neg \all \neg \varphi, w)=
\begin{cases}
1 & \quad \mathit{if}  \;  \exists u  (  wRu \; and \;  (v_{\mathcal{M}}(\varphi, u)=1)) \\
0 & \quad otw.
\end{cases}
\end{split}
\end{equation*}
Lastly, by using the syntactic definition of $\some$ one obtains
\begin{equation*}
\begin{split}
v_{\mathcal{M}}(\some \varphi, w)=
\begin{cases}
1 & \quad \mathit{if}  \;  \exists u  (  wRu \; and \;  (v_{\mathcal{M}}(\varphi, u)=1)) \\
0 & \quad otw.
\end{cases}
\end{split}
\end{equation*}

\section*{Exercise 11}
\begin{quote}
Find Kripke models in which the following formulas are true in some world. If possible also find Kripke models in which the formulas are true in every world.
(Try to use as few worlds as possible.)
\begin{itemize}
\item $\some p \land \some \all p \land \neg \all p$
\item $p \land \all p \land \neg \some p$
\item $(p \lto q) \land \some (p \land \neg q)$
\item $\neg p \land \some \some p \land \neg \all \some p \land \some \all  \neg p$
\end{itemize}
\end{quote}

\begin{itemize}
\item $\some p \land \some \all p \land \neg \all p$. Consider $\mathcal{M}:=\langle \{s,t\}, \{(s,s),(s,t)\}, V\rangle$ such that $V(p):=\{t\}$. Now consider
\begin{equation*}
\begin{split}
&\mathcal{M}, s \models \some p \land \some \all p \land \neg \all p  \iff \\
&\mathcal{M}, s \models \some p \sand \mathcal{M}, s \models \some \all p \sand \mathcal{M}, s \models \neg \all p \iff \\
&(\exists u \; sRu \sand \mathcal{M}, u \models  p) \sand \\
&( \exists u \; sRu \sand (\forall w \; uRw \sto \mathcal{M}, w \models  p)) \sand \\
&\sneg (\forall u \; sRu \sto \mathcal{M}, u \models p)
\end{split}
\end{equation*}
For $(\exists u \; (sRu \sand \mathcal{M}, u \models  p))$ consider $u$ to be $t$, then $sRt$ and $\mathcal{M}, u \models  p$ because $t \in V(p)$. For $( \exists u \; (sRu \sand (\forall w \; uRw \sto \mathcal{M}, w \models  p)))$  consider $u$ to be $t$. Clearly, $sRt$ and since $t$ is a terminal state $(\forall w \; uRw \sto \mathcal{M}, w \models  p)$ is vacuously true. Lastly, from $\sneg (\forall u \; sRu \sto \mathcal{M}, u \models p)$ one obtains $ (\exists u \; sRu \sand( \sneg \mathcal{M}, u \models p))$. Consider $u$ to be $s$ and since $s \notin V(p)$ it follows that $\mathcal{M}, s \nvDash p$, satisfying the last part of the conjunction. 

Moreover, assume $\mathcal{N}$, such that for all $s \in W$ $\mathcal{N},s \models \some p \land \some \all p \land \neg \all p$. According to $\some \all p $ there $\exists$ a state $a$ such that $\forall w$ reachable from $a$, i.e. $aRw$ it must be that $w \in V(p)$. That is, $\mathcal{N},a \models \all p$. However, this is directly in contradiction with $\mathcal{N},a \models  \neg \all p$. Hence, $\mathcal{N}$ can not exist.

Lastly, there can not be a model where $|W|\leq 1$, as there must be a state where $p$ holds and one where $\neg p$ holds. Having only one state this is not possible.

\item $p \land \all p \land \neg \some p$. Consider $\mathcal{M}:=\langle \{s\}, \{\}, V\}\rangle$ such that $V(p):=\{s\}$. Now consider
\begin{equation*}
\begin{split}
&\mathcal{M}, s \models p \land \all p \land \neg \some p \iff \\
&\mathcal{M}, s \models p \sand  \\
&\forall u \; (sRu \sto \mathcal{M}, u \models p \sand ) \\
&\sneg( \exists u \;( sRu \sand \mathcal{M}, u \models p))
\end{split}
\end{equation*}
The fist part of the conjunction holds because $s \in V(p)$. The second part is vacuously true due to $R=\{\}$. The third part is equivalent to  $\forall u \; ((\sneg sRu) \sor (\sneg \mathcal{M}, u \models p))$ which again is vacuously true due to $sRu$ constantly evaluating to false. 


Obviously, the whole formula holds in all worlds. Moreover, since $W>0$  by definition, this model is minimal.


\item $(p \lto q) \land \some (p \land \neg q)$. Consider $\mathcal{M}:=\langle \{s,t\}, \{(s,t)\}, V\}\rangle$ such that $V(p):=\{t\}$. Now consider
\begin{equation*}
\begin{split}
&\mathcal{M}, s \models (p \lto q) \land \some (p \land \neg q)  \iff \\
&\mathcal{M}, s \models (p \lto q) \sand  \\
&\exists u \; (sRu \sand \mathcal{M}, u \models (p \land \neg q)) \iff \\
&\mathcal{M}, s \models p \sto \mathcal{M}, s \models q \sand  \\
& \exists u \; (sRu \sand (\mathcal{M}, u \models p \sand (\sneg \mathcal{M}, u \models q))) \\
\end{split}
\end{equation*}
Since $s \notin V(p)$ it follows that $\mathcal{M}, s \nmodels p $, thus by semantics of material implication, the first part of the conjunction holds. As for the second part. Consider $u$ to be $t$. Because $t\in V(p)$ and $t \notin V(q)$ it follows that  $\mathcal{M}, t \models p$ and $\mathcal{M}, t \nmodels q $. Thereby, satisfying the second part of the conjunction.

Moreover, assume there exists a model $\mathcal{N}$ such that the formula holds in every state. Let $s$ be an arbitrary state. The statement $\mathcal{N}, s \models \some (p \land \neg q)$ demands the existence of a state $a$ such that $\mathcal{N}, a \models (p \land \neg q)$. However, this requires  $\mathcal{N}, a \nmodels (p \lto q)$. Hence, by contradiction, $\mathcal{N}$ does not exist.


Lastly, there can not be a model where $|W|\leq 1$, as otherwise $p \land \neg q$ and $p \lto q$
must hold in the same state, which as established above can not be the case.

\item $\neg p \land \some \some p \land \neg \all \some p \land \some \all  \neg p$. Consider $\mathcal{M}:=\langle \{s,t\}, \{(s,t),(s,s)\}, V\}\rangle$ such that $V(p):=\{t\}$. Now consider
\begin{equation*}
\begin{split}
&\mathcal{M}, s \models \neg p \land \some \some p \land \neg \all \some p \land \some \all  \neg p \iff \\
&\mathcal{M}, s \nmodels p \sand \\
&\exists u \; sRu \sand (\exists w \; uRw \sand \mathcal{M}, w \models  p) \sand \\
&\sneg ( \forall u \; sRu \sto (\exists w \; uRw \sand \mathcal{M}, w \models p)) \sand  \\
&\exists u \; sRu \sand (\forall w \; uRw \sto \mathcal{M}, w \nmodels  p)
\end{split}
\end{equation*}
Since $s \notin V(p)$ it follows that $\mathcal{M}, s \nmodels p $ satisfying the first part of the conjunction. Let $u$ be $s$ and let $w$ be $t$. Clearly $sRs$ and $sRt$. Moreover, since $t \in V(p)$ $\mathcal{M},t\models p$. Hence, satisfying the second part. After some transformation on the meta-level one obtains 
\begin{equation*}
\begin{split}
\exists u \; sRu \sand (\forall w \; \sneg uRw \sor \mathcal{M}, w \nmodels p))
\end{split}
\end{equation*}
Consider $u$ to be $s$. Since, $sRt$ and $\mathcal{M}, w \models p$. Neither $\sneg uRw$ nor $\mathcal{M}, t \nmodels p$ hold. Hence, by the semantics of $\forall$, the third conjunct is satisfied. 
%Consider $u$ to be $s$. Since $s \notin V(p)$ implies $\mathcal{M},s \nmodels p$ and since $t$ is only related to itself, $\forall w \; \sneg uRw \sor \mathcal{M}, w \nmodels p$ holds. Hence, the third conjunct is satisfied. 
The last part of the conjunction is satisfied as well. This is because, $sRt$ and because $t$ is terminal. Making $\mathcal{M}, t \models \all \neg p$, hold trivially.

Lastly, assume there exists a model $\mathcal{N}$ such that the formula holds in every state. Let $s$ be an arbitrary state. The statement $\mathcal{N}, s \models \some \some p $ demands the existence of a state $a$ such that $\mathcal{N}, a \models p$. However, this requires  $\mathcal{N}, a \nmodels  \neg p$. Hence, the assumption of $\mathcal{N}$ is contradicted. Moreover, there can not be a model where $|W|\leq 1$, as otherwise $p$ and $\neg p$ must hold in the same state.

\end{itemize}

%
%\section*{Exercise 12}
%\begin{quote}
%Find 3-5 further (interesting) examples of modal formulas with one schematic variable that are valid in F, above, such that removal of some (which?) accessibilities leads to invalidity.
%\end{quote}
%The frame in question is $\mathcal{F}:=\langle \{r,t\}, \{(r,r),(s,s)\}\rangle$
%\begin{itemize}
%\item $\all p \lto \all \all p$. Take an arbitrary state $s \in W$. Now consider
%\begin{equation*}
%\begin{split}
%&\mathcal{F}, s \models \all p \lto \all \all p \iff \\
%&(\forall u \; sRu \sto \mathcal{F}, u \models p) \sto (\forall u \; sRu \sto (\forall w \; uRw \sto \mathcal{F}, w \models p)) \iff \\
%\end{split}
%\end{equation*}
%Consider 
%
%
%\item $\neg \all p \lto \all \neg \all p$
%\item $\neg (\all p \land \all \neg p)$
%\item $(\some p \land \all(p \lto \all p)) \lto p$
%\end{itemize}



\section*{Exercise 13}
\begin{quote}
Show that the intersection of two logics is also a logic. What about unions of logics?
\end{quote}
Let $\mathcal{L}$ be a logic and $X \subseteq \mathcal{L}$. Firstly, it has to be established that the closure of $X$, written as $\overline{X}$, is also a subset of $\mathcal{L}$, i.e. $X \subseteq \mathcal{L} \sto \overline{X}\subseteq \mathcal{L}$. Assume the contrary, i.e. $\exists \varphi \in \overline{X}$ and $\varphi \notin \mathcal{L}$. However, by definition this can not be the case.

Let $\mathcal{L}_1$ and $\mathcal{L}_2$ be two logics and let $\mathcal{L}_{\cap} = \mathcal{L}_1 \cap \mathcal{L}_2$.  
%Assume $\mathcal{L}^*$ to be not a logic. That is, $\exists \varphi \in \overline{\mathcal{L}^*} \setminus \mathcal{L}^*$. 
Clearly $\mathcal{L}_{\cap} \subseteq \overline{\mathcal{L}_{\cap}} \subseteq \mathcal{L}_x$ for $x \in \{1,2\}$. That is, $\mathcal{L}_{\cap}$ is a subset of $\mathcal{L}_1$ and of $\mathcal{L}_{2}$. Moreover, as already established. If both $\mathcal{L}_1$ and $\mathcal{L}_{2}$ contain $\mathcal{L}_{\cap}$, then they must contain $\overline{\mathcal{L}_{\cap}}$. Hence, $\overline{\mathcal{L}_{\cap}}=\mathcal{L}_1 \cap \mathcal{L}_2$. Therefore, $\mathcal{L}_{\cap} = \overline{\mathcal{L}_{\cap}}$. \\


Let $\mathcal{L}_1:=\overline{\{a \land b\}}$ and $\mathcal{L}_2:=\overline{\{a \lor b\}}$. The atom $b$ does not occur in either logic. Because, in both cases the only closure operation applicable is substitution. That is, there is no implication where (MP) could be applied. Consider $\mathcal{L}_{\cup}:= \mathcal{L}_1 \cup \mathcal{L}_2$. Moreover, it can be observed that $\varphi \in \mathcal{L}_{\cup}$ is of the form
\begin{equation*}
\varphi = \begin{cases}
\bigwedge_{i} \psi_i  & \quad \mathit{if} \varphi \in \mathcal{L}_1 \\
\bigvee_{i} \psi_i & \quad \mathit{if} \varphi \in \mathcal{L}_2 \\
\end{cases}
\end{equation*}
with $\psi_i$ being some formula from the respective logic. 
Lastly, by substitution, $\overline{\mathcal{L}_{\cup}}$ contains $a \lor (a \land b)$. Hence, $\mathcal{L}_{\cup}$ does not satisfy the closure condition and is not considered a logic.


%\section*{Exercise 15}
%\begin{quote}
%The set of formulas that are valid in a particular Kripke frame do always form a logic that extends CL.
%Prove this fact.
%\end{quote}



\section*{Exercise 16}
\begin{quote}
\begin{itemize}
\item $\some (A \land B) \lto \some  A$
\item $\some \neg A \lto  \neg \all  A$
\item $\all (A \land B) \lto (\all A \land \all B) $
\item $\all (A \lto B) \lto (\all A \lto \all B) $
\end{itemize}
Prove these facts.
\end{quote}

\begin{itemize}
\item $\some (\varphi \land \psi) \lto \some  \varphi$. Let $\mathcal{F}$ be a Kripke frame such that 
\begin{equation*}
\mathcal{F} \nmodels \some (\varphi \land \psi) \lto \some  \varphi
\end{equation*}
This corresponds to  
\begin{equation*}
\sneg (\mathcal{F} \models \some (\varphi \land \psi) \sto  \mathcal{F} \models \some  \varphi)
\end{equation*}
By reasoning on the semantic level one can transform this statement into 
\begin{equation*}
\begin{split}
&\sneg ((\sneg \mathcal{F} \models \some (\varphi \land \psi)) \sor  \mathcal{F} \models \some  \varphi) \iff \\
&\mathcal{F} \models \some (\varphi \land \psi) \sand  \sneg \mathcal{F} \models \some  \varphi 
\end{split}
\end{equation*}
Moreover, $\mathcal{F} \models \chi$ is equivalent to for all $w \in W \; \mathcal{F},w \models \chi$. Hence, consider an arbitrary $s \in W$.
\begin{equation*}
\begin{split}
&\mathcal{F},s \models \some (\varphi \land \psi) \sand  \sneg \mathcal{F},s \models \some  \varphi \iff\\
& \exists u \; sRu \sand \mathcal{F},u \models (\varphi \land \psi) \sand  \\
&\sneg (\exists u \; sRu \sand \mathcal{F},u \models \varphi) \iff  \\
& \exists u \; sRu \sand (\mathcal{F},u \models \varphi \sand \mathcal{F},u \models \psi) \sand  \\
& (\forall u \; \sneg sRu \sor  not \mathcal{F},u \models \varphi)
\end{split}
\end{equation*}
Hence, from $\mathcal{F},s \models \some (\varphi \land \psi)$ one obtains the existence of a state $u$ reachable from s, where $\mathcal{F},u \models \varphi$. However, $\sneg \mathcal{F},s \models \some  \varphi$ requires that all states are not reachable from $s$ or have to reject $\varphi$. That is, for every state $u$ reachable from $s$, $\mathcal{F},u \nmodels \varphi$. Thus $\mathcal{F}$ can not exists. Hence, $\some (\varphi \land \psi) \lto \some  \varphi$ holds for every frame.


\item $\some \neg \varphi \lto  \neg \all  \varphi$. Consider an arbitrary frame $\mathcal{F}$. Starting with the semantic evaluation. Let $s \in W$ be an arbitrary world.
\begin{equation*}
\begin{split}
&\mathcal{F},s \models \some \neg \varphi \lto  \neg \all  \varphi \iff \\
& (\exists u \; sRu \sand \sneg \mathcal{F},u \models \varphi k)\sto (\sneg (\forall u \; sRu \sto \mathcal{F},u \models \varphi)) \iff \\
& (\exists u \; sRu \sand \sneg \mathcal{F},u \models \varphi )\sto (\exists u \;  sRu \sand \sneg \mathcal{F},u \models \varphi) 
\end{split}
\end{equation*} 
Thus, given the semantics of implication, it is known that an assertion implies itself. Moreover, by choosing an arbitrary frame, one can conclude that the formula $\some \neg \varphi \lto  \neg \all  \varphi$ holds in arbitrary frames.


\item $\all (\varphi \land \psi) \lto (\all \varphi \land \all \psi)$. Consider an frame $\mathcal{F}$ such that 
\begin{equation*}
\begin{split}
\mathcal{F} \models \all (\varphi \land \psi)
\end{split}
\end{equation*} 
Let $s \in W$ be an arbitrary world. 
\begin{equation*}
\begin{split}
&\mathcal{F},s \models \all (\varphi \land \psi) \iff  \\
&\forall u \; ( \sneg sRu) \sor (\mathcal{F},u \models \varphi \sand \mathcal{F},u \models \psi) \\
\end{split}
\end{equation*} 
Now, using the semantic notion of the classical tautology of distributivity to reason on the meta level, one obtains 

\begin{equation*}
\begin{split}
&\forall u \; (( \sneg sRu \lor \mathcal{F},u \models \varphi) \sand ( \sneg sRu \lor \mathcal{F},u \models \psi))  \\
\end{split}
\end{equation*}
Moreover, in this case one can safely distribute $\forall$. That is,
\begin{equation*}
\begin{split}
&\forall u \; ( \sneg sRu \lor \mathcal{F},u \models \varphi) \sand \forall u \; ( \sneg sRu \lor \mathcal{F},u \models \psi)  \iff \\
&\forall u \; (  sRu \lto \mathcal{F},u \models \varphi) \sand \forall u \; ( sRu \lto \mathcal{F},u \models \psi)  \\
\end{split}
\end{equation*}
Now, by condensing the semantic notions described above back into the corresponding syntactic form one obtains.
\begin{equation*}
\begin{split}
\mathcal{F},s \models (\all \varphi \land \all \psi)
\end{split}
\end{equation*}
Hence, if $\all (\varphi \land \psi)$ holds in a frame, then $(\all \varphi \land \all \psi)$ must hold as well. Therefore, $\all (\varphi \land \psi) \lto (\all \varphi \land \all \psi)$ holds in any frame.

%
%\item $\all (\varphi \lto \psi) \lto (\all \varphi \lto \all \psi)$ Consider an frame $\mathcal{F}$ such that 
%\begin{equation*}
%\begin{split}
%\mathcal{F} \models \all (\varphi \lto \psi)
%\end{split}
%\end{equation*} 
%Let $s \in W$ be an arbitrary world. 
%\begin{equation*}
%\begin{split}
%&\mathcal{F},s \models \all (\varphi \lto \psi) \iff  \\
%&\forall u \; ( \sneg sRu) \sor (\mathcal{F},u \models \varphi \sto \mathcal{F},u \models \psi) \iff  \\
%&\forall u \; ( \sneg sRu) \sor (\sneg \mathcal{F},u \models \varphi \sor \mathcal{F},u \models \psi)
%\end{split}
%\end{equation*} 
%Consider the fact that $\chi$ is equivalent to $\chi \sor \chi$. Moreover, since it is known that $\sor$ adheres to commutativity and associativity. One obtains
%
%
%%Now, from classical logic, it is known that $\chi$ is equivalent to $1 \sand \chi$, i.e. an conjunction can be extended by assertions that always evaluate to true. Furthermore, . Moreover, every tautology evaluates always to $1$. That is, consider $\chi \sor \neg \chi$. Furthermore, having a disjunction where at least one part evaluates always to true, allows for arbitrary extensions of the disjunction, while retaining the evaluation. Hence, one obtains
%\begin{equation*}
%\begin{split}
%&\forall u \; ((\sneg sRu) \sor (\sneg \mathcal{F},u \models \varphi ) \sor (\mathcal{F},u \models \psi)) \iff \\
%&\forall u \; ((\sneg sRu) \sor (\sneg sRu) \sor (\sneg \mathcal{F},u \models \varphi ) \sor (\mathcal{F},u \models \psi)) \iff \\
%&\forall u \; ((\sneg sRu) \sor (\sneg \mathcal{F},u \models \varphi ) \sor (\sneg sRu) \sor  (\mathcal{F},u \models \psi)) \\
%&\forall u \; ((1 \sand \sneg sRu) \sor (\sneg \mathcal{F},u \models \varphi ) \sor (\mathcal{F},u \models \psi))) \iff \\
%&\forall u \; ((sRu \sor \sneg sRu) \sand \\
%&((\sneg sRu) \sor (\sneg \mathcal{F},u \models \varphi ) \sor (\mathcal{F},u \models \psi))) \iff \\
%&\forall u \; ((sRu \sor (\sneg sRu) \sor \mathcal{F}, s \models \psi) \\
%&\sand ((\sneg sRu) \sor (\sneg \mathcal{F},u \models \varphi ) \sor (\mathcal{F},u \models \psi))) \\
%\end{split}
%\end{equation*} 
%By using the semantic notion of distributivity, one can factor this statement into 
%\begin{equation*}
%\begin{split}
%&\forall u \; ((sRu \sand (\sneg \mathcal{F},u \models \varphi )) \sor (\sneg sRu \sor \mathcal{F},u \models \psi )) \iff \\
%&\forall u \; (\sneg (\sneg sRu \sor  \mathcal{F},u \models \varphi ) \sor (\sneg sRu \sor \mathcal{F},u \models \psi )) \iff \\
%&\forall u \; (\sneg (sRu \sto  \mathcal{F},u \models \varphi ) \sor (sRu \sto \mathcal{F},u \models \psi )) \iff \\
%&\forall u \; (sRu \sto  \mathcal{F},u \models \varphi ) \sto (sRu \sto \mathcal{F},u \models \psi ))\\
%\end{split}
%\end{equation*} 
%Now consider the classical first order validity $\forall x (P(x) \lto Q(x)) \lto (\forall x P(x) \lto \forall x  Q(x)))$. That is, assume there exists an interpretation $\mathcal{I}$ such that  $\mathcal{I}\nmodels \forall x (P(x) \lto Q(x)) \lto (\forall x P(x) \lto \forall x  Q(x)))$. To that extend it must be that $\forall x (P(x) \lto Q(x)) $ must hold under $\mathcal{I}$ and $(\forall x P(x) \lto \forall x  Q(x))$ does not. If there exists a $c$ in the domain such that $v_{\mathcal{I}}(P(c))=0$ it would follow that $\mathcal{I} \models (\forall x P(x) \lto \forall x  Q(x))$. Hence, $\mathcal{I} \models \forall x P(x)$. However, now in order to satisfy the premise of the big implication, it must be that $\mathcal{I} \models \forall x Q(x)$. Hence, allowing the conclusion that $\mathcal{I}$ can not exist. Using this reasoning on the meta-level. One obtains
%\begin{equation*}
%\begin{split}
%&\forall u \; (sRu \sto  \mathcal{F},u \models \varphi ) \sto (sRu \sto \mathcal{F},u \models \psi )) \implies\\
%&\forall u \; (sRu \sto  \mathcal{F},u \models \varphi ) \sto  \forall u \; (sRu \sto \mathcal{F},u \models \psi ))
%\end{split}
%\end{equation*} 
%Now translating the semantic notion described above, back into syntax one obtains
%\begin{equation*}
%\begin{split}
%\mathcal{F},s \models \all \varphi \lto \all \psi
%\end{split}
%\end{equation*}
%Hence, by a similar argument to the previous point, one can conclude $\all (\varphi \lto \psi) \lto (\all \varphi \lto \all \psi)$ holds in arbitrary frames.
\end{itemize}

\section*{Exercise 17}
\begin{quote}
Which of the above implicative formulas can/cannot be inverted? Provide either a proof or a counter-example in each case.
\end{quote}
\begin{itemize}
\item $\some  \varphi \lto \some (\varphi \land \psi)$. Consider the Kripke model $\mathcal{M}:=\langle \{s,t\}, \{(s,t)\}, V \rangle$ such that $V(\varphi):=\{t\}$. Clearly 
\begin{equation*}
\begin{split}
&\mathcal{M},s \models \some  \varphi \iff \\
&\exists u \; sRu \sand \mathcal{M},u \models  \varphi  \\
\end{split}
\end{equation*} 
holds, due to $t \in V(\varphi)$. However,  
\begin{equation*}
\begin{split}
&\mathcal{M},s \models \some  (\varphi \land \psi) \iff \\
&\exists u \; sRu \sand (\mathcal{M},u \models  \varphi \sand \mathcal{M},u \models  \psi ) \\
\end{split}
\end{equation*} 
can not hold. That is, $\mathcal{M},u \nmodels  \psi$, because $t \notin V(\psi)$.



\item $ \neg \all  \varphi \lto  \some \neg \varphi $ As established in the previous exercise 
\begin{equation*}
\begin{split}
&\mathcal{F},s \models \some \neg \varphi \lto  \neg \all  \varphi \iff \\
& (\exists u \; sRu \sand \sneg \mathcal{F},u \models \varphi )\sto (\exists u \;  sRu \sand \sneg \mathcal{F},u \models \varphi)
\end{split}
\end{equation*} 
Hence, allowing the conclusion 
\begin{equation*}
\begin{split}
& (\exists u \; sRu \sand \sneg \mathcal{F},u \models \varphi )\sto (\exists u \;  sRu \sand \sneg \mathcal{F},u \models \varphi) \iff \\
&\mathcal{F},s \models  \neg \all  \varphi \lto \some \neg \varphi\\
\end{split}
\end{equation*} 
Hence, $ \neg \all  \varphi \lto  \some \neg \varphi $ holds in arbitrary frames.


\item $(\all \varphi \land \all \psi)  \lto  \all (\varphi \land \psi)$. By observing the proof in the exercise above, one observe that every transformation made was bidirectional. Hence, from $\mathcal{F} \models  \all (\varphi \land \psi)$ one can trace the arguments back to $\mathcal{F} \models  (\all \varphi \land \all \psi)$.  Therefore, $(\all \varphi \land \all \psi)  \lto  \all (\varphi \land \psi)$ holds in every frame.

\item $(\all \varphi \lto \all \psi) \lto \all (\varphi \lto \psi)$. In contrast to the previous formula, there are Kripke models where this formula does not hold. Consider $\mathcal{M}:=\langle \{r,s,t\}, \{(s,r),(s,t)\}, V\rangle$ where $V(\varphi):=\{r\}$ and $V(\psi):=\{\}$. Now, one can observe
\begin{equation*}
\begin{split}
& \mathcal{M}, s \models \all \varphi \lto \all \psi \iff \\
& (\forall u \; sRu \sto \mathcal{M}, u \models \varphi) \sto (\forall u \; sRu \sto \mathcal{M}, u \models \psi) \\
\end{split}
\end{equation*} 
Clearly, the premise of the implication evaluates to false. That is, consider $u$ to be $t$ as a world reachable from $s$ where $\mathcal{M}, t \nmodels \varphi$. Hence, by the semantics of implication in classical logic, the whole statement holds. Hence, given the assumption that the whole formula holds, it must be that 
\begin{equation*}
\begin{split}
&\mathcal{M}, s \models (\all \varphi \lto \all \psi) \lto \all (\varphi \lto \psi) \iff \\
& 1 \sto (\forall u \; sRu \sto (\mathcal{M}, u \models \varphi \sto \mathcal{M}, u \models \psi)
\end{split}
\end{equation*} 
Now, try to evaluate the right part of the implication at state $r$. Firstly, $r$ is reachable from $s$. Secondly, $\mathcal{M}, r \models \varphi$ as well as  $\mathcal{M}, r \nmodels \psi$ by construction. 
Therefore, $\mathcal{M}, r \nmodels \varphi \lto \psi$. Hence, $\mathcal{M}, s \nmodels (\all \varphi \lto \all \psi) \lto \all (\varphi \lto \psi)$.
\end{itemize}


\section*{Exercise 18}
\begin{quote}
For any Kripke interpretation $\mathcal{M}$:
\begin{itemize}
\item If $\mathcal{M} \models A$ then $\mathcal{M} \models \all A$
\end{itemize}
It follows, that also for all frames $\mathcal{F}$:
\begin{itemize}
\item If $\mathcal{F} \models A$ then $\mathcal{F} \models \all A$
\end{itemize}
\end{quote}

Firstly, let $\mathcal{M}$ be an arbitrary Kripke interpretation and $\varphi$ a formula. Given $\mathcal{M} \models \varphi$ actually expresses, for all $w \in W \; \mathcal{M},w \models \varphi$. Consider an arbitrary state $s$. By assumption $\mathcal{M},s \models \varphi$. If $s$ is a terminal state, then $\mathcal{M} \models \all \varphi$. Otherwise let $t$ be an arbitrary state satisfying $sRt$.
By assumption $\mathcal{M},t \models \varphi$. However, since this holds for an arbitrary state it follows that $\forall u \; sRu \sto \mathcal{M}, u \models \varphi$. However, this is precisely the semantics of $\mathcal{M}, s \models \all \varphi$. Moreover, since $s$ was chosen arbitrarily, $\mathcal{M}\models \all \varphi$ follows. Furthermore, at no point in the proof the variable assignment $V$ was used. Hence, the same result holds for arbitrary frames $\mathcal{F}:=\langle W,R \rangle$.

\section*{Exercise 19}
\begin{quote}
Prove the soundness of $\mathbf{K}$ (using mentioned facts).
\end{quote}

This can be done by induction on the length of the proof. To that end, consider the induction hypothesis, a proof in $\mathbf{K}$ of length $n$ is sound.

\begin{itemize}
\item Induction base, i.e.\ $n=0$. Since, the system described on the slides does not include reasoning from a theory $\Gamma$, it suffices to show that the axioms of $\mathbf{K}$ are valid. Hence, let $\mathcal{F}$ be an arbitrary frame and $s \in W$ be an arbitrary state. Let $\varphi$ be an axiom of $\mathbf{K}$.
\begin{enumerate}
\item Case: $\varphi$ is a tautology in classical logic. It holds that $\mathcal{F},s\models \varphi$, because the semantic of the connectives $\land, \lor, \lto$ and $\neg$ is the same as in classical logic.
\item Case: $\varphi= \all (\psi \lto \chi) \lto (\all \psi \lto \all \chi)$. For a proof of its validity in arbitrary frames consult exercise 16.
\end{enumerate}
\item Induction step, i.e.\ $n=m+1$. Let $\varphi$ be a formula with a proof of length $m+1$ in $\mathbf{K}$. Within the system, there are two derivation rules. Hence, the last step in the derivation of $\varphi$ is either an application of \emph{(MP)} or \emph{(NC)} (necessitation).
\begin{enumerate}
\item Case: (MP). If the last step of derivation is (MP) then the proof is of the form
     \begin{prooftree}
        
              \AxiomC{$\vdots$}
              \noLine
              \UnaryInfC{$\psi$}
                   \AxiomC{$\vdots$}
                   \noLine
                   \UnaryInfC{$\psi \lto \varphi$}
              \RightLabel{\scriptsize(MP)}
              \BinaryInfC{$\varphi$}
     \end{prooftree}
Clearly, $\psi$ and $\psi \lto \varphi$ have a derivation of length $\leq m$. Hence, by induction hypothesis, one can conclude that for an arbitrary frame $\mathcal{F}$,  $\mathcal{F} \models \psi $ and $\mathcal{F} \models \psi \lto \varphi$ hold. Now consider for an arbitrary $s\in W$
\begin{equation*}
\begin{split}
&\mathcal{F}, s \models  \psi \sand \mathcal{F}, s \models  \psi \lto \varphi  \iff \\
&\mathcal{F}, s \models  \psi \sand (\mathcal{F}, s \models  \psi \sto \mathcal{F}, s \models \varphi) 
\end{split}
\end{equation*} 
Hence, by the semantics of implication one can conclude $\mathcal{F}, s \models \varphi$, i.e. $\mathcal{F}, s \models  \psi$ and $\mathcal{F}, s \nmodels  \psi$ can not hold at the same time. Lastly, because $s$ arbitrary, it follows $\mathcal{F}\models \varphi$. Thus, (MP) is sound.

\item Case: (NC). If the last step of derivation is (NC) then $\varphi = \all \psi$ and the proof has the form
     \begin{prooftree}
              \AxiomC{$\vdots$}
              \noLine
              \UnaryInfC{$\psi$}
              \RightLabel{\scriptsize(MP)}
              \UnaryInfC{$\all \psi$}
     \end{prooftree}
Again, the derivation of $\psi$ is smaller than $m+1$, thus by induction hypothesis it is a sound derivation. Thereby, for an arbitrary frame $\mathcal{F}$, $\mathcal{F}\models \psi$ holds, meaning that $\psi$ holds in every state. Picking an arbitrary state $s \in W$. Obviously, $\mathcal{F},s\models \psi$. Moreover, since $\psi$ holds in every state, it holds in particular in any state reachable from $s$ (if $s$ is terminal, $\all \psi$ holds vacuously). Hence,
\begin{equation*}
\begin{split}
&\forall u \; sRu \sto \mathcal{F},u \models \psi \iff \\
& \mathcal{F},s \models \all \psi
\end{split}
\end{equation*}
Since $s$ arbitrary $\mathcal{F}\models \all \psi$. Hence, the soundness of (NC) is demonstrated.
\end{enumerate} 
\end{itemize}
Hence, one can conclude that $\mathbf{K}$ is sound.





\end{document}
