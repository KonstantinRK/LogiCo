

\documentclass[11pt,a4paper]{article}
\usepackage{amsmath}
\usepackage{amssymb}
\usepackage{enumitem}
\usepackage{amsthm}
\usepackage{MnSymbol}
\setlength{\parindent}{0pt}
\usepackage[utf8]{inputenc}
\usepackage{listings} [python]
\usepackage{url}
\usepackage{bussproofs}
\usepackage{rotating}
 
\usepackage{hyperref}

\newtheorem{theorem}{Theorem}[section]
\newtheorem{corollary}{Corollary}[theorem]
\newtheorem{lemma}[theorem]{Lemma}
\newtheorem{mydef}{Definition}

%opening


\newcommand{\lto}{\supset}
\newcommand{\some}{\Diamond}
\newcommand{\all}{\Box}

\newcommand{\sand}{\; and \;}
\newcommand{\sor}{ \; or \;}
\newcommand{\sneg}{not \;}
\newcommand{\sto}{\Rightarrow}
\newcommand{\negmodels}{\nvDash}

\begin{document}

%\maketitle


\section*{Exercise 9}
\begin{quote}
Derive the above in detail from $\some = \neg \all \neg $ and from the semantics for $\all F$ and $\neg F$ classical (first order) reasoning (at the meta-level), indicating at each step what is used.
\end{quote}

Let $\varphi$ be a formula, $\mathcal{M}$ be an arbitrary Kripke model. and $w$ an arbitrary world. Given the syntactic definition of $\some$, the formula $\some \varphi$ can be rewritten as such
\begin{equation*}
\begin{split}
v_{\mathcal{M}}(\some \varphi, w) \stackrel{syntax}{=} v_{\mathcal{M}}(\neg \all \neg \varphi,w)
\end{split}
\end{equation*}

One has to switch onto a semantic level.

\begin{equation*}
\begin{split}
v_{\mathcal{M}}(\neg \all \neg \varphi, w)=
\begin{cases}
1 & \quad \mathit{if}  \; \mathit{not} \; (v_{\mathcal{M}}(\all \neg \varphi, w)=1) \\
0 & \quad otw.
\end{cases}
\end{split}
\end{equation*}
iff
\begin{equation*}
\begin{split}
v_{\mathcal{M}}(\neg \all \neg \varphi, w)=
\begin{cases}
1 & \quad \mathit{if}  \; \mathit{not} \; (\forall u ( wRu  \Rightarrow (v_{\mathcal{M}}(\neg \varphi, u)=1))) \\
0 & \quad otw.
\end{cases}
\end{split}
\end{equation*}
iff
\begin{equation*}
\begin{split}
v_{\mathcal{M}}(\neg \all \neg \varphi, w)=
\begin{cases}
1 & \quad \mathit{if}  \; \mathit{not} \; (\forall u ( wRu  \Rightarrow (not \; (v_{\mathcal{M}}(\varphi, u)=1)))) \\
0 & \quad otw.
\end{cases}
\end{split}
\end{equation*}
If something does not hold for all elements, then there exists an element for which its negation holds. Therefore, 
\begin{equation*}
\begin{split}
v_{\mathcal{M}}(\neg \all \neg \varphi, w)=
\begin{cases}
1 & \quad \mathit{if}  \;  \exists u (not \; ( wRu  \Rightarrow (not \; (v_{\mathcal{M}}(\varphi, u)=1)))) \\
0 & \quad otw.
\end{cases}
\end{split}
\end{equation*}
Since reasoning on the meta level is classic, one can rewrite $A \Rightarrow B$ as $(not \; A) \; or \; B$. That is, an implication is always satisfied, except if $A$ holds and $B$ does not. Hence,
\begin{equation*}
\begin{split}
v_{\mathcal{M}}(\neg \all \neg \varphi, w)=
\begin{cases}
1 & \quad \mathit{if}  \;  \exists u (not \; ( (not \; wRu) \; or \; (not \; (v_{\mathcal{M}}(\varphi, u)=1)))) \\
0 & \quad otw.
\end{cases}
\end{split}
\end{equation*}
Again, reasoning on the meta-level is done with respect to classical logic, allowing the appeal to the semantic form of the DeMorgan laws. That is, if a disjunction $A \; or \; B$ does not hold. Then $(not \; A) \; and \; (not \; B)$ must be the case, and vice versa. 
\begin{equation*}
\begin{split}
v_{\mathcal{M}}(\neg \all \neg \varphi, w)=
\begin{cases}
1 & \quad \mathit{if}  \;  \exists u  ( (not \; (not \; wRu)) \; and \; (not \;(not \; (v_{\mathcal{M}}(\varphi, u)=1)))) \\
0 & \quad otw.
\end{cases}
\end{split}
\end{equation*}
Now consider the fact that (in classical logic) by negating an already negated assertion one obtains the original assertion.
\begin{equation*}
\begin{split}
v_{\mathcal{M}}(\neg \all \neg \varphi, w)=
\begin{cases}
1 & \quad \mathit{if}  \;  \exists u  (  wRu \; and \;  (v_{\mathcal{M}}(\varphi, u)=1)) \\
0 & \quad otw.
\end{cases}
\end{split}
\end{equation*}
Lastly, by using the syntactic definition of $\some$ one obtains
\begin{equation*}
\begin{split}
v_{\mathcal{M}}(\some \varphi, w)=
\begin{cases}
1 & \quad \mathit{if}  \;  \exists u  (  wRu \; and \;  (v_{\mathcal{M}}(\varphi, u)=1)) \\
0 & \quad otw.
\end{cases}
\end{split}
\end{equation*}

\section*{Exercise 11}
\begin{quote}
Find Kripke models in which the following formulas are true in some world. If possible also find Kripke models in which the formulas are true in every world.
(Try to use as few worlds as possible.)
\begin{itemize}
\item $\some p \land \some \all p \land \neg \all p$
\item $p \land \all p \land \neg \some p$
\item $(p \lto q) \land \some (p \land \neg q)$
\item $\neg p \land \some \some p \land \neg \all \some p \land \some \all  \neg p$
\end{itemize}
\end{quote}

\begin{itemize}
\item $\some p \land \some \all p \land \neg \all p$. Consider $\mathcal{M}:=\langle \{s,t\}, \{(s,s),(s,t)\}, V\rangle$ such that $V(p):=\{t\}$. Now consider
\begin{equation*}
\begin{split}
&\mathcal{M}, s \models \some p \land \some \all p \land \neg \all p  \iff \\
&\mathcal{M}, s \models \some p \sand \mathcal{M}, s \models \some \all p \sand \mathcal{M}, s \models \neg \all p \iff \\
&(\exists u \; sRu \sand \mathcal{M}, u \models  p) \sand \\
&( \exists u \; sRu \sand (\forall w \; uRw \sto \mathcal{M}, w \models  p)) \sand \\
&\sneg (\forall u \; sRu \sto \mathcal{M}, u \models p)
\end{split}
\end{equation*}
For $(\exists u \; (sRu \sand \mathcal{M}, u \models  p))$ consider $u$ to be $t$, then $sRt$ and $\mathcal{M}, u \models  p$ because $t \in V(p)$. For $( \exists u \; (sRu \sand (\forall w \; uRw \sto \mathcal{M}, w \models  p)))$  consider $u$ to be $t$. Clearly, $sRt$ and since $t$ is a terminal state $(\forall w \; uRw \sto \mathcal{M}, w \models  p)$ is vacuously true. Lastly, from $\sneg (\forall u \; sRu \sto \mathcal{M}, u \models p)$ one obtains $ (\exists u \; sRu \sand( \sneg \mathcal{M}, u \models p))$. Consider $u$ to be $s$ and since $s \notin V(p)$ it follows that $\mathcal{M}, s \nvDash p$, satisfying the last part of the conjunction. 

Moreover, assume $\mathcal{N}$, such that for all $s \in W$ $\mathcal{N},s \models \some p \land \some \all p \land \neg \all p$. According to $\some \all p $ there $\exists$ a state $a$ such that $\forall w$ reachable from $a$, i.e. $aRw$ it must be that $w \in V(p)$. That is, $\mathcal{N},a \models \all p$. However, this is directly in contradiction with $\mathcal{N},a \models  \neg \all p$. Hence, $\mathcal{N}$ can not exist.

Lastly, there can not be a model where $|W|\leq 1$, as there must be a state where $p$ holds and one where $\neg p$ holds. Having only one state this is not possible.

\item $p \land \all p \land \neg \some p$. Consider $\mathcal{M}:=\langle \{s\}, \{\}, V\}\rangle$ such that $V(p):=\{s\}$. Now consider
\begin{equation*}
\begin{split}
&\mathcal{M}, s \models p \land \all p \land \neg \some p \iff \\
&\mathcal{M}, s \models p \sand  \\
&\forall u \; (sRu \sto \mathcal{M}, u \models p \sand ) \\
&\sneg( \exists u \;( sRu \sand \mathcal{M}, u \models p))
\end{split}
\end{equation*}
The fist part of the conjunction holds because $s \in V(p)$. The second part is vacuously true due to $R=\{\}$. The third part is equivalent to  $\forall u \; ((\sneg sRu) \sor (\sneg \mathcal{M}, u \models p))$ which again is vacuously true due to $sRu$ constantly evaluating to false. 


Obviously, the whole formula holds in all worlds. Moreover, since $W>0$  by definition, this model is minimal.


\item $(p \lto q) \land \some (p \land \neg q)$. Consider $\mathcal{M}:=\langle \{s,t\}, \{(s,t)\}, V\}\rangle$ such that $V(p):=\{t\}$. Now consider
\begin{equation*}
\begin{split}
&\mathcal{M}, s \models (p \lto q) \land \some (p \land \neg q)  \iff \\
&\mathcal{M}, s \models (p \lto q) \sand  \\
&\exists u \; (sRu \sand \mathcal{M}, u \models (p \land \neg q)) \iff \\
&\mathcal{M}, s \models p \sto \mathcal{M}, s \models q \sand  \\
& \exists u \; (sRu \sand (\mathcal{M}, u \models p \sand (\sneg \mathcal{M}, u \models q))) \\
\end{split}
\end{equation*}
Since $s \notin V(p)$ it follows that $\mathcal{M}, s \nmodels p $, thus by semantics of material implication, the first part of the conjunction holds. As for the second part. Consider $u$ to be $t$. Because $t\in V(p)$ and $t \notin V(q)$ it follows that  $\mathcal{M}, t \models p$ and $\mathcal{M}, t \nmodels q $. Thereby, satisfying the second part of the conjunction.

Moreover, assume there exists a model $\mathcal{N}$ such that the formula holds in every state. Let $s$ be an arbitrary state. The statement $\mathcal{N}, s \models \some (p \land \neg q)$ demands the existence of a state $a$ such that $\mathcal{N}, a \models (p \land \neg q)$. However, this requires  $\mathcal{N}, a \nmodels (p \lto q)$. Hence, by contradiction, $\mathcal{N}$ does not exist.


Lastly, there can not be a model where $|W|\leq 1$, as otherwise $p \land \neg q$ and $p \lto q$
must hold in the same state, which as established above can not be the case.

\item $\neg p \land \some \some p \land \neg \all \some p \land \some \all  \neg p$. Consider $\mathcal{M}:=\langle \{s,t\}, \{(s,t),(s,s)\}, V\}\rangle$ such that $V(p):=\{t\}$. Now consider
\begin{equation*}
\begin{split}
&\mathcal{M}, s \models \neg p \land \some \some p \land \neg \all \some p \land \some \all  \neg p \iff \\
&\mathcal{M}, s \nmodels p \sand \\
&\exists u \; sRu \sand (\exists w \; uRw \sand \mathcal{M}, w \models  p) \sand \\
&\sneg ( \forall u \; sRu \sto (\exists w \; uRw \sand \mathcal{M}, w \models p)) \sand  \\
&\exists u \; sRu \sand (\forall w \; uRw \sto \mathcal{M}, w \nmodels  p)
\end{split}
\end{equation*}
Since $s \notin V(p)$ it follows that $\mathcal{M}, s \nmodels p $ satisfying the first part of the conjunction. Let $u$ be $s$ and let $w$ be $t$. Clearly $sRs$ and $sRt$. Moreover, since $t \in V(p)$ $\mathcal{M},t\models p$. Hence, satisfying the second part. After some transformation on the meta-level one obtains 
\begin{equation*}
\begin{split}
\exists u \; sRu \sand (\forall w \; \sneg uRw \sor \mathcal{M}, w \nmodels p))
\end{split}
\end{equation*}
Consider $u$ to be $s$. Since, $sRt$ and $\mathcal{M}, w \models p$. Neither $\sneg uRw$ nor $\mathcal{M}, t \nmodels p$ hold. Hence, by the semantics of $\forall$, the third conjunct is satisfied. 
%Consider $u$ to be $s$. Since $s \notin V(p)$ implies $\mathcal{M},s \nmodels p$ and since $t$ is only related to itself, $\forall w \; \sneg uRw \sor \mathcal{M}, w \nmodels p$ holds. Hence, the third conjunct is satisfied. 
The last part of the conjunction is satisfied as well. This is because, $sRt$ and because $t$ is terminal. Making $\mathcal{M}, t \models \all \neg p$, hold trivially.

Lastly, assume there exists a model $\mathcal{N}$ such that the formula holds in every state. Let $s$ be an arbitrary state. The statement $\mathcal{N}, s \models \some \some p $ demands the existence of a state $a$ such that $\mathcal{N}, a \models p$. However, this requires  $\mathcal{N}, a \nmodels  \neg p$. Hence, the assumption of $\mathcal{N}$ is contradicted. Moreover, there can not be a model where $|W|\leq 1$, as otherwise $p$ and $\neg p$ must hold in the same state.

\end{itemize}


\section*{Exercise 12}
\begin{quote}
Find 3-5 further (interesting) examples of modal formulas with one schematic variable that are valid in F, above, such that removal of some (which?) accessibilities leads to invalidity.
\end{quote}
The frame in question is $\mathcal{F}:=\langle \{u,w\}, \{(u,u),(w,w),(u,w),(w,u)\}\rangle$
\begin{itemize}
\item $\some p \lto (\all p \lto p)$, by removing $(w,w)$ the assignment $V(p):\{u\}$ induces a model where the formula does no longer hold in $w$.

\item $((\all p \land p) \lto \all p) \lto (\all \some \all \some p \lor \neg p)$, by removing $(w,u), (u,u)$ the assignment $V(p):\{w, u\}$ induces a model where the formula does no longer hold in $w$.
\item $\all p \lto \some p$, by removing all edges the formula does no longer hold.
\item $(\some p \land \all(p \lto \all p)) \lto p$, by removing $(w,u)$ the assignment $V(p):\{u\}$ induces a model where the formula does no longer hold in $w$.
\item $(\all (\some p \lto p) \land \some p) \lto \all p$, by removing $(w,u)$ the assignment $V(p):\{w\}$ induces a model where the formula does no longer hold in $w$.
\end{itemize}



\section*{Exercise 13}
\begin{quote}
Show that the intersection of two logics is also a logic. What about unions of logics?
\end{quote}
Let $\mathcal{L}$ be a logic and $X \subseteq \mathcal{L}$. Firstly, it has to be established that the closure of $X$, written as $\overline{X}$, is also a subset of $\mathcal{L}$, i.e. $X \subseteq \mathcal{L} \sto \overline{X}\subseteq \mathcal{L}$. Assume the contrary, i.e. $\exists \varphi \in \overline{X}$ and $\varphi \notin \mathcal{L}$. However, by definition this can not be the case, because otherwise the closure condition would be violated.

Let $\mathcal{L}_1$ and $\mathcal{L}_2$ be two logics and let $\mathcal{L}_{\cap} = \mathcal{L}_1 \cap \mathcal{L}_2$.  
%Assume $\mathcal{L}^*$ to be not a logic. That is, $\exists \varphi \in \overline{\mathcal{L}^*} \setminus \mathcal{L}^*$. 
Clearly $\mathcal{L}_{\cap} \subseteq \overline{\mathcal{L}_{\cap}} \subseteq \mathcal{L}_x$ for $x \in \{1,2\}$. That is, $\mathcal{L}_{\cap}$ is a subset of $\mathcal{L}_1$ and of $\mathcal{L}_{2}$. Moreover, as already established. If both $\mathcal{L}_1$ and $\mathcal{L}_{2}$ contain $\mathcal{L}_{\cap}$, then they must contain $\overline{\mathcal{L}_{\cap}}$. Hence, $\overline{\mathcal{L}_{\cap}}=\mathcal{L}_1 \cap \mathcal{L}_2$. Therefore, $\mathcal{L}_{\cap} = \overline{\mathcal{L}_{\cap}}$. \\


Let $\mathcal{L}_1:=\overline{\{a \land b\}}$ and $\mathcal{L}_2:=\overline{\{a \lor b\}}$.  In both cases the only closure operation applicable is substitution, i.e. (MP) requires an implication and there is no implication where (MP) could be applied. Consider $\mathcal{L}_{\cup}:= \mathcal{L}_1 \cup \mathcal{L}_2$. Moreover, it can be observed that $\varphi \in \mathcal{L}_{\cup}$ is of the form
\begin{equation*}
\varphi = \begin{cases}
\bigwedge_{i} p_i  & \quad \mathit{if} \varphi \in \mathcal{L}_1 \\
\bigvee_{i} p_i & \quad \mathit{if} \varphi \in \mathcal{L}_2 \\
\end{cases}
\end{equation*}
with $p_i$ being some propositional atom. 
Lastly, by substitution, $\overline{\mathcal{L}_{\cup}}$ contains $a \lor (a \land b)$. Hence, $\mathcal{L}_{\cup}$ does not satisfy the closure condition and is not considered a logic.\\

Note: The logics $\mathcal{L}_1:=\overline{\{a\}}$ and $\mathcal{L}_2:=\overline{\{a \lto b\}}$ with $\mathcal{L}_{\cup}:=\mathcal{L}_1 \cup \mathcal{L}_2$ together with the argument that $b \nin \mathcal{L}_{\cup}$ but $b \in \overline{\mathcal{L}_{\cup}}$ is not correct. Due to the fact that by substitution $(a \lto b) \lto b \in \mathcal{L}_2$, thus resulting (by MP) in $b \in \mathcal{L}_2$.

\section*{Exercise 14}
\begin{quote}
Find an $\mathcal{M}$ s.t. $\{A \mid \mathcal{M} \models A\}$ is a logic. (Prove your claim.)
\end{quote}

Consider the model $\mathcal{M}:= \langle \{s\}, \{(s,s)\}, V\rangle$ where $\forall p \in PV \; (V(p):=\{s\})$ and the language $\mathcal{L}:=\{A \mid \mathcal{M} \models A\}$. Claim $\mathcal{L}$ is a logic.

Firstly, as there is only one state, choosing an arbitrary state is the same as choosing a particular state. Moreover, one of the intuitive reasons why every propositional variable is set to true is to prevent: That from $\neg p \in \mathcal{L}$ by substitution $\neg \neg p \in \overline{\mathcal{L}}$ and by tautology and (MP) $p \in \mathcal{L}$ but $\mathcal{M} \nmodels p$.  \\

Secondly, it has to be demonstrated that $\mathcal{M} \models \all \varphi$ if and 
only if $\mathcal{M} \models \varphi$ and $\mathcal{M} \models \some\varphi$ if and only if $\mathcal{M} \models \varphi$. Since $\mathcal{M} \models \all \varphi$ is equivalent to $\forall t (sRt \sto \mathcal{M},t \models \varphi)$, which in this case is equivalent to $sRs \sto \mathcal{M},s \models \varphi$, again given the specific model this is equivalent to $\mathcal{M} \models \varphi$. Analogue for the other equivalence. Therefore, every formula that is satisfied in $\mathcal{M}$ \\

Thirdly, show that $\mathcal{L}$ is closed under (MP). To do so, consider $\varphi \in \mathcal{L}$, and $\varphi \lto \psi \in \mathcal{L}$. By definition $\mathcal{M} \models \varphi$  and $\mathcal{M} \models \varphi \lto \psi$. Now by semantics of $\lto$ one obtains
\begin{equation*}
\mathcal{M} \models \varphi \sand (\mathcal{M} \models \varphi \sto \mathcal{M} \models \psi)
\end{equation*}
Therefore, $\mathcal{M} \models \psi$, which implies that $\psi \in \mathcal{M}$. Hence, $\mathcal{M}$ is closed under (MP). \\

Lastly, show that $\mathcal{L}$ is closed under substitution. The intuitive idea behind this proof is the following. Every propositional variable in a formula evaluates to true in $\mathcal{M}$, i.e. $\mathcal{M} \models p$. Moreover, by definition for every $\varphi \in \mathcal{L}$ it holds that $\mathcal{M} \models \varphi$. Hence, any substitution $\pi$ will substitute a propositional variable with a propositional formula of the same truth value. Hence, the evaluation of the whole formula can not be effected by this substitution. Moreover, the truth value of formulas with modality as most external "connective" (?), is solely determined by the truth value of its internal formula. Therefore, one can neglect to engage with modalities in particular. \\

Let $\varphi \in \mathcal{L}$ and let the family $(p_i)_{i \in I}$ such that $\{ p_i \mid i \in I\} = PV$. Moreover, let $\pi_n$ be a substitution such that
\begin{itemize}
\item $\pi_n(\chi \land \psi) = \pi_n(\chi) \land \pi_n(\psi) $
\item $\pi_n(\chi \lor \psi) = \pi_n(\chi) \lor \pi_n(\psi) $
\item $\pi_n(\chi \lto \psi) = \pi_n(\chi) \lto \pi_n(\psi) $
\item $\pi_n(\neg \psi) =\neg \pi_n(\psi) $
\item $\pi_n(\all \psi) =\all \pi_n(\psi) $
\item $\pi_n(\some \psi) =\some \pi_n(\psi) $
\item \begin{equation*}
\pi_n(q) = \begin{cases}
x  & \quad \mathit{for} \; x \in \{q, \psi\} \; \mathit{with} \; \psi \in \mathcal{L} \quad if \; q=p_i \land i\leq n\\
q & \quad otw.\\
\end{cases}
\end{equation*}
\end{itemize}

Show by induction on $n$, i.e. on the length of parallel replacement of propositional variables, that $\mathcal{M} \models \pi_n(\varphi)$. 
\begin{itemize}
\item \textbf{IH:}  for $\varphi \in \mathcal{L}$, $\mathcal{M} \models \pi_n(\varphi)$.
\item \textbf{IB:}  $n=0$. All propositional variables in $\varphi$ are "replaced" by them self. Hence, $\pi_0(\varphi)=\varphi$. By definition $\mathcal{M} \models \varphi$, thus  $\mathcal{M} \models \pi_0(\varphi)$
\item \textbf{IS:} If all propositional variables of $\varphi$ are in $\{p_i \mid i < n+1\}$ then $\pi_{n+1}$ can be restricted to $\pi_n$. Otherwise, $p_{n+1}$ occurs in $\varphi$. If $\pi_{n+1}(p_{n+1})=p_{n+1}$, the same restriction can be made. Lastly, if $\pi_{n+1}(p_{n+1})=\psi$. Firstly, rename all occurrences of $p_{n+1}$ in $\varphi$ to some fresh variable $v$ (a propositional variable not used in the formula), let this operation be called $\tau$. This is, possible since the formula is finite and by definition all propositional variables are assigned the value true. Secondly, restrict $\pi_{n+1}$ to $\pi_n$ and apply it to $\varphi$. By IH we thus have $\mathcal{M} \models \pi_n(\tau(\varphi))$. Thirdly, replace all occurrences of $v$ by $\psi$, let this operation be called $\rho$. Clearly, $\pi_{n+1}=(\rho \circ \pi_n \circ \tau)$. Now, by definition $\mathcal{M}\models \psi$ and $\mathcal{M} \models v$, as well as $\mathcal{M} \models \varphi$. Hence, by replacing one propositional variable with the truth value true, by a formula with the truth value true, the truth value of the resulting formula remains the same (truth functionality). Hence, $\mathcal{M} \models \pi_{n+1}(\varphi)$
\end{itemize}

Therefore, it can be concluded that $\mathcal{M}$ is closed under substitution.

\section*{Exercise 15}
\begin{quote}
The set of formulas that are valid in a particular Kripke frame do always form a logic that extends CL.
Prove this fact.
\end{quote}

\begin{enumerate}
\item Show for an arbitrary frame $\mathcal{F}$ that $\mathcal{L}:=\{\varphi \mid \mathcal{M} \models \varphi\}$ is a logic.
\begin{enumerate}
\item Show that $\mathcal{L}$ is closed under (MP). To do so, consider $\varphi \in \mathcal{L}$, and $\varphi \lto \psi \in \mathcal{L}$. By definition $\mathcal{F} \models \varphi$  and $\mathcal{F} \models \varphi \lto \psi$. Now by semantics of $\lto$ one obtains
\begin{equation*}
\mathcal{F} \models \varphi \sand (\mathcal{F} \models \varphi \sto \mathcal{F} \models \psi)
\end{equation*}
Therefore, $\mathcal{F} \models \psi$, which implies that $\psi \in \mathcal{L}$. Hence, $\mathcal{L}$ is closed under (MP).
\item Show that $\mathcal{L}$ is closed under substitution. Starting with the following intuitive argument. Consider $\varphi \in \mathcal{L}$, clearly $\varphi$ is valid in $\mathcal{F}$. Hence, for every possible variable assignment the formula $\varphi$ will evaluate to true. This holds especially for the a variable assignment assigning all atoms the truth value true. Moreover, as $\varphi$ arbitrary, every formula in $\mathcal{L}$ evaluates to true. Hence, by replacing the atoms in a formula $\varphi$, by formulas from $\mathcal{L}$ one (for a lack of a better word) "fixes" the truth assignment of those propositional variables to truth, since the truth value of $\varphi$ is not contingent on the variable assignment, the truth value of the remaining variables can be chosen arbitrarily, and $\varphi$ would remain true. Hence, $\varphi$ remains valid even after substitution. \\

Let $\varphi \in \mathcal{L}$ and let the family $(p_i)_{i \in I}$ such that $\{ p_i \mid i \in I\} = PV$. Moreover, let $\pi_n$ be a substitution such that
\begin{itemize}
\item $\pi_n(\chi \land \psi) = \pi_n(\chi) \land \pi_n(\psi) $
\item $\pi_n(\chi \lor \psi) = \pi_n(\chi) \lor \pi_n(\psi) $
\item $\pi_n(\chi \lto \psi) = \pi_n(\chi) \lto \pi_n(\psi) $
\item $\pi_n(\neg \psi) =\neg \pi_n(\psi) $
\item $\pi_n(\all \psi) =\all \pi_n(\psi) $
\item $\pi_n(\some \psi) =\some \pi_n(\psi) $
\item \begin{equation*}
\pi_n(q) = \begin{cases}
x  & \quad \mathit{for} \; x \in \{q, \psi\} \; \mathit{with} \; \psi \in \mathcal{L} \quad if \; q=p_i \land i\leq n\\
q & \quad otw.\\
\end{cases}
\end{equation*}
\end{itemize}

Show by induction on $n$, i.e. on the length of parallel replacement of propositional variables, that $\mathcal{F} \models \pi_n(\varphi)$. 
\begin{itemize}
\item \textbf{IH:}  for $\varphi \in \mathcal{L}$, $\mathcal{F} \models \pi_n(\varphi)$.
\item \textbf{IB:}  $n=0$. All propositional variables in $\varphi$ are "replaced" by them self. Hence, $\pi_0(\varphi)=\varphi$. By definition $\mathcal{F} \models \varphi$, thus  $\mathcal{F} \models \pi_0(\varphi)$
\item \textbf{IS:} If all propositional variables of $\varphi$ are in $\{p_i \mid i < n+1\}$ then $\pi_{n+1}$ can be restricted to $\pi_n$. Otherwise, $p_{n+1}$ occurs in $\varphi$. If $\pi_{n+1}(p_{n+1})=p_{n+1}$, the same restriction can be made. Lastly, if $\pi_{n+1}(p_{n+1})=\psi$. Firstly, rename all occurrences of $p_{n+1}$ in $\varphi$ to some fresh variable $v$, let this operation be called $\tau$. Uniform renaming preserves validity. Secondly, restrict $\pi_{n+1}$ to $\pi_n$ and apply it to $\varphi$. This preserves validity by IH. Thirdly, replace all occurrences of $v$ by $\psi$, let this operation be called $\rho$. Clearly, $\pi_{n+1}=(\rho \circ \pi_n \circ \tau)$. Moreover, $\mathcal{F} \models  \pi_n(\tau(\varphi))$. By definition $\psi$ is also valid in $\mathcal{F}$. Now, as $\pi_n(\tau(\varphi))$ valid, the formula evaluates to true if the variable assignment of $v$ is fixed to true. Moreover, for all propositional variables in $\pi_n(\tau(\varphi))$ that are not $v$ can still be chosen arbitrarily. We know the variable assignment of the propositional variables in $\psi$ can be chosen arbitrarily as well. Hence, $\mathcal{F} \models \pi_{n+1}(\varphi)$.
\end{itemize}

Since, it is possible to replace an arbitrary amount of propositional variables in a valid formula by other valid formulas without infringing on validity, and since all formulas are finite, it can be concluded that $\mathcal{L}$ is closed under substitution.
\end{enumerate}

\item Show that $CL \subset \mathcal{L}$.
\begin{enumerate}
\item Show that $CL \subseteq \mathcal{L}$. Take $\varphi \in CL$. By definition $\varphi$ is a tautology in classical propositional logic. Hence, it is true under every variable assignment.  Moreover, there are no modalities present in $\varphi$. Hence, the evaluation of the formula, is independent of the world in which the formula is evaluated. Additionally, the semantics of the remaining logical connectives/operators is the same in predicate logic and modal logic. Therefore, $\varphi$ is valid in arbitrary worlds and models. Hence, $\mathcal{F} \models \varphi$ and thus $\varphi \in \mathcal{L}$. Hence, $CL \subseteq \mathcal{L}$ \\

\item Moreover, for any given Kripke frame $\mathcal{F}$ there are at least two formulas, $\varphi$ and $\psi$, such that $\mathcal{F} \models \varphi$ and $\mathcal{F} \models \psi$ and both of them are equivalent in modal logic, but not equivalent in classical logic, i.e. $\varphi \cong_M \psi$ and $\varphi \ncong_{CL} \psi$. Let $\varphi:= \all(A \lto B) \lto \all A \all B$ and $\psi:= \all(C \lto D) \lto \all C \all D$. As shown in the exercise below, both instances of K are valid in all frames. However, the formulas $p_{\all(A \lto B)} \lto p_{\all A} \lto p_{\all B}$ and  $p_{\all(C \lto D)} \lto p_{\all C} \lto p_{\all D}$ are clearly not equivalent. Hence, at least one of them can not be a tautology. Therefore, there is at least one formula in $\mathcal{L}$, to which there is no CL equivalent.
\end{enumerate}


\end{enumerate}


\section*{Exercise 16}
\begin{quote}
\begin{itemize}
\item $\some (A \land B) \lto \some  A$
\item $\some \neg A \lto  \neg \all  A$
\item $\all (A \land B) \lto (\all A \land \all B) $
\item $\all (A \lto B) \lto (\all A \lto \all B) $
\end{itemize}
Prove these facts.
\end{quote}

\begin{itemize}
\item $\some (\varphi \land \psi) \lto \some  \varphi$. Let $\mathcal{F}$ be a Kripke frame such that 
\begin{equation*}
\mathcal{F} \nmodels \some (\varphi \land \psi) \lto \some  \varphi
\end{equation*}
This corresponds to  
\begin{equation*}
\sneg (\mathcal{F} \models \some (\varphi \land \psi) \sto  \mathcal{F} \models \some  \varphi)
\end{equation*}
By reasoning on the semantic level one can transform this statement into 
\begin{equation*}
\begin{split}
&\sneg ((\sneg \mathcal{F} \models \some (\varphi \land \psi)) \sor  \mathcal{F} \models \some  \varphi) \iff \\
&\mathcal{F} \models \some (\varphi \land \psi) \sand  \sneg \mathcal{F} \models \some  \varphi 
\end{split}
\end{equation*}
Moreover, $\mathcal{F} \models \chi$ is equivalent to for all $w \in W \; \mathcal{F},w \models \chi$. Hence, consider an arbitrary $s \in W$.
\begin{equation*}
\begin{split}
&\mathcal{F},s \models \some (\varphi \land \psi) \sand  \sneg \mathcal{F},s \models \some  \varphi \iff\\
& \exists u \; sRu \sand \mathcal{F},u \models (\varphi \land \psi) \sand  \\
&\sneg (\exists u \; sRu \sand \mathcal{F},u \models \varphi) \iff  \\
& \exists u \; sRu \sand (\mathcal{F},u \models \varphi \sand \mathcal{F},u \models \psi) \sand  \\
& (\forall u \; \sneg sRu \sor  not \mathcal{F},u \models \varphi)
\end{split}
\end{equation*}
Hence, from $\mathcal{F},s \models \some (\varphi \land \psi)$ one obtains the existence of a state $u$ reachable from s, where $\mathcal{F},u \models \varphi$. However, $\sneg \mathcal{F},s \models \some  \varphi$ requires that all states are not reachable from $s$ or have to reject $\varphi$. That is, for every state $u$ reachable from $s$, $\mathcal{F},u \nmodels \varphi$. Thus $\mathcal{F}$ can not exists. Hence, $\some (\varphi \land \psi) \lto \some  \varphi$ holds for every frame.


\item $\some \neg \varphi \lto  \neg \all  \varphi$. Consider an arbitrary frame $\mathcal{F}$. Starting with the semantic evaluation. Let $s \in W$ be an arbitrary world.
\begin{equation*}
\begin{split}
&\mathcal{F},s \models \some \neg \varphi \lto  \neg \all  \varphi \iff \\
& (\exists u \; sRu \sand \sneg \mathcal{F},u \models \varphi k)\sto (\sneg (\forall u \; sRu \sto \mathcal{F},u \models \varphi)) \iff \\
& (\exists u \; sRu \sand \sneg \mathcal{F},u \models \varphi )\sto (\exists u \;  sRu \sand \sneg \mathcal{F},u \models \varphi) 
\end{split}
\end{equation*} 
Thus, given the semantics of implication, it is known that an assertion implies itself. Moreover, by choosing an arbitrary frame, one can conclude that the formula $\some \neg \varphi \lto  \neg \all  \varphi$ holds in arbitrary frames.


\item $\all (\varphi \land \psi) \lto (\all \varphi \land \all \psi)$. Consider an frame $\mathcal{F}$ such that 
\begin{equation*}
\begin{split}
\mathcal{F} \models \all (\varphi \land \psi)
\end{split}
\end{equation*} 
Let $s \in W$ be an arbitrary world. 
\begin{equation*}
\begin{split}
&\mathcal{F},s \models \all (\varphi \land \psi) \iff  \\
&\forall u \; ( \sneg sRu) \sor (\mathcal{F},u \models \varphi \sand \mathcal{F},u \models \psi) \\
\end{split}
\end{equation*} 
Now, using the semantic notion of the classical tautology of distributivity to reason on the meta level, one obtains 

\begin{equation*}
\begin{split}
&\forall u \; (( \sneg sRu \sor \mathcal{F},u \models \varphi) \sand ( \sneg sRu \sor \mathcal{F},u \models \psi))  \\
\end{split}
\end{equation*}
Moreover, in this case one can safely distribute $\forall$. That is,
\begin{equation*}
\begin{split}
&\forall u \; ( \sneg sRu \sor \mathcal{F},u \models \varphi) \sand \forall u \; ( \sneg sRu \sor \mathcal{F},u \models \psi)  \iff \\
&\forall u \; (  sRu \sto \mathcal{F},u \models \varphi) \sand \forall u \; ( sRu \sto \mathcal{F},u \models \psi)  \\
\end{split}
\end{equation*}
Now, by condensing the semantic notions described above back into the corresponding syntactic form one obtains.
\begin{equation*}
\begin{split}
\mathcal{F},s \models (\all \varphi \land \all \psi)
\end{split}
\end{equation*}
Hence, if $\all (\varphi \land \psi)$ holds in a frame, then $(\all \varphi \land \all \psi)$ must hold as well. Therefore, $\all (\varphi \land \psi) \lto (\all \varphi \land \all \psi)$ holds in any frame.
\item $\all (\varphi \lto \psi) \lto (\all \varphi \lto \all \psi)$. Assume that there exists a frame $\mathcal{F}$ where the formula does not hold. This means there exists a Kripke model $\mathcal{M}$ based on this frame $\mathcal{F}$ where the formula does not hold.
Propositional classical logic, dictates that the formula $A \lto (B \lto C)$ can only evaluate to false if $A$ and $B$ hold and $C$ does not. Hence, within the assumed model $\mathcal{M}$ it must be that there exists one world $s \in W$ where
\begin{equation*}
\begin{split}
\mathcal{M},s \models \all (\varphi \lto \psi) \sand \mathcal{M},s \models \all \varphi \sand \mathcal{M},s \nmodels \all \psi
\end{split}
\end{equation*}
However, it is the case that 
\begin{equation*}
\begin{split}
&\mathcal{M},s \models \all (\varphi \lto \psi) \sand \mathcal{M},s \models \all \varphi \iff \\
&\forall u \; (sRu \sto (\mathcal{M},u \models \varphi \sto \mathcal{M},u \models \psi)) \sand \forall u \; (sRu \sto \mathcal{M},u \models \varphi)
\end{split}
\end{equation*}
Meaning that, for any world $u$ reachable from $s$ it is that case that $\mathcal{M},u \models \varphi \lto \psi$ implying that if $\mathcal{M},u \models \varphi$ it must also be that $\mathcal{M},u \models \psi$. Now given the fact that $\mathcal{M},u \models \varphi$ holds in every $u$ reachable by $s$ as well, one concludes that $\mathcal{M},u \models \psi$. Implying that $\mathcal{M},s \models \all \psi$ and thus contradicting the assumption. Hence, is must be that in every Kripke model build upon the arbitrary frame $\mathcal{F}$ the formula $\all (\varphi \lto \psi) \lto (\all \varphi \lto \all \psi)$ holds. Therefore, it holds in $\mathcal{F}$. Moreover, since $\mathcal{F}$ arbitrary, it is the case that the formula holds in every Kripke frame.
%
%%
%\item $\all (\varphi \lto \psi) \lto (\all \varphi \lto \all \psi)$ Consider an frame $\mathcal{F}$ such that 
%\begin{equation*}
%\begin{split}
%\mathcal{F} \models \all (\varphi \lto \psi)
%\end{split}
%\end{equation*} 
%Let $s \in W$ be an arbitrary world. 
%\begin{equation*}
%\begin{split}
%&\mathcal{F},s \models \all (\varphi \lto \psi) \iff  \\
%&\forall u \; ( \sneg sRu) \sor (\mathcal{F},u \models \varphi \sto \mathcal{F},u \models \psi) \iff  \\
%&\forall u \; ( \sneg sRu) \sor (\sneg \mathcal{F},u \models \varphi \sor \mathcal{F},u \models \psi)
%\end{split}
%\end{equation*} 
%Consider the fact that $\chi$ is equivalent to $\chi \sor \chi$. Moreover, since it is known that $\sor$ adheres to commutativity and associativity. One obtains
%
%
%%Now, from classical logic, it is known that $\chi$ is equivalent to $1 \sand \chi$, i.e. an conjunction can be extended by assertions that always evaluate to true. Furthermore, . Moreover, every tautology evaluates always to $1$. That is, consider $\chi \sor \neg \chi$. Furthermore, having a disjunction where at least one part evaluates always to true, allows for arbitrary extensions of the disjunction, while retaining the evaluation. Hence, one obtains
%\begin{equation*}
%\begin{split}
%&\forall u \; ((\sneg sRu) \sor (\sneg \mathcal{F},u \models \varphi ) \sor (\mathcal{F},u \models \psi)) \iff \\
%&\forall u \; ((\sneg sRu) \sor (\sneg sRu) \sor (\sneg \mathcal{F},u \models \varphi ) \sor (\mathcal{F},u \models \psi)) \iff \\
%&\forall u \; ((\sneg sRu) \sor (\sneg \mathcal{F},u \models \varphi ) \sor (\sneg sRu) \sor  (\mathcal{F},u \models \psi)) \\
%&\forall u \; ((1 \sand \sneg sRu) \sor (\sneg \mathcal{F},u \models \varphi ) \sor (\mathcal{F},u \models \psi))) \iff \\
%&\forall u \; ((sRu \sor \sneg sRu) \sand \\
%&((\sneg sRu) \sor (\sneg \mathcal{F},u \models \varphi ) \sor (\mathcal{F},u \models \psi))) \iff \\
%&\forall u \; ((sRu \sor (\sneg sRu) \sor \mathcal{F}, s \models \psi) \\
%&\sand ((\sneg sRu) \sor (\sneg \mathcal{F},u \models \varphi ) \sor (\mathcal{F},u \models \psi))) \\
%\end{split}
%\end{equation*} 
%By using the semantic notion of distributivity, one can factor this statement into 
%\begin{equation*}
%\begin{split}
%&\forall u \; ((sRu \sand (\sneg \mathcal{F},u \models \varphi )) \sor (\sneg sRu \sor \mathcal{F},u \models \psi )) \iff \\
%&\forall u \; (\sneg (\sneg sRu \sor  \mathcal{F},u \models \varphi ) \sor (\sneg sRu \sor \mathcal{F},u \models \psi )) \iff \\
%&\forall u \; (\sneg (sRu \sto  \mathcal{F},u \models \varphi ) \sor (sRu \sto \mathcal{F},u \models \psi )) \iff \\
%&\forall u \; (sRu \sto  \mathcal{F},u \models \varphi ) \sto (sRu \sto \mathcal{F},u \models \psi ))\\
%\end{split}
%\end{equation*} 
%Now consider the classical first order validity $\forall x (P(x) \lto Q(x)) \lto (\forall x P(x) \lto \forall x  Q(x)))$. That is, assume there exists an interpretation $\mathcal{I}$ such that  $\mathcal{I}\nmodels \forall x (P(x) \lto Q(x)) \lto (\forall x P(x) \lto \forall x  Q(x)))$. To that extend it must be that $\forall x (P(x) \lto Q(x)) $ must hold under $\mathcal{I}$ and $(\forall x P(x) \lto \forall x  Q(x))$ does not. If there exists a $c$ in the domain such that $v_{\mathcal{I}}(P(c))=0$ it would follow that $\mathcal{I} \models (\forall x P(x) \lto \forall x  Q(x))$. Hence, $\mathcal{I} \models \forall x P(x)$. However, now in order to satisfy the premise of the big implication, it must be that $\mathcal{I} \models \forall x Q(x)$. Hence, allowing the conclusion that $\mathcal{I}$ can not exist. Using this reasoning on the meta-level. One obtains
%\begin{equation*}
%\begin{split}
%&\forall u \; (sRu \sto  \mathcal{F},u \models \varphi ) \sto (sRu \sto \mathcal{F},u \models \psi )) \implies\\
%&\forall u \; (sRu \sto  \mathcal{F},u \models \varphi ) \sto  \forall u \; (sRu \sto \mathcal{F},u \models \psi ))
%\end{split}
%\end{equation*} 
%Now translating the semantic notion described above, back into syntax one obtains
%\begin{equation*}
%\begin{split}
%\mathcal{F},s \models \all \varphi \lto \all \psi
%\end{split}
%\end{equation*}
%Hence, by a similar argument to the previous point, one can conclude $\all (\varphi \lto \psi) \lto (\all \varphi \lto \all \psi)$ holds in arbitrary frames.
\end{itemize}

\section*{Exercise 17}
\begin{quote}
Which of the above implicative formulas can/cannot be inverted? Provide either a proof or a counter-example in each case.
\end{quote}
\begin{itemize}
\item $\some  \varphi \lto \some (\varphi \land \psi)$. Consider the Kripke model $\mathcal{M}:=\langle \{s,t\}, \{(s,t)\}, V \rangle$ such that $V(\varphi):=\{t\}$. Clearly 
\begin{equation*}
\begin{split}
&\mathcal{M},s \models \some  \varphi \iff \\
&\exists u \; sRu \sand \mathcal{M},u \models  \varphi  \\
\end{split}
\end{equation*} 
holds, due to $t \in V(\varphi)$. However,  
\begin{equation*}
\begin{split}
&\mathcal{M},s \models \some  (\varphi \land \psi) \iff \\
&\exists u \; sRu \sand (\mathcal{M},u \models  \varphi \sand \mathcal{M},u \models  \psi ) \\
\end{split}
\end{equation*} 
can not hold. That is, $\mathcal{M},u \nmodels  \psi$, because $t \notin V(\psi)$.



\item $ \neg \all  \varphi \lto  \some \neg \varphi $ As established in the previous exercise 
\begin{equation*}
\begin{split}
&\mathcal{F},s \models \some \neg \varphi \lto  \neg \all  \varphi \iff \\
& (\exists u \; sRu \sand \sneg \mathcal{F},u \models \varphi )\sto (\exists u \;  sRu \sand \sneg \mathcal{F},u \models \varphi)
\end{split}
\end{equation*} 
Hence, allowing the conclusion 
\begin{equation*}
\begin{split}
& (\exists u \; sRu \sand \sneg \mathcal{F},u \models \varphi )\sto (\exists u \;  sRu \sand \sneg \mathcal{F},u \models \varphi) \iff \\
&\mathcal{F},s \models  \neg \all  \varphi \lto \some \neg \varphi\\
\end{split}
\end{equation*} 
Hence, $ \neg \all  \varphi \lto  \some \neg \varphi $ holds in arbitrary frames.


\item $(\all \varphi \land \all \psi)  \lto  \all (\varphi \land \psi)$. By observing the proof in the exercise above, one observe that every transformation made was bidirectional. Hence, from $\mathcal{F} \models  \all (\varphi \land \psi)$ one can trace the arguments back to $\mathcal{F} \models  (\all \varphi \land \all \psi)$.  Therefore, $(\all \varphi \land \all \psi)  \lto  \all (\varphi \land \psi)$ holds in every frame.

\item $(\all \varphi \lto \all \psi) \lto \all (\varphi \lto \psi)$. In contrast to the previous formula, there are Kripke models where this formula does not hold. Consider $\mathcal{M}:=\langle \{r,s,t\}, \{(s,r),(s,t)\}, V\rangle$ where $V(\varphi):=\{r\}$ and $V(\psi):=\{\}$. Now, one can observe
\begin{equation*}
\begin{split}
& \mathcal{M}, s \models \all \varphi \lto \all \psi \iff \\
& (\forall u \; sRu \sto \mathcal{M}, u \models \varphi) \sto (\forall u \; sRu \sto \mathcal{M}, u \models \psi) \\
\end{split}
\end{equation*} 
Clearly, the premise of the implication evaluates to false. That is, consider $u$ to be $t$ as a world reachable from $s$ where $\mathcal{M}, t \nmodels \varphi$. Hence, by the semantics of implication in classical logic, the whole statement holds. Hence, given the assumption that the whole formula holds, it must be that 
\begin{equation*}
\begin{split}
&\mathcal{M}, s \models (\all \varphi \lto \all \psi) \lto \all (\varphi \lto \psi) \iff \\
& 1 \sto (\forall u \; sRu \sto (\mathcal{M}, u \models \varphi \sto \mathcal{M}, u \models \psi)
\end{split}
\end{equation*} 
Now, try to evaluate the right part of the implication at state $r$. Firstly, $r$ is reachable from $s$. Secondly, $\mathcal{M}, r \models \varphi$ as well as  $\mathcal{M}, r \nmodels \psi$ by construction. 
Therefore, $\mathcal{M}, r \nmodels \varphi \lto \psi$. Hence, $\mathcal{M}, s \nmodels (\all \varphi \lto \all \psi) \lto \all (\varphi \lto \psi)$.
\end{itemize}


\section*{Exercise 18}
\begin{quote}
For any Kripke interpretation $\mathcal{M}$:
\begin{itemize}
\item If $\mathcal{M} \models A$ then $\mathcal{M} \models \all A$
\end{itemize}
It follows, that also for all frames $\mathcal{F}$:
\begin{itemize}
\item If $\mathcal{F} \models A$ then $\mathcal{F} \models \all A$
\end{itemize}
\end{quote}

Firstly, let $\mathcal{M}$ be an arbitrary Kripke interpretation and $\varphi$ a formula. Given $\mathcal{M} \models \varphi$ actually expresses, for all $w \in W \; \mathcal{M},w \models \varphi$. Consider an arbitrary state $s$. By assumption $\mathcal{M},s \models \varphi$. If $s$ is a terminal state, then $\mathcal{M} \models \all \varphi$. Otherwise let $t$ be an arbitrary state satisfying $sRt$.
By assumption $\mathcal{M},t \models \varphi$. However, since this holds for an arbitrary state it follows that $\forall u \; sRu \sto \mathcal{M}, u \models \varphi$. However, this is precisely the semantics of $\mathcal{M}, s \models \all \varphi$. Moreover, since $s$ was chosen arbitrarily, $\mathcal{M}\models \all \varphi$ follows. Furthermore, at no point in the proof the variable assignment $V$ was used. Hence, the same result holds for arbitrary frames $\mathcal{F}:=\langle W,R \rangle$.

\section*{Exercise 19}
\begin{quote}
Prove the soundness of $\mathbf{K}$ (using mentioned facts).
\end{quote}

This can be done by induction on the length of the proof. To that end, consider the induction hypothesis, a proof in $\mathbf{K}$ of length $n$ is sound.

\begin{itemize}
\item Induction base, i.e.\ $n=0$. Since, the system described on the slides does not include reasoning from a theory $\Gamma$, it suffices to show that the axioms of $\mathbf{K}$ are valid. Hence, let $\mathcal{F}$ be an arbitrary frame and $s \in W$ be an arbitrary state. Let $\varphi$ be an axiom of $\mathbf{K}$.
\begin{enumerate}
\item Case: $\varphi$ is a tautology in classical logic. It holds that $\mathcal{F},s\models \varphi$, because the semantic of the connectives $\land, \lor, \lto$ and $\neg$ is the same as in classical logic.
\item Case: $\varphi= \all (\psi \lto \chi) \lto (\all \psi \lto \all \chi)$. For a proof of its validity in arbitrary frames consult exercise 16.
\end{enumerate}
\item Induction step, i.e.\ $n=m+1$. Let $\varphi$ be a formula with a proof of length $m+1$ in $\mathbf{K}$. Within the system, there are two derivation rules. Hence, the last step in the derivation of $\varphi$ is either an application of \emph{(MP)} or \emph{(NC)} (necessitation).
\begin{enumerate}
\item Case: (MP). If the last step of derivation is (MP) then the proof is of the form
     \begin{prooftree}
        
              \AxiomC{$\vdots$}
              \noLine
              \UnaryInfC{$\psi$}
                   \AxiomC{$\vdots$}
                   \noLine
                   \UnaryInfC{$\psi \lto \varphi$}
              \RightLabel{\scriptsize(MP)}
              \BinaryInfC{$\varphi$}
     \end{prooftree}
Clearly, $\psi$ and $\psi \lto \varphi$ have a derivation of length $\leq m$. Hence, by induction hypothesis, one can conclude that for an arbitrary frame $\mathcal{F}$,  $\mathcal{F} \models \psi $ and $\mathcal{F} \models \psi \lto \varphi$ hold. Now consider for an arbitrary $s\in W$
\begin{equation*}
\begin{split}
&\mathcal{F}, s \models  \psi \sand \mathcal{F}, s \models  \psi \lto \varphi  \iff \\
&\mathcal{F}, s \models  \psi \sand (\mathcal{F}, s \models  \psi \sto \mathcal{F}, s \models \varphi) 
\end{split}
\end{equation*} 
Hence, by the semantics of implication one can conclude $\mathcal{F}, s \models \varphi$, i.e. $\mathcal{F}, s \models  \psi$ and $\mathcal{F}, s \nmodels  \psi$ can not hold at the same time. Lastly, because $s$ arbitrary, it follows $\mathcal{F}\models \varphi$. Thus, (MP) is sound.

\item Case: (NC). If the last step of derivation is (NC) then $\varphi = \all \psi$ and the proof has the form
     \begin{prooftree}
              \AxiomC{$\vdots$}
              \noLine
              \UnaryInfC{$\psi$}
              \RightLabel{\scriptsize(MP)}
              \UnaryInfC{$\all \psi$}
     \end{prooftree}
Again, the derivation of $\psi$ is smaller than $m+1$, thus by induction hypothesis it is a sound derivation. Thereby, for an arbitrary frame $\mathcal{F}$, $\mathcal{F}\models \psi$ holds, meaning that $\psi$ holds in every state. Picking an arbitrary state $s \in W$. Obviously, $\mathcal{F},s\models \psi$. Moreover, since $\psi$ holds in every state, it holds in particular in any state reachable from $s$ (if $s$ is terminal, $\all \psi$ holds vacuously). Hence,
\begin{equation*}
\begin{split}
&\forall u \; sRu \sto \mathcal{F},u \models \psi \iff \\
& \mathcal{F},s \models \all \psi
\end{split}
\end{equation*}
Since $s$ arbitrary $\mathcal{F}\models \all \psi$. Hence, the soundness of (NC) is demonstrated.
\end{enumerate} 
\end{itemize}
Hence, one can conclude that $\mathbf{K}$ is sound.



\section*{Exercise 22}
\begin{quote}
Find and prove some relations between E1 – E10.
\end{quote}
\begin{itemize}
\item $E5 \& E2 \lto E4$, i.e. euclidian and symmetry implies transitivity. Consider an arbitrary euclidian and symmetric frame $\mathcal{F}$. Moreover, consider three states $x,y,z$ (not necessarily distinct), such that $xRy \sand yRz$. If no such states exists the statement $\forall s \forall t \forall u ((sRt \sand tRu) \sto sRu)$ holds vacuously. Hence, from $xRy$ by symmetry one obtains $yRx$. Hence, from $xRy \sand yRz$ one obtains $yRx \sand yRz$. Now due to $\mathcal{F}$ being euclidian one obtains $xRz$, i.e. $((yRx \sand yRz) \sand ((yRx \sand yRz) \sto xRz)) \sto xRz$. Since, $x,y,z$ arbitrary, it follows that $\forall s \forall t \forall u ((sRt \sand tRu) \sto sRu)$.
 
\item $(E1 \& E4 \& E5) \sto E2$. This claim can be strengthened to $E1 \& E5 \sto E2$. Consider an arbitrary reflexive and euclidian frame $\mathcal{F}$. Consider two (no necessarily distinct) states $x$ and $y$, such that $xRy$ (due to reflexivity such states always exist). Now given reflexivity $xRx$ must hold. Hence, one obtains $xRy \sand xRx$, now given $\mathcal{F}$ being euclidian, i.e.$\forall s \forall t \forall u ((sRt \sand sRu) \sto tRu)$, one obtains via $((xRy \sand xRx) \sand ((xRy \sand xRx) \sto yRx)) \sto yRx$, $yRx$. Hence, if $xRy$ holds, one obtains $yRx$. Due to the arbitrary nature of $x,y$ and $\mathcal{F}$, it follows that $\forall s \forall t (sRt \sto tRs)$.

\item $(E4 \& E2) \sto E5$. Consider an arbitrary symmetric and transitive frame $\mathcal{F}$. Consider the arbitrary not necessary distinct states $x,y,z$, such that $xRy$ and $xRz$. If no such states exists, the frame already satisfies $\forall s \forall t \forall u ((sRt \sand sRu) \sto tRu)$. Otherwise, since $\mathcal{F}$ is symmetric and since $xRy$ it must be that $yRx$. Hence, one obtains $yRx \sand xRz$. Now given the fact that $\mathcal{F}$ is transitive, it must be the case that $yRz$. Hence, for arbitrary states $x,y,z$ it holds due to symmetry and transitivity that $(xRy \sand xRz)\sto yRz$, and thus $\forall s \forall t \forall u ((sRt \sand sRu) \sto tRu)$. 

\item $E7 \& E6$. Consider a functional frame $\mathcal{F}$. Let $x,y,z$ be three arbitrary, not necessarily distinct states, such that $xRy$ and $xRz$. If no such states exist, the statement $\forall s \forall t \forall u ((sRt \sand sRu) \sto t=u)$ holds vacuously. Take $xRy$, by functionality of $\mathcal{F}$ it holds that $xRy \sto \forall s (xRs) \sto y=s$. Now given the particular case for $s$ being $z$, i.e. $xRy \sto ((xRz) \sto y=z$, it follows that since $xRy$ and $xRz$ both hold, it must be the case that $y=z$. Hence, from $(xRy \sand xRz) \sto y=z$ holds due to the functionality of $\mathcal{F}$. Moreover, due to the fact that $x,y,z$ are arbitrary one obtains $\forall s \forall t \forall u ((sRt \sand sRu) \sto t=u$.

\item $E5 \& E9$ Consider an euclidian frame $\mathcal{F}$. Let $x,y,z$ be three arbitrary, not necessarily distinct states, such that $xRy$ and $xRz$. If those doe not exists, the statement $\forall s \forall t \forall u ((sRt \sand sRu) \sto (tRu \sor t=u \sor uRt))$ holds vacuously. Since $\mathcal{F}$ is euclidian one obtains from $(xRy \sand xRz) \sand ((xRy \sand xRz) \sto yRz)$ the fact that $yRz$.Clearly, if $A$ is true then $A \sor B$ is true, i.e. $A \sto (A \sor B)$. Hence, by the same reasoning one obtains $yRz \sor x=z \sor zRy$ from $yRz$. Thus, due to $\mathcal{F}$ being euclidian one obtains from $(xRy \sand xRz)$ the statement $(yRz \sor x=z \sor zRy)$. Moreover, as $x,y,z$ are arbitrary, it follows that $\mathcal{F}$ satisfies the property $\forall s \forall t \forall u ((sRt \sand sRu) \sto (tRu \sor t=u \sor uRt))$.
\end{itemize}



\section*{Exercise 23}
\begin{quote}
Prove some (more) parts of the theorem.
\end{quote}

Let $\mathcal{F}$ be a frame, and let $\mathcal{M}$ be a model corresponding to the frame $\mathcal{F}$. Unfortunately, I realised to late that I.....!!!!!!! 
\begin{itemize}
\item $\mathcal{F} \models \varphi \lto \all \some \varphi \iff \mathcal{F} \mathit{symmetric}$ 
\begin{itemize}
\item "$\Rightarrow$" Assume $\mathcal{F} \models p \lto \all \some p $ and $\exists s \exists t (sRt \land \neg tRs)$. Let $V(p):=\{s\}$. Hence, $\mathcal{M},s \models p$. Moreover, since $sRt$ it must be that $\mathcal{M},t \models \some p$. However, the only state where $p$ holds is $s$ and by assumption $\neg tRs$, thus $\mathcal{M},t \models \some p$ can not hold. Hence, $\mathcal{M}, s \nmodels \all \some p$ and thus $\mathcal{M}, s \nmodels p \lto \all \some p$. Having found a model such that $\mathcal{M} \nmodels  p \lto \all \some p $, it follows that $\mathcal{F} \nmodels  p \lto \all \some p $, which contradicts the assumption.
\item "$\Leftarrow$". It is known that $\forall s \forall t (sRt \sto tRs)$. Take an arbitrary state $s$. If $\mathcal{M},s \nmodels \varphi$ then $\varphi \lto \all \some \varphi $ holds vacuously. If $s$ is isolated then $\mathcal{M},s \models \all \some \varphi$ holds vacuously. Otherwise, let $t$ be an arbitrary from s accessible state, i.e. $sRt$. Moreover, due to symmetry it must also be that $tRs$. Therefore, $\exists v tRv \sand \mathcal{M}, v \models \varphi $, and thus $\mathcal{M}, t \models \some \varphi$. Moreover, as $t$ was chosen arbitrary, $\forall v sRv \sto \mathcal{M}, v \models \some \varphi$, which leads to $\mathcal{M}, s \models \varphi \lto \all \some \varphi$. As $s$ is arbitrary, this is the case in every state, and since no variable assignment was fixed, this holds in every model, and therefore it holds in $\mathcal{F}$.
\end{itemize}

\item $\mathcal{F} \models \all p \lto \some p \iff \mathcal{F} \mathit{serial}$ 
\begin{itemize}
\item "$\Rightarrow$". Assume $\mathcal{F} \models \all p \lto \some p$ and $\exists s \forall t \neg sRt$. This implies that $s$ is an isolated state. Which implies $\forall v (sRv \sto \mathcal{M}, v\models p)$ holds vacuously, and thus $\mathcal{M}, s \models \all p$. However, for the same reason $\exists v sRv \sand \mathcal{M}, v\models p$ can never hold, i.e. $\mathcal{M}, s \nmodels \some p$. Hence, one obtains $\mathcal{M}, s \nmodels  \all p \lto \some p$. Having found a model such that $\mathcal{M} \nmodels  \all p \lto \some p$, it follows that $\mathcal{F} \nmodels  \all p \lto \some p$, which contradicts the assumption.
\item "$\Leftarrow$". By assumption $\mathcal{F}$ is serial, i.e. $\forall s \exists t sRt$. Let $s$ be an arbitrary state. If $\mathcal{M},s \nmodels \all \varphi$, then $\mathcal{M},s \models \all\varphi \lto \some \varphi$ holds vacuously. Since, $\mathcal{M}$ serial, there must be a $t$ such that $sRt$. Now given  $\mathcal{M},s \models \all \varphi$ it must be by semantics of $\all$ that $\mathcal{M}, t \models \varphi$. Therefore, due to $sRt$ one obtains $\exists v (sRv \land \mathcal{M}, v \models \varphi)$, which leads to $\mathcal{M}, s\models  \some \varphi $. Since $s$ is arbitrary and no variable assignment was fixed, it follows $\mathcal{F} \models \all \varphi \lto \some \varphi$.
\end{itemize}

%\item $\mathcal{F} \models \all \varphi \lto \all \all \varphi \iff \mathcal{F} \mathit{transitive}$ 
%\begin{itemize}
%\item "$\Rightarrow$"
%\item "$\Leftarrow$"
%\end{itemize}
%
%\item $\mathcal{F} \models \some \varphi \lto \all \some \varphi \iff \mathcal{F} \mathit{euclidian}$ 
%\begin{itemize}
%\item "$\Rightarrow$"
%\item "$\Leftarrow$"
%\end{itemize}

\item $\mathcal{F} \models \some \varphi \lto \all \varphi \iff \mathcal{F} \mathit{partially \; functional}$ 
\begin{itemize}
\item "$\Rightarrow$". Assume $\mathcal{F} \models \some p \lto \all p $ and $\exists s \exists t \exists u (sRt \land sRu \land t \neq u)$. Consider a variable assignment where $t \in V(p)$ and $u \nin V(p)$, which is possible due to $t \neq u$. Since, $sRt$ it is clearly the case that $\exists v (sRv \land \mathcal{M}, v \models p)$, and since $sRu$ $\forall v (sRv \sto \mathcal{M}, v \models p)$ can not be the case. Therefore, $\mathcal{M}, s \models \some p$ and  $\mathcal{M}, s \nmodels \all p$. Having found a formula $p$ and a model such that $\mathcal{M} \nmodels  \some p \lto \all p$, it follows that $\mathcal{F} \nmodels  \some p \lto \all p$, which contradicts the assumption.
\item "$\Leftarrow$" By assumption $\mathcal{F}$ is a partially functional frame, i.e. \\
$\forall s \forall t \forall u ((sRt \sand sRu) \sto t = u)$. Take an arbitrary state $s$. If $\mathcal{M}, s \nmodels \some \varphi$ then $\mathcal{M},s \models \some \varphi \lto \all \varphi$. If both $\neg sRt$ and $\neg sRu$ for any $t$ and $u$, then $\mathcal{M}, s \nmodels \some \varphi$ vacuously. If there exists states $t,u$, such that $sRt$ and $sRu$. By partial functionality, it follows that $t = u$. Therefore, there can only be one state accessible from $s$. Hence, if there is only one state $t$ such that $sRt$, the fact that $\mathcal{M}, s \models \some \varphi$ implies that $\forall v (sRv \sto \mathcal{M}, v \models \varphi)$. That is, there exists one state where $\varphi$ holds, and there is only one accessible state implies that $\varphi$ holds in every accessible state. Allowing the conclusion $\mathcal{M},s \models \some \varphi \lto \all \varphi $. As $s$ is arbitrary, this is the case in every state, and since no variable assignment was fixed, this holds in every model, and therefore it holds in $\mathcal{F}$. 
\end{itemize}


%\item $\mathcal{F} \models \all \all \varphi \lto \all \varphi \iff \mathcal{F} \mathit{weakly \; dense}$ 
%\begin{itemize}
%\item "$\Rightarrow$"  Assume $\mathcal{F} \models \all \all \varphi \lto \all \varphi  $ and $\exists s \exists t (sRt \sand \forall u (\neg sRu \sor \neg uRt))$
%\item "$\Leftarrow$"
%\end{itemize}

\end{itemize}


\section*{Exercise 24}
\begin{quote}
\begin{enumerate}
\item  Find at least one appropriate internet resource for
‘bisimulation’ as well as for ‘bounded morphism’
(sometimes also called ‘p-morphism’).
\item Summarize the central definition and fact(s) precisely.
\item Give non-trivial examples of bisimilar models.
\item Apply a bounded morphism to show that asymmetry is not
characterizable.
\end{enumerate}
\end{quote}


\begin{enumerate}
\item The definition of bounded morphism and bisimulation is taken from \url{http://ai.stanford.edu/~epacuit/classes/modal-spr2012/ml-overview-handout.pdf} with slight terminological adaptation from \url{http://ali.cmi.ac.in/isla2010/slides/vg-lec3.pdf}. While the first lemma is taken from the prior source, the second lemma is taken from \url{http://flolac.iis.sinica.edu.tw/flolac09/lib/exe/modal_logic_4on1.pdf}.

\item 
\begin{mydef}
A bounded morphism between models $\mathcal{M}:=\langle W, R, V \rangle$ and $\mathcal{M}':=\langle W', R', V' \rangle$, is a function $f: W \to W'$ such that
\begin{itemize}
\item \textbf{Atom:} For each atom $p$, $s \in V(p)$ if and only if $f(s) \in V'(p)$;
\item \textbf{Forth:} If $sRt$ then $f(s)R'f(t)$;
\item \textbf{Back:} If $f(s)R't'$ then $\exists t\in W$ such that $f(t)=t'$ and $sRt$.
\end{itemize}
\end{mydef}



\begin{mydef}
A bisimulation between models $\mathcal{M}:=\langle W, R, V \rangle$ and $\mathcal{M}':=\langle W', R', V' \rangle$ is a non-empty binary relation $\mathfrak{B} \subseteq W \times W'$ such that if $s\mathfrak{B}s'$
\begin{itemize}
\item \textbf{Atom:} For each atom $p$, $s \in V(p)$ if and only if $f(s) \in V'(p)$;
\item \textbf{Forth:} If $sRt$, then $\exists t' \in W'$ such that $t\mathfrak{B}t'$ and $s'R't'$;
\item \textbf{Back:}  If $s'R't'$, then $\exists t \in W$ such that $t\mathfrak{B}t'$ and $sRt$.
\end{itemize}
\end{mydef}


\begin{lemma}
If $f: W \to W'$ is a bounded morphism and $\varphi$ is a wff modal logic formula, then
\begin{itemize}
\item for all $s \in W$, $\mathcal{M},s \models \varphi$ iff $\mathcal{M}', f(s) \models \varphi$
\item if $f$ is surjective then $\mathcal{M}\models \varphi$ iff $\mathcal{M}' \models \varphi$
\end{itemize}
\end{lemma}

\begin{lemma}
If $f: W \to W'$ is a bounded morphism then for all $s \in W$, $s\mathfrak{B}f(s)$ is a bisimulation.
\end{lemma}
\item 
\begin{itemize}
\item $\mathcal{M}:= \langle \{a,b\}, \{(a,b),(b,a)\},V\rangle$ and $\mathcal{M}':= \langle \{1\}, \{(1,1)\},V'\rangle$, where $V(p):=\{a,b\}$ and $V'(p):=\{1\}$ Let $\mathfrak{B}:=\{(a,1),(b,1)\}$.
\begin{itemize}
\item $a\mathfrak{B}1$
\begin{itemize}
\item $a \in V(p)$ and $1  \in V'(p)$
\item $aRb$ and $b\mathfrak{B}1$ and $1R'1$
\item $1R1$ and $b\mathfrak{B}1$ and $aRb$
\end{itemize}
\item $b\mathfrak{B}1$
\begin{itemize}
\item $b \in V(p)$ and $1  \in V'(p)$
\item $bRa$ and $a\mathfrak{B}1$ and $1R'1$
\item $1R1$ and $a\mathfrak{B}1$ and $bRa$
\end{itemize}
\end{itemize}


\item $\mathcal{M}:= \langle \{a,b,c,d\}, \{(a,b),(b,c),(c,d),(d,b)\},V\rangle$ and $\mathcal{M}':= \langle \{1,2\}, \{(1,2),(2,2)\},V'\rangle$, where $V(p):=\{a,b,c,d\}$ and $V'(p):=\{1,2\}$ Let $f:=\{(a,1),(b,2),(c,2),(d,2)\}$. 
\begin{itemize}
\item $a \in W$
\begin{itemize}
\item $a \in V(p)$ and $1  \in V'(p)$
\item $aRb$ and $f(a)=1$ and $f(b)=2$ and $1R'2$
\item $f(a)=1$ and $1R'2$ and $f(b)=2$ and $aRb$
\end{itemize}

\item $b \in W$
\begin{itemize}
\item $b \in V(p)$ and $f(b)=2$ and $2  \in V'(p)$
\item $bRc$ and $f(b)=2$ and $f(c)=2$ and $2R'2$
\item $f(b)=2$ and $2R'2$ and $f(c)=2$ and $bRc$
\end{itemize}


\item $c \in W$
\begin{itemize}
\item $c \in V(p)$ and $f(c)=2$ and $2  \in V'(p)$
\item $cRd$ and $f(c)=2$ and $f(d)=2$ and $2R'2$
\item $f(c)=2$ and $2R'2$ and $f(d)=2$ and $cRd$
\end{itemize}


\item $d \in W$
\begin{itemize}
\item $d \in V(p)$ and $f(d)=2$ and $2  \in V'(p)$
\item $dRb$ and $f(d)=2$ and $f(b)=2$ and $2R'2$
\item $f(d)=2$ and $2R'2$ and $f(b)=2$ and $dRb$
\end{itemize}
\end{itemize}
\end{itemize}


\item
Assume there exists a formula $\psi$ such that $\mathcal{F} \models \psi$ if and only if $\mathcal{F}$ satisfies $\forall x \forall y (xRy \sto \neg y R x)$ (i.e. $\mathcal{F}$ is asymmetric).

Hence, $\psi$ must hold in all models that are based on an asymmetric frame and $\psi$ can not hold in any model that is based on a non-asymmetric frame.

Consider the models 
$\mathcal{M}:= \langle \{a,b,c,d\}, \{(a,b),(b,c),(c,d),(d,b)\},V\rangle$ and $\mathcal{M}':= \langle \{1,2\}, \{(1,2),(2,2)\},V'\rangle$, where $V(p):=\{a,b,c,d\}$ and $V'(p):=\{1,2\}$.

One can observe that $\mathcal{M}$ satisfies asymmetry. However, $\mathcal{M}'$ does not. That is, since $2R'2$, $2R'2 \sto \neg 2R'2$ can clearly not be the case.
As shown in the point above, there exists a bounded morphism between those two models. Moreover, due to the fact that $f$ is also surjective, it must be the case that for each modal logic formula $\varphi$  $\mathcal{M}\models \varphi$ iff $\mathcal{M}' \models \varphi$. That is, there is no formula that can distinguish between this asymmetric model and this non-asymmetric model. Hence, $\psi$ can not exist.

\end{enumerate}



\section*{Exercise 26}
\begin{quote}
Exercise 26:
Derive the following formulas in the system for K:
\begin{itemize}
\item $\all (A \land B) \lto (\all A \land \all B)$
\item $(\all A \land \all B) \lto \all (A \land B)  $
\end{itemize}
HINT: Show first that further rules are admissible and use them!
\end{quote}



\begin{center}
\begin{sidewaysfigure}
\small
\begin{prooftree}
				\AxiomC{$(\varphi \land \psi) \lto \varphi$}
				\UnaryInfC{$\all((\varphi \land \psi) \lto \varphi) $}
			\AxiomC{$\all((\varphi \land \psi) \lto \varphi) \lto (\all (\varphi \land \psi) \lto \all \varphi)$}
		\BinaryInfC{$\all (\varphi \land \psi) \lto \all \varphi$}	
		\UnaryInfC{\tiny $(2)$}
\end{prooftree}

\begin{prooftree}
				\AxiomC{$(\varphi \land \psi) \lto \psi$}
				\UnaryInfC{$\all((\varphi \land \psi) \lto \psi) $}
			\AxiomC{$\all((\varphi \land \psi) \lto \psi) \lto (\all (\varphi \land \psi) \lto \all \psi)$}
		\BinaryInfC{$\all (\varphi \land \psi) \lto \all \psi$}	
		\UnaryInfC{\tiny $(1)$}
\end{prooftree}

\begin{prooftree}
\def\defaultHypSeparation{\hskip 0.01in}
		
		\AxiomC{\tiny $(1)$}
		
		\AxiomC{\tiny $(2)$}
				\AxiomC{ $(\all (\varphi \land \psi) \lto \all \varphi) \lto ((\all (\varphi \land \psi) \lto \all \psi) \lto (\all (\varphi \land \psi) \lto \all \varphi \land \all (\varphi \land \psi) \lto 	\all \psi) )$}
		
		\BinaryInfC{$(\all (\varphi \land \psi) \lto \all \psi) \lto (\all (\varphi \land \psi) \lto \all \varphi \land \all (\varphi \land \psi) \lto 	\all \psi) $}

	\BinaryInfC{$\all (\varphi \land \psi) \lto \all \varphi \land \all (\varphi \land \psi) \lto 	\all \psi$}
	\AxiomC{$(\all (\varphi \land \psi) \lto \all \varphi \land \all (\varphi \land \psi) \lto \all \psi) \lto (\all (\varphi \land \psi) \lto (\all \varphi \land \all \psi))$}
\BinaryInfC{$\all (\varphi \land \psi) \lto (\all \varphi \land \all \psi)$}
\end{prooftree}
\end{sidewaysfigure}
\end{center}

%\newpage


%Secondly, the proof of $(\all A \land \all B) \lto \all (A \land B)  $ is 

\begin{center}

\begin{sidewaysfigure}
\scriptsize
\begin{prooftree}
\def\defaultHypSeparation{\hskip 0.001in}

\AxiomC{$\varphi \lto (\psi \lto (\varphi \land \psi)) $}
\UnaryInfC{$\all( \varphi \lto (\psi \lto (\varphi \land \psi)))$}
\AxiomC{$\all( \varphi \lto (\psi \lto (\varphi \land \psi)))\lto (\all \varphi \lto \all (\psi \lto (\varphi \land \psi)))$}
\BinaryInfC{$\all \varphi \lto \all (\psi \lto (\varphi \land \psi))$}		
\AxiomC{ $\all \varphi \lto \all (\psi \lto (\varphi \land \psi)) \lto (\all \varphi \lto (\all \psi \lto \all(\varphi \land \psi)))$} 
		
\BinaryInfC{$\all \varphi \lto (\all \psi \lto \all(\varphi \land \psi))$}		
\AxiomC{ $(\all \varphi \lto (\all \psi \lto \all(\varphi \land \psi))) \lto ((\all \varphi \land \all \psi) \lto \all (\varphi \land \psi))$}
\BinaryInfC{$(\all \varphi \land \all \psi) \lto \all (\varphi \land \psi)$}
\end{prooftree}
\end{sidewaysfigure}
\end{center}

\newpage
\section*{Exercise 27}
\begin{quote}
Derive (D) in system T
\end{quote}


\begin{center}

\begin{sidewaysfigure}
\scriptsize
\begin{prooftree}
\def\defaultHypSeparation{\hskip 0.001in}
	\AxiomC{$\neg \all \neg \varphi \lto \some \varphi$}
	
				\AxiomC{$\all \neg \varphi \lto \neg \varphi$}
						
						\AxiomC{$ \all \varphi \lto \varphi$}
				
						\AxiomC{$( \all \varphi  \lto \varphi)\lto (\neg \varphi \lto \neg \all \varphi)$}
	
					\BinaryInfC{$\neg \varphi \lto \neg \all \varphi$}
					\AxiomC{$(\neg \varphi \lto \neg \all \varphi) \lto ((\all \neg \varphi \lto \neg \varphi )\lto (\all \neg \varphi \lto  \neg \all \varphi))$}
					
				\BinaryInfC{$(\all \neg \varphi \lto \neg \varphi )\lto (\all \neg \varphi \lto  \neg \all \varphi)$}
				
		
			\BinaryInfC{$\all \neg \varphi \lto  \neg \all \varphi$}
			\AxiomC{$(\all \neg \varphi \lto  \neg \all \varphi) \lto (\all \varphi \lto \neg \all \neg \varphi)$}
	
		\BinaryInfC{$\all \varphi \lto \neg \all \neg \varphi$}
		\AxiomC{$(\all \varphi \lto \neg \all \neg \varphi)\lto((\neg \all \neg \varphi \lto \some \varphi) \lto (\all \varphi \lto \some \varphi)) $}
	\BinaryInfC{$(\neg \all \neg \varphi \lto \some \varphi) \lto (\all \varphi \lto \some \varphi)$}
\BinaryInfC{$\all \varphi \lto \some \varphi$}
\end{prooftree}
\end{sidewaysfigure}
\end{center}


\newpage
\section*{Exercise 28}
\begin{quote}
Derive (4) in system S5
\end{quote}



\begin{center}



\begin{sidewaysfigure}
\tiny

\begin{prooftree}
\def\defaultHypSeparation{\hskip 0.001in}
		
		
		
		\AxiomC{$\some \neg \varphi \lto \all \some \neg \varphi$}
				
					\AxiomC{$\some \neg \varphi \lto \neg \all \neg \neg \varphi$}	
					
							\AxiomC{$ \varphi \lto \neg \neg \varphi$}
							\UnaryInfC{$\all( \varphi \lto \neg \neg \varphi)$}
							\AxiomC{$\all( \varphi \lto \neg \neg \varphi) \lto ( \all  \varphi \lto \all \neg \neg \varphi)$}
							\BinaryInfC{$\all  \varphi \lto \all \neg \neg \varphi$}
							\AxiomC{$( \all  \varphi \lto \all \neg \neg \varphi) \lto (\neg \all \neg \neg \varphi \lto \neg \all \varphi)$}		
						\BinaryInfC{$\neg \all \neg \neg \varphi \lto \neg \all \varphi$}
						\AxiomC{$(\neg \all \neg \neg \varphi \lto \neg \all \varphi) \lto ((\some \neg \varphi \lto \neg \all \neg \neg \varphi) \lto (\some \neg \varphi \lto \neg \all \varphi))$}
					\BinaryInfC{$(\some \neg \varphi \lto \neg \all \neg \neg \varphi) \lto (\some \neg \varphi \lto \neg \all \varphi)$}
				\BinaryInfC{$\some \neg \varphi \lto \neg \all \varphi$}
\BinaryInfC{$(4)$}
\end{prooftree}

\begin{prooftree}
\def\defaultHypSeparation{\hskip 0.001in}
		
		
		
		\AxiomC{$\some \neg \varphi \lto \all \some \neg \varphi$}
				
					\AxiomC{$(4)$}	
					\noLine		
				\UnaryInfC{$\some \neg \varphi \lto \neg \all \varphi$}
				\UnaryInfC{$\all(\some \neg \varphi \lto \neg \all \varphi)$}
				\AxiomC{$\all(\some \neg \varphi \lto \neg \all \varphi) \lto (\all \some \neg \varphi \lto \all \neg \all \varphi)$}
			\BinaryInfC{$\all \some \neg \varphi \lto \all \neg \all \varphi$}
			\AxiomC{$(\all \some \neg \varphi \lto \all \neg \all \varphi)\lto ((\some \neg \varphi \lto \all \some \neg \varphi) \lto (\some \neg \varphi \lto \all \neg \all \varphi))$}
		\BinaryInfC{$(\some \neg \varphi \lto \all \some \neg \varphi) \lto (\some \neg \varphi \lto \all \neg \all \varphi)$}
	\BinaryInfC{$\some \neg \varphi \lto \all \neg \all \varphi$}
	\AxiomC{$(\some \neg \varphi \lto \all \neg \all \varphi) \lto ((\neg \all \varphi \lto \some \neg  \varphi)\lto (\neg \all \varphi \lto \all \neg \all \varphi))$}
\BinaryInfC{$(3)$}
\end{prooftree}

\begin{prooftree}
\def\defaultHypSeparation{\hskip 0.001in}
		
		\AxiomC{$(3)$}
		\noLine
		\UnaryInfC{$(\neg \all \varphi \lto \some \neg  \varphi)\lto (\neg \all \varphi \lto \all \neg \all \varphi)$}
		
			\AxiomC{$\neg \all \neg \neg \varphi \lto \some \neg \varphi$}
							
							\AxiomC{$\neg \neg  \varphi  \lto \varphi$}
						\UnaryInfC{$\all (\neg \neg  \varphi  \lto \varphi)$}
						\AxiomC{$\all (\neg \neg  \varphi  \lto \varphi)\lto (\all  \neg \neg  \varphi  \lto \all \varphi)$}
					\BinaryInfC{$\all  \neg \neg  \varphi  \lto \all \varphi$}
					\AxiomC{$(\all  \neg \neg  \varphi  \lto \all \varphi)\lto (\neg \all \varphi  \lto \neg \all \neg \neg \varphi)$}
				\BinaryInfC{$\neg \all \varphi  \lto \neg \all \neg \neg \varphi$}			
				\AxiomC{$(\neg \all \varphi  \lto \neg \all \neg \neg \varphi) \lto ((\neg \all \neg \neg \varphi \lto \some \neg \varphi) \lto (\neg \all \varphi \lto \some \neg \varphi))$}
			\BinaryInfC{$(\neg \all \neg \neg \varphi \lto \some \neg \varphi) \lto (\neg \all \varphi \lto \some \neg \varphi)$}
		\BinaryInfC{$\neg \all \varphi \lto \some \neg \varphi$}
	\BinaryInfC{$\neg \all \varphi \lto \all \neg \all \varphi$}
	\AxiomC{$(\neg \all \varphi \lto \all \neg \all \varphi) \lto (\neg \all \neg \all \varphi \lto \all \varphi)$}
\BinaryInfC{$(2)$}
\end{prooftree}
\begin{prooftree}
\def\defaultHypSeparation{\hskip 0.001in}

				\AxiomC{$\neg \all \neg \all \varphi \lto \some \all \varphi$}
					\AxiomC{\tiny$(2)$}
					\noLine
					\UnaryInfC{$\neg \all \neg \all \varphi \lto \all \varphi$}
					\AxiomC{$(\neg \all \neg \all \varphi \lto \all \varphi)\lto((\some \all \varphi \lto \neg \all \neg \all \varphi ) \lto (\some \all\varphi \lto \all \varphi))$}
				\BinaryInfC{$(\some \all \varphi \lto \neg \all \neg \all \varphi ) \lto (\some \all\varphi \lto \all \varphi)$}
			\BinaryInfC{$\some \all\varphi \lto \all \varphi$}
		\UnaryInfC{$\all (\some \all\varphi \lto \all \varphi)$}		
		\AxiomC{$\all (\some \all\varphi \lto \all \varphi)\lto (\all \some \all\varphi \lto \all \all \varphi) $}
	\BinaryInfC{$\all \some \all\varphi \lto \all \all \varphi$}
	\AxiomC{$(\all \some \all \varphi \lto \all \all \varphi)\lto ((\all \varphi \lto \all \some \all \varphi)\lto (\all \varphi \lto \all \all \varphi))$}
\BinaryInfC{$(1)$}
\end{prooftree}

\begin{prooftree}
\def\defaultHypSeparation{\hskip 0.001in}
	\AxiomC{\tiny$(1)$}
	\noLine
	\UnaryInfC{\tiny$(\all \varphi \lto \all \some \all \varphi)\lto (\all \varphi \lto \all \all \varphi)$}
		\AxiomC{$\some \all \varphi \lto \all \some \all \varphi$}
			
				\AxiomC{$\neg \all \neg \varphi \lto \some \all \varphi$}
						\AxiomC{$\all \neg\all \varphi \lto \neg \all \varphi$}
						\AxiomC{\tiny$(\all \neg\all \varphi \lto \neg \all \varphi )\lto(\all \varphi \lto \neg\all \neg\all \varphi)$}
						
						\BinaryInfC{$\all \varphi \lto \neg\all \neg\all \varphi$}
					\AxiomC{\tiny$(\all \varphi \lto \neg\all \neg\all \varphi)\lto((\neg \all \neg \varphi \lto \some \all \varphi)\lto(\all \varphi \some \all \varphi))$}
				\BinaryInfC{\tiny$(\neg \all \neg \varphi \lto \some \all \varphi)\lto(\all \varphi \some \all \varphi)$}
			\BinaryInfC{$\all \varphi \some \all \varphi$}
			\AxiomC{\tiny$(\all \varphi \some \all \varphi)\lto((\some \all \varphi \lto \all \some \all \varphi)\lto (\all \varphi \lto \all \some \all \varphi)) $}
		\BinaryInfC{\tiny$(\some \all \varphi \lto \all \some \all \varphi)\lto (\all \varphi \lto \all \some \all \varphi) $}
	\BinaryInfC{$\all \varphi \lto \all \some \all \varphi$}
\BinaryInfC{$\all \varphi \lto \all \all \varphi$}
\end{prooftree}
\end{sidewaysfigure}
\end{center}

\newpage
\section*{Exercise 29}
\begin{quote}
Derive $\some \all \some \varphi \lto \some \varphi$ in system S4.
\end{quote}


\begin{center}

\begin{sidewaysfigure}
\scriptsize

\begin{prooftree}
\def\defaultHypSeparation{\hskip 0.001in}


						
						\AxiomC{$\some \varphi \lto \neg \all \neg \varphi$}
							\AxiomC{$\all \some \varphi \lto \some \varphi$}
							\AxiomC{$(\all \some \varphi \lto \some \varphi)\lto ((\some \varphi \lto \neg \all \neg \varphi)\lto (\all \some\varphi \lto  \neg \all \neg \varphi)) $}
							
						\BinaryInfC{$(\some \varphi \lto \neg \all \neg \varphi)\lto (\all \some\varphi \lto  \neg \all \neg \varphi) $}
						\BinaryInfC{$\all \some\varphi \lto  \neg \all \neg \varphi $}
						\AxiomC{$(\all \some\varphi \lto  \neg \all \neg \varphi )\lto (\all \neg \varphi \lto  \neg \all \some\varphi)$}
					\BinaryInfC{$\all \neg \varphi \lto  \neg \all \some\varphi$}

\UnaryInfC{\tiny$(2)$}
\end{prooftree}

\begin{prooftree}
\def\defaultHypSeparation{\hskip 0.001in}

		\AxiomC{$\all \neg \varphi \lto \all \all \neg \varphi $}
						
				\AxiomC{\tiny$(2)$}
				\noLine
				\UnaryInfC{$\all (\all \neg \varphi \lto  \neg \all \some\varphi)$}
				\AxiomC{$\all (\all \neg \varphi \lto  \neg \all \some\varphi)\lto (\all \all \neg \varphi \lto \all \neg \all \some\varphi)$}
			\BinaryInfC{$\all \all \neg \varphi \lto \all \neg \all \some \varphi$}
			\AxiomC{$(\all \all \neg \varphi \lto \all \neg \all \some\varphi)\lto((\all \neg \varphi \lto \all \all \neg \varphi) \lto (\all \neg \varphi \lto \all \neg \all \some \varphi))$}
		\BinaryInfC{$(\all \neg \varphi \lto \all \all \neg \varphi) \lto (\all \neg \varphi \lto \all \neg \all \some \varphi)$}
	\BinaryInfC{$\all \neg \varphi \lto \all \neg \all \some \varphi$}
	\AxiomC{$(\all \neg \varphi \lto \all \neg \all \some \varphi) \lto (\neg \all \neg \all \some \varphi  \lto \neg \all \neg \varphi)$}
\BinaryInfC{\tiny$(1)$}
\end{prooftree}


\begin{prooftree}
\def\defaultHypSeparation{\hskip 0.001in}
	\AxiomC{$\some \all \some \varphi \lto \neg \all \neg \all \some \varphi$}
		\AxiomC{$\neg \all \neg \varphi \lto \some \varphi$}
			\AxiomC{\tiny $(1)$}
			\noLine
			\UnaryInfC{$\neg \all \neg \all \some \varphi  \lto \neg \all \neg \varphi $}
			\AxiomC{$(\neg \all \neg \all \some \varphi  \lto \neg \all \neg \varphi) \lto ((\neg \all \neg \varphi \lto \some \varphi) \lto (\neg \all \neg \all \some \varphi \lto \some \varphi))$}	
					
		\BinaryInfC{$(\neg \all \neg \varphi \lto \some \varphi) \lto (\neg \all \neg \all \some \varphi \lto \some \varphi)$}
	
	\BinaryInfC{$\neg \all \neg \all \some \varphi \lto \some \varphi$}
	\AxiomC{$(\neg \all \neg \all \some \varphi \lto \some \varphi) \lto ((\some \all \some \varphi \lto \neg \all \neg \all \some \varphi) \lto (\some \all \some \varphi \lto \some \varphi))$}
	\BinaryInfC{$(\some \all \some \varphi \lto \neg \all \neg \all \some \varphi) \lto (\some \all \some \varphi \lto \some \varphi)$}
	
\BinaryInfC{$\some \all \some \varphi \lto \some \varphi$}
\end{prooftree}
\end{sidewaysfigure}
\end{center}

\newpage


\section*{Exercise 30}
\begin{quote}
$A \in \mathbf{S5}$ iff $A$ is valid in all frames where the accessibility relation is
an equivalence relation.\\

Explain, why ‘equivalence relation’ (using the theorem below).
\end{quote}

By definition $\mathbf{S5}$ has the axioms (K), (T), (5). Hence, by specifying the general theorem on the slides one obtains.

\begin{quote}
The logic  $\mathbf{S5}$ with axioms (T) and (5) (in
addition to (K) and CL axioms) is sound and complete for frames,
where the accessibility relation satisfies the properties (E1) and (E5).
\end{quote}

If $A \in \mathbf{S5}$ then $A$ is derivable in the system $\mathbf{S5}$. Given the theorem above, one can use soundness to conclude that $A$ is valid in all frames that satisfy the properties (E1) and E5. Moreover, given a frame that satisfies (E1) and (E5), in which $A$ is valid, one can use the  completeness side of the theorem above to conclude that $A$ is derivable in $\mathbf{S5}$, i.e. $A \in \mathbf{S5}$.

Now the last thing to show is that a frame $\mathcal{F}$ is reflexive and euclidian if and only if its accessibility relation is an equivalence relation. As for "$\Rightarrow$". Consider Exercise 22. Here it is demonstrated that if the accessibility relation of $\mathcal{F}$ is reflexive and euclidian, then it is symmetric. Moreover, it was also demonstrated that if the accessibility relation of $\mathcal{F}$ is symmetric and euclidian, then it is transitive. Hence, one can conclude that from reflexivity (E1) and from euclidian (E5) one obtains reflexivity, symmetry and transitivity, which is precisely the axiomatisation of an equivalence relation.  As for "$\Leftarrow$". It is known that an equivalence relation is reflexive, transitive and symmetric. Clearly, reflexivity implies reflexivity hence, (E1) is satisfied. Moreover, as seen in Exercise 22 symmetry and transitivity imply (E5). Therefore, if the accessibility relation is an equivalence relation then it satisfies (E1) and (E5).



\end{document}
