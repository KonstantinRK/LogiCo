      \documentclass [11pt]{article}
   \usepackage{latexsym}
   \usepackage{amssymb}
      \usepackage{amsmath}
      \usepackage{amsthm}
   \usepackage{url}

   \textwidth      15cm
   \textheight     23cm
   \oddsidemargin 0.5cm
   \topmargin    -0.5cm
   \evensidemargin\oddsidemargin

% \newcommand{\nop}[1]{#1}
 \newcommand{\nop}[1]{}


   \pagestyle{plain}
   \bibliographystyle{plain}


%
%  \newtheorem{theorem}{Theorem}
%  \newtheorem{lemma}[theorem]{Lemma}
%  \newtheorem{exercise}{Exercise}

  \newcommand{\ra}{\rightarrow}
  \newcommand{\Ra}{\Rightarrow}
  \newcommand{\La}{\Leftarrow}
  \newcommand{\la}{\leftarrow}
  \newcommand{\LR}{\Leftrightarrow}

  \renewcommand{\phi}{\varphi}
  \renewcommand{\theta}{\vartheta}


\newcommand{\ccfont}[1]{\protect\mathsf{#1}}
\newcommand{\NP}{\ccfont{NP}}
\newcommand{\Ptime}{\ccfont{P}}
\newcommand{\phs}[1]{\Sigma_{#1}^{\Ptime}}
\newcommand{\php}[1]{\Pi_{#1}^{\Ptime}}
\newcommand{\pha}[1]{\Lambda_{#1}^{\Ptime}}
\newcommand{\QSAT}[1]{\ccfont{QSAT}_{#1}}
\newcommand{\MINSAT}{\mbox{\bf MINIMAL MODEL SAT}}
\newcommand{\True}{\mathbf{true}}
\newcommand{\False}{\mathbf{false}}
%\newcommand{\QSAT}[1]{\mathit{QBF}_{\exists, 2}}

\renewcommand{\labelenumi}{(\alph{enumi})}



  \newcommand{\lto}{\rightarrow}
  \newcommand{\sto}{\Rightarrow}
  \newcommand{\snto}{\Leftarrow}
  \newcommand{\lnto}{\leftarrow}
  \newcommand{\siff}{\Leftrightarrow}
   \newcommand{\liff}{\leftrightarrow}
   \newcommand{\nmodels}{\not\models}

\newcommand{\clau}{\mathit{Clause}}
\newcommand{\var}{\mathit{Var}}
\newcommand{\lit}{\mathit{Lit}}
\newcommand{\dom}{\mathit{Dom}}

  \newtheorem{theorem}{Theorem}
  \newtheorem{lemma}[theorem]{Lemma}
  \newtheorem{corollary}[theorem]{Corollary}
    \newtheorem{observation}[theorem]{Observation}
  \newtheorem{proposition}[theorem]{Proposition}
  \newtheorem{conjecture}[theorem]{Conjecture}
  \newtheorem{definition}[theorem]{Definition}
  \newtheorem{example}[theorem]{Example}
  \newtheorem{remark}[theorem]{Remark}
  \newtheorem{exercise}[theorem]{Exercise}
 

%%%%%%%%%%%%%%%%%%%%%%%%%%%%%%%%%%%%%%%%%%%%%%%%%%%%%%%%%%%%%%%%%%%

\begin{document}




\noindent
\begin{exercise}[5 credits]
{\em 
Recall the following characterizations of the complexity classes 
$\phs{i}$ and $\php{i}$ for $i \geq 1$. 
}%em


\medskip

\noindent
\begin{theorem}
\label{thm:main}
Let $L$ be a language and $i \geq 1$. 
\begin{itemize}
\item Then $L \in \phs{i}$ iff there is
a polynomially balanced relation $R$ such that the language 
$\{x\# y \mid (x,y) \in R\}$ is in $\php{i-1}$ and 
\[
L = \{x \mid \mbox{there exists a } y 
\mbox{ with } |y| \leq |x|^k
\mbox{ s.t. } (x,y) \in R\}
\]
\item Then $L \in \php{i}$ iff there is
a polynomially balanced relation $R$ such that the language 
$\{x\# y \mid (x,y) \in R\}$ is in $\phs{i-1}$ and 
\[
L = \{x \mid \mbox{for all } y \mbox{ with } |y| \leq |x|^k, (x,y) \in R\}
\]
\end{itemize}
\end{theorem}

\medskip
\begin{corollary}
Let $L$ be a language and $i \geq 1$.
\begin{itemize}
\item  Then $L \in \phs{i}$ iff there is
a polynomially balanced, polynomial-time decidable $(i+1)$-ary
relation $R$ such that 
\[
L = \{x \mid \exists y_1 \forall y_2\exists y_3\cdots Q y_i \mbox{ such
that } 
(x,y_1,\ldots,y_i) \in R\}
\]
where $Q$ is $\forall$ if $i$ is even and $\exists$ if $i$ is odd. 

\item 
Then $L \in \php{i}$ iff there is
a polynomially balanced, polynomial-time decidable $(i+1)$-ary
relation $R$ such that 
\[
L = \{x \mid \forall y_1 \exists y_2\forall y_3\cdots Q y_i \mbox{ such
that } 
(x,y_1,\ldots,y_i) \in R\}
\]
where $Q$ is $\exists$ if $i$ is even and $\forall$ if $i$ is odd. 
\end{itemize}
\end{corollary}


\medskip
\noindent
{\em 
Give a rigorous proof of this corollary.
}%em
\end{exercise}

\noindent
{\bf Hint.} Use the above theorem and proceed by induction on $i$.
It suffices to prove the correctness of the characterization of 
$\phs{i}$. You may use the characterization of $\php{i}$ in the induction step.



\medskip

\paragraph*{Solution}

To make the proof Corollary \label{cor:main} of more concise consider the following.
\begin{definition}
For some $n$ let $\mathcal{R}^n$ represent the set of all $n$-ary relations.
Moreover, if $R \in \mathcal{R}^2$, then let $\mathcal{L}(R):= \{x\# y \mid (x,y) \in R\}$.
Furthermore, as an abbreviation let p.b. stand for polynomially balanced and let p.d. stand for polynomial-time decidable
\end{definition}
Before delving into the proof of Corollary \label{cor:main} consider the following remark.
\begin{remark}
As proposed in the slides. It is possible to omit the condition $|y| \leq |x|^k$,  due to fact that $R$ is a polynomially balanced relation.
\end{remark}

Moreover, some preliminary results. 

\begin{lemma}
\label{lem:lang-rel}
For some $R \in \mathcal{R}^2$ with $R$ p.b. it follows that $(x,y) \in R =_{\Ptime} x\#y \in \mathcal{L}(R)$.
\end{lemma}
\begin{proof}
Let $\tau$ be the bijection $\tau(x,y) = x\#y$ (and its inverse $\tau^{-1}(x\#y) = (x,y)$).

Firstly, $\tau$ can be computed in polynomial time with respect to $|x|$\footnote{This is stronger than required.}. That is, given the input $(x,y)$ one can easily create the string $x\#y$ by copying $x$ adding a $\#$ and copying $y$, and since $R$ is p.b. it must be that $|y| \leq |x|^k$ for some $k >0$. Hence, it follows that this transformation can be done in linear time with respect to $|x|$.

Secondly, $\tau^{-1}$ can be computed in polynomial time with respect to $|x|$. That is, given the input $x\#y$ one can easily create the tuple $(x,y)$ iterating over $x\#y$ until $\#$ is reached. Then one merely copies everything before the separator into the first position of the tuple and everything after into the second position. Since $R$ is p.b. it must be that $|y| \leq |x|^k$ for some $k >0$. Hence, it follows that this transformation can be done in linear time with respect to $|x|$.


Thirdly, it must be established that 
\begin{equation*}
(x,y) \in R \iff \tau(x,y) \in \mathcal{L}(R) \land \tau^{-1}(x\#y) \iff x\#y \in \mathcal{L}(R)
\end{equation*}
However, since $\tau(x,y)=x\#y$ and $\tau^{-1}(x\#y)=(x,y)$ this follows by construction of $\mathcal{L}$.
\end{proof}
Finally, allowing the demonstration of the following corollary.

\begin{corollary}
\label{cor:main}
Let $L$ be a language and $i \geq 1$. 
\begin{itemize}
\item  $L \in \phs{i}$ iff there is
a polynomially balanced, polynomial-time decidable $(i+1)$-ary
relation $R$ such that 
\[
L = \{x \mid \exists y_1 \forall y_2\exists y_3\cdots Q y_i \mbox{ such
that } 
(x,y_1,\ldots,y_i) \in R\}
\]
where $Q$ is $\forall$ if $i$ is even and $\exists$ if $i$ is odd. 
\item $L \in \php{i}$ iff there is
a polynomially balanced, polynomial-time decidable $(i+1)$-ary
relation $R$ such that 
\[
L = \{x \mid \forall y_1 \exists y_2\forall y_3\cdots Q y_i \mbox{ such
that } 
(x,y_1,\ldots,y_i) \in R\}
\]
where $Q$ is $\exists$ if $i$ is even and $\forall$ if $i$ is odd. 
\end{itemize}

\end{corollary}
\begin{proof}
Firstly, in the subsequent proof a p.b. relation $R \in \mathcal{R}^2$ will be obtained  by applying Theorem \ref{thm:main} w.l.o.g. assume that $\forall (x,y) \in R \; |x|>1$. As for any polynomially balanced relation $R$, where there exists an $(x,y) \in R$ such that $x\leq 1$ it is possible to construct the relation $R':=\{( x\boxtimes , y) \mid (x,y) \in R\}$ (where $\boxtimes$ is some new character) that still is polynomially balanced, i.e. increasing the size of $x$ can not invalidate the condition for being polynomially balanced. Moreover, this construction can clearly be done in polynomial time. Moreover, using $R'$ one can just as easily reconstruct $R$. Hence, $\mathcal{L}(R')$ will live at the same level of the polynomial hierarchy as $\mathcal{L}(R)$.


Secondly, the claim can be demonstrated by induction on $i$.

\begin{itemize}
\item \textbf{IH:} For a fixed $i>0$.

\begin{equation*}
\begin{split}
L \in \phs{i} \iff & \exists R  \in  \mathcal{R}^{i+1} \; R \; p.b. \land R \; p.d. \land  \\
&L = \{x \mid \exists y_1 \forall y_2\exists y_3\dots Q y_i  \; (x,y_1,\ldots,y_i) \in R\}
\\
& (i \text{ even}  \sto Q=\forall) \land (i \text{ odd}  \sto Q=\exists) 
\end{split}
\end{equation*}
%where $Q$ is $\forall$ if $i$ is even and $\exists$ if $i$ is odd.
and 
\begin{equation*}
\begin{split}
L \in \php{i} \iff & \exists R  \in  \mathcal{R}^{i+1} \; R \; p.b. \land R \; p.d. \land  \\
&L = \{x \mid \forall y_1 \exists y_2\forall y_3\dots Q y_i  \; (x,y_1,\ldots,y_i) \in R\} \land \\
& (i \text{ even}  \sto Q=\exists) \land (i \text{ odd}  \sto Q=\forall) 
\end{split}
\end{equation*}
%where $Q$ is $\exists$ if $i$ is even and $\forall$ if $i$ is odd. 


\item \textbf{IB:} For $i=1$.
Firstly, starting from $L \in \phs{1}$, by Theorem \ref{thm:main}, on obtains
\begin{equation*}
\begin{split}
L \in \phs{1} \iff \exists R \in \mathcal{R}^{2} \; R \; p.b.  \land \mathcal{L}(R) \in \php{0} \land L = \{x \mid \exists y \; (x,y) \in R\}
\end{split}
\end{equation*}
Since, $\php{0}=\Ptime$ this is equivalent to
\begin{equation*}
\begin{split}
L \in \phs{1} \iff  \exists R \in \mathcal{R}^{2} \; R \; p.b.  \land \mathcal{L}(R) \in \Ptime \land L = \{x \mid \exists y \; (x,y) \in R\}
\end{split}
\end{equation*}
By Lemma \ref{lem:lang-rel}, since $\mathcal{L}(R) \in \Ptime$, it follows that $(x,y) \in R$  can be decided in polynomial time. Hence, the previous equality is equivalent to 
\begin{equation*}
\begin{split}
L \in \phs{1} \iff  \exists R \in \mathcal{R}^{2} \; R \; p.b.  \land R \; p.d. \land L = \{x \mid \exists y \; (x,y) \in R\}
\end{split}
\end{equation*}
which is precisely what was desired.


Secondly, starting from $L \in \php{1}$ this is done completely in analogue. That is, by Theorem \ref{thm:main}, on obtains
\begin{equation*}
\begin{split}
L \in \php{1} \iff \exists R \in \mathcal{R}^{2} \; R \; p.b.  \land \mathcal{L}(R) \in \phs{0} \land L = \{x \mid \forall  y \; (x,y) \in R\}
\end{split}
\end{equation*}
Since, $\phs{0}=\Ptime$ this is equivalent to
\begin{equation*}
\begin{split}
L \in \php{1} \iff  \exists R \in \mathcal{R}^{2} \; R \; p.b.  \land \mathcal{L}(R) \in \Ptime \land L = \{x \mid \forall y \; (x,y) \in R\}
\end{split}
\end{equation*}
By Lemma \ref{lem:lang-rel}, since $\mathcal{L}(R) \in \Ptime$, it follows that $(x,y) \in R$  can be decided in polynomial time. Hence, the previous equality is equivalent to 
\begin{equation*}
\begin{split}
L \in \php{1} \iff  \exists R \in \mathcal{R}^{2} \; R \; p.b.  \land R \; p.d. \land L = \{x \mid \forall y \; (x,y) \in R\}
\end{split}
\end{equation*}
which is precisely what was desired.


\item \textbf{IS:} Let $i=n+1$. Observe the following
\begin{equation*}
\begin{split}
&L \in \phs{n+1}\\
\stackrel{\text{(i)}}{\iff}& \exists R_2 \in \mathcal{R}^{2} \; R_2 \; p.b.  \land \mathcal{L}(R_2) \in \php{n} \land L = \{x \mid \exists y \; (x,y) \in R_2\} \\
\stackrel{\text{(ii)}}{\iff} &\exists R_2 \in \mathcal{R}^{2} \; R_2 \; p.b.  \land  \exists R_{n+1}  \in  \mathcal{R}^{n+1} \; R_{n+1} \; p.b. \land R_{n+1} \; p.d. \land  \\
&\mathcal{L}(R_2) = \{x\#y \mid \forall y_1 \exists y_2\forall y_3\dots Q y_n  \; (x\#y,y_1,\ldots,y_n) \in R_{n+1}\}  \land \\
& L = \{x \mid \exists y \; (x,y) \in R_2\}  \land \\
&  (n \text{ even}  \sto Q=\exists )\land (n \text{ odd}  \sto Q=\forall )\\
\stackrel{\text{(iii)}}{\iff} &\exists R_2 \in \mathcal{R}^{2} \; R_2 \; p.b.  \land  \exists R_{n+2}  \in  \mathcal{R}^{n+2} \; R_{n+2} \; p.b. \land R_{n+2} \; p.d. \land  \\
&\mathcal{L}(R_2) = \{x\#y \mid \forall y_1 \exists y_2\forall y_3\dots Q y_n  \; (x,y,y_1,\ldots,y_n) \in R_{n+2}\}  \land \\ 
& L = \{x \mid \exists y \; (x,y) \in R_2\} \land  \\ 
&  (n \text{ even}  \sto Q=\exists )\land (n \text{ odd}  \sto Q=\forall )\\
\stackrel{\text{(iv)}}{\iff} &\exists R_2 \in \mathcal{R}^{2} \; R_2 \; p.b.  \land  \exists R_{n+2}  \in  \mathcal{R}^{n+2} \; R_{n+2} \; p.b. \land R_{n+2} \; p.d. \land  \\
&R_2 = \{(x,y) \mid \forall y_1 \exists y_2\forall y_3\dots Q y_n  \; (x,y,y_1,\ldots,y_n) \in R_{n+2}\}  \land\\ 
& L = \{x \mid \exists y \; (x,y) \in  \{(x,y) \mid \forall y_1 \exists y_2\forall y_3\dots Q y_n  \; (x,y,y_1,\ldots,y_n) \in R_{n+2}\}\}  \land  \\ 
&  (n \text{ even}  \sto Q=\exists )\land (n \text{ odd}  \sto Q=\forall )\\
\stackrel{\text{(v)}}{\iff} & \exists R_{n+2}  \in  \mathcal{R}^{n+2} \; R_{n+2} \; p.b. \land R_{n+2} \; p.d. \land  \\
& L = \{x \mid \exists y \; (x,y) \in  \{(x,y) \mid \forall y_1 \exists y_2\forall y_3\dots Q y_n  \; (x,y,y_1,\ldots,y_n) \in R_{n+2}\}\} \land  \\ 
&  (n \text{ even}  \sto Q=\exists )\land (n \text{ odd}  \sto Q=\forall )\\
\stackrel{\text{(vi)}}{\iff} & \exists R_{n+2}  \in  \mathcal{R}^{n+2} \; R_{n+2} \; p.b. \land R_{n+2} \; p.d. \land  \\
& L = \{x \mid \exists y \forall y_1 \exists y_2\forall y_3\dots Q y_n  \; (x,y,y_1,\ldots,y_n) \in R_{n+2}\}\land  \\ 
&  (n \text{ even}  \sto Q=\exists )\land (n \text{ odd}  \sto Q=\forall )\\
\stackrel{\text{(vii)}}{\iff} & \exists R_{n+2}  \in  \mathcal{R}^{n+2} \; R_{n+2} \; p.b. \land R_{n+2} \; p.d. \land  \\
& L = \{x \mid \exists y_1 \forall y_2 \exists y_3 \dots Q y_{n+1}  \; (x,y,y_1,\ldots,y_n) \in R_{n+2}\}  \land \\ 
&  (n+1 \text{ even}  \sto Q=\forall) \land (n+1 \text{ odd}  \sto Q=\exists) \\
\end{split}
\end{equation*}
\end{itemize}

\begin{enumerate}
\item[(i)] Here Theorem \ref{thm:main} was applied.
\item[(ii)] Here the \textbf{IH} was applied, i.e. 
\begin{equation*}
\begin{split}
\mathcal{L}(R_2) \in \php{n} \iff& \exists R_{n+1}  \in  \mathcal{R}^{n+1} \; R_{n+1} \; p.b. \land R_{n+1} \; p.d. \land  \\
&\mathcal{L}(R_2) = \{x\#y \mid \forall y_1 \exists y_2\forall y_3\dots Q y_n  \; (x\#y,y_1,\ldots,y_n) \in R_{n+1}\} \land  \\
 & (n \text{ even}  \sto Q=\exists) \land (n \text{ odd}  \sto Q=\forall )
\end{split}
\end{equation*}


\item[(iii)] Firstly, $\Rightarrow$. Starting from 
 \begin{equation*}
\begin{split}
&\exists R_2 \in \mathcal{R}^{2} \; R_2 \; p.b.  \land  \exists R_{n+1}  \in  \mathcal{R}^{n+1} \; R_{n+1} \; p.b. \land R_{n+1} \; p.d. \land  \\
&\mathcal{L}(R_2) = \{x\#y \mid \forall y_1 \exists y_2\forall y_3\dots Q y_n  \; (x\#y,y_1,\ldots,y_n) \in R_{n+1}\} 
\end{split}
\end{equation*}
Taking the relation $R_{n+1}$ one can construct the relation $R_{n+2} \in \mathcal{R}^{n+2}$ such that 
 \begin{equation*}
\begin{split}
 R_{n+2}=\{ (x,y,y_1,\ldots,y_n) \mid (x\#y,y_1,\ldots,y_n) \in R_{n+1}\} 
\end{split}
\end{equation*}
To do so one merely has to split the first entry in $(x\#y,y_1,\ldots,y_n) \in R_{n+1}$ into two, which can be done in polynomial time (similar argument as in Lemma \ref{lem:lang-rel}). Moreover, by construction it clearly holds that
 \begin{equation*}
\begin{split}
 (x,y,y_1,\ldots,y_n) \in R_{n+2} \iff  (x\#y,y_1,\ldots,y_n) \in R_{n+1}
\end{split}
\end{equation*}
Since by assumption $R_2$ is polynomially balanced it follows that there exists a $k$ such that for any $(x,y) \in R_{2}$ one has $|y| \leq |x|^k$. Furthermore, it is known that $R_{n+1}$ is p.b., thus there exists a $k'$ such that for any $1 \leq i \leq n$ one has $|y_i| \leq |x\#y|^{k'}\leq |x| +1 +|x|^k$. By assumption, i.e. $|x|>1$, it follows that there exists a $k^* \geq k$ such that $|y_i| \leq |x|^{k*}$ and $|y| \leq |x|^{k*}$. Hence, $R_{n+2}$ is polynomially balanced. 
Additionally, one knows that $R_{n+1}$ is p.d., thus $R_{n+2}$ can be decided by concatenating the first two entries and querying $R_{n+1}$. Both operations can be done in polynomial time, thus $R_{n+2}$ is p.d..
Hence, one obtains 
 \begin{equation*}
\begin{split}
&\exists R_2 \in \mathcal{R}^{2} \; R_2 \; p.b.  \land  \exists R_{n+2}  \in  \mathcal{R}^{n+2} \; R_{n+2} \; p.b. \land R_{n+2} \; p.d. \land  \\
&\mathcal{L}(R_2) = \{x\#y \mid \forall y_1 \exists y_2\forall y_3\dots Q y_n  \; (x,y,y_1,\ldots,y_n) \in R_{n+2}\} 
\end{split}
\end{equation*}

\medskip

Secondly, $\Leftarrow$.  This argument is essentially the same as the previous one, but in reverse (and with slight alterations in the complexity arguments). That is, starting from
 \begin{equation*}
\begin{split}
&\exists R_2 \in \mathcal{R}^{2} \; R_2 \; p.b.  \land  \exists R_{n+2}  \in  \mathcal{R}^{n+2} \; R_{n+2} \; p.b. \land R_{n+2} \; p.d. \land  \\
&\mathcal{L}(R_2) = \{x\#y \mid \forall y_1 \exists y_2\forall y_3\dots Q y_n  \; (x,y,y_1,\ldots,y_n) \in R_{n+2}\} 
\end{split}
\end{equation*}
Taking the relation $R_{n+2}$ one can construct the relation $R_{n+1} \in \mathcal{R}^{n+1}$ such that 
 \begin{equation*}
\begin{split}
 R_{n+1}=\{ (x\#y,y_1,\ldots,y_n) \mid (x,y,y_1,\ldots,y_n) \in R_{n+2}\} 
\end{split}
\end{equation*}
To do so one merely has to concatenate the first two entries in $(x,y,y_1,\ldots,y_n) \in R_{n+1}$ using the separator $\#$, which can be done in polynomial time (similar argument as in Lemma \ref{lem:lang-rel}). Moreover, it clearly holds that
 \begin{equation*}
\begin{split}
(x\#y,y_1,\ldots,y_n) \in R_{n+1} \iff   (x,y,y_1,\ldots,y_n) \in R_{n+2} 
\end{split}
\end{equation*}
It is known that $R_{n+2}$ is p.b., thus there exists a $k$ such that for $1 \leq i \leq n$, $|y_i| \leq |x|^k$ and $|y| \leq |x|^k$. Now since $|x| <|x\#y|$ it must be that $R_{n+1}$ is p.b. as well.
Additionally, one knows that $R_{n+2}$ is p.d., thus $R_{n+1}$ can be decided by splitting the first entry on $\#$ and querying $R_{n+2}$. Both operations can be done in polynomial time, thus $R_{n+1}$ is p.d..
Hence, one obtains 
 \begin{equation*}
\begin{split}
&\exists R_2 \in \mathcal{R}^{2} \; R_2 \; p.b.  \land  \exists R_{n+1}  \in  \mathcal{R}^{n+1} \; R_{n+1} \; p.b. \land R_{n+1} \; p.d. \land  \\
&\mathcal{L}(R_2) = \{x\#y \mid \forall y_1 \exists y_2\forall y_3\dots Q y_n  \; (x\#y,y_1,\ldots,y_n) \in R_{n+1}\} 
\end{split}
\end{equation*}


\item[(iv)] This equality is guaranteed by the following. Take an arbitrary relation $R$. Clearly, $(x,y) \in R \iff x\#y \in \mathcal{L}(R)$. Hence, in this particular case one has $(x,y) \in R_2 \iff x\#y \in \mathcal{L}(R_2)$. Now starting from  
\begin{equation*}
\begin{split}
\mathcal{L}(R_2) &= \{x\#y \mid \forall y_1 \exists y_2\forall y_3\dots Q y_n  \; (x,y,y_1,\ldots,y_n) \in R_{n+2}\}  \land \\ 
L &= \{x \mid \exists y \; (x,y) \in R_2\}  \\ 
\end{split}
\end{equation*}
due to 
 \begin{equation*}
\begin{split}
(\alpha,\beta) \in R_2 &\iff \alpha\#\beta \in\mathcal{L}(R_2) \\
&\iff \alpha\#\beta \in \{ x\#y \mid \forall y_1 \exists y_2\forall y_3\dots Q y_n  \; (x,y,y_1,\ldots,y_n) \in R_{n+2}\}  \\
&\iff (\alpha,\beta) \in \{(x,y) \mid \forall y_1 \exists y_2\forall y_3\dots Q y_n  \; (x,y,y_1,\ldots,y_n) \in R_{n+2}\} 
\end{split}
\end{equation*}
one obtains the equivalent statement
\begin{equation*}
\begin{split}
R_2 &= \{(x,y) \mid \forall y_1 \exists y_2\forall y_3\dots Q y_n  \; (x,y,y_1,\ldots,y_n) \in R_{n+2}\}  \land \\ 
L &= \{x \mid \exists y \; (x,y) \in  \{(x,y) \mid \forall y_1 \exists y_2\forall y_3\dots Q y_n  \; (x,y,y_1,\ldots,y_n) \in R_{n+2}\}\}  \\ 
\end{split}
\end{equation*}
%$\exists R \in \mathcal{R}^n \{x\#y \mid  \forall y_1 \exists y_2\forall y_3\dots Q y_n  \; (x\#y,y_1,\ldots,y_n) \in R_{n+1}\}$



\item[(v)] Firstly, $\Rightarrow$. Starting from 

\begin{equation*}
\begin{split}
&\exists R_2 \in \mathcal{R}^{2} \; R_2 \; p.b.  \land  \exists R_{n+2}  \in  \mathcal{R}^{n+2} \; R_{n+2} \; p.b. \land R_{n+2} \; p.d. \land  \\
&R_2 = \{(x,y) \mid \forall y_1 \exists y_2\forall y_3\dots Q y_n  \; (x,y,y_1,\ldots,y_n) \in R_{n+2}\}  \land\\ 
& L = \{x \mid \exists y \; (x,y) \in  \{(x,y) \mid \forall y_1 \exists y_2\forall y_3\dots Q y_n  \; (x,y,y_1,\ldots,y_n) \in R_{n+2}\}\}  \\ 
\end{split}
\end{equation*}
one can simply use weakening to obtain the part of the statement, where $R_2$ does not occur. 
\begin{equation*}
\begin{split}
& \exists R_{n+2}  \in  \mathcal{R}^{n+2} \; R_{n+2} \; p.b. \land R_{n+2} \; p.d. \land  \\
& L = \{x \mid \exists y \; (x,y) \in  \{(x,y) \mid \forall y_1 \exists y_2\forall y_3\dots Q y_n  \; (x,y,y_1,\ldots,y_n) \in R_{n+2}\}\}  \\ 
\end{split}
\end{equation*}
Thereby, eradicating all references of $R_2$. 

\medskip
Secondly, $\Leftarrow$. Starting from 
\begin{equation*}
\begin{split}
& \exists R_{n+2}  \in  \mathcal{R}^{n+2} \; R_{n+2} \; p.b. \land R_{n+2} \; p.d. \land  \\
& L = \{x \mid \exists y \; (x,y) \in  \{(x,y) \mid \forall y_1 \exists y_2\forall y_3\dots Q y_n  \; (x,y,y_1,\ldots,y_n) \in R_{n+2}\}\}  \\ 
\end{split}
\end{equation*}
One can define the relation $R_2:=\{(x,y) \mid \forall y_1 \exists y_2\forall y_3\dots Q y_n  \; (x,y,y_1,\ldots,y_n) \in R_{n+2}\} $.
Since it is known that $R_{n+2}$ is p.b. this implies that there exists a $k$ such that $|y| \leq |x|^k$, thus implying that $R_2$ is p.b..
Allowing one to conclude that
\begin{equation*}
\begin{split}
&\exists R_2 \in \mathcal{R}^{2} \; R_2 \; p.b.  \land  \exists R_{n+2}  \in  \mathcal{R}^{n+2} \; R_{n+2} \; p.b. \land R_{n+2} \; p.d. \land  \\
&R_2 = \{(x,y) \mid \forall y_1 \exists y_2\forall y_3\dots Q y_n  \; (x,y,y_1,\ldots,y_n) \in R_{n+2}\}  \land\\ 
& L = \{x \mid \exists y \; (x,y) \in  \{(x,y) \mid \forall y_1 \exists y_2\forall y_3\dots Q y_n  \; (x,y,y_1,\ldots,y_n) \in R_{n+2}\}\}  \\ 
\end{split}
\end{equation*}

\item[(vi)] Starting from $\{x \mid \exists y \; (x,y) \in  \{(x,y) \mid \forall y_1 \exists y_2\forall y_3\dots Q y_n  \; (x,y,y_1,\ldots,y_n) \in R_{n+2}\}\}$. Notice that 
\begin{equation*}
\begin{split}
 &(\alpha, \beta) \in \{(x,y) \mid \forall y_1 \exists y_2\forall y_3\dots Q y_n  \; (x,y,y_1,\ldots,y_n) \in R_{n+2}\} \\ 
 &\iff \forall y_1 \exists y_2\forall y_3\dots Q y_n  \; (\alpha, \beta,y_1,\ldots,y_n) \in R_{n+2}
\end{split}
\end{equation*}
From this it follows that 
\begin{equation*}
\begin{split}
 &\exists y \; (\alpha,y) \in  \{(x,y) \mid \forall y_1 \exists y_2\forall y_3\dots Q y_n  \; (x,y,y_1,\ldots,y_n) \in R_{n+2}\} \\ 
 &\iff \exists y \forall y_1 \exists y_2\forall y_3\dots Q y_n  \; (\alpha, y,y_1,\ldots,y_n) \in R_{n+2}
\end{split}
\end{equation*}
and therefore
\begin{equation*}
\begin{split}
 &\{x \mid \exists y \; (x,y) \in  \{(x,y) \mid \forall y_1 \exists y_2\forall y_3\dots Q y_n  \; (x,y,y_1,\ldots,y_n) \in R_{n+2}\}\}\\ 
 &= \{x \mid \exists y \forall y_1 \exists y_2\forall y_3\dots Q y_n  \; (x,y,y_1,\ldots,y_n) \in R_{n+2}\}
\end{split}
\end{equation*}


\item[(vii)] This particular renaming of bound variables is clearly an equivalence transformation. Hence, 
\begin{equation*}
\begin{split}
n \text{ even}  \sto Q=\exists \land n \text{ odd}  \sto Q=\forall  \iff  n+1 \text{ even}  \sto Q=\forall \land n+1 \text{ odd}  \sto Q=\exists 
\end{split}
\end{equation*}
remains to be established. However, this follows directly from the fact that $n$ is even if and only if $n+1$ is odd. That is, 
if $n$ was even, one has $Q=\exists$. However, this implies that for $Q=\exists$ for $n+1$ being odd and if $n+1$ is odd then $Q=\exists$, meaning that $Q=\exists$ for $n$ is even.
Analogous for the other case.
\end{enumerate}
\bigskip
The other case, i.e. for $L \in \php{n+1}$ is done completely analogously (see Appendix).
\end{proof}




























%
%
%
%
%To do so it suffices to prove Lemma XX and Lemma YY.
%
%\begin{lemma}
%\label{lem:sig}
%There exists a  a polynomially balanced relation $R$ with $\{x\# y \mid (x,y) \in R\}$ is in $\php{i-1}$, if and only if there exists a polynomially balanced, polynomial-time decidable $(i+1)$-ary
%relation $R'$ such that 
%\begin{equation*}
%\begin{split}
% \{x \mid \exists y \; |y| \leq |x|^k \land  (x,y) \in R\} = \{x \mid \exists y_1 \forall y_2\exists y_3\cdots Q y_i \;  (x,y_1,\ldots,y_i) \in R'\}
%\end{split}
%\end{equation*}
%where $Q$ is $\forall$ if $i$ is even and $\exists$ if $i$ is odd.
%\end{lemma}
%
%%
%
%\begin{definition}
%Let $R$ be a 2-ary relation, then $\mathcal{L}(R):= \{x\# y \mid (x,y) \in R\}$.
%\end{definition}
%
%
%\begin{remark}
%As proposed in the slides. It is possible to omit the condition $|y| \leq |x|^k$,  due to fact that $R$ is a polynomially balanced relation.
%\end{remark}
%
%\begin{lemma}
%\label{lem:to}
%For all $i>0$.
%\begin{itemize}
%\item If $R$ is a polynomially balanced relation such that the language $\{x\# y \mid (x,y) \in R\}$ is in $\php{i-1}$, then there exists a polynomially balanced, polynomial-time decidable $(i+1)$-ary
%relation $R'$ such that 
%\begin{equation*}
%\begin{split}
% \{x \mid \exists y \;   (x,y) \in R\} = \{x \mid \exists y_1 \forall y_2\exists y_3\cdots Q y_i \;  (x,y_1,\ldots,y_i) \in R'\}
%\end{split}
%\end{equation*}
%where $Q$ is $\forall$ if $i$ is even and $\exists$ if $i$ is odd.
%\item If $R$ is a polynomially balanced relation such that the language $\{x\# y \mid (x,y) \in R\}$ is in $\phs{i-1}$, then there exists a polynomially balanced, polynomial-time decidable $(i+1)$-ary
%relation $R'$ such that 
%\begin{equation*}
%\begin{split}
% \{x \mid \forall y \;   (x,y) \in R\} = \{x \mid \forall y_1 \exists y_2\forall y_3\cdots Q y_i \;  (x,y_1,\ldots,y_i) \in R'\}
%\end{split}
%\end{equation*}
%where $Q$ is $\exists$ if $i$ is even and $\forall$ if $i$ is odd.
%\end{itemize}
%\end{lemma}
%
%\begin{proof}
%\begin{itemize}
%\item \textbf{IH:} For $i>0$.
%\begin{itemize}
%\item If $R$ is a polynomially balanced relation such that the language $\{x\# y \mid (x,y) \in R\}$ is in $\php{i-1}$, then there exists a polynomially balanced, polynomial-time decidable $(i+1)$-ary
%relation $R'$ such that 
%\begin{equation*}
%\begin{split}
% \{x \mid \exists y \;   (x,y) \in R\} = \{x \mid \exists y_1 \forall y_2\exists y_3\cdots Q y_i \;  (x,y_1,\ldots,y_i) \in R'\}
%\end{split}
%\end{equation*}
%where $Q$ is $\forall$ if $i$ is even and $\exists$ if $i$ is odd.
%\item If $R$ is a polynomially balanced relation such that the language $\{x\# y \mid (x,y) \in R\}$ is in $\phs{i-1}$, then there exists a polynomially balanced, polynomial-time decidable $(i+1)$-ary
%relation $R'$ such that 
%\begin{equation*}
%\begin{split}
% \{x \mid \forall y \;   (x,y) \in R\} = \{x \mid \forall y_1 \exists y_2\forall y_3\cdots Q y_i \;  (x,y_1,\ldots,y_i) \in R'\}
%\end{split}
%\end{equation*}
%where $Q$ is $\exists$ if $i$ is even and $\forall$ if $i$ is odd.
%\end{itemize}
%
%\item \textbf{IB:}
%\begin{itemize}
%\item 
%Let $i=1$. Assume that one has $R$ polynomially balanced such that $\mathcal{L}(R) \in \php{0}$. Since $\php{0}=\Ptime$ it follows that $R$ is not only polynomially balanced, but also polynomially decidable. Finally, with $R$ being 2-ary, it follows that $R$ is the desired $R'$ and since $R'=R$ it is clearly the case that
%\begin{equation*}
%\begin{split}
% \{x \mid \exists y \;  (x,y) \in R\} = \{x \mid \exists y_1  \;  (x,y_1) \in R'\}
%\end{split}
%\end{equation*}
%\item  Let $i=1$. Assume that one has $R$ polynomially balanced such that $\mathcal{L}(R) \in \phs{0}$. Since $\phs{0}=\Ptime$ it follows that $R$ is not only polynomially balanced, but also polynomially decidable. Finally, with $R$ being 2-ary, it follows that $R$ is the desired $R'$ and since $R'=R$ it is clearly the case that
%\begin{equation*}
%\begin{split}
% \{x \mid \forall y \;  (x,y) \in R\} = \{x \mid \forall y_1  \;  (x,y_1) \in R'\}
%\end{split}
%\end{equation*}
%\end{itemize}
%
%
%\item \textbf{IS:} 
%Firstly, w.l.o.g. assume that $x>1$. As for any polynomially balanced relation $R$, where there exists an $(x,y) \in R$ such that $x\leq 1$ it is possible to construct the relation $R':=\{( x\boxtimes , y) \mid (x,y) \in R\}$ that still is polynomially balanced, i.e. by increasing the size of $x$ can not invalidate the condition for being polynomially balanced. Moreover, this construction can clearly be done in polynomial time. 
%Secondly, the actual claim can be demonstrated by induction on $i$.
%
%\bigskip
%
%Let $i=n+1$. Consider a polynomially balanced (p.b.) relation $R$ such that $\mathcal{L}(R) \in \php{n}$.  By Theorem \ref{thm:main}, it follows that there exists a p.b. relation $R_{\Pi}$ with $\mathcal{L}(R_{\Pi}) \in \phs{n-1}$ such that 
%\begin{equation*}
%\begin{split}
%\mathcal{L}(R)=\{x_R\#y_{\Pi} \mid (x_R,y_{\Pi}) \in R\}=\{ x_R\#y_{\Pi} \mid \forall y_{\Sigma} \; ( x_R\#y_{\Pi},y_{\Sigma}) \in R_{\Pi}\}
%\end{split}
%\end{equation*}
%By \textbf{IH}, it follows that there exists a polynomially balanced, polynomially decidable, $n$-ary relation $R_*^{n}$ such that
% \begin{equation*}
%\begin{split}
%&\{ x_R\#y_{\Pi} \mid \forall y_{\Sigma} \; ( x_R\#y_{\Pi},y_{\Sigma}) \in R_{\Pi}\} \\
%&=\{x_R\#y_{\Pi} \mid \forall y_1 \exists y_2\forall y_3\cdots Q y_{n-1 }\;  (x_R\#y_{\Pi},y_1,\ldots,y_{n-1}) \in R_*^{n}\}
%\end{split}
%\end{equation*}
%where $Q$ is $\exists$ if $n-1$ is even and $\forall$ if $n-1$ is odd. Hence, $Q$ is $\forall$ if $n$ is even and $\exists$ if $n$ is even.
%By equality, one obtains 
% \begin{equation*}
%\begin{split}
%&\mathcal{L}(R)=\{x_R\#y_{\Pi} \mid (x_R,y_{\Pi}) \in R\}= \\
%&=\{x_R\#y_{\Pi} \mid \forall y_1 \exists y_2\forall y_3\cdots Q y_{n-1 }\;  (x_R\#y_{\Pi},y_1,\ldots,y_{n-1}) \in R_*^{n}\}
%\end{split}
%\end{equation*}
%Using $R_*^{n}$ one can construct the relation 
% \begin{equation*}
%\begin{split}
%R_*^{n+1}:=\{ (x_R,y_{\Pi},y_1,\ldots,y_{n-2} ) \mid (x_R\#y_{\Pi},y_1,\ldots,y_{n-2} ) \in R_*^{n}\}
%\end{split}
%\end{equation*}
%Clearly, by construction 
% \begin{equation*}
%\begin{split}
%\mathcal{L}(R) &= \{x_R\#y_{\Pi} \mid \forall y_1 \exists y_2\forall y_3\cdots Q y_{n-1 }\;  (x_R\#y_{\Pi},y_1,\ldots,y_{n-1}) \in  R_*^{n}\} \\
%&= \{x_R\#y_{\Pi} \mid \forall y_1 \exists y_2\forall y_3\cdots Q y_{n-1 }\;  (x_R,y_{\Pi},y_1,\ldots,y_{n-1}) \in R_*^{n+1}\} 
%\end{split}
%\end{equation*}
%Now since $(x,y) \in R \iff x\#y \in \mathcal{L}(R)$ it follows
% \begin{equation*}
%\begin{split}
%(x,y) \in R &\iff (x,y) \in \{(x_R,y_{\Pi} )\mid \forall y_1 \exists y_2\forall y_3\cdots Q y_{n-1 }\;  (x_R,y_{\Pi},y_1,\ldots,y_{n-1}) \in R_*^{n+1}\}  \\
%&\iff \forall y_1 \exists y_2\forall y_3\cdots Q y_{n-1 }\;  (x,y,y_1,\ldots,y_{n-1})   \in R_*^{n+1}
%\end{split}
%\end{equation*}
%From there it follows that 
%\begin{equation*}
%\begin{split}
% \{x_R& \mid \exists y_{\Pi} \;   (x_R,y_{\Pi}) \in R\} = \\
% &= \{x_R \mid \exists y_{\Pi} \;   (x,y) \in \{(x_R,y_{\Pi} )\mid \forall y_1 \exists y_2\forall y_3\cdots Q y_{n-1 }\;  (x_R,y_{\Pi},y_1,\ldots,y_{n-1}) \in R_*^{n+1}\} \}     \\
% &= \{x_R \mid \exists y_{\Pi} \forall y_1 \exists y_2\forall y_3\cdots Q y_{n-1 }\;  (x_R,y_{\Pi},y_1,\ldots,y_{n-1}) \in R_*^{n+1}\}  \\
% &= \{x_R \mid \exists y_{1} \forall y_2 \exists y_3 \cdots Q y_{n }\;  (x_R,y_{1},y_1,\ldots,y_{n-1}) \in R_*^{n+1}\}  
%\end{split}
%\end{equation*}
%Now, by \textbf{IH}, it follows that $Q$ is $\forall$ if $n$ is even and $\exists$ if $n$ is odd. 
%
%
%Lastly, it has to be established that $R_*^{n+1}$ is both polynomially balanced and polynomially decidable.
%It is known that $\mathcal{R}_*^{n}$ is both polynomially balanced and polynomially decidable.
%By constriction, one knows that 
%\begin{equation*}
%\begin{split}
%R_*^{n+1}=\{(x_R,y_{\Pi},y_1,\ldots,y_{n-1})  \mid  (x_R\#y_{\Pi},y_1,\ldots,y_{n-1}) \in R_*^{n}\}
%\end{split}
%\end{equation*}
%Hence, $R_*^{n}$ is clearly polynomially decidable, because to solve $(x,y_1,\ldots,y_{n}) \in R_*^{n}$ it suffices to concatenate $x$ and $y_1$ and query $R_*^{n}$, which by \textbf{IH} can be done in polynomial time. 
%
%
%Moving on towards establishing that $R_*^{n+1}$ is polynomially balanced. Firstly, observe that by assumption $R$ is polynomially balanced. Meaning that there exists a $k_R\in \mathbb{N}$ such that for all $(x_R,y_{\Pi}) \in R$ one knows 
%$|y_{\Pi}|\leq |x_R|^{k_R}$. By \textbf{IH}, one knows that $R_*^{n}$ is p.b., thus it follows that for $i \in \{1,\dots, n-1\}$ one has $|y_i |\leq |x_R\#y_{\Pi}|^{k_{\Pi}}$. Again, it is known that $|x_R\#y_{\Pi}| \leq  |x_R| +1 + |x_R|^{k_{R}}$, thus by
% assumption one knows that $x>1$ hence it follows that there must exists a $k_*$ such that for $i \in \{1,\dots, n-1\}$ one has $|y_i |\leq |x_R|^{k_*}$. Hence, it follows that $|y_1|,\dots ,|y_{n-2}|, |y_{\Pi}|\leq  |x_R|^{k_*}$, thus it follows that $R_*^{n+1}$ is polynomially balanced. Thereby, establishing the claim.
%\end{itemize}
%
%\bigskip
%
%The other case is done in analogue. !!!!!!!!!!!!!!!!!!!!!!
%\end{proof}
%
%
%
%
%
%
%\begin{lemma}
%\label{lem:to}
%For all $i>0$. If $R$ is a polynomially balanced, polynomial-time decidable $(i+1)$-ary relation. Then there exists a polynomially balanced relation 
%\begin{itemize}
%\item $R_{\Pi}$ such that $\mathcal{L}(R_{\Pi}) \in \php{i-1}$, such that
%\begin{equation*}
%\begin{split}
%\{x \mid \exists y_1 \forall y_2\exists y_3\cdots Q y_i \;  (x,y_1,\ldots,y_i) \in R\}
% =\{x \mid \exists y \;  (x,y) \in R_{\Pi}\} 
%\end{split}
%\end{equation*}
%where $Q$ is $\forall$ if $i$ is even and $\exists$ if $i$ is odd.
%\item $R_{\Sigma}$ such that $\mathcal{L}(R_{\Sigma}) \in \phs{i-1}$, such that
%\begin{equation*}
%\begin{split}
%\{x \mid \forall y_1 \exists y_2\forall y_3\cdots Q y_i \;  (x,y_1,\ldots,y_i) \in R\}
% =\{x \mid \exists y  (x,y) \in R_{\Sigma}\} 
%\end{split}
%\end{equation*}
%where $Q$ is $\exists$ if $i$ is even and $\forall$ if $i$ is odd.
%\end{itemize}
%\end{lemma}
%
%\begin{proof}
%Proof by induction over $i$.
%\begin{itemize}
%\item \textbf{IH:} For $i>0$. If $R$ is a polynomially balanced, polynomial-time decidable $(i+1)$-ary relation. Then there exists a polynomially balanced relation 
%\begin{itemize}
%\item $R_{\Pi}$ such that $\mathcal{L}(R_{\Pi}) \in \php{i-1}$, such that
%\begin{equation*}
%\begin{split}
%\{x \mid \exists y_1 \forall y_2\exists y_3\cdots Q y_i \;  (x,y_1,\ldots,y_i) \in R\}
% =\{x \mid \exists y \;  (x,y) \in R_{\Pi}\} 
%\end{split}
%\end{equation*}
%where $Q$ is $\forall$ if $i$ is even and $\exists$ if $i$ is odd.
%\item $R_{\Sigma}$ such that $\mathcal{L}(R_{\Sigma}) \in \phs{i-1}$, such that
%\begin{equation*}
%\begin{split}
%\{x \mid \forall y_1 \exists y_2\forall y_3\cdots Q y_i \;  (x,y_1,\ldots,y_i) \in R\}
% =\{x \mid \exists y  (x,y) \in R_{\Sigma}\} 
%\end{split}
%\end{equation*}
%where $Q$ is $\exists$ if $i$ is even and $\forall$ if $i$ is odd.
%\end{itemize}
%
%\item \textbf{IB:(MISSING)}
%\begin{itemize}
%\item 
%Let $i=1$. Assume that one has $R$ polynomially balanced such that $\mathcal{L}(R) \in \php{0}$. Since $\php{0}=\Ptime$ it follows that $R$ is not only polynomially balanced, but also polynomially decidable. Finally, with $R$ being 2-ary, it follows that $R$ is the desired $R'$ and since $R'=R$ it is clearly the case that
%\begin{equation*}
%\begin{split}
% \{x \mid \exists y \;  (x,y) \in R\} = \{x \mid \exists y_1  \;  (x,y_1) \in R'\}
%\end{split}
%\end{equation*}
%\item  Let $i=1$. Assume that one has $R$ polynomially balanced such that $\mathcal{L}(R) \in \phs{0}$. Since $\phs{0}=\Ptime$ it follows that $R$ is not only polynomially balanced, but also polynomially decidable. Finally, with $R$ being 2-ary, it follows that $R$ is the desired $R'$ and since $R'=R$ it is clearly the case that
%\begin{equation*}
%\begin{split}
% \{x \mid \forall y \;  (x,y) \in R\} = \{x \mid \forall y_1  \;  (x,y_1) \in R'\}
%\end{split}
%\end{equation*}
%\end{itemize}
%
%
%\item \textbf{IS:} 
%Let $i=n+1$. Assume that there exists the polynomially decidable, polynomially balanced, $n+1$-ary relation $R$. 
%To that end let $R^{n}$ be defined as
%\begin{equation*}
%R^n := \{(x\#y , y_1, \dots, y_{n-1}) \mid (x,y,y_1,\dots y_{n-1}) \in R\}
%\end{equation*} 
%Clearly, this relation can be decided in polynomial-time. Moreover, since $x < x\#y$, $R^n$ must be polynomially balanced.
%Hence, by \textbf{IH}, it follows that there exists a polynomially balanced relation 
%$R_{\Pi}$ such that $\mathcal{L}(R_{\Pi}) \in \php{n-1}$, such that
%\begin{equation*}
%\begin{split}
%\{x_R\#y_{\Pi} \mid \exists y_1 \forall y_2\exists y_3\cdots Q y_i \;  (x_R\#y_{\Pi},y_1,\ldots,y_{n-1}) \in R^n\}
% =\{x_R\#y_{\Pi} \mid \exists y_{\Pi} \;  (x_R\#y,y_{\Pi}) \in R_{\Pi}\} 
%\end{split}
%\end{equation*}
%Due to $(x,y) \in R \iff x\#y \in \mathcal{L}(R)$ it follows that
%\begin{equation*}
%\begin{split}
%\{(x_R,y_{\Pi})\mid \exists y_1 \forall y_2\exists y_3\cdots Q y_i \;  (x_R\#y_{\Pi},y_1,\ldots,y_{n-1}) \in R^n\}
% =\{(x_R,y_{\Pi}) \mid \exists y_{\Pi} \;  (x_R\#y,y_{\Pi}) \in R_{\Pi}\} 
%\end{split}
%\end{equation*}
%This implies that 
%\begin{equation*}
%\begin{split}
%&\{x_R \mid \forall y_{\Pi} \exists y_1\forall y_3\cdots Q y_n \;  (x,y_{\Pi},\dots,y_n) \in R^{n+1}\} \\
%& =\{x_R \mid \forall y_{\Pi} \exists y_1\forall y_3\cdots Q y_n \;  (x\#y_{\Pi},\dots,y_n) \in R^{n}\}
%\end{split}
%\end{equation*}
%Moreover, one can construct the relation $R_2^{n}$ 
%\begin{equation*}
%\begin{split}
%R_2^{n}:=\{x_R\#y_{\Pi}  \mid \forall y_{\Pi} \exists y_1\forall y_3\cdots Q y_n \;  (x_R\#y_{\Pi},\dots,y_n) \in R^{n}\}
%\end{split}
%\end{equation*}
%Hence, one knows that 
%\begin{equation*}
%\begin{split}
%(x,y) \in R_2^{n} \iff  \forall y_{\Pi} \exists y_1\forall y_3\cdots Q y_n \;  (x\#y,\dots,y_n) \in R^{n}
%\end{split}
%\end{equation*}
%Moreover, 
%where $Q$ is $\forall$ if $i$ is even and $\exists$ if $i$ is odd.
%
%
%
%
%
%\end{itemize}
%\end{proof}
%
%%%%%%%%%%%%%%%%%%%%%%%%%%%%%%%%%%%%%%%%%%%%%%%%%%%%%%%%%%%%%%%%
%%%%%%%%%%%%%%%%%%%%%%%%%%%%%%%%%%%%%%%%%%%%%%%%%%%%%%%%%%%%%%%%
%%%%%%%%%%%%%%%%%%%%%%%%%%%%%%%%%%%%%%%%%%%%%%%%%%%%%%%%%%%%%%%%
%%%%%%%%%%%%%%%%%%%%%%%%%%%%%%%%%%%%%%%%%%%%%%%%%%%%%%%%%%%%%%%%
%%%%%%%%%%%%%%%%%%%%%%%%%%%%%%%%%%%%%%%%%%%%%%%%%%%%%%%%%%%%%%%%
%%%%%%%%%%%%%%%%%%%%%%%%%%%%%%%%%%%%%%%%%%%%%%%%%%%%%%%%%%%%%%%%
%\newpage
%
%\begin{lemma}
%\label{lem:sig-to}
%For all $i>0$, if $R$ is a polynomially balanced relation such that the language $\{x\# y \mid (x,y) \in R\}$ is in $\php{i-1}$, then there exists a polynomially balanced, polynomial-time decidable $(i+1)$-ary
%relation $R'$ such that 
%\begin{equation*}
%\begin{split}
% \{x \mid \exists y \;   (x,y) \in R\} = \{x \mid \exists y_1 \forall y_2\exists y_3\cdots Q y_i \;  (x,y_1,\ldots,y_i) \in R'\}
%\end{split}
%\end{equation*}
%where $Q$ is $\forall$ if $i$ is even and $\exists$ if $i$ is odd.
%\end{lemma}
%\begin{proof}
%Firstly, w.l.o.g. assume that $x>1$. As for any polynomially balanced relation $R$, where there exists an $(x,y) \in R$ such that $x\leq 1$ it is possible to construct the relation $R':=\{( x\boxtimes , y) \mid (x,y) \in R\}$ that still is polynomially balanced, i.e. by increasing the size of $x$ can not invalidate the condition for being polynomially balanced. Moreover, this construction can clearly be done in polynomial time. 
%Secondly, the actual claim can be demonstrated by induction on $i$.
%\begin{itemize}
%\item \textbf{IH:} For an $i>0$, if $R$ is a polynomially balanced relation with $\mathcal{L}(R)\in \php{i-1}$, then there exists a polynomially balanced, polynomial-time decidable $(i+1)$-ary
%relation $R'$ such that 
%\begin{equation*}
%\begin{split}
% \{x \mid \exists y \;  (x,y) \in R\} = \{x \mid \exists y_1 \forall y_2\exists y_3\cdots Q y_i \;  (x,y_1,\ldots,y_i) \in R'\}
%\end{split}
%\end{equation*}
%where $Q$ is $\forall$ if $i$ is even and $\exists$ if $i$ is odd.
%
%\item \textbf{IB:} 
%Let $i=1$. Assume that one has $R$ polynomially balanced such that $\mathcal{L}(R) \in \php{0}$. Since $\php{0}=\Ptime$ it follows that $R$ is not only polynomially balanced, but also polynomially decidable. Finally, with $R$ being 2-ary, it follows that $R$ is the desired $R'$ and since $R'=R$ it is clearly the case that
%\begin{equation*}
%\begin{split}
% \{x \mid \exists y \;  (x,y) \in R\} = \{x \mid \exists y_1  \;  (x,y_1) \in R'\}
%\end{split}
%\end{equation*}
%
%
%\item \textbf{IS:} 
%Let $i=n+1$. Consider a polynomially balanced (p.b.) relation $R$ such that $\mathcal{L}(R) \in \php{n}$.  By Theorem \ref{thm:main}, it follows that there exists a p.b. relation $R_{\Pi}$ with $\mathcal{L}(R_{\Pi}) \in \phs{n-1}$ such that 
%\begin{equation*}
%\begin{split}
%\mathcal{L}(R)=\{x_R\#y_{\Pi} \mid (x_R,y_{\Pi}) \in R\}=\{ x_R\#y_{\Pi} \mid \forall y_{\Sigma} \; ( x_R\#y_{\Pi},y_{\Sigma}) \in R_{\Pi}\}
%\end{split}
%\end{equation*}
%By Theorem \ref{thm:main} one knows that there exists a p.b. relation $R_{\Sigma}$ with $\mathcal{L}(R_{\Sigma}) \in \php{n-2}$ such that 
%\begin{equation*}
%\begin{split}
%\mathcal{L}(R_{\Pi})=\{x\#y_{\Pi}\#y_{\Sigma} \mid (x\#y_{\Pi},y_{\Sigma}) \in R_{\Pi}\}=
%\{ x_R\#y_{\Pi}\#y_{\Sigma}  \mid \exists y \; (x_R\#y_{\Pi}\#y_{\Sigma} ,y) \in R_{\Sigma}\}
%\end{split}
%\end{equation*}
%Now, by applying the \textbf{IH} it is guaranteed that there exists a polynomially balanced, polynomially decidable, $n-1$-ary relation $R_*^{n-1}$ such that
% \begin{equation*}
%\begin{split}
%&\{ x_R\#y_{\Pi}\#y_{\Sigma}  \mid \exists y \; (x_R\#y_{\Pi}\#y_{\Sigma} ,y) \in R_{\Sigma}\}  \\
%&=\{x_R\#y_{\Pi}\#y_{\Sigma} \mid \exists y_1 \forall y_2\exists y_3\cdots Q y_{n-2 }\;  (x_R\#y_{\Pi}\#y_{\Sigma},y_1,\ldots,y_{n-2}) \in R_*^{n-1}\}
%\end{split}
%\end{equation*}
%Notice that $Q$ is $\forall$ if $n-2$ is even, and $\exists$ is it is odd.
%By simple equality one obtains 
%\begin{equation*}
%\begin{split}
%\mathcal{L}(R_{\Pi})&=\{x_R\#y_{\Pi}\#y_{\Sigma} \mid (x_R\#y_{\Pi},y_{\Sigma}) \in R_{\Pi}\}\\
%%\{ \mid \exists y_1 \forall y_2\exists y_3\cdots Q y_{n-2 }\;  (x,y_1,\ldots,y_{n-2}) \in R_*^{n-1}\}\\
%&=\{x_R\#y_{\Pi}\#y_{\Sigma} \mid \exists y_1 \forall y_2\exists y_3\cdots Q y_{n-2 }\;  (x_R\#y_{\Pi}\#y_{\Sigma},y_1,\ldots,y_{n-2}) \in R_*^{n-1}\}\\
%\end{split}
%\end{equation*}
%From there let $R_*^{n}$ be defined as 
%\begin{equation*}
%\begin{split}
%R_*^{n}&:=\{(x\#y_{\Pi},y_{\Sigma},y_1,\ldots,y_{n-2})  \mid  (x\#y_{\Pi}\#y_{\Sigma},y_1,\ldots,y_{n-2}) \in R_*^{n-1}\}
%\end{split}
%\end{equation*}
%Since $(x,y) \in R_{\Pi} \iff x\#y  \in \mathcal{L}(R_{\Pi}) $, it follows that  
%\small
%\begin{equation*}
%\begin{split}
%(x,y) \in R_{\Pi}& \iff  x\#y \in \{x_R\#y_{\Pi}\#y_{\Sigma}\mid \exists y_1 \forall y_2\exists y_3\cdots Q y_{n-2 }\;  (x_R\#y_{\Pi}\#y_{\Sigma},y_1,\ldots,y_{n-2}) \in R_*^{n-1}\}\\
%& \iff  (x,y) \in \{(x_R\#y_{\Pi},y_{\Sigma}) \mid \exists y_1 \forall y_2\exists y_3\cdots Q y_{n-2 }\;  (x_R\#y_{\Pi},y_{\Sigma},y_1,\ldots,y_{n-2}) \in R_*^{n}\}
%\end{split}
%\end{equation*}
%\normalsize
%%Using this one can define $R_{\Pi}':=\{ (x,y) \mid \exists y_1 \forall y_2\exists y_3\cdots Q y_{n-2 }\;  (x\#y,y_1,\ldots,y_{n-2}) \in R_*^{n-1}\}$ such that 
%%\begin{equation*}
%%\begin{split}
%%(x,y) \in R_{\Pi} \iff  (x,y) \in R_{\Pi}'
%%\end{split}
%%\end{equation*}
%From this one obtains that 
%\small
%\begin{equation*}
%\begin{split}
%&\mathcal{L}(R)=\\
%&=\{x_R\#y_{\Pi} \mid (x_R,y_{\Pi}) \in R\}=\{ x_R\#y_{\Pi} \mid \forall y_{\Sigma} \; (x_R\#y_{\Pi},y_{\Sigma}) \in R_{\Pi}\} \\
%&=\{ x_R\#y_{\Pi}  \mid \forall y_{\Sigma} \; (x_R\#y_{\Pi},y_{\Sigma}) \in  \{(x_R\#y_{\Pi},y_{\Sigma}) \mid \exists y_1 \forall y_2\exists y_3\cdots Q y_{n-2 }\;  (x\#y_{\Pi},y_{\Sigma},y_1,\ldots,y_{n-2}) \in R_*^{n}\} \\
%&\stackrel{(1)}{=}\{  x_R\#y_{\Pi} \mid \forall y_{\Sigma} \exists y_1 \forall y_2\exists y_3\cdots Q y_{n-2 }\;  (x_R\#y_{\Pi},y_{\Sigma},y_1,\ldots,y_{n-2}) \in R_*^{n}\} 
%%&= \{ x\#y \mid \forall y_0 \exists y_1 \forall y_2\exists y_3\cdots Q y_{n-2 }\;  (x\#y,y_0,y_1,\ldots,y_{n-2}) \in R_*^{n}\} 
%\end{split}
%\end{equation*}
%\normalsize
%As before, let $R_*^{n+1}$ be the relation 
%\begin{equation*}
%\begin{split}
%R_*^{n+1}&:=\{(x_R,y_{\Pi},y_{\Sigma},y_1,\ldots,y_{n-2})  \mid  (x_R\#y_{\Pi},y_{\Sigma},y_1,\ldots,y_{n-2}) \in R_*^{n}\}
%\end{split}
%\end{equation*}
%Since $(x,y) \in R \iff x\#y  \in \mathcal{L}(R) $, it follows that  
%\begin{equation*}
%\begin{split}
%(x,y) \in R& \iff  x\#y \in \{x_R\#y_{\Pi} \mid\forall y_{\Sigma} \exists y_1 \forall y_2 \cdots Q y_{n-2 }\;  (x_R\#y_{\Pi},y_{\Sigma},\ldots,y_{n-2}) \in R_*^{n}\}\\
%& \iff  (x,y) \in \{(x_R,y_{\Pi}) \mid \forall y_0 \exists y_1\forall y_2\cdots Q y_{n-2 }\;  (x_R,y_{\Pi},y_{\Sigma},\ldots,y_{n-2}) \in R_*^{n+1}\}
%\end{split}
%\end{equation*}
%Therefore, one obtains 
%\begin{equation*}
%\begin{split}
% &\{x_R \mid \exists y_{\Pi} \;  (x_R,y_{\Pi}) \in R\} =\\
% &= \{x_R \mid \exists y_{\Pi} \; (x_R,y_{\Pi}) \in \{(x_R,y_{\Pi}) \mid \forall y_{\Sigma} \exists y_1\forall y_2\cdots Q y_{n-2 }\;  ((x_R,y_{\Pi}, y_{\Sigma},y_1,\ldots,y_{n-2}) \in R_*^{n+1}\}\} \\
% &\stackrel{(1)}{=} \{x_R \mid \exists y_{\Pi}  \forall y_{\Sigma} \exists y_1\forall y_2\cdots Q y_{n-2 }\;  (x_R,y_{\Pi}, y_{\Sigma},y_1,\ldots,y_{n-2}) \in R_*^{n+1}\}\\
% &=  \{x \mid \exists y_1  \forall y_2 \cdots Q y_{n }\;  (x,y_1,\ldots,y_{n}) \in R_*^{n+1}\}
%\end{split}
%\end{equation*}
%Observe, that if $n-2$ is even if and only if $n$ is even. Hence, the $Q$ is the appropriate quantifier.
%Lastly, it has to be established that $R_*^{n+1}$ is both polynomially balanced and polynomially decidable.
%It is known that $\mathcal{R}_*^{n-1}$ is both polynomially balanced and polynomially decidable.
%By constriction, one knows that 
%\begin{equation*}
%\begin{split}
%R_*^{n}=\{(x,y,y_1,\ldots,y_{n-2})  \mid  (x\#y,y_1,\ldots,y_{n-2}) \in R_*^{n-1}\}
%\end{split}
%\end{equation*}
%Hence, $R_*^{n}$ is clearly polynomially decidable, because to solve $(x,y_1,\ldots,y_{n-1}) \in R_*^{n}$ it suffices to concatenate $x$ and $y_1$ and query $R_*^{n-1}$, which by \textbf{IH} can be done in polynomial time. 
%Analogously, by construction
%\begin{equation*}
%\begin{split}
%R_*^{n+1}&:=\{(x_R,y_{\Pi},y_{\Sigma},y_1,\ldots,y_{n-2})  \mid  (x_R\#y_{\Pi},y_{\Sigma},y_1,\ldots,y_{n-2}) \in R_*^{n}\}
%\end{split}
%\end{equation*}
%Hence, by \textbf{IH}  and the fact that concatenation can be done in polynomial time, it follows that $R_{*}^{n+1}$ can be decided in polynomial time. 
%
%Moving on towards establishing that $R_*^{n+1}$ is polynomially balanced. Firstly, observe that by assumption $R$ is polynomially balanced. Meaning that there exists a $k_R\in \mathbb{N}$ such that for all $(x_R,y_{\Pi}) \in R$ one knows 
%$|y_{\Pi}|\leq |x_R|^{k_R}$. By Theorem \ref{cor:main-sig}, it is known that for any $ ( x_R\#y_{\Pi},y_{\Sigma}) \in R_{\Pi}$ one has $|y_{\Sigma}|\leq |x_R\#y_{\Pi}|^{k_{\Pi}}$. However, since $x_R\#y_{\Pi} \in \mathcal{L}(R)$ and since $|y_{\Pi}|\leq |x_R|^{k_R}$ follows that $|x_R\#y_{\Pi}| \leq |x_R| +|x_R|^{k_R} + 1$. Now by assumption, i.e. $x>1$, there must be a $k_R'$
%such that $|x_R\#y_{\Pi}| \leq | x_R|^{k_R'}$. Hence, it follows that $|y_{\Sigma}|\leq (|x_R|^{k_r'})^{k_{\Pi}'}= |x_R|^{k_r' +k_{\Pi} }=|x_R|^{k_{\Pi} '}$. Now by, \textbf{IH}, one knows that $R_*^{n-1}$ is p.b., thus it follows that for $i \in \{1,\dots, n-2\}$ one has $|y_i |\leq |x_R\#y_{\Pi}\#y_{\Pi}|^{k_{\Pi}}$. Again it is known that $|x_R\#y_{\Pi}\#y_{\Pi}| \leq |x_R|^{k_{R}} + |x_R|^{k_{\Pi} '} \leq |x_R|^{2*k_{\Pi} '}= |x_R|^{k_*}$. Hence, it follows that $|y_1|,\dots ,|y_{n-2}|,|y_{\Sigma}| , |y_{\Pi}|\leq  |x_R|^{k_*}$, thus it follows that $R_*^{n+1}$ is polynomially balanced. Thereby, establishing the claim.
%
%
%
%%Moreover, with respect to polynomially balanced, as  $R_*^{n-1}$ is polynomially balanced it follows for $(x\#y,y_1,\ldots,y_{n-2})$ and an $1 \leq i \leq n-2$ one has $|y_i| \leq (|x|+|y|)^k$ for some $k>0$.
%
%%
%%Let $i=n+1$. Assume that one has the polynomially balanced relation $R$ such that $\mathcal{L}(R) \in \php{n}$. By Theorem \ref{thm:main} this implies that there exists a polynomially balanced relation $R_{L}^{n-1}$ such that $\mathcal{L}(R_{L}^{n-1}) \in \phs{n-1}$ and that
%%\begin{equation*}
%%\begin{split}
%%\mathcal{L}(R)=\{x\# y \mid (x,y) \in R\}=\{x \mid \forall y \;  (x,y) \in R_{L}^{n-1}\}
%%\end{split}
%%\end{equation*}
%%Now by \textbf{IH}, it follows that there exists an $n$-ary, polynomially balanced and polynomially decidable relation $R_{\Pi}$ such that 
%%\begin{equation*}
%%\begin{split}
%%\mathcal{L}(R_{\Pi})&=  \{x\# y \mid (x,y) \in R_{\Pi}\}\\
%%&=\{x \mid \exists y_1 \forall y_2\exists y_3\cdots Q y_{n-1} \;  (x,y_1,\ldots,y_{n-1}) \in R_{\Pi}^{n}\}\\
%%&=\{x\#y \mid \exists y_1 \forall y_2\exists y_3\cdots Q y_{n-1} \;  (x\#y,y_1,\ldots,y_{n-1}) \in R_{\Pi}^{n}\}
%%\end{split}
%%\end{equation*}
%%Furthermore, let $P_{\Pi}^{n+1}$ be defined such that
%%\begin{equation*}
%%\begin{split}
%%(x,y_1, \dots y_n) \in R_{\Pi}^{n+1} \iff  (x\#y_1,\ldots,y_{n}) \in R_{\Pi}^{n}
%%\end{split}
%%\end{equation*}
%%Thus on obtains
%%\begin{equation*}
%%\begin{split}
%%(x,y) \in R_{\Pi} &\iff (x,y) \in \{ (x,y) \mid \exists y_1 \forall y_2\exists y_3\cdots Q y_{n-1} \;  (x\#y,y_1,\ldots,y_{n-1}) \in R_{\Pi}^{n}\} \\
%%&\iff  (x,y) \in \{ (x,y) \mid \exists y_1 \forall y_2\exists y_3\cdots Q y_{n-1} \;  (x,y,y_1,\ldots,y_{n-1}) \in R_{\Pi}^{n+1}\}
%%\end{split}
%%\end{equation*}
%%From this it follows that 
%%\begin{equation*}
%%\begin{split}
%%\mathcal{L}(R)&=\{x \#y \mid \forall y_0 \;  (x\#y ,y_0) \in R_{\Pi}\} \\
%%&=\{x \#y \mid \forall y_0 \;  (x\#y,y_0) \in \{ (x',y') \mid \exists y_1 \forall y_2\exists y_3\cdots Q y_{n-1} \;  (x',y',y_1,\ldots,y_{n-1}) \in R_{\Pi}^{n+1}\}\}  \\
%%&\stackrel{(!)}{=}\{x \#y \mid \forall y_0 \exists y_1 \forall y_2\exists y_3\cdots Q y_{n-1} \;  (x\#y,y_0,y_1,\ldots,y_{n-1}) \in R_{\Pi}^{n+1}\}  \\
%%\end{split}
%%\end{equation*}
%%\begin{equation*}
%%\begin{split}
%%\mathcal{L}(R)&=\{x \mid \forall y \;  (x,y) \in \{ (x,y) \mid \exists y_1 \forall y_2\exists y_3\cdots Q y_{n-1} \;  (x\#y,y_1,\ldots,y_{n-1}) \in R_{\Pi}^{n}\}\} \\
%%&= \{x \mid \forall y \exists y_1 \forall y_2\exists y_3\cdots Q y_{n-1} \;  (x\#y,y_1,\ldots,y_{n-1}) \in R_{\Pi}^{n}\}  \\
%%&= \{x\#y' \mid \forall y \exists y_1 \forall y_2\exists y_3\cdots Q y_{n-1} \;  (x\#y' \#y,y_1,\ldots,y_{n-1}) \in R_{\Pi}^{n}\} 
%%\end{split}
%%\end{equation*}
%%As before one can construct the following relation.
%%
%%Hence, 
%%\begin{equation*}
%%\begin{split}
%%(x,y) \in R \iff (x,y) \in \{ (x,y) \mid \exists y_1 \forall y_2\exists y_3\cdots Q y_{n-1} \;  (x\#y,y_1,\ldots,y_{n-1}) \in R_{\Pi}^{n}\}
%%\end{split}
%%\end{equation*}
%\end{itemize}
%\end{proof}
%
%
%\begin{lemma}
%\label{lem:sig-from}
%Let $R$ be a polynomially balanced, polynomial-time decidable $(i+1)$-ary relation, then there exists a polynomially balanced relation $R'$ with $\{x\# y \mid (x,y) \in R'\}$ being in $\php{i-1}$, such that
%\begin{equation*}
%\begin{split}
%\{x \mid \exists y_1 \forall y_2\exists y_3\cdots Q y_i \;  (x,y_1,\ldots,y_i) \in R\}
% =\{x \mid \exists y \; |y| \leq |x|^k \land  (x,y) \in R'\} 
%\end{split}
%\end{equation*}
%where $Q$ is $\forall$ if $i$ is even and $\exists$ if $i$ is odd.
%\end{lemma}
%\begin{proof}
%\begin{itemize}
%\item \textbf{IH:} For an $i>0$, if $R$ is a polynomially balanced, polynomial-time decidable $(i+1)$-ary relation, then there exists a polynomially balanced relation $R'$ with $\{x\# y \mid (x,y) \in R'\}$ being in $\php{i-1}$, such that
%\begin{equation*}
%\begin{split}
%\{x \mid \exists y_1 \forall y_2\exists y_3\cdots Q y_i \;  (x,y_1,\ldots,y_i) \in R\}
% =\{x \mid \exists y \; |y| \leq |x|^k \land  (x,y) \in R'\} 
%\end{split}
%\end{equation*}
%where $Q$ is $\forall$ if $i$ is even and $\exists$ if $i$ is odd.
%
%
%\item \textbf{IB:} Let $i=1$. Assume that one has $R$ polynomially balanced, polynomially decidable  such that $\mathcal{L}(R) \in \php{0}$. Since $\php{0}=\Ptime$ it follows that $R$ is not only polynomially balanced, but also polynomially decidable. Finally, with $R$ being 2-ary, it follows that $R$ is the desired $R'$ and since $R'=R$ it is clearly the case that
%\begin{equation*}
%\begin{split}
% \{x \mid \exists y \;  (x,y) \in R\} = \{x \mid \exists y_1  \;  (x,y_1) \in R'\}
%\end{split}
%\end{equation*}
%\item \textbf{IS:} 
%
%\end{itemize}
%\end{proof}

\newpage

\noindent
\begin{exercise}[5 credits]
{\em Recall the $\phs{2}$-hardness proof of \MINSAT\ by reduction from the $\QSAT{2}$-problem: 
%
Let an arbitrary instance of $\QSAT{2}$ be given by the 
QBF
%
$$\psi = (\exists x_1, \dots, x_k) 
(\forall y_1, \dots, y_\ell) \phi
$$
%
%
Now let $\{x'_1, \dots, x'_k, z\}$ be fresh propositional variables.
Then we construct an instance of 
\MINSAT\ by the {\em variable $z$} and the {\em formula} 
%
%\smallskip
$$\chi =  
\big(\bigwedge_{i=1}^k (\neg x_i \leftrightarrow x'_i) \big)
\wedge \big( \neg \phi \vee (y_1 \wedge \dots 
\wedge y_\ell \wedge z)\big)
$$
%
Recall from the lecture that we have already proved the following 
implication:  \\
$\psi$ is $\True$ (in every interpretation) $\Ra$ $z$ is $\True$ in a minimal model of $\chi$.


\smallskip
\noindent
Give a rigorous proof also of the opposite direction, i.e.: \\
$z$ is $\True$ in a minimal model of $\chi$ 
$\Ra$ $\psi$ is $\True$ (in every interpretation).
} % em
\end{exercise}


\noindent
{\bf Hint.} Let ${\cal J}$ be 
a minimal model of $\chi$ and let 
$z$ be $\True$ in ${\cal J}$. 
\begin{itemize}
\item First show that then 
${\cal J} (y_j) = \True$ for every $j$.
\item
Second, let ${\cal I}$ be the truth assignment obtained by 
restricting ${\cal J}$ to the variables 
$\{x_1, \dots, x_k \}$. Show that (by the minimality of ${\cal J}$)
${\cal I}$ is indeed a 
partial assigment on $\{x_1, \dots, x_k\}$ s.t.\ for any values assigned to $\{y_1, \dots, y_\ell\}$, the formula $\phi$ is $\True$. 
\end{itemize}

\paragraph*{Solution}

Firstly, a restatement of the reduction, to unify with the notation used in the solution. 

\begin{definition}
Let $\varphi:=(\exists x_1, \dots, x_k) (\forall y_1, \dots, y_\ell) \psi$ be a $\mathbf{QBF}_{2,\exists}$-formula, then let
\begin{equation*}
\chi(\varphi):=\big(\bigwedge_{i=1}^k (\neg x_i \leftrightarrow x'_i) \big)
\wedge \big( \neg \psi \vee (y_1 \wedge \dots 
\wedge y_\ell \wedge z)\big)
\end{equation*}
Moreover, $\mathcal{Y}(\varphi):= \{y_1, \dots, y_\ell\}$, $\mathcal{X}(\varphi):= \{x_1, \dots, x_k\}$ and $\mathcal{X}'(\varphi):=\{x'_1, \dots, x'_k\}$. Lastly, let $\tau: \mathcal{X}(\varphi) \to \mathcal{X}'(\varphi)$ a bijection such that $x_i \mapsto \tau(x_i)=x'_i$. 
\end{definition}

\begin{remark}
The function $\tau$ is thus merely a function that given $x\in \mathcal{X}(\varphi)$ allows one to access the corresponding $x' \in \mathcal{X}'(\varphi)$. Moreover, corresponding in this case merely means they occur in the sub-formula $\neg x_i \leftrightarrow x'_i$ of $\chi(\varphi)$. Lastly, by construction of the sub-formula $\bigwedge_{i=1}^k (\neg x_i \leftrightarrow x'_i) $ one can be sure that $\tau$ is actually bijective.
\end{remark}
Secondly, in this proof the notion of subset minimality is required. To wield the usual notion of subset minimality, it is necessary to conceptualise an interpretation as a set of atoms, where an atom is true under this interpretation if and only if it is part of the set. Here, a marginally different approach shall be chosen.


\begin{definition}
Let $\mathcal{I}$ be an interpretation over the set of atoms $A_{\mathcal{I}}$. Then $\mathfrak{S}(\mathcal{I}):=\{x \mid \forall x\in A_{\mathcal{I}} \; \mathcal{I}(x)=\True\}$. Moreover, an interpretation $\mathcal{J}$ then $\mathcal{I} \subseteq \mathcal{J}$ if and only if $\mathfrak{S}(\mathcal{I}) \subseteq \mathfrak{S}(\mathcal{J})$ and $\mathcal{I} \subset \mathcal{J}$ if and only if $\mathfrak{S}(\mathcal{I}) \subset \mathfrak{S}(\mathcal{J})$
Moreover, $\mathcal{I}$ is a subset minimal if and only if $\nexists \mathcal{I}' \;  \mathcal{I}' \subset \mathcal{I}$.
Similarly, $\mathcal{I}$ is a subset minimal interpretation of a formula $\varphi$ if and only if $\mathcal{I}\models \varphi \land \nexists \mathcal{I}' \;  \mathcal{I}' \subset \mathcal{I} \land \mathcal{I}'\models \varphi$.
\end{definition}

\begin{remark}
Notice that if $\mathcal{I}$ is a subset minimal interpretation of the formula $\varphi$, then $\mathfrak{S}(\mathcal{I}) \subseteq \var(\varphi)$. That is, if there would exists an $x \in \mathfrak{S}(\mathcal{I})$ such that $x\notin \var(\varphi)$, $\mathcal{I}$ would not be subset minimal.
\end{remark}

%\begin{definition}
%Let $\mathcal{I}$ be an interpretation over the set of atoms $A_{\mathcal{I}}$ and $\mathcal{J}$ be an interpretation over the set of atoms $A_{\mathcal{J}}$
%Then  $\mathcal{I} \subseteq \mathcal{J}$ if and only if $A_{\mathcal{I}}\subseteq A_{\mathcal{J}}$ and $\forall a \in A_{\mathcal{I}}\; \mathcal{I}(a) \sto \mathcal{J}(a)$.
%Furthermore, $\mathcal{I} \subset \mathcal{J}$ is as usual simply $\mathcal{I} \subseteq \mathcal{J}$ and  $\mathcal{I} \neq \mathcal{J}$.
%Moreover, $\mathcal{I}$ is subset minimal if and only if $\nexists \mathcal{I}' \;  \mathcal{I}' \subset \mathcal{I}$.
%\end{definition}


Thirdly, the notion of extension is required.

\begin{definition}
Let $\mathcal{I}$ be an interpretation. Then an extension of $\mathcal{I}$ by the atoms $X$, is any interpretation $\mathcal{J}$ such that $\mathcal{I} \subseteq \mathcal{J}$ and $\forall x \in X \; \mathcal{J}(x)=\True \lor \mathcal{J}(x)=\False$.
\end{definition}
%The subsequent solution does not adhere to the route outlined by the hints of the exercise statement. 
%\begin{lemma}
%Let $\varphi:=(\exists x_1, \dots, x_k) (\forall y_1, \dots, y_\ell) \psi$ be a $\mathbf{QBF}_{2,\exists}$-formula, such that there exists a minimal model $\mathcal{J}$, where $\mathcal{J} \models z$, (i.e. $ \mathcal{J}(z)=\True$) and $\mathcal{J} \models \chi(\varphi)$. Then $\forall y \in \mathcal{Y}(\varphi) \; \mathcal{J} \models y$.
%\end{lemma}
%


As suggested in the given hint, to demonstrate Lemma XX, two precursory results are demonstrated.

\begin{lemma}
\label{lem:all-y}
Let $\varphi:=(\exists x_1, \dots, x_k) (\forall y_1, \dots, y_\ell) \psi$ be a $\mathbf{QBF}_{2,\exists}$-formula, such that there exists a minimal model $\mathcal{J}$ of $\chi(\varphi)$, where $\mathcal{J} \models z$. Then $\forall y \in \mathcal{Y}(\varphi) \; \mathcal{J} \models y$.
\end{lemma}
\begin{proof}
Assume that $\mathcal{J}$ exists, thus it is known that $\mathcal{J} \models z$ and that $\mathcal{J}$ is a subset minimal interpretation of $\chi(\varphi)$. Assume that there exists a $y \in \mathcal{Y}(\varphi)$ such that $\mathcal{J} \nmodels y$. If this is the case, then clearly $\mathcal{J} \nmodels  (y_1 \wedge \dots \wedge y_\ell \wedge z)$. However, since $\mathcal{J} \models \chi(\varphi)$, it must be that $\mathcal{J}\models \neg \psi$. By construction it is known that $z \notin \var(\varphi)$. Therefore, the only occurrence of $z$ in $\chi(\varphi)$ is in the sub-formula $ (y_1 \wedge \dots \wedge y_\ell \wedge z)$. Now, with the one $y$ evaluating to $\False$ under $\mathcal{J}$, the truth value of $z$ is immaterial in the evaluation of $\chi(\varphi)$. Hence, one can construct the interpretation $\mathcal{J}'$ such that $\forall x \neq z\; \mathcal{J}'(x):=\mathcal{J}(x)$ and $\mathcal{J}'(z):=\False$ that satisfies $\chi(\varphi)$. Hence, by definition one obtains $\mathcal{J}' \subset \mathcal{J} $, which clearly violates the assumed subset minimality of $\mathcal{J}$. Therefore, one can conclude that $\forall y \in \mathcal{Y}(\varphi) \; \mathcal{J} \models y$.
\end{proof}

Guided by the hint, the second relevant lemma.

\begin{lemma}
\label{lem:sat-phi}
Let $\varphi:=(\exists x_1, \dots, x_k) (\forall y_1, \dots, y_\ell) \psi$ be a $\mathbf{QBF}_{2,\exists}$-formula, such that there exists a minimal model $\mathcal{J}$ of $\chi(\varphi)$, where $\mathcal{J} \models z$. Let $\mathcal{I}_{|_X}$ be the interpretation $\forall x \in \mathcal{X}(\varphi)\; \mathcal{I}_{|_X}(x)=\mathcal{J}(x) \land \mathfrak{S}(\mathcal{I}_{|_X}) \subseteq \mathcal{X}(\varphi)$, i.e. it is $\mathcal{J}$ restricted to the variables in $\mathcal{X}(\varphi)$. Then it holds that for any arbitrary extension $\mathcal{I}_{|_{X\cup Y}}$ of $\mathcal{I}_{|_X}$ by the variables in $\mathcal{Y}(\varphi)$, it must be that $\mathcal{I}_{|_{X\cup Y}}\models \psi$. 
\end{lemma}
\begin{proof}
Towards a contradiction, assume that there exists an extension $\mathcal{I}_{|_{X\cup Y}}$ of $\mathcal{I}_{|_{X}}$ by the variables in $\mathcal{Y}(\varphi)$ such that $\mathcal{I}_{|_{X\cup Y}} \nmodels \psi$. Hence, by semantics this implies that $\mathcal{I}_{|_{X\cup Y}} \models  \neg\psi$. Now using $\mathcal{I}_{|_{X\cup Y}}$ an interpretation $\mathcal{J}'$ will be constructed such that $\mathcal{J}' \subset \mathcal{J}$ and $\mathcal{J}' \models \chi(\varphi)$. The sought after interpretation is defined such that
\begin{itemize}
\item $\forall x \in \mathcal{X}(\varphi) \; \mathcal{J}'(x):=\mathcal{I}_{|_{X\cup Y}}(x)$;
\item $\forall x \in \mathcal{X}(\varphi) \; \mathcal{J}'(\tau(x)):=\neg \mathcal{I}_{|_{X\cup Y}}(x)$;
\item $\forall y \in \mathcal{Y}(\varphi) \; \mathcal{J}'(y):=\mathcal{I}_{|_{X\cup Y}}(y)$;
\item $ \mathcal{J}'(z):=\False$.
\end{itemize}
Notice that $\mathcal{J}'$ was constructed using $\mathcal{I}_{|_{X\cup Y}}$, which is an extension of $\mathcal{I}_{|_{X}}$, which itself is merely a restriction of $\mathcal{J}$ to the variables in $\mathcal{X}(\varphi)$. Hence,
it follows that $\forall x \in \mathcal{X}(\varphi) \; \mathcal{J}'(x)=\mathcal{J}(x)$. 
Moreover, together with the fact that $\mathcal{J} \models \bigwedge_{i=1}^k (\neg x_i \leftrightarrow x'_i)$ (and the construction of $\tau$) it follows that $\forall x' \in \mathcal{X}'(\varphi) \; \mathcal{J}'(x')=\mathcal{J}(x')$. Therefore, $\mathcal{J}$ and $\mathcal{J}'$ agree on the variables in $\mathcal{X}(\varphi)\cup \mathcal{X}'(\varphi)$, i.e. $\mathfrak{S}(\mathcal{J}') \cap (\mathcal{X}(\varphi)\cup \mathcal{X}'(\varphi))=\mathfrak{S}(\mathcal{J}) \cap (\mathcal{X}(\varphi)\cup \mathcal{X}'(\varphi))$. 
Now, by Lemma \ref{lem:all-y}, it is known that for any $y \in \mathcal{Y}(\varphi)$ it must be that $\mathcal{J}\models y$, i.e. $\mathfrak{S}(\mathcal{J}) \cap  \mathcal{Y}(\varphi)= \mathcal{Y}(\varphi)$. Hence, $\mathfrak{S}(\mathcal{J}') \cap \mathcal{Y}(\varphi) \subseteq \mathfrak{S}(\mathcal{J}) \cap \mathcal{Y}(\varphi)$. Furthermore, by assumption it is known that $\mathcal{J} \models z$ while $\mathcal{J}'$ does not, thus $\mathfrak{S}(\mathcal{J}') \cap \{z\} \subset \mathfrak{S}(\mathcal{J}) \cap\{z\}$ holds. To summarise, 
\begin{enumerate}
\item[(i)] $\mathfrak{S}(\mathcal{J}') \cap (\mathcal{X}(\varphi)\cup \mathcal{X}'(\varphi)=\mathfrak{S}(\mathcal{J}) \cap (\mathcal{X}(\varphi)\cup \mathcal{X}'(\varphi)$,
\item[(ii)] $\mathfrak{S}(\mathcal{J}') \cap \mathcal{Y}(\varphi) \subseteq \mathfrak{S}(\mathcal{J}) \cap \mathcal{Y}(\varphi)$ and 
\item[(iii)] $\mathfrak{S}(\mathcal{J}') \cap \{z\} \subset \mathfrak{S}(\mathcal{J}) \cap\{z\}$.
\end{enumerate}

As this covers all variables assigned in $\mathcal{J}$ by subset minimality and all variables assigned in $\mathcal{J}'$ by construction, one can conclude that $\mathfrak{S}(\mathcal{J}')  \subset \mathfrak{S}(\mathcal{J})$ which by definition implies that $\mathcal{J}'\subset\mathcal{J}$.


What remains to be shown is that $\mathcal{J}' \models \chi(\varphi)$. From (i) and the fact that $\mathcal{J} \models \chi(\varphi)$ one obtains $\mathcal{J}' \models \bigwedge_{i=1}^k (\neg x_i \leftrightarrow x'_i)$. Furthermore, by assumption it is known that $\mathcal{I}_{|_{X\cup Y}} \models \neg \varphi$, by construction one knows that $\forall x \in \mathcal{X}(\varphi)\cup \mathcal{Y}(\varphi)\; \mathcal{J}'(x)=\mathcal{I}_{|_{X\cup Y}}(x) $, as well as  $\mathcal{X}(\varphi)\cup \mathcal{Y}(\varphi)= \var(\varphi)=\var(\psi)$. Hence, one can conclude that $\mathcal{J}' \models \neg \psi$.
Which thereby, forces that $\mathcal{J}' \models \chi(\varphi)$, thus clearly contradicting the subset minimality of $\mathcal{J}$. 

%Assume that $\mathcal{J}$ exists, thus it is known that $\mathcal{J} \models z$ and $\mathcal{J} \models \chi(\varphi)$ and that $\mathcal{J}$ is subset minimal. Assume that there exists a $y \in \mathcal{Y}(\varphi)$ such that $\mathcal{J} \nmodels y$. If this is the case then clearly $\mathcal{J} \nmodels  (y_1 \wedge \dots \wedge y_\ell \wedge z)$. However, since $\mathcal{J} \models \chi(\varphi)$, it must be that $\mathcal{J}\models \neg \psi$. However, by construction it is known that $z \notin \var(\varphi)$. Therefore, the only occurrence of $z$ in $\chi(\varphi)$ is in $ (y_1 \wedge \dots \wedge y_\ell \wedge z)$. Now, with the one $y$ evaluating to $\False$ under $\mathcal{J}$, the truth value of $z$ is immaterial. Hence, one can construct the interpretation $\mathcal{J}'$ such that $\forall x \neq z\; \mathcal{J}'(x)=\mathcal{J}(x)$ and $\mathcal{J}'(z)=\False$ that satisfies $\chi(\varphi)$. Hence, by definition one obtains $\mathcal{J}' \subset \mathcal{J} $, which clearly violates the assumed subset minimality of $\mathcal{J}$. Therefore, one can conclude that $\forall y \in \mathcal{Y}(\varphi) \; \mathcal{J} \models y$.
\end{proof}


Finally, allowing the proof of the main result.

\begin{lemma}
Let $\varphi:=(\exists x_1, \dots, x_k) (\forall y_1, \dots, y_\ell) \psi$ be a $\mathbf{QBF}_{2,\exists}$-formula.
$z$ is $\True$ in a minimal model of $\chi(\varphi) \implies \varphi$ is $\True$.
\end{lemma}
\begin{proof}
If the antecedent is not satisfied, the statement holds vacuously. 
Hence, to demonstrate this claim, under the assumption that $z$ is $\True$ in a minimal model of $\chi(\varphi)$, the formula $\varphi$ is $\True$. This is precisely the case if there exists a partial assignment $\mathcal{I}$ of the variables $\mathcal{X}(\varphi)$ such that any extension $\mathcal{I}'$ by the  variables $\mathcal{Y}(\varphi)$ satisfies $\psi$, i.e. $\mathcal{I}' \models \psi$.  However, this is precisely what Lemma \ref{lem:sat-phi} provides. That is, using this lemma it is possible to construct such a partial truth assignment for the variables in $\mathcal{X}(\varphi)$. 
\end{proof}


\newpage

\paragraph*{Appendix}

Let $i=n+1$. Observe the following
\begin{equation*}
\begin{split}
&L \in \php{n+1}\\
\stackrel{\text{(i)}}{\iff}& \exists R_2 \in \mathcal{R}^{2} \; R_2 \; p.b.  \land \mathcal{L}(R_2) \in \phs{n} \land L = \{x \mid \forall y \; (x,y) \in R_2\} \\
\stackrel{\text{(ii)}}{\iff} &\exists R_2 \in \mathcal{R}^{2} \; R_2 \; p.b.  \land  \exists R_{n+1}  \in  \mathcal{R}^{n+1} \; R_{n+1} \; p.b. \land R_{n+1} \; p.d. \land  \\
&\mathcal{L}(R_2) = \{x\#y \mid \exists y_1 \forall y_2\exists y_3\dots Q y_n  \; (x\#y,y_1,\ldots,y_n) \in R_{n+1}\}  \land \\
& L = \{x \mid \forall y \; (x,y) \in R_2\}  \land \\
&  (n \text{ even}  \sto Q=\forall )\land (n \text{ odd}  \sto Q=\exists )\\
\stackrel{\text{(iii)}}{\iff} &\exists R_2 \in \mathcal{R}^{2} \; R_2 \; p.b.  \land  \exists R_{n+2}  \in  \mathcal{R}^{n+2} \; R_{n+2} \; p.b. \land R_{n+2} \; p.d. \land  \\
&\mathcal{L}(R_2) = \{x\#y \mid \exists y_1 \forall y_2\exists y_3\dots Q y_n  \; (x,y,y_1,\ldots,y_n) \in R_{n+2}\}  \land \\ 
& L = \{x \mid \forall y \; (x,y) \in R_2\} \land  \\ 
&  (n \text{ even}  \sto Q=\forall )\land (n \text{ odd}  \sto Q=\exists )\\
\stackrel{\text{(iv)}}{\iff} &\exists R_2 \in \mathcal{R}^{2} \; R_2 \; p.b.  \land  \exists R_{n+2}  \in  \mathcal{R}^{n+2} \; R_{n+2} \; p.b. \land R_{n+2} \; p.d. \land  \\
&R_2 = \{(x,y) \mid \exists y_1 \forall y_2\exists y_3\dots Q y_n  \; (x,y,y_1,\ldots,y_n) \in R_{n+2}\}  \land\\ 
& L = \{x \mid \forall y \; (x,y) \in  \{(x,y) \mid \exists y_1 \forall y_2\exists y_3\dots Q y_n  \; (x,y,y_1,\ldots,y_n) \in R_{n+2}\}\}  \land  \\ 
&  (n \text{ even}  \sto Q=\forall )\land (n \text{ odd}  \sto Q=\exists )\\
\stackrel{\text{(v)}}{\iff} & \exists R_{n+2}  \in  \mathcal{R}^{n+2} \; R_{n+2} \; p.b. \land R_{n+2} \; p.d. \land  \\
& L = \{x \mid \forall y \; (x,y) \in  \{(x,y) \mid \exists y_1 \forall y_2\exists y_3\dots Q y_n  \; (x,y,y_1,\ldots,y_n) \in R_{n+2}\}\} \land  \\ 
&  (n \text{ even}  \sto Q=\forall )\land (n \text{ odd}  \sto Q=\exists )\\
\stackrel{\text{(vi)}}{\iff} & \exists R_{n+2}  \in  \mathcal{R}^{n+2} \; R_{n+2} \; p.b. \land R_{n+2} \; p.d. \land  \\
& L = \{x \mid \forall y \exists y_1 \forall y_2\exists y_3\dots Q y_n  \; (x,y,y_1,\ldots,y_n) \in R_{n+2}\}\land  \\ 
&  (n \text{ even}  \sto Q=\forall )\land (n \text{ odd}  \sto Q=\exists )\\
\stackrel{\text{(vii)}}{\iff} & \exists R_{n+2}  \in  \mathcal{R}^{n+2} \; R_{n+2} \; p.b. \land R_{n+2} \; p.d. \land  \\
& L = \{x \mid \forall y_1 \exists y_2 \forall y_3 \dots Q y_{n+1}  \; (x,y,y_1,\ldots,y_n) \in R_{n+2}\}  \land \\ 
&  (n+1 \text{ even}  \sto Q=\exists) \land (n+1 \text{ odd}  \sto Q=\forall) \\
\end{split}
\end{equation*}

\begin{enumerate}
\item[(i)] Here Theorem \ref{thm:main} was applied.
\item[(ii)] Here the \textbf{IH} was applied, i.e. 
\begin{equation*}
\begin{split}
\mathcal{L}(R_2) \in \phs{n} \iff& \exists R_{n+1}  \in  \mathcal{R}^{n+1} \; R_{n+1} \; p.b. \land R_{n+1} \; p.d. \land  \\
&\mathcal{L}(R_2) = \{x\#y \mid \exists y_1 \forall y_2\exists y_3\dots Q y_n  \; (x\#y,y_1,\ldots,y_n) \in R_{n+1}\} \land  \\
 & (n \text{ even}  \sto Q=\forall) \land (n \text{ odd}  \sto Q=\exists )
\end{split}
\end{equation*}


\item[(iii)] Firstly, $\Rightarrow$. Starting from 
 \begin{equation*}
\begin{split}
&\exists R_2 \in \mathcal{R}^{2} \; R_2 \; p.b.  \land  \exists R_{n+1}  \in  \mathcal{R}^{n+1} \; R_{n+1} \; p.b. \land R_{n+1} \; p.d. \land  \\
&\mathcal{L}(R_2) = \{x\#y \mid \exists y_1 \forall y_2\exists y_3\dots Q y_n  \; (x\#y,y_1,\ldots,y_n) \in R_{n+1}\} 
\end{split}
\end{equation*}
Taking the relation $R_{n+1}$ one can construct the relation $R_{n+2} \in \mathcal{R}^{n+2}$ such that 
 \begin{equation*}
\begin{split}
 R_{n+2}=\{ (x,y,y_1,\ldots,y_n) \mid (x\#y,y_1,\ldots,y_n) \in R_{n+1}\} 
\end{split}
\end{equation*}
To do so one merely has to split the first entry in $(x\#y,y_1,\ldots,y_n) \in R_{n+1}$ into two, which can be done in polynomial time (similar argument as in Lemma \ref{lem:lang-rel}). Moreover, by construction it clearly holds that
 \begin{equation*}
\begin{split}
 (x,y,y_1,\ldots,y_n) \in R_{n+2} \iff  (x\#y,y_1,\ldots,y_n) \in R_{n+1}
\end{split}
\end{equation*}
Since by assumption $R_2$ is polynomially balanced it follows that there exists a $k$ such that for any $(x,y) \in R_{2}$ one has $|y| \leq |x|^k$. Furthermore, it is known that $R_{n+1}$ is p.b., thus there exists a $k'$ such that for any $1 \leq i \leq n$ one has $|y_i| \leq |x\#y|^{k'}\leq |x| +1 +|x|^k$. By assumption, i.e. $|x|>1$, it follows that there exists a $k^* \geq k$ such that $|y_i| \leq |x|^{k*}$ and $|y| \leq |x|^{k*}$. Hence, $R_{n+2}$ is polynomially balanced. 
Additionally, one knows that $R_{n+1}$ is p.d., thus $R_{n+2}$ can be decided by concatenating the first two entries and querying $R_{n+1}$. Both operations can be done in polynomial time, thus $R_{n+2}$ is p.d..
Hence, one obtains 
 \begin{equation*}
\begin{split}
&\exists R_2 \in \mathcal{R}^{2} \; R_2 \; p.b.  \land  \exists R_{n+2}  \in  \mathcal{R}^{n+2} \; R_{n+2} \; p.b. \land R_{n+2} \; p.d. \land  \\
&\mathcal{L}(R_2) = \{x\#y \mid \exists y_1 \forall y_2\exists y_3\dots Q y_n  \; (x,y,y_1,\ldots,y_n) \in R_{n+2}\} 
\end{split}
\end{equation*}

\medskip

Secondly, $\Leftarrow$.  This argument is essentially the same as the previous one, but in reverse (and with slight alterations in the complexity arguments). That is, starting from
 \begin{equation*}
\begin{split}
&\exists R_2 \in \mathcal{R}^{2} \; R_2 \; p.b.  \land  \exists R_{n+2}  \in  \mathcal{R}^{n+2} \; R_{n+2} \; p.b. \land R_{n+2} \; p.d. \land  \\
&\mathcal{L}(R_2) = \{x\#y \mid \exists y_1 \forall y_2\exists y_3\dots Q y_n  \; (x,y,y_1,\ldots,y_n) \in R_{n+2}\} 
\end{split}
\end{equation*}
Taking the relation $R_{n+2}$ one can construct the relation $R_{n+1} \in \mathcal{R}^{n+1}$ such that 
 \begin{equation*}
\begin{split}
 R_{n+1}=\{ (x\#y,y_1,\ldots,y_n) \mid (x,y,y_1,\ldots,y_n) \in R_{n+2}\} 
\end{split}
\end{equation*}
To do so one merely has to concatenate the first two entries in $(x,y,y_1,\ldots,y_n) \in R_{n+1}$ using the separator $\#$, which can be done in polynomial time (similar argument as in Lemma \ref{lem:lang-rel}). Moreover, it clearly holds that
 \begin{equation*}
\begin{split}
(x\#y,y_1,\ldots,y_n) \in R_{n+1} \iff   (x,y,y_1,\ldots,y_n) \in R_{n+2} 
\end{split}
\end{equation*}
It is known that $R_{n+2}$ is p.b., thus there exists a $k$ such that for $1 \leq i \leq n$, $|y_i| \leq |x|^k$ and $|y| \leq |x|^k$. Now since $|x| <|x\#y|$ it must be that $R_{n+1}$ is p.b. as well.
Additionally, one knows that $R_{n+2}$ is p.d., thus $R_{n+1}$ can be decided by splitting the first entry on $\#$ and querying $R_{n+2}$. Both operations can be done in polynomial time, thus $R_{n+1}$ is p.d..
Hence, one obtains 
 \begin{equation*}
\begin{split}
&\exists R_2 \in \mathcal{R}^{2} \; R_2 \; p.b.  \land  \exists R_{n+1}  \in  \mathcal{R}^{n+1} \; R_{n+1} \; p.b. \land R_{n+1} \; p.d. \land  \\
&\mathcal{L}(R_2) = \{x\#y \mid \exists y_1 \forall y_2\exists y_3\dots Q y_n  \; (x\#y,y_1,\ldots,y_n) \in R_{n+1}\} 
\end{split}
\end{equation*}


\item[(iv)] This equality is guaranteed by the following. Take an arbitrary relation $R$. Clearly, $(x,y) \in R \iff x\#y \in \mathcal{L}(R)$. Hence, in this particular case one has $(x,y) \in R_2 \iff x\#y \in \mathcal{L}(R_2)$. Now starting from  
\begin{equation*}
\begin{split}
\mathcal{L}(R_2) &= \{x\#y \mid \exists y_1 \forall y_2\exists y_3\dots Q y_n  \; (x,y,y_1,\ldots,y_n) \in R_{n+2}\}  \land \\ 
L &= \{x \mid \forall y \; (x,y) \in R_2\}  
\end{split}
\end{equation*}
due to 
 \begin{equation*}
\begin{split}
(\alpha,\beta) \in R_2 &\iff \alpha\#\beta \in\mathcal{L}(R_2) \\
&\iff \alpha\#\beta \in \{ x\#y \mid  \exists y_1 \forall y_2\exists y_3\dots Q y_n  \; (x,y,y_1,\ldots,y_n) \in R_{n+2}\}  \\
&\iff (\alpha,\beta) \in \{(x,y) \mid \exists y_1 \forall y_2\exists y_3\dots Q y_n   \; (x,y,y_1,\ldots,y_n) \in R_{n+2}\} 
\end{split}
\end{equation*}
one obtains the equivalent statement
\begin{equation*}
\begin{split}
R_2 &= \{(x,y) \mid  \exists y_1 \forall y_2\exists y_3\dots Q y_n   \; (x,y,y_1,\ldots,y_n) \in R_{n+2}\}  \land \\ 
L &= \{x \mid \forall y \; (x,y) \in  \{(x,y) \mid  \exists y_1 \forall y_2\exists y_3\dots Q y_n  \; (x,y,y_1,\ldots,y_n) \in R_{n+2}\}\} 
\end{split}
\end{equation*}
%$\exists R \in \mathcal{R}^n \{x\#y \mid  \forall y_1 \exists y_2\forall y_3\dots Q y_n  \; (x\#y,y_1,\ldots,y_n) \in R_{n+1}\}$



\item[(v)] Firstly, $\Rightarrow$. Starting from 

\begin{equation*}
\begin{split}
&\exists R_2 \in \mathcal{R}^{2} \; R_2 \; p.b.  \land  \exists R_{n+2}  \in  \mathcal{R}^{n+2} \; R_{n+2} \; p.b. \land R_{n+2} \; p.d. \land  \\
&R_2 = \{(x,y) \mid \exists y_1 \forall y_2\exists y_3\dots Q y_n  \; (x,y,y_1,\ldots,y_n) \in R_{n+2}\}  \land\\ 
& L = \{x \mid \forall y \; (x,y) \in  \{(x,y) \mid  \exists y_1 \forall y_2\exists y_3\dots Q y_n  \; (x,y,y_1,\ldots,y_n) \in R_{n+2}\}\}  \\ 
\end{split}
\end{equation*}
one can simply use weakening to obtain the part of the statement, where $R_2$ does not occur. 
\begin{equation*}
\begin{split}
& \exists R_{n+2}  \in  \mathcal{R}^{n+2} \; R_{n+2} \; p.b. \land R_{n+2} \; p.d. \land  \\
& L = \{x \mid \forall y \; (x,y) \in  \{(x,y) \mid  \exists y_1 \forall y_2\exists y_3\dots Q y_n   \; (x,y,y_1,\ldots,y_n) \in R_{n+2}\}\}  \\ 
\end{split}
\end{equation*}
Thereby, eradicating all references of $R_2$. 

\medskip
Secondly, $\Leftarrow$. Starting from 
\begin{equation*}
\begin{split}
& \exists R_{n+2}  \in  \mathcal{R}^{n+2} \; R_{n+2} \; p.b. \land R_{n+2} \; p.d. \land  \\
& L = \{x \mid \forall y \; (x,y) \in  \{(x,y) \mid  \exists y_1 \forall y_2\exists y_3\dots Q y_n  \; (x,y,y_1,\ldots,y_n) \in R_{n+2}\}\}  \\ 
\end{split}
\end{equation*}
One can define the relation $R_2:=\{(x,y) \mid  \exists y_1 \forall y_2\exists y_3\dots Q y_n   \; (x,y,y_1,\ldots,y_n) \in R_{n+2}\} $.
Since it is known that $R_{n+2}$ is p.b. this implies that there exists a $k$ such that $|y| \leq |x|^k$, thus implying that $R_2$ is p.b..
Allowing one to conclude that
\begin{equation*}
\begin{split}
&\exists R_2 \in \mathcal{R}^{2} \; R_2 \; p.b.  \land  \exists R_{n+2}  \in  \mathcal{R}^{n+2} \; R_{n+2} \; p.b. \land R_{n+2} \; p.d. \land  \\
&R_2 = \{(x,y) \mid  \exists y_1 \forall y_2\exists y_3\dots Q y_n  \; (x,y,y_1,\ldots,y_n) \in R_{n+2}\}  \land\\ 
& L = \{x \mid \forall y \; (x,y) \in  \{(x,y) \mid  \exists y_1 \forall y_2\exists y_3\dots Q y_n  \; (x,y,y_1,\ldots,y_n) \in R_{n+2}\}\}  \\ 
\end{split}
\end{equation*}

\item[(vi)] Starting from $\{x \mid \forall y \; (x,y) \in  \{(x,y) \mid \exists y_1 \forall y_2\exists y_3\dots Q y_n  \; (x,y,y_1,\ldots,y_n) \in R_{n+2}\}\}$. Notice that 
\begin{equation*}
\begin{split}
 &(\alpha, \beta) \in \{(x,y) \mid  \exists y_1 \forall y_2\exists y_3\dots Q y_n   \; (x,y,y_1,\ldots,y_n) \in R_{n+2}\} \\ 
 &\iff  \exists y_1 \forall y_2\exists y_3\dots Q y_n  \; (\alpha, \beta,y_1,\ldots,y_n) \in R_{n+2}
\end{split}
\end{equation*}
From this it follows that 
\begin{equation*}
\begin{split}
 &\forall y \; (\alpha,y) \in  \{(x,y) \mid  \exists y_1 \forall y_2\exists y_3\dots Q y_n   \; (x,y,y_1,\ldots,y_n) \in R_{n+2}\} \\ 
 &\iff \forall y  \exists y_1 \forall y_2\exists y_3\dots Q y_n  \; (\alpha, y,y_1,\ldots,y_n) \in R_{n+2}
\end{split}
\end{equation*}
and therefore
\begin{equation*}
\begin{split}
 &\{x \mid \forall y \; (x,y) \in  \{(x,y) \mid \exists y_1 \forall y_2\exists y_3\dots Q y_n  \; (x,y,y_1,\ldots,y_n) \in R_{n+2}\}\}\\ 
 &= \{x \mid \forall y  \exists y_1 \forall y_2\exists y_3\dots Q y_n   \; (x,y,y_1,\ldots,y_n) \in R_{n+2}\}
\end{split}
\end{equation*}


\item[(vii)] This particular renaming of bound variables is clearly an equivalence transformation. Hence, 
\begin{equation*}
\begin{split}
n \text{ even}  \sto Q=\forall \land n \text{ odd}  \sto Q=\exists  \iff  n+1 \text{ even}  \sto Q=\exists \land n+1 \text{ odd}  \sto Q=\forall 
\end{split}
\end{equation*}
remains to be established. However, this follows directly from the fact that $n$ is even if and only if $n+1$ is odd. That is, 
if $n$ was even, one has $Q=\forall$. However, this implies that for $Q=\forall$ for $n+1$ being odd and if $n+1$ is odd then $Q=\forall$, meaning that $Q=\forall$ for $n$ is even.
Analogous for the other case.
\end{enumerate}

\end{document}


