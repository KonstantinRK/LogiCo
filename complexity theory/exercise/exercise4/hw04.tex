      \documentclass [11pt]{article}
   \usepackage{latexsym}
   \usepackage{amssymb}
      \usepackage{amsmath}
      \usepackage{amsthm}
   \usepackage{url}

   \textwidth      15cm
   \textheight     23cm
   \oddsidemargin 0.5cm
   \topmargin    -0.5cm
   \evensidemargin\oddsidemargin

 \newcommand{\nop}[1]{}


   \pagestyle{plain}
   \bibliographystyle{plain}



%  \newtheorem{theorem}{Theorem}
%  \newtheorem{lemma}[theorem]{Lemma}
%  \newtheorem{exercise}{Exercise}
  \newtheorem{reduction}{Reduction}

  \newcommand{\ra}{\rightarrow}
  \newcommand{\Ra}{\Rightarrow}
  \newcommand{\La}{\Leftarrow}
  \newcommand{\la}{\leftarrow}
  \newcommand{\LR}{\Leftrightarrow}

  \renewcommand{\phi}{\varphi}
  \renewcommand{\theta}{\vartheta}



\newcommand{\ccfont}[1]{\protect\mathsf{#1}}
\newcommand{\NP}{\ccfont{NP}}
\newcommand{\SAT}{\mbox{\bf SAT}}
\newcommand{\Ptime}{\ccfont{P}}
\newcommand{\phs}[1]{\Sigma_{#1}\Ptime}
\newcommand{\php}[1]{\Pi_{#1}\Ptime}
\newcommand{\pht}[1]{\Theta_{#1}\Ptime}
\newcommand{\QSAT}[1]{\ccfont{QSAT}_{#1}}
\newcommand{\MINSAT}{\mbox{\bf MINIMAL MODEL SAT}}
\newcommand{\True}{\mathbf{true}}
\newcommand{\False}{\mathbf{false}}


\newcommand{\MINCARDSAT}{\mbox{\bf CARD-MINIMAL MODEL SAT}}
\newcommand{\MINWEIGHTSAT}{\mbox{\bf WEIGHT-MINIMAL MODEL SAT}}
\newcommand{\MAXCARDSAT}{\mbox{\bf CARD-MAXIMAL MODEL SAT}}
\newcommand{\MAXWEIGHTSAT}{\mbox{\bf WEIGHT-MAXIMAL MODEL SAT}}
\newcommand{\MINLEXSAT}{\mbox{\bf LEX-MINIMAL MODEL SAT}}
\newcommand{\MAXLEXSAT}{\mbox{\bf LEX-MAXIMAL MODEL SAT}}
\newcommand{\LogMAXLEXSAT}{\mbox{\bf LogLEX-MAXIMAL MODEL SAT}}
\newcommand{\LogMINLEXSAT}{\mbox{\bf LogLEX-MINIMAL MODEL SAT}}



\renewcommand{\labelenumi}{(\alph{enumi})}

%\newcommand{\True}{\mathbf{true}}
%\newcommand{\False}{\mathbf{false}}
%\newcommand{\QSAT}[1]{\mathit{QBF}_{\exists, 2}}

\renewcommand{\labelenumi}{(\alph{enumi})}



  \newcommand{\lto}{\rightarrow}
  \newcommand{\sto}{\Rightarrow}
  \newcommand{\snto}{\Leftarrow}
  \newcommand{\lnto}{\leftarrow}
  \newcommand{\siff}{\Leftrightarrow}
   \newcommand{\liff}{\leftrightarrow}
   \newcommand{\nmodels}{\not\models}

\newcommand{\clau}{\mathit{Clause}}
\newcommand{\var}{\mathit{Var}}
\newcommand{\lit}{\mathit{Lit}}
\newcommand{\dom}{\mathit{Dom}}

  \newtheorem{theorem}{Theorem}
  \newtheorem{lemma}[theorem]{Lemma}
  \newtheorem{corollary}[theorem]{Corollary}
    \newtheorem{observation}[theorem]{Observation}
  \newtheorem{proposition}[theorem]{Proposition}
  \newtheorem{conjecture}[theorem]{Conjecture}
  \newtheorem{definition}[theorem]{Definition}
  \newtheorem{example}[theorem]{Example}
  \newtheorem{remark}[theorem]{Remark}
  \newtheorem{exercise}[theorem]{Exercise}

%%%%%%%%%%%%%%%%%%%%%%%%%%%%%%%%%%%%%%%%%%%%%%%%%%%%%%%%%%%%%%%%%%%

\begin{document}

\noindent
Recall from the lecture the following variants of the \SAT -problem:
\begin{itemize}
\item \MAXLEXSAT
\item \MAXWEIGHTSAT
\item \LogMAXLEXSAT
\item \MAXCARDSAT
\item \MINCARDSAT
\end{itemize}

\noindent
And also recall the following problem reductions. 

\begin{reduction}
\label{red:1}
{\em 
From \MAXLEXSAT\ to 
\\
\MAXWEIGHTSAT: 

\smallskip

\noindent
Consider an arbitrary instance 
$\phi; (x_1, \dots, x_n)$ of 
\MAXLEXSAT, \linebreak
where
$\phi$ is a Boolean formula over variables $X$ and 
$(x_1, \dots, x_n)$  is an ordering  
of the variables in $X$. 

\smallskip

\noindent
We define the instance 
$\phi; (x_1, \dots, x_n); (w(x_1), \dots, w(x_n)); z$ of 
{\bf WEIGHT-MAXIMAL MODEL SAT} as follows:

\begin{itemize}
\item Formula $\phi$ (and, hence, also variable set $X$) is left unchanged.
\item For every $i \in \{1, \dots, n\}$, we define
the weight $w(x_i) = 2^{n-i}$. 
\item We set $z = x_n$.
\end{itemize}

}
\end{reduction}

%%%%%%%%%%%%%%%%%%%%%%%%%%%%%%%%%%%%%%%%%%%%%%%%

\begin{reduction}
\label{red:2}
{\em 
From \LogMAXLEXSAT\ to 
\\
\MAXCARDSAT: 

\smallskip

\noindent
Consider an arbitrary instance 
$\phi; (x_1, \dots, x_n)$ of 
\LogMAXLEXSAT, \linebreak
where
$\phi$ is a Boolean formula 
and 
$(x_1, \dots, x_n)$ is an ordering 
of logarithmically many variables in $\phi$;
moreover, let $Y = \{y_1, \dots, y_m\}$ denote the remaining variables in $\phi$.

\smallskip

\noindent
We add the following fresh variables: 

\begin{itemize}
\item ``copies'' of each variable $x_i$, i.e.\, for every $i \in \{1, \dots, n\}$,  we introduce $2^{n-i}-1$ new variables 
$x_i^{(1)}, x_i^{(2)},  \dots, x_i^{(r_i)}$ with $r_i = 2^{n-i}-1$. 
%
\item a primed copy of each variable in $Y$: 
$Y' = \{y'_1, \dots, y'_m\}$. 
\end{itemize}

\noindent
Then we construct 
the instance $\psi; z$ of \MAXCARDSAT\ as follows:
We set $z = x_n$ and we define $\psi$ as

\[
\psi  =  \phi  \wedge   
%
\bigwedge_{i=1}^n 
\big( (x_i \leftrightarrow x_i^{(1)}) \wedge \dots
\wedge (x_i \leftrightarrow x_i^{(r_i)})
\big) \wedge  
%
\bigwedge_{i=1}^m 
y_i  \leftrightarrow \neg y'_i 
\]

}
\end{reduction}


%%%%%%%%%%%%%%%%%%%%%%%%%%%%%%%%%%%%%%%%%%%%%%%%%%%%%%%


\begin{reduction}
\label{red:3}
{\em 
From \MAXCARDSAT\ to 
\\
\MINCARDSAT: 

\smallskip

\noindent
Consider an arbitrary instance 
$\phi; x_i$ of \MAXCARDSAT\
where  $\phi$ is a Boolean formula
over the variables $X = \{x_1, \dots, x_n\}$.


\smallskip

\noindent
We add 
primed and double-primed copies of the variables, i.e.\, 
$X'= \{x'_1, \dots, x'_n\}$ and $X''= \{x''_1, \dots, x''_n\}$. 

\smallskip

\noindent
Then we construct 
the instance 
$\psi;  x_i$
 of \MINCARDSAT\ as follows:

\[
\psi  =   \phi \wedge 
\bigwedge_{i=1}^n 
\big( ( x_i  \leftrightarrow \neg x'_i) 
\wedge
( x_i  \leftrightarrow \neg x''_i)
\big)
\]


}
\end{reduction}


\medskip
\newpage


\begin{exercise}[4 credits]
{\em 
Prove the correctness of the above reduction
from \\
\MAXLEXSAT\ to 
\MAXWEIGHTSAT.
}%em
\end{exercise}


\paragraph*{Solution}

%The transformation as defined above.
%\begin{definition}
%\label{def:red}
%Let $\varphi$ be a propositional formula over the set of variables $X$ , let $(x_1,\dots,x_n)$ be a linear ordering of $X$. The let $\tau$ a transformation such that
%\begin{equation*}
%\tau(\varphi; (x_1,\dots,x_n)):=(\varphi; (w(x_1),\dots,w(x_n)); x_n)
%\end{equation*}
%where for some $i \in \{1,\dots ,n \}$ one has $w(x_i) := \mapsto 2^{n-i}$. 
%Moreover, in a small abuse of notation let $w(x_1, \dots , x_n):=(w(x_1),\dots,w(x_n))$.
%\end{definition}

In a small abuse of notation let $w(x_1, \dots , x_n):=(w(x_1),\dots,w(x_n))$. Moreover, consider the following

\begin{definition}
Let $X$ be some set (or ordering) of variables, let $\mathcal{I}$ be an interpretation and let $w$ be the weight function as introduced in Reduction \ref{red:1} over the variables in $X$, then let $w^{\mathcal{I}}$ be defined as
\begin{equation*}
w^{\mathcal{I}}(x):= \begin{cases}
w(x) \quad & \mathit{if}  \; \mathcal{I} \models x  \\
0 \quad & \mathit{otw.}   \\
\end{cases}
\end{equation*}
Moreover, for some $X' \subseteq X$ let $\omega^{\mathcal{I}}(X'):=\sum_{x \in X'} w^{\mathcal{I}}(x)$.
\end{definition}

%Subsequently, notation will further be mangled by using the !!!!!!!!
%The following lemma demonstrates the correctness of Reduction \ref{red:1}
\begin{lemma}
Let $\varphi$ be a propositional formula over the set of variables $X$ , let $\overline{x}:=(x_1,\dots,x_n)$ be a linear ordering of $X$. Then
\small
\begin{equation*}
(\varphi;\overline{x}) \in \MAXLEXSAT \iff \tau(\varphi;\overline{x}) \in \MAXWEIGHTSAT
\end{equation*}
where $\tau(\varphi;\overline{x})=(\varphi;w(\overline{x}); x_n)$, i.e. the transformation as presented in Reduction \ref{red:1}.
\end{lemma}
\begin{proof}
 Let $I:= \{1,\dots ,n\}$. Each direction is shown separately.
\begin{itemize}
\item[$\Rightarrow$] Assume that $(\varphi;\overline{x}) \in \MAXLEXSAT$. Hence, there exists a lexicographical maximal model $\mathcal{I}$ of $\varphi$ such that $\mathcal{I}\models x_n$. Hence, for any other model $\mathcal{J}$ of $\varphi$ there must exist a $k \in I$ such that  $\forall i  \in \{1,\dots , k-1\}$, $\mathcal{I}(x_i)=\mathcal{J}(x_i)$ and $\mathcal{I} \models x_k$ and $\mathcal{J} \nmodels x_k$. If $\mathcal{J}$ does not exists, $\mathcal{I}$ is trivially weight maximal. Otherwise, it must be that
\begin{equation*}
\omega^{\mathcal{I}}(x_{k}, \dots , x_n) \geq w^{\mathcal{I}}(x_k)=2^{n-k}> \sum_{k+1}^n 2^{n-i}\geq \omega^{\mathcal{J}}(x_{k}, \dots , x_n)
\end{equation*}
That is, even if $\forall i \in \{k+1, \dots, n\}$ $\mathcal{J} \models x_i $ one would have at most $\omega^{\mathcal{J}}(x_{k}, \dots , x_n)= \sum_{k+1}^n 2^{n-i}$ and since both interpretations agree on all $x_i$ smaller $k$ it must be that$\omega^{\mathcal{I}}(\overline{x})> \omega^{\mathcal{J}}(\overline{x})$. Therefore, from $\mathcal{J}$ being arbitrary, it follows that $\mathcal{I}$ is weight maximal. Furthermore, since $\mathcal{I} \models x_n$ one can conclude that $x_n$ is true in a weight maximal model (w.r.t. $w(\overline{x})$). Hence, $\tau(\varphi;\overline{x})\in \MAXWEIGHTSAT$.

\item[$\Leftarrow$]
Assume that $\tau(\varphi;\overline{x}) \in \MAXWEIGHTSAT$. Hence, there exists a weight maximal model $\mathcal{I}$ of $\varphi$ such that $\mathcal{I}\models x_n$. Let $\mathcal{J}$ be an arbitrary model of $\varphi$ other than $\mathcal{I}$. If $\mathcal{J}$ does not exists, $\mathcal{I}$ is trivially lexicographically maximal. Consider the order imposed by the vector $\overline{x}$. Since $\mathcal{J}$ differs from $\mathcal{I}$ it must be that there exists a $k \in I $ where $\mathcal{I}(x_k)\neq \mathcal{J}(x_k)$ and where $\forall i  \in \{1,\dots , k-1 \}$, $\mathcal{I}(x_i)=\mathcal{J}(x_i)$. Now there are two cases. If $\mathcal{I} \models x_k $ and $\mathcal{J} \nmodels x_k$, then $\mathcal{I}$ is lexicographically greater than $\mathcal{J}$. If $\mathcal{I} \nmodels x_k $ and $\mathcal{J} \models x_k$, then
\begin{equation*}
\omega^{\mathcal{J}}(x_{k}, \dots , x_n) \geq w^{\mathcal{J}}(x_k)=2^{n-k}> \sum_{k+1}^n 2^{n-i}\geq \omega^{\mathcal{I}}(x_{k}, \dots , x_n)
\end{equation*}
However, as above, this directly implies that $\omega^{\mathcal{I}}(\overline{x}) < \omega^\mathcal{J}(\overline{x})$ causing a contradiction.
With $\mathcal{J}$ being arbitrary, it follows that $\mathcal{I}$ is the lexicographically maximal model. Moreover, since $\mathcal{I}\models x_n$ it follows that $x_n$ is satisfied in the lexicographically maximal interpretation of $\varphi$, and thus $(\varphi;\overline{x}) \in \MAXLEXSAT$.


%Let $I:= \{1,\dots ,n\}$. Assume that $\tau(\varphi;\overline{x}) \in \MAXWEIGHTSAT$. Hence, there exists a lexicographical model $\mathcal{I}$ of $\varphi$ such that $\mathcal{I}\models x_n$. 
%Towards a contradiction, assume that there exists a model $\mathcal{J}$ of $\varphi$ such that $\mathcal{J}$ is lexicographically greater than $\mathcal{I}$ (w.r.t $\overline{x}$). Since, $\mathcal{J}$ is lexicographically greater than $\mathcal{I}$ it must be that there exist an $k \in I$ where $\mathcal{J} \models x_k $ and $\mathcal{I} \nmodels x_k$ and where $\forall i  \in I$ such that $i < k$, $\mathcal{I}(x_i)=\mathcal{J}(x_i)$. However, 
%\begin{equation*}
%\omega^{\mathcal{J}}(x_{k}, \dots , x_n) \geq \omega^{\mathcal{J}}(x_k)=2^{n-k}> \sum_{k+1}^n 2^{n-i}\geq \omega^{\mathcal{I}}(x_{k}, \dots , x_n)
%\end{equation*}
%However, this directly implies that $\omega^{\mathcal{I}}(\overline{x}) < \omega^\mathcal{J}(\overline{x})$, causing the desired contradiction. Therefore, $\mathcal{I}$ is the lexicographically maximal relation. Moreover, since $\mathcal{I}\models x_n$ it follows that $x_n$ is satisfied in the lexicographically maximal interpretation of $\varphi$, and thus $(\varphi;\overline{x}) \in \MAXLEXSAT$.
\end{itemize}

\end{proof}
%\begin{proof}
%Let $I:= \{1,\dots ,n\}$. Assume that $(\varphi;\overline{x}) \in \MAXLEXSAT$. Hence, there exists a lexicographical model $\mathcal{I}$ of $\varphi$ such that $\mathcal{I}\models x_n$. Towards a contradiction, assume that there exists a model $\mathcal{J}$ of $\varphi$ where $\mathcal{J} \models x_n$ such that $\omega^{\mathcal{I}}(\overline{x}) < \omega^{\mathcal{J}}(\overline{x})$. Since, $\mathcal{I}$ is lexicographical maximal and since $\omega^{\mathcal{I}}(\overline{x}) < \omega^{\mathcal{J}}(\overline{x})$, there must exist an $k \in I$ where $\mathcal{I} \models (x_k) $ and $\mathcal{J} \nmodels x_k$ and where $\forall i  \in I$ such that $i < k$, $\mathcal{I}(x_i)=\mathcal{J}(x_i)$.  However, 
%\begin{equation*}
%\omega^{\mathcal{I}}(x_{k}, \dots , x_n) \geq \omega^{\mathcal{I}}(x_k)=2^{n-k}> \sum_{k+1}^n 2^{n-i}\geq \omega^{\mathcal{J}}(x_{k}, \dots , x_n)
%\end{equation*}
%Hence, for $\mathcal{J}$ to be weight maximal it must be that $\mathcal{J} \models x_k$. Therefore, such a $k$ can not exists. Meaning that $\forall i \in I \; \mathcal{I}(x_i)=\mathcal{J}(x_i)$, contradicting $\omega^{\mathcal{I}}(\overline{x}) < \omega^{\mathcal{J}}(\overline{x})$. Thus, among all the models of $\varphi$ that satisfy $x_n$, $\mathcal{I}$ is weight maximal.
%\end{proof}



\newpage



%%%%%%%%%%%%%%%%%%%%%%%%%%%%%%%%%%%%%%%%%%%%%%%%%%%%%%%%%%%%%%%%%%%%%%%%%%%%%%%%%%%%%%%%%%%%%%%%%%%%%%%%%%%%%%%%%%%%%%%%%%%%%%%%%%%%%%%%%%%%%%%%%%%%%%%%%%%%%%%%%%%%%%%%%%%%%%%%%%%%%%%%%%%%%%%%%%%%%%%%%%%%%%%%%%%%%%%%%%%%%%%%%%%%%%%%%%%%%%%%%%%%%%%%%%%%%%%%%%%%%%%%%%%%%%%%%%%%%%%%%%%%%%%%%%%%%%%%%%%%%%%%%%%%%%%%%%%%%%%%%%%%%%%%%%%%%%%%%%%%%%%%%%%%%%%%%%%%%%%%%%%%%%%%%%%%%%%%%%%%%%%%%%%%%%%%%%%%%%%%%%%%%%%%




\begin{exercise}[3 credits]
{\em 
Prove the correctness of the above reduction
from \\
\LogMAXLEXSAT\ to 
\MAXCARDSAT.
}%em
\end{exercise}


\paragraph*{Solution}
\begin{definition}
Let $(\varphi;\overline{x}) \in \LogMAXLEXSAT$ then $\tau(\varphi;\overline{x})$ is defined as $\tau(\varphi;\overline{x}):= (\tau(\varphi), x_n)$, with $\tau(\varphi)$ being the construction from Reduction \ref{red:2}. Moreover, for some $k \in \{1, \dots , n\}$ let $\mathcal{X}_k:=\{ x_k, x_k^{(1)}, \dots, x_k^{(r_k)}\}$, where $r_k:=2^{n-k}-1$ (as defined in Reduction \ref{red:2}). Furthermore, let $\mathcal{X}_{\leq k}:= \bigcup_{i \in \{1, \dots, k\}} \mathcal{X}_k$. Analogously for $\mathcal{X}_{> k}$.
\end{definition}

\begin{definition}
Let $\mathcal{I}$ be an interpretation, then $\mathfrak{S}(\mathcal{I}):=\{x \mid \mathcal{I}(x)=\True\}$. Moreover, if $\mathcal{I}$ is a model of the formula $\varphi$ then one can construct the interpretation $\tau(\mathcal{I})$ such that $\forall i \in \{1, \dots , m\} \;\tau(\mathcal{I})(y_i):=\mathcal{I}(y_i) \land \tau(\mathcal{I})(y_i'):=\neg \mathcal{I}(y_i)$, as well as $\forall i \in \{1, \dots , n\}  \; \tau(\mathcal{I})(x_i):=\mathcal{I}(x_i) \land \forall j \in \{1, \dots 2^{n-i}-1\} \; \tau(\mathcal{I})(x_i^{(j)}):=\mathcal{I}(x_i)$.
%\begin{itemize}
%\item $\forall i \in \{1, \dots , m\} \;\tau(\mathcal{I})(y_i):=\mathcal{I}(y_i)$
%\item $\forall i \in \{1, \dots , m\} \; \tau(\mathcal{I})(y_i'):=\neg \mathcal{I}(y_i)$,
%\item $\forall i \in \{1, \dots , n\}  \; \tau(\mathcal{I})(x_i):=\mathcal{I}(x_i)$ and
%\item $\forall i \in \{1, \dots , n\} \forall j \in \{1, \dots 2^{n-i}-1\} \; \tau(\mathcal{I})(x_i^{(j)}):=\mathcal{I}(x_i)$.
%\end{itemize}
\end{definition}


\begin{observation}
\label{obs:1a}
For an arbitrary $\mathcal{I}$ such that $ \mathcal{I} \models \bigwedge_{i=1}^n 
\big( (x_i \leftrightarrow x_i^{(1)}) \wedge \dots
\wedge (x_i \leftrightarrow x_i^{(r_i)})
\big) $ then for any $i \in \{1, \dots , n\}$ one has $\mathcal{I} \models x_i$ if and only if $\mathcal{I} \models  x_i \land x_i^{(1)} \land \dots x_i^{(r_i)}$. 
\end{observation}

\begin{observation}
\label{obs:1b}
Take an arbitrary model $\mathcal{I}$ of $\tau(\varphi)$. For any $i \in \{1, \dots,n\}$ one has $\mathcal{I} \models y_i$ if and only if $\mathcal{I} \nmodels y_i'$. Hence, for any other model $\mathcal{J}$ of $\tau(\varphi)$ it must be that
\begin{equation*}
|\mathfrak{S}(\mathcal{J}) \cap (Y \cup Y')|=|\mathfrak{S}(\mathcal{I}) \cap (Y \cup Y')|
\end{equation*}
\end{observation}

\begin{lemma}
\label{lem:1c}
For the models $\mathcal{I}$ and $\mathcal{J}$ of $\tau(\varphi)$, consider the order $\overline{x}:=(x_1, \dots , x_n)$ on the basis of which $\tau(\varphi)$ was constructed. Then it holds that, there exists $k \in \{1, \dots , n \}$ such that $\forall i \in  \{1, \dots , k-1 \}$ one has $\mathcal{I}(x_i)=\mathcal{J}(x_i)$, as well as $\mathcal{J} \nmodels x_k$ and $\mathcal{I}  \models x_k$ if and only if $|\mathfrak{S}(\mathcal{J})| < |\mathfrak{S}(\mathcal{I}) |$.
\end{lemma}
\begin{proof}

\begin{itemize}
\item[$\Rightarrow$]
By Observation \ref{obs:1a}, this implies that $\mathcal{X}_k \subseteq \mathfrak{S}(\mathcal{I})$ and $\mathcal{X}_k \cap \mathfrak{S}(\mathcal{J})= \emptyset$. Meaning that, 
\begin{equation*}
|\mathcal{X}_{\leq k} \cap \mathfrak{S}(\mathcal{I})| = |\mathcal{X}_{\leq k} \cap \mathfrak{S}(\mathcal{J})|+ 2^{n-k}
\end{equation*}
However, by construction $|\mathcal{X}_{> k} |<2^{n-k}$. Thus it follows that $|\mathfrak{S}(\mathcal{J}) \cap \mathcal{X}_{\leq n}| < |\mathfrak{S}(\mathcal{I}) \cap \mathcal{X}_{\leq n} |$. By Observation \ref{obs:1b}, this implies that $|\mathfrak{S}(\mathcal{J})| < |\mathfrak{S}(\mathcal{I}) |$.

\item[$\Leftarrow$] If $|\mathfrak{S}(\mathcal{J})| < |\mathfrak{S}(\mathcal{I}) |$, by Observation \ref{obs:1b}, there must exists an $k \in \{1, \dots , n\}$ such that $\mathcal{I} \models x_k$ and $\mathcal{J} \nmodels x_k$.  With respect to the ordering $\overline{x}$ let this be the first of its occurrences. Assume that there exists a $l< k$ such that $\mathcal{I} \nmodels x_l$ and $\mathcal{J} \models x_l$ If this is the case then this implies that $\mathcal{X}_{l} \subseteq \mathfrak{S}(\mathcal{J})$ and $\mathcal{X}_l \cap \mathfrak{S}(\mathcal{I})= \emptyset$. Meaning that, 
\begin{equation*}
|\mathcal{X}_{\leq l} \cap \mathfrak{S}(\mathcal{J})| \geq |\mathcal{X}_{\leq l} \cap \mathfrak{S}(\mathcal{I})|+ 2^{n-l}
\end{equation*}
However, by construction $|\mathcal{X}_{> l} |<2^{n-l}$, thus even if $k$ is satisfied by $\mathcal{I}$ this would not matter, as it would still be the case that $|\mathfrak{S}(\mathcal{I})| < |\mathfrak{S}(\mathcal{J}) |$, which is a contradiction. Hence, $l$ does not exist. Now with $k$ being the first of its occurrences, it must be the case that $\forall i \in  \{1, \dots , k-1 \}$ one has $\mathcal{I}(x_i)=\mathcal{J}(x_i)$. 
\end{itemize}
\end{proof}

\begin{observation}
\label{obs:1d}
For the models $\mathcal{J}$ and $\mathcal{I}$ of $\tau(\varphi)$. Then $|\mathfrak{S}(\mathcal{I})|=|\mathfrak{S}(\mathcal{J})|$ if and only if $\mathfrak{S}(\mathcal{I}) \cap X= \mathfrak{S}(\mathcal{J}) \cap X$.
\end{observation}
\begin{proof}
\begin{itemize}
\item[$\Rightarrow$]
Assume that $|\mathfrak{S}(\mathcal{I})|=|\mathfrak{S}(\mathcal{J})|$ and $\mathfrak{S}(\mathcal{I}) \cap X\neq \mathfrak{S}(\mathcal{J}) \cap X$.  Consider the order $\overline{x}$ on the basis of which $\tau(\varphi)$ was constructed. Given this order, and from Observation \ref{obs:1b}, there must be a $k \in \{1, \dots , n \}$ such that $\forall i \in  \{1, \dots , k-1 \}$ one has $\mathcal{I}(x_i)=\mathcal{J}(x_i)$, as well as $\mathcal{J}(x_k) \neq \mathcal{I}(x_k)$. W.l.o.g.\  $\mathcal{J} \nmodels x_k$ and $\mathcal{I}  \models x_k$, thus from Lemma \ref{lem:1c} it follows that $|\mathfrak{S}(\mathcal{J})| < |\mathfrak{S}(\mathcal{I}) |$, which is a contradiction.

\item[$\Leftarrow$] This follows trivially from Observation \ref{obs:1a} \& \ref{obs:1b}.
\end{itemize}

\end{proof}


%The following lemma demonstrates the correctness of Reduction \ref{red:2}

\begin{lemma}
Let $\varphi$ be a propositional formula over the set of variables $X \cup Y$ , let $\overline{x}:=(x_1,\dots,x_n)$ be a linear ordering of $X$ and let $Y:=\{y_1, \dots, y_m\}$. Then
\small
\begin{equation*}
(\varphi;\overline{x}) \in \LogMAXLEXSAT \iff \tau(\varphi;\overline{x}) \in \MAXCARDSAT
\end{equation*}
%where $\tau(\varphi;\overline{x}):= (\tau(\varphi), x_n)$, with $\tau(\varphi)$ being the construction from the Reduction \ref{red:2}.
\end{lemma}

\begin{proof}
Each direction is shown separately. 
\begin{itemize}
\item[$\Rightarrow$] Assume that $(\varphi;\overline{x}) \in \LogMAXLEXSAT $.  Hence, there exists a lexicographically maximal (w.r.t. $\overline{x}$) model $\mathcal{I}$ of $\varphi$ where $x_n$ evaluates to true. 
%Using $\mathcal{I}$ one can construct $\mathcal{I}_{\tau}$ such that 
%\begin{itemize}
%\item $\forall i \in \{1, \dots , m\} \; \mathcal{I}_{\tau}(y_i):=\mathcal{I}(y_i)$,
%\item $\forall i \in \{1, \dots , m\} \; \mathcal{I}_{\tau}(y_i'):=\neg \mathcal{I}(y_i)$,
%\item $\forall i \in \{1, \dots , n\}  \; \mathcal{I}_{\tau}(x_i):=\mathcal{I}(x_i)$ and
%\item $\forall i \in \{1, \dots , n\} \forall j \in \{1, \dots 2^{n-i}-1\} \; \mathcal{I}_{\tau}(x_i^{(j)}):=\mathcal{I}(x_i)$.
%\end{itemize}
Let $\mathcal{I}_{\tau}:=\tau(\mathcal{I})$. By construction $\mathcal{I}_{\tau} \models \tau(\varphi)$. Hence, it remains to be shown that $\mathcal{I}_{\tau}$ is a cardinal maximal 
model. Take an arbitrary model $\mathcal{J}_{\tau}$ of $\tau(\varphi)$ other than $\mathcal{I}_{\tau}$. If none exists, then $\mathcal{I}_{\tau}$ is trivially maximal in cardinality. Firstly, by Observation \ref{obs:1b} it must be that
$|\mathfrak{S}(\mathcal{J}_{\tau}) \cap (Y \cup Y')|=|\mathfrak{S}(\mathcal{I}_{\tau}) \cap (Y \cup Y')|$.
Secondly, 
%since $\mathcal{I}$ is a lexicographically maximal model of $\varphi$ w.r.t.\ $\overline{x}$, by construction this carries over to $\mathcal{I}_{\tau}$. 
%Moreover, by construction it must be that $\mathcal{J} \models \varphi$.
 two cases ought to be considered. On the one hand, if $\forall i \in \{1, \dots , n\}\; \mathcal{I}_{\tau}(x_i)=\mathcal{J}_{\tau}(x_i)$ it follows that $|\mathfrak{S}(\mathcal{I}_{\tau})|=|\mathfrak{S}(\mathcal{J}_{\tau})|$ by Observation \ref{obs:1d}. On the other hand, since $\mathcal{J}_{\tau}$ must model $\varphi$, one can restrict it to the variables in $X \cup Y$ to obtain $\mathcal{J}$ which still models $\varphi$. Since, $\mathcal{I}$ is lexicographically maximal with respect to $\overline{x}$, it follows that there must be a $k \in \{1, \dots , n \}$ such that $\forall i \in  \{1, \dots , k-1 \}$ one has $\mathcal{I}(x_i)=\mathcal{J}(x_i)$, as well as  $\mathcal{J} \nmodels x_k$ and $\mathcal{I} \models x_k$. However, since $\mathcal{I}_{\tau}$ agrees with $\mathcal{I}$ and since $\mathcal{J}_{\tau}$ agrees with $\mathcal{J}$ on all $X$'s (and $Y$'s), the same must hold for $\mathcal{I}_{\tau}$ and $\mathcal{J}_{\tau}$. Hence, from Lemma \ref{lem:1c} one obtains $|\mathfrak{S}(\mathcal{J}_{\tau})| < |\mathfrak{S}(\mathcal{I}_{\tau}) |$. Therefore, from the two cases one obtains $|\mathfrak{S}(\mathcal{J}_{\tau})|\leq |\mathfrak{S}(\mathcal{I}_{\tau}) |$ for some arbitrary model of $\tau(\varphi)$, and thus $\mathcal{I}_{\tau}$ is cardinal maximal. Finally, since $\mathcal{I}_{\tau} \models x_n$ it follows that $(\tau(\varphi), x_n) \in \MAXCARDSAT$.
%
%By Observation \ref{obs:1a}, this implies that $\mathcal{X}_k \subseteq \mathfrak{S}(\mathcal{I}_{\tau})$ and $\mathcal{X}_k \cap \mathfrak{S}(\mathcal{J}_{\tau})= \emptyset$. Meaning that, 
%\begin{equation*}
%|\mathcal{X}_{\leq k} \cap \mathfrak{S}(\mathcal{I}_{\tau})| = |\mathcal{X}_{\leq k} \cap \mathfrak{S}(\mathcal{J}_{\tau})|+ 2^{n-k}
%\end{equation*}
%However, by construction $|\mathcal{X}_{> k} |<2^{n-k}$, thus it follows that $|\mathfrak{S}(\mathcal{J}_{\tau})| < |\mathfrak{S}(\mathcal{I}_{\tau}) |$.  
%

\item[$\Leftarrow$] Assume that $(\tau(\varphi), x_n)  \in \MAXCARDSAT $.  Hence, there must exits a model $\mathcal{I}_{\tau}$ of $\tau(\varphi)$ that is maximal in its cardinality. Clearly, by restricting $\mathcal{I}_{\tau}$ to the variables in $X\cup Y$, one can construct an interpretation $\mathcal{I}$ such that $\mathcal{I} \models \varphi$. Now, take any arbitrary model $\mathcal{J}$ of $\varphi$ other than $\mathcal{I}$. Again, if $\mathcal{J}$ would not exists then $\mathcal{I}$ must be lexicographically maximal (w.r.t. $\overline{x}$). Then $\mathcal{J}_{\tau}:=\tau(\mathcal{J})$ is a model of $\tau(\varphi)$. Moreover, by maximality of $\mathcal{I}_{\tau}$ it is known that $|\mathfrak{S}(\mathcal{J}_{\tau})| \leq |\mathfrak{S}(\mathcal{I}_{\tau})|$.
Clearly, by Observation \ref{obs:1b} it must be that $|\mathfrak{S}(\mathcal{J}_{\tau}) \cap (Y \cup Y')|=|\mathfrak{S}(\mathcal{I}_{\tau}) \cap (Y \cup Y')|$.
Hence, by Observation \ref{obs:1a}, the only difference in cardinality can occur due to the respective evaluation of the variables in $X$.  Thereby, inducing two cases. On the one hand, if $|\mathfrak{S}(\mathcal{J}_{\tau})|=|\mathfrak{S}(\mathcal{I}_{\tau})|$, then it follows by Observation \ref{obs:1d}, that $\mathfrak{S}(\mathcal{J}_{\tau}) \cap X=\mathfrak{S}(\mathcal{I}_{\tau}) \cap X$, implying that $\mathcal{I}$ is lexicographically equivalent to $\mathcal{J}$, i.e. they only differ on their assignments on the variables in $Y$. On the other hand, if $|\mathfrak{S}(\mathcal{J}_{\tau})|<|\mathfrak{S}(\mathcal{I}_{\tau})|$, then it follows from Lemma \ref{lem:1c} that there exists a $k \in \{1, \dots , n \}$ such that $\forall i \in  \{1, \dots , k-1 \}$ one has $\mathcal{I}_{\tau}(x_i)=\mathcal{J}_{\tau}(x_i)$, as well as $\mathcal{J}_{\tau} \nmodels x_k$ and $\mathcal{I}_{\tau}  \models x_k$. However, this means that $\mathcal{I}_{\tau}$ is lexicographically greater than $\mathcal{J}_{\tau}$ w.r.t.\ the order $\overline{x}$. Since this carries over to $\mathcal{I}$ and $\mathcal{J}$ and since $\mathcal{I} \models x_n$ by assumption, it follows that $(\varphi, \overline{x}) \in  \LogMAXLEXSAT$.


%
%
%Assume towards a contradiction that there exists a model $\mathcal{J}$ of $\tau(\varphi)$ that is lexicographically greater that $\mathcal{I}$ on the variables $\overline{x}$. Hence, there must be a $k \in \{1, \dots , n \}$ such that $\forall i \in  \{1, \dots , n \}$ with $i<k$ one has $\mathcal{I}(x_i)=\mathcal{J}(x_i)$, as well as  $\mathcal{J} \models x_k$ and $\mathcal{I}  \nmodels x_k$. However, this implies that $\{ x_k, x_k^{(1)} , \dots, x_k^{(r_k)}\} \subseteq \mathfrak{S}(\mathcal{J})$ and $\{ x_k, x_k^{(1)} , \dots, x_k^{(r_k)}\} \cap \mathfrak{S}(\mathcal{I})= \emptyset$. Meaning that, 
%\begin{equation*}
%|\mathcal{X}_{\leq k} \cap \mathfrak{S}(\mathcal{J})| = |\mathcal{X}_{\leq k} \cap \mathfrak{S}(\mathcal{I})|+ 2^{n-k}
%\end{equation*}
%However, by construction $|\mathcal{X}_{> k} |<2^{n-k}$, thus it follows that $|\mathfrak{S}(\mathcal{J})| > |\mathfrak{S}(\mathcal{I}) |$ which is clearly a contradiction. Therefore, the $\mathcal{I}$ is a lexicographically maximal model of $\tau(\varphi)$ with respect to the ordering $\overline{x}$. Now since $\mathcal{I} \models \varphi$, one can conclude that $\mathcal{I}$ is a lexicographically maximal model of $\varphi$ that satisfies $x_n$. Hence, $(\varphi, \overline{x}) \in  \LogMAXLEXSAT$.
\end{itemize}

\end{proof}


\newpage


\begin{exercise}[3 credits]
{\em 
Prove the correctness of the above 
reduction
from \\
\MAXCARDSAT\ to 
\MINCARDSAT.
}%em
\end{exercise}
\paragraph*{Solution}

\begin{definition}
Let $\varphi$ be a propositional formula over the set of variables $X:=\{x_1,\dots , x_n\}$, then $\tau(\varphi)$ the formula as defined in Reduction \ref{red:3}.
Moreover, consider a model $\mathcal{I}$ of $\varphi$, then $\tau(\mathcal{I})$ is the interpretation $ \forall i \in \{1,\dots, n\}$ $\tau(\mathcal{I})(x_i):=\mathcal{I}(x_i) \;  \land  \; \tau(\mathcal{I})(x_i'):=\neg \mathcal{I}(x_i)  \;  \land \;  \tau(\mathcal{I})(x_i''):=\neg \mathcal{I}(x_i)$. 

\end{definition}

\begin{observation}
\label{obs:2}
Let $\varphi$ be a propositional formula over the set of variables $X:=\{x_1,\dots , x_n\}$. Let $\mathcal{I}$ be a model of $\tau(\varphi)$. Then $|\mathfrak{S}(\mathcal{I})\cap\tau(X)| = 2n-|\mathfrak{S}(\mathcal{I}) \cap X|$
\end{observation}
\begin{proof}
Clearly, $\forall i \in \{1,\dots , n\}  \; x_i \notin \mathfrak{S}(\mathcal{I}) \iff x_i' \in \mathfrak{S}(\mathcal{I}) \land x_i'' \in \mathfrak{S}(\mathcal{I})$. 
Hence, 
\begin{equation*}
|\mathfrak{S}(\mathcal{I})| = |\mathfrak{S}(\mathcal{I})\cap\tau(X)|  = |\mathfrak{S}(\mathcal{I}) \cap X| + 2(|X| -  |\mathfrak{S}(\mathcal{I}) \cap X|) = 2n- |\mathfrak{S}(\mathcal{I}) \cap X|
\end{equation*}
where the first equality holds, if one considers only interpretations restricted to the variables in $\tau(X)$.\footnote{as is required to obtain any notion of maximality}
\end{proof}

The following lemma demonstrates the correctness of Reduction \ref{red:3}

\begin{lemma}
Let $\varphi$ be a propositional formula over the set of variables $X:=\{x_1,\dots , x_n\}$. Then,
\small
\begin{equation*}
(\varphi;x_i) \in \MAXCARDSAT \iff (\tau(\varphi); x_i) \in \MINCARDSAT
\end{equation*}

\end{lemma}
\begin{proof}
Each direction is shown separately.
\begin{itemize}
\item[$\Rightarrow$] Assume that $(\varphi;x_i) \in \MAXCARDSAT$. Hence, $x_i$ is satisfied in a cardinal maximal model $\mathcal{I}$ of $\varphi$. Using this, construct the interpretation $\mathcal{I}_{\tau}:=\tau(\mathcal{I})$. By construction, $\mathcal{I}_{\tau} \models  \tau(\varphi)$, thus it remains to be shown that it is cardinal minimal. Consider an arbitrary model $\mathcal{J}_{\tau}$ of $\tau(\varphi)$. Restrict $\mathcal{J}_{\tau}$ to the variables in $X$ in order to create $\mathcal{J}$. Clearly, $\mathcal{J} \models \varphi$. By maximality of $\mathcal{I}$, any interpretation that satisfies $\varphi$ must satisfy fewer or equally as many $x \in X$, i.e.
\begin{equation*}
r=|\mathfrak{S}(\mathcal{J}_{\tau}) \cap X| = |\mathfrak{S}(\mathcal{J})| \leq |\mathfrak{S}(\mathcal{I})| =|\mathfrak{S}(\mathcal{I}_{\tau})\cap X|= o
\end{equation*}
However, by Observation \ref{obs:2}, this implies that 
\begin{equation*}
|\mathfrak{S}(\mathcal{J}_{\tau})| =|\mathfrak{S}(\mathcal{J}_{\tau}) \cap \tau(X)|=2n-r \geq 2n-o = |\mathfrak{S}(\mathcal{I}_{\tau}) \cap \tau(X) |=|\mathfrak{S}(\mathcal{I}_{\tau})|
\end{equation*}
Now with $\mathcal{J}_{\tau}$ being arbitrary $\mathcal{I}_{\tau}$ is a cardinal minimal model of $\tau(\varphi)$. Moreover, by assumption $x_i$ is satisfied by $\mathcal{I}$ and thus also by $\mathcal{I}_{\tau}$. 

\noindent
Hence,  $(\tau(\varphi); x_i) \in \MINCARDSAT$.


\item[$\Leftarrow$] Assume that $(\tau(\varphi); x_i) \in \MINCARDSAT$. Hence, $x_i$ is satisfied in a cardinal minimal model $\mathcal{I}_{\tau}$ of $\tau(\varphi)$. Using this construct the interpretation $\mathcal{I}$ by restricting $\mathcal{I}_{\tau}$ to the set of variables in $X$. By construction, $\mathcal{I} \models \varphi$, thus it remains to be shown that $\mathcal{I}$ is cardinal maximal.
By minimality of $\mathcal{I}_{\tau}$, any interpretation that satisfies $\tau(\varphi)$ must satisfy more or equally as many $x \in \tau(X)$. Now consider an arbitrary model $\mathcal{J}$ of $\varphi$ and construct $\mathcal{J}_{\tau}:=\tau(\mathcal{J})$. Clearly, $\mathcal{J}_{\tau} \models \tau(\varphi)$ and thus
\begin{equation*}
2n-|\mathfrak{S}(\mathcal{J}_{\tau}) \cap X| =|\mathfrak{S}(\mathcal{J}_{\tau}) \cap \tau(X)|=|\mathfrak{S}(\mathcal{J}_{\tau})|   \geq |\mathfrak{S}(\mathcal{I}_{\tau})|  =|\mathfrak{S}(\mathcal{I}_{\tau}) \cap \tau(X) | = 2n-|\mathfrak{S}(\mathcal{I}_{\tau}) \cap X|
\end{equation*}
which clearly is equivalent to $|\mathfrak{S}(\mathcal{J}_{\tau}) \cap X| \leq |\mathfrak{S}(\mathcal{I}_{\tau}) \cap X|$. However, by construction this is equivalent to $|\mathfrak{S}(\mathcal{J}) | \leq |\mathfrak{S}(\mathcal{I}) |$.
Therefore, $\mathcal{I}$ is a cardinal maximal model of $\varphi$. Moreover, by assumption $x_i$ is satisfied by $\mathcal{I}_{\tau}$ and thus also by $\mathcal{I}$. Hence,  $(\varphi;x_i) \in \MAXCARDSAT$.

%
%\item[$\Leftarrow$] Assume that $(\tau(\varphi); x_i) \in \MINCARDSAT$. Hence, $x_i$ is satisfied in a cardinal minimal model $\mathcal{I}$ of $\tau(\varphi)$. Let $\mathcal{I}'$ be the interpretation $\forall x \in X\; \mathcal{I}'(x):= \mathcal{I}(x)$. Clearly, $\mathcal{I}' \models \varphi$. Hence, it remains to be shown that $\mathcal{I}'$ is cardinal maximal. Assume that there exists a model $\mathcal{J}$ of $\varphi$, such that $|\mathfrak{S}(\mathcal{J})|> |\mathfrak{S}(\mathcal{I}')|$.
%This implies that there exists a model $\mathcal{J}':=\tau(\mathcal{J})$
%However, since $|\mathfrak{S}(\mathcal{J})|>|\mathfrak{S}(\mathcal{I}')|$, it follows, by Observation \ref{obs:2}, that $|\mathfrak{S}(\mathcal{J}')|= 2n - |\mathfrak{S}(\mathcal{J})| < 2n - |\mathfrak{S}(\mathcal{I}')| =|\mathfrak{S}(\mathcal{I})|$, which clearly is a contradiction.
%Hence, there exists cardinal maximal model of $\varphi$ under which $x_i$ is satisfied, i.e.\ $(\varphi;x_i) \in \MAXCARDSAT$.
\end{itemize}

\end{proof}
\end{document}


