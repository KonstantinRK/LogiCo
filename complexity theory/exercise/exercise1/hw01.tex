   \documentclass [11pt]{article}
   \usepackage{latexsym}
   \usepackage{amssymb}
   \usepackage{amsmath}
      \usepackage{amsthm}
   \usepackage{url}

   \textwidth      15cm
   \textheight     23cm
   \oddsidemargin 0.5cm
   \topmargin    -0.5cm
   \evensidemargin\oddsidemargin


   \pagestyle{plain}
   \bibliographystyle{plain}




  \newtheorem{theorem}{Theorem}
  \newtheorem{lemma}[theorem]{Lemma}
  \newtheorem{corollary}[theorem]{Corollary}
    \newtheorem{observation}[theorem]{Observation}
  \newtheorem{proposition}[theorem]{Proposition}
  \newtheorem{conjecture}[theorem]{Conjecture}
  \newtheorem{definition}[theorem]{Definition}
  \newtheorem{example}[theorem]{Example}
  \newtheorem{remark}[theorem]{Remark}
  \newtheorem{exercise}[theorem]{Exercise}
 
 


  \newcommand{\ra}{\rightarrow}
  \newcommand{\Ra}{\Rightarrow}
  \newcommand{\La}{\Leftarrow}
  \newcommand{\la}{\leftarrow}
  \newcommand{\LR}{\Leftrightarrow}

  \newcommand{\lto}{\rightarrow}
  \newcommand{\sto}{\Rightarrow}
  \newcommand{\snto}{\Leftarrow}
  \newcommand{\lnto}{\leftarrow}
  \newcommand{\siff}{\Leftrightarrow}
   \newcommand{\liff}{\leftrightarrow}
   \newcommand{\nmodels}{\not\models}


  \renewcommand{\phi}{\varphi}
  \renewcommand{\theta}{\vartheta}


\newcommand{\bigO}{\mathrm{O}}
\newcommand{\bigOmega}{\Omega}
\newcommand{\bigTheta}{\Theta}
\newcommand{\tuple}[1]{\langle#1\rangle}

%%% useful macros for Turing machines:
\newcommand{\blank}{\sqcup}
\newcommand{\ssym}{\triangleright}
\newcommand{\esym}{\triangleleft}
\newcommand{\halt}{\mbox{h}}
\newcommand{\yess}{\mbox{``yes''}}
\newcommand{\nos}{\mbox{``no''}}
\newcommand{\lmove}{\leftarrow}
\newcommand{\rmove}{\rightarrow}
\newcommand{\stay}{-}
\newcommand{\diverge}{\nearrow}
\newcommand{\yields}[1]{\stackrel{#1}{\rightarrow}}

\newcommand{\HALTING}{\mbox{\bf HALTING}}

\newcommand{\ccfont}[1]{\protect\mathsf{#1}}
\newcommand{\NP}{\ccfont{NP}}
\newcommand{\SAT}{\mbox{\bf SAT}}
\newcommand{\sym}[3]{\textit{symbol}_{#1}[#2,#3]}
\newcommand{\cursor}[2]{\textit{cursor}[#1,#2]}
\newcommand{\state}[2]{\textit{state}_{#1}[#2]}
\newcommand{\accept}{\textit{accept}}
\newcommand{\naccept}{\textit{naccept}}
\newcommand{\conf}{\textit{conf}}
\newcommand{\trans}[1]{\mathfrak{T}_{\mathcal{#1}}}
\newcommand{\sequ}[1]{\mathfrak{C}_{#1}}
\newcommand{\sequint}[2]{ \mathcal{#1}_{\mathfrak{C}_{#2}}}
\renewcommand{\labelenumi}{(\alph{enumi})}

%%%%%%%%%%%%%%%%%%%%%%%%%%%%%%%%%%%%%%%%%%%%%%%%%%%%%%%%%%%%%%%%%%%

\begin{document}


%\maketitle


\medskip

\noindent
\begin{exercise}[5 credits]
{\em Recall the problem reduction from an arbitrary language $L \in \NP$ 
to the \SAT-problem given in the lecture. 
In particular, recall the construction of an instance $R(x)$ of
\SAT\ from an arbitrary instance $x$ of $L$.
Give a rigorous proof of the correctness of this reduction, i.e.
$x \in L \LR R(x) \in \SAT$.
} % em
\end{exercise}


\noindent
{\bf Hint.} Prove both directions of the equivalence separately. 
The intended meaning of
the propositional atoms in $R(x)$ is clear. You have to be careful, what is given, what is constructed (or defined), and what has to be proved. 

\begin{itemize}
\item Suppose that $x \in L$. 
\begin{itemize}
\item given: Then we know that there exists a successful computation of the NTM $T$ on input $x$. By our assumption, this computation consists of exactly $N$ steps. Let $\conf_0, \dots, \conf_N$ denote the configurations of the NTM $T$ along this computation.
\item constructed/defined:  We define a truth assignment ${\cal I}$ appropriate to $R(x)$ according to the intended meaning of the propositional atoms in $R(x)$. 
\item to be proved: It remains to show that all conjuncts in $R(x)$ are indeed satisfied by ${\cal I}$. For this purpose, you have to inspect all groups of conjuncts in $R(x)$ and argue that each of them is true in ${\cal I}$.
\end{itemize}

\item Suppose that $R(x) \in \SAT$.
\begin{itemize}
\item given: Then there exists a satisfying truth assignment ${\cal I}$ of $R(x)$. 
\item constructed/defined: We can construct a sequence $\conf_0, \dots, \conf_N$ of configurations 
of the NTM $T$ according to the intended meaning of the propositional atoms in $R(x)$. 
\item to be proved: First argue that the configurations $\conf_0, \dots, \conf_N$ are well-defined (by using the the fact that ${\cal I}$ satisfies all conjuncts of $R(x)$).

It remains to show that there exists a computation of $T$ on input $x$ which produces exactly this sequence of configurations $\conf_0, \dots, \conf_N$. Note that, in particular, this is a successful computation by the conjunct $\state{s_m}{N}$
(with $s_m  = \yess$) in $R(x)$ and by the intended meaning of $\state{s_m}{N}$.
For this purpose, you have to  show by induction on $\tau$ that there exists 
a computation of $T$ on input $x$ whose first $\tau$ 
configurations are $\conf_0, \dots, \conf_\tau$.

\end{itemize}


\end{itemize}



\noindent
{\bf For your convenience.} The groups of conjuncts in $R(x)$ are recalled below.

\smallskip
\noindent
{\bf 1. Initialization facts.} \\
$\sym{\ssym}{0}{0}$  \\
$\sym{\sigma}{0}{\pi}$
  \quad\quad for $1\leq \pi \leq |x|$, where $x_\pi=\sigma$ \\
$\sym{\blank}{0}{\pi}$
  \quad\quad for $|x| < \pi\leq N$ \\
$\cursor{0}{0}$\\
$\state{s_0}{0}$

\smallskip
\noindent
{\bf 2. Transition rules.} For each pair $(s,\sigma)$ of state $s$ and symbol $\sigma$ let $\tuple{s,\sigma,s'_1,\sigma'_1,d_1}$, \dots, 
$\tuple{s,\sigma,s'_k,\sigma'_k,d_k}$ denote all possible transitions according
to the  transition relation $\Delta$ (for the cursor movements, we write $d_i \in \{-1,0,1\}$ rather than $d_i \in \{\lmove,\stay,\rmove\}$).
 \\
Then $R(x)$ contains the 
following conjuncts for each value of $\tau$ and $\pi$ such that $0\leq \tau<N$ and $0
\leq \pi < N$

\smallskip
\noindent
%
$\state{s}{\tau} \wedge \sym{\sigma}{\tau}{\pi} \wedge \cursor{\tau}{\pi}$
$\ra$ \\ 
\mbox{}\ \  $[({\state{s'_1}{\tau+1}}\wedge 
\sym{\sigma'_1}{\tau+1}{\pi} \wedge {\cursor{\tau+1}{\pi+d_1}} 
) \vee \dots \vee $ \\
\mbox{}\ \  $({\state{s'_k}{\tau+1}} \wedge 
\sym{\sigma'_k}{\tau+1}{\pi} \wedge {\cursor{\tau+1}{\pi+d_k}} 
) ] $ 


\smallskip
\noindent
{\bf 3. Uniqueness constraints.} Let 
$K = \{s_0, \dots, s_m\}$ and
$\Sigma = \{\sigma_1, \dots, \sigma_n\}$. \\
Then $R(x)$ contains the following formulae for each value of $\tau$ and $\pi$ such that $0\leq \tau \leq N$, $0 \leq \pi \leq  N$, 
$0 \leq i \leq m$, and $1 \leq j \leq n$.

\smallskip
\noindent
$\state{s_i}{\tau} \leftrightarrow
(\neg \state{s_0}{\tau} \wedge \dots \wedge 
\neg \state{s_{i-1}}{\tau} \wedge \mbox{}$ \\
\mbox{} \quad \ $\mbox{}\wedge \neg \state{s_{i+1}}{\tau} \wedge \dots \wedge 
\neg \state{s_{m}}{\tau})$

\noindent
$\cursor{\tau}{\pi} \leftrightarrow
(\neg \cursor{\tau}{0} \wedge \dots \wedge \neg \cursor{\tau}{\pi - 1} 
\wedge \mbox{}$ \\
\mbox{} \quad \ $\mbox{}\wedge 
\neg \cursor{\tau}{\pi+1} \wedge \dots \wedge \neg \cursor{\tau}{N}$ 


\noindent
$\sym{\sigma_j}{\tau}{\pi} \leftrightarrow
(\neg \sym{\sigma_1}{\tau}{\pi} \wedge \dots \wedge
\neg \sym{\sigma_{j-1}}{\tau}{\pi} \wedge \mbox{}$ \\
\mbox{} \quad \ $\mbox{}\wedge 
\neg \sym{\sigma_{j+1}}{\tau}{\pi} \wedge \dots \wedge
\neg \sym{\sigma_{n}}{\tau}{\pi}$


\smallskip 
\noindent
{\bf 4. Inertia rules.} 
$R(x)$ contains the
following conjuncts for each value $\tau, \pi, \pi', \sigma$,
where $0 \leq \tau < N$, $0 \leq \pi < \pi' \leq N$, and
$\sigma \in \Sigma$, 


\smallskip 

\noindent
$\sym{\sigma}{\tau}{\pi}, \cursor{\tau}{\pi'} \ra 
\sym{\sigma}{\tau+1}{\pi}$

\noindent
$\sym{\sigma}{\tau}{\pi'}, \cursor{\tau}{\pi} \ra 
\sym{\sigma}{\tau+1}{\pi'}$


\smallskip 
\noindent



{\bf 5. Acceptance.} 
Let $s_m = \yess$. \\
Then  $R(x)$ contains the following atom as a conjunct:

\smallskip

\noindent
$\state{s_m}{N})$.



\newpage


\paragraph*{Solution}


The following external theorems are used.
\begin{theorem}[FMI, Woltran]
\label{thrm:fm-ndt-deciding}
A non-deterministic Turing machine $T$ decides a language $L$ iff for any $x \in \Sigma^*$, the following holds:
\begin{equation*}
x \in L \iff (s,\triangleright,x) \stackrel{T}{\lto}^* (\yess, w,u) 
\end{equation*}
for some strings $w$ and $u$. 
\end{theorem}


What follows are some notational conventions.

\begin{definition}
Consider a NTM $T:=\langle K, \Sigma, \Delta, s \rangle$. 
\begin{itemize}
\item For some $0 \leq n$ let $\Sigma^n$ be the set of string of length $n$ over the alphabet $\Sigma$.
\item For some $0 \leq n$ let $\Sigma^{\leq n}$ be $\bigcup_{i \in \{0 , \dots , n\}} \Sigma^i$.
\item Let $\Sigma^*$ be $\bigcup_{0 \leq i} \Sigma^i$.
\end{itemize}
\end{definition}


\begin{definition}
Consider a NTM $T:=\langle K, \Sigma, \Delta, s \rangle$. A sequence of configuration $C:= c_0, \dots, c_m$ for some $ 0 \leq m$ is called valid.
If for every $0\leq i < m$ there exists a $\delta	\in \Delta$ such that $c_i \to_{\delta} c_{i+1}$.
\end{definition}


\begin{definition}
Consider a NTM $T:=\langle K, \Sigma, \Delta, s \rangle$. Given a string $s \in \Sigma^*$ such that $s = s_0 \dots s_n$ for some $0\leq n$, then $s[i] = s_{i}$ if $0 \leq i \leq n$. Otherwise, $s[i]=\sqcup$.
\end{definition}


\begin{definition}
Consider a NTM $T:=\langle K, \Sigma, \Delta, s \rangle$. Given the strings $s,t \in \Sigma^*$ such that $s = s_0 \dots s_m$ and $t = t_0 \dots t_n$ for some $0\leq m,n$, then $st= s\cdot t = s_0 \dots s_m\, t_0 \dots t_n$.
\end{definition}





%
%Moreover, w.l.o.g. it can be assumed that all computations of $T$ require $N$ steps. 
%\begin{remark}
%This assumption is permitted, as one can easily use a construct a NTM $T$ satisfying this assumption, from a NTM $T'$ that does not satisfy this assumption. Roughly sketched, one can  MISSING
%\end{remark}

$"\Longrightarrow"$
\bigskip

To show that $x \in L \Ra R(x) \in \SAT$, it will be assumed that $x \in L$.  Thereby, essentially expressing that the NTM $T$ deciding $L$, terminates ins state "yes". Using this, it will be inferred, that there must exists at least one
 valid sequence of configurations terminating in an accepting state. The task at hand is to use this valid sequence of configurations, as well as the semantics of non-deterministic Turing machines, to construct an interpretation that satisfies the formula $R(x)$.
 
 
 
To be more precise, as $L \in \NP$ there must exist a NTM, let it be called $T:=\langle K, \Sigma, \Delta, s \rangle$, that decides $L$ for any input $x$ in a polynomial number of steps. Moreover, w.l.o.g. it can be assumed that all computations of $T$ require $N$ steps. 
 Assuming that $x\in L$, one can use Theorem \ref{thrm:fm-ndt-deciding} from the Formal Methods-lecture, to infer that there exists a valid sequence of configuration ranging from the initial configuration of input $x$ to a configuration with sate $\yess$. Thereby, allowing the following definition.
 
 \begin{definition}
 Let $L \in \NP$, let $x \in L$ and let $T$ be a NTM deciding $L$ such that all computations require $N$ steps. Let $\sequ{x}$ be a valid sequence of configuration of the form $\sequ{x}:=(c_i)_{i \in \{ 0,\dots ,N\}}=(s_0,\triangleright,x), \dots , (\yess, w,u)$
Moreover, as a notational convention for some $0\leq i \leq N$ let $\sequ{x}(i):=c_i$.
 \end{definition}
 
%  such that there exists a transition $\delta \in  \Delta$ such that $c_i \to_{\delta} c_{i+1}$.
% 
% if one considered for all $i \in  \{ 0,\dots ,N\}$,  $d_i=(p,a,a')=(p,a_1 \dots a_k,a_{k+1} \dots a_m) $ and $d_{i+1}=(q,b,b')=(p,b_1 \dots b_l, b_{l+1} \dots b_n) $  where $p \in K$, $q \in K \cup \{h, \mathit{"yes"}, \mathit{"no"}\}$, $a,b,a',b' \in \Sigma^*$ and $1 < l,k \leq m,n \leq N$, it must be the case that $( p, a_k, q ,b_i, d ) \in \Delta$. Hence, it must be the case that $d_{i+1}=(q,b,b')=(p,a_1 \dots b_{l+d}, b_{l+d+1} \dots b_n)$


 Using this, an interpretation can be constructed.  
 
 \begin{definition}
 Let $L \in \NP$, let $x \in L$ and let $T$ be a NTM deciding $L$ such that all computations require $N$ steps. For $0 \leq \tau \leq N$,  $0 \leq \pi \leq N$ and $\forall \sigma \in \Sigma$,
 \begin{itemize}
 \item let $\sequint{I}{x}(\sym{\sigma}{\tau}{\pi}) := 1$ if and only if for $\sequ{x}(\tau)=(p,a,b)$ it holds that $ab[\pi]=\sigma  $ (otherwise let $\sequint{I}{x}(\sym{\sigma}{\tau}{\pi})$ be $0$);
  \item let $\sequint{I}{x}(\cursor{\tau}{\pi}) := 1$ if and only if for $\sequ{x}(\tau)=(p,a,b)$ it holds that $|a|-1=\pi$ (otherwise let $\sequint{I}{x}(\cursor{\tau}{\pi}) $ be $0$);
   \item let $\sequint{I}{x}(\state{s}{\tau}) := 1$ if and only if for $\sequ{x}(\tau)=(p,a,b)$ it holds that $s=p$ (otherwise let $\sequint{I}{x}(\state{s}{\tau})$ be $0$).
 \end{itemize}
 
 \end{definition}
 
% For $0 \leq \tau \leq N$,  $0 \leq \pi \leq N$ and $\sigma \in \Sigma$,
% 
% \begin{itemize}
% \item let $\mathcal{I}(\sym{\sigma}{\tau}{\pi}) := 1$ if and only if for $c_{\tau}=(p,a,a') \in C$ it holds that $aa'[\pi]=\sigma  $ (otherwise let $\mathcal{I}(\sym{\sigma}{\tau}{\pi})$ be $0$);
%  \item let $\mathcal{I}(\cursor{\tau}{\pi}) := 1$ if and only if for $c_{\tau}=(p,a,a') \in C$ it holds that $|a|-1=\pi$ (otherwise let $\mathcal{I}(\cursor{\tau}{\pi}) $ be $0$);
%   \item let $\mathcal{I}(\state{s}{\tau}) := 1$ if and only if for $c_{\tau}=(p,a,a') \in C$ it holds that $s=p$ (otherwise let $\mathcal{I}(\state{s}{\tau})$ be $0$).
% \end{itemize}
 
As $\sequint{I}{x}$ assigns truth values to every atom in $R(x)$, an appropriate interpretation (i.e.\ truth assignment) was constructed. Hence, it remains to demonstrate that $\sequint{I}{x} \models R(x)$.
To that end, each block of conjuncts will be evaluated separately under $\sequint{I}{x}$. \\


Starting with the verification of \emph{Initialisation Facts} in $R(x)$.
\begin{proposition}
\label{prop:r-init}
Let $L \in \NP$, let $x \in L$ and let $T$ be a NTM deciding $L$ such that all computations require $N$ steps. Then all \emph{Initialisation Facts} in $R(x)$ are satisfied under $\sequint{I}{x}$.

\end{proposition}
\begin{proof}
 It is known that $\sequ{x}(0)=c_0$ has the form $(s_0,\triangleright,x)=(s_0,\triangleright,x_1 \dots x_n)$, where $\ssym x \in \Sigma^{\leq N}$. 
\begin{itemize}
\item By definition, $\triangleright \cdot x [0]=\triangleright$ and thus, by construction, $\sequint{I}{x} \models \sym{\ssym}{0}{0}$.
\item For all $1 \leq \pi \leq |x|$ where $x_{\pi}=\sigma$ it holds that  $\ssym \cdot x [\pi]=x_{\pi}$ which is precisely $\sigma$. Hence, by construction, $\sequint{I}{x} \models\sym{\sigma}{0}{\pi}$.
\item For all $|x| < \pi \leq N$ it holds, by definition, that  $\ssym \cdot x [\pi]=\blank$. Hence, by construction, $\sequint{I}{x} \models \sym{\blank}{0}{\pi}$.
\item By construction, $\sequint{I}{x} \models \cursor{0}{0}$  iff $|\ssym|-1=0$. Which is precisely the case.
\item By construction, $\sequint{I}{x} \models \state{s_0}{0}$ iff $s_0=s_0$. Which is precisely the case.
\end{itemize}
Thereby, verifying the claim.
\end{proof}

Subsequently, all \emph{Transition Rules} must be checked. 

\begin{proposition}
\label{prop:r-trans}
Let $L \in \NP$, let $x \in L$ and let $T$ be a NTM deciding $L$ such that all computations require $N$ steps. Then all \emph{Transition Rules} in $R(x)$ are satisfied under $\sequint{I}{x}$.
\end{proposition}
\begin{proof}
By definition, it is known that $R(x)$ contains the following.   For each pair $(s,\sigma)$ of state $s$ and symbol $\sigma$ let 
$(s,\sigma,s'_1,\sigma'_1,d_1)$, \dots, $(s,\sigma,s'_k,\sigma'_k,d_k)$ denote all possible transitions according to the  transition relation $\Delta$.
For each value of $\tau$ and $\pi$ such that $0\leq \tau<N$ and $0\leq \pi < N$
\begin{equation*}
\begin{split}
\state{s}{\tau}& \wedge \sym{\sigma}{\tau}{\pi} \wedge \cursor{\tau}{\pi} \to \\
& \Big(({\state{s'_1}{\tau+1}}\wedge \sym{\sigma'_1}{\tau+1}{\pi} \wedge {\cursor{\tau+1}{\pi+d_1}} ) \vee \dots \vee \\
& \; \; \,  ({\state{s'_k}{\tau+1}} \wedge \sym{\sigma'_k}{\tau+1}{\pi} \wedge {\cursor{\tau+1}{\pi+d_k}} ) \Big)
\end{split}
\end{equation*}

Firstly, the semantics of implications allows one to focus solely on the case $\sequint{I}{x} \models \state{s}{\tau} \wedge \sym{\sigma}{\tau}{\pi} \wedge \cursor{\tau}{\pi}$.
Secondly, as $\tau < N$ one can exclude the case $\tau=N$, thus ensuring that there must be another step of computation, i.e. the disjunction can not be empty.
Thirdly, take a $0 \leq \tau < N$ such that $\sequint{I}{x}$ satisfies its premiss of an $\tau$-transition rule, i.e. 
\begin{equation*}
\sequint{I}{x} \models \state{s}{\tau} \land \sym{\sigma}{\tau}{\pi} \land \cursor{\tau}{\pi}
\end{equation*}
for some $\sigma \in \Sigma$, some $0 \leq \pi <N$ and some $s \in K$.
Hence, by construction of $\sequint{I}{x} $, it must be that from
\begin{itemize}
\item $\sequint{I}{x} \models  \state{s}{\tau}$ it follows that $\sequ{x}(\tau)$ contains the state $s$;
\item  $\sequint{I}{x} \models \sym{\sigma}{\tau}{\pi}$ it follows that  $\sequ{x}(\tau)= (s,a,b)=(s,a_0\dots a_{\pi-1}\sigma, b)$ for $ab \in \Sigma^{\leq N}$ and $s \in K$ ;
\item  $\sequint{I}{x} \models \cursor{\tau}{\pi}$ it follows that $\sequ{x}(\tau)= (s,a,b)= (s,a_0\dots a_{\pi}, b)$ for $ab \in \Sigma^{\leq N}$ and $s \in K$;.
\end{itemize}
Hence, $\sequ{x}(\tau)=(s,a_0\dots a_{\pi-1}\sigma, b)$.
Moreover, consider the fact that this particular transition rule in $R(x)$ was constructed using the pair $(s,\sigma)$ and all valid transitions $(s,\sigma , s'_i,\sigma'_i, d_i) \in \Delta$ for some $1\leq i\leq k$ where $k$ is the number of all $\delta \in \Delta$ starting with the tuple $(s,\sigma)$.
That is, the form of the transition rule in question is 
\begin{equation*}
\begin{split}
\state{s}{\tau}& \wedge \sym{\sigma}{\tau}{\pi} \wedge \cursor{\tau}{\pi} \to  \\
&\bigvee_{i \in \{1,\dots ,k\}} ({\state{s'_i}{\tau+1}} \wedge \sym{\sigma'_i}{\tau+1}{\pi} \wedge {\cursor{\tau+1}{\pi+d_i}} ) 
\end{split}
\end{equation*}
However, consider that $\sequ{x}(\tau)$ has a successor, i.e. $\sequ{x}(\tau+1)$. This implies further that there must exist an $\delta \in \Delta$ such that $\sequ{x}(\tau) \to_{\delta} \sequ{x}(\tau+1)$.
Due to the above observation it must be that this particular $\delta$ is of the form $(s,\sigma, s', \sigma',d')$ implying that there must exist $1\leq i\leq k$ such that $s'_i=s'$, $\sigma'_i=\sigma'$ and $d'_i=d'$.
Moreover, it also follows that $\sequ{x}(\tau+1)=(s', a',b')$ where $|a'|=\pi +d$ and $a'b'[\pi]=\sigma$. However, by construction of $\sequint{I}{x}$ it thereby follows that 
$\sequint{I}{x} \models {\state{s'}{\tau+1}} \wedge \sym{\sigma'}{\tau+1}{\pi} \wedge {\cursor{\tau+1}{\pi+d'}}$, which as already established is one part of the disjunction of the consequent of the transition rule in question.
Hence, resulting in 
\begin{equation*}
\begin{split}
\sequint{I}{x} \models \state{s}{\tau}& \wedge \sym{\sigma}{\tau}{\pi} \wedge \cursor{\tau}{\pi} \to  \\
&\bigvee_{i \in \{1,\dots ,k\}} ({\state{s'_i}{\tau+1}} \wedge \sym{\sigma'_i}{\tau+1}{\pi} \wedge {\cursor{\tau+1}{\pi+d_i}} ) 
\end{split}
\end{equation*}
As this works for arbitrary $0 \leq \tau <N$ it was demonstrated that all transition relation are satisfied.
%all transition relation are satisfied.
%By construction, there exists only a single configuration at position $\tau$.  Hence, for this $\tau$
%\begin{itemize}
%\item there exists exactly one $\state{s}{\tau}$ in $R(x)$ such that $\sequint{I}{x} \models  \state{s}{\tau}$, namely the one where $s$ is the sate of $\sequ{x}(\tau)$;
%\item there exists exactly one $\sym{\sigma}{\tau}{\pi}$ in $R(x)$ such that $\sequint{I}{x} \models \sym{\sigma}{\tau}{\pi}$, namely exactly if $\sequ{x}(\tau)= (s,a_0\dots a_{\pi-1}\sigma, b)$;
%\item there exists exactly one $\cursor{\tau}{\pi}$ in $R(x)$ such that $\sequint{I}{x} \models \cursor{\tau}{\pi}$, namely exactly if $\sequ{x}(\tau)= (s,a_0\dots a_{\pi}, b)$.
%\end{itemize}
%Hence, for this $\tau$ there exists only a single transition rule where all premisses are satisfied. Or more concisely the conjunction 
%$\state{s}{\tau} \land \sym{\sigma}{\tau}{\pi} \land \cursor{\tau}{\pi}$ is by construction of $\sequint{I}{x}$ only satisfied if and only if the configuration $\sequ{x}(\tau)$ has the form $(s,a_0\dots a_{\pi-1}\sigma, b)$.
%What remains to show is to demonstrate that $\sequint{I}{x}$ satisfies at least one of the disjuncts. 
%By construction, all transitions $\delta \in \Delta $ containing $s$ and $\sigma$ as first and second element are used to construct this rule. That is, let $k$ be the number of rules containing $s$ and $\sigma$ as first and second element, then the rule in question has the form
%\begin{equation*}
%\begin{split}
%\state{s}{\tau}& \wedge \sym{\sigma}{\tau}{\pi} \wedge \cursor{\tau}{\pi} \to  \\
%&\bigvee_{i \in \{1,\dots ,k\}} ({\state{s'_i}{\tau+1}} \wedge \sym{\sigma'_i}{\tau+1}{\pi} \wedge {\cursor{\tau+1}{\pi+d_i}} ) 
%\end{split}
%\end{equation*}
%Now considering that there exists a $\delta=(s,\sigma,s',\sigma',d) \in \Delta$ connecting $\sequ{x}(\tau)$ with$\sequ{x}(\tau+1)=(s', a',b')$ with $|a'|=\pi+d$ and $a'b'[\pi]=\sigma$, this rule must also be present in the transition rule in question.
%Hence, by construction of $\sequint{I}{x}$ there exists one $i \in \{1,\dots , k\}$ such that $\sequint{I}{x} \models {\state{s'_i}{\tau+1}} \wedge \sym{\sigma'_i}{\tau+1}{\pi} \wedge {\cursor{\tau+1}{\pi+d_i}} $. As this works for arbitrary $0 \leq \tau <N$ it was demonstrated that
%all transition relation are satisfied.
\end{proof}

Moving on towards the \emph{Uniqueness Constraints}.

%
%\noindent
%{\bf 2. } For each pair $(s,\sigma)$ of state $s$ and symbol $\sigma$ let $\tuple{s,\sigma,s'_1,\sigma'_1,d_1}$, \dots, 
%$\tuple{s,\sigma,s'_k,\sigma'_k,d_k}$ denote all possible transitions according
%to the  transition relation $\Delta$ (for the cursor movements, we write $d_i \in \{-1,0,1\}$ rather than $d_i \in \{\lmove,\stay,\rmove\}$).
% \\
%Then $R(x)$ contains the 
%following conjuncts for each value of $\tau$ and $\pi$ such that $0\leq \tau<N$ and $0
%\leq \pi < N$
%
%\smallskip
%\noindent
%%
%$\state{s}{\tau} \wedge \sym{\sigma}{\tau}{\pi} \wedge \cursor{\tau}{\pi}$
%$\ra$ \\ 
%\mbox{}\ \  $[({\state{s'_1}{\tau+1}}\wedge 
%\sym{\sigma'_1}{\tau+1}{\pi} \wedge {\cursor{\tau+1}{\pi+d_1}} 
%) \vee \dots \vee $ \\
%\mbox{}\ \  $({\state{s'_k}{\tau+1}} \wedge 
%\sym{\sigma'_k}{\tau+1}{\pi} \wedge {\cursor{\tau+1}{\pi+d_k}} 
%) ] $ 

\begin{proposition}
\label{prop:r-uniqueness}
Let $L \in \NP$, let $x \in L$ and let $T$ be a NTM deciding $L$ such that all computations require $N$ steps. Then all \emph{Uniqueness Constraints} in $R(x)$ are satisfied under $\sequint{I}{x}$.
\end{proposition}
\begin{proof}
%Consider the values $\tau$ and $\pi$ such that $0\leq \tau \leq N$, $0 \leq \pi \leq  N$, 
%$0 \leq i \leq m$, and $1 \leq j \leq n$.
Even though the proof is analogues for all three kinds of constraints, all shall be demonstrated in detail. 
\begin{itemize}
\item
Consider an arbitrary $0\leq \tau \leq N$ and an arbitrary $0 \leq i \leq m$. Let $\varphi:=\state{s_i}{\tau} \leftrightarrow (\neg \state{s_0}{\tau} \wedge \dots \wedge \neg \state{s_{i-1}}{\tau} \wedge  \neg \state{s_{i+1}}{\tau} \wedge \dots \wedge \neg \state{s_{m}}{\tau})$.
Observe that there exists only a single entry at position $\tau$ in the accepting sequence of configurations $\sequ{x}$.
Hence, as there exist only one state $s \in K$ that is contained in $\sequ{x}(\tau)$, if follows by construction of $\sequint{I}{x}$ that there can be only one state-atom at step $\tau$ that is satisfied under $\sequint{I}{x}$, i.e. $\sequint{I}{x} \models \state{s}{\tau} $. Therefore, on the one hand, if $s_i=s$ it follows that $\sequint{I}{x} \models  \state{s_i}{\tau}$ and that for all $0 \leq j \leq m$ such that $j \neq i$ $\sequint{I}{x} \nmodels  state{s_j}{\tau}$. Thereby, inducing the conclusion $\sequint{I}{x} \models  \varphi$. On the other hand, if $s_i \neq s$ it must be that
$\sequint{I}{x} \nmodels \state{s_i}{\tau}$. Moreover, this requires that there exists a $j$ such that $j\neq i$ and  $0 \leq j \leq m$ where $s_j = s$, forcing by construction of the interpretation $\sequint{I}{x} \models \state{s_i}{\tau}$ and further leading to $\sequint{I}{x} \models \varphi$ . 

\item 
Consider an arbitrary $0\leq \tau \leq N$ and $0 \leq \pi \leq  N$.
Let $\varphi := \cursor{\tau}{\pi} \leftrightarrow (\neg \cursor{\tau}{0} \wedge \dots \wedge \neg \cursor{\tau}{\pi - 1}  \wedge \neg \cursor{\tau}{\pi+1} \wedge \dots \wedge \neg \cursor{\tau}{N})$.
Again, there exists only a single entry at position $\tau$ in the sequence $\sequ{x}$, namely $\sequ{x}(\tau)=(s,a,a')$. As $\sequint{I}{x} \models   \cursor{\tau}{\pi} $ if and only if $|a|-1=\pi$, there can only be a single $\tau$-cursor-atom satisfied under $\sequint{I}{x}$.
Hence, if $\pi=|a|-1$ then $\sequint{I}{x} \models   \cursor{\tau}{\pi}$ and for all other $0 \leq \mu \leq  N$ s.t. $\mu \neq \pi$ it must be $\sequint{I}{x} \nmodels  \cursor{\tau}{\mu}$. Therefore, from the usual semantics if follows that $\sequint{I}{x} \models  \varphi$.
Alternatively, if $\pi\neq|a|-1$ then $\sequint{I}{x} \nmodels  \cursor{\tau}{\pi}$. But, this implies that there exists the atom $ \cursor{\tau}{|a|-1}$ in the big conjunct, which by construction clearly evaluates to $1$.  Therefore, from the usual semantics if follows that $\sequint{I}{x} \models  \varphi$.

\item Consider an arbitrary $0\leq \tau \leq N$ and $0 \leq \pi \leq  N$, as well as an arbitrary $0 \leq j \leq n$.
Let $\varphi := \sym{\sigma_j}{\tau}{\pi} \leftrightarrow (\neg \sym{\sigma_1}{\tau}{\pi} \wedge \dots \wedge \neg \sym{\sigma_{j-1}}{\tau}{\pi} \wedge \neg \sym{\sigma_{j+1}}{\tau}{\pi} \wedge \dots \wedge \neg \sym{\sigma_{n}}{\tau}{\pi})$.
Similarly as before. $\sequint{I}{x} \models\sym{\sigma_j}{\tau}{\pi}$ if and only if  for $\sequ{x}(\tau)=(p,a,a')$ it holds that $aa'[\pi]=\sigma_j$. Assume that this is actually the case, i.e. that $\sequ{x}(\tau)=(p,a,a')$ such that $aa'[\pi]=\sigma_i$, thus by construction  $\sequint{I}{x} \models  \sym{\sigma_j}{\tau}{\pi}$. Hence, due to $aa'[\pi]=\sigma_j$, any atom of the form $\sym{\sigma_i}{\tau}{\pi}$ for $i\neq j$ and $0 \leq i \leq n$ can not be satisfied. Thereby, $\sequint{I}{x} \models  \varphi$ follows from the usual semantics.
If this is not the case, i.e. $aa'[\pi]\neq\sigma_j$ then  $\sequint{I}{x} \nmodels  \sym{\sigma_j}{\tau}{\pi}$. It is known that $\varphi$ must contain the atom $sym{\sigma_i}{\tau}{\pi}$ in its big conjunct, which  will clearly be satisfied by $\mathcal{I}$. 
Thereby, $\sequint{I}{x} \models  \varphi $ follows from the usual semantics.
\end{itemize}
\end{proof}

Furthermore, the same ought to be done of the \emph{Inertia Rules}.
\begin{proposition}
\label{prop:r-inertia}
Let $L \in \NP$, let $x \in L$ and let $T$ be a NTM deciding $L$ such that all computations require $N$ steps. Then all  \emph{Inertia Rules} in $R(x)$ are satisfied under $\sequint{I}{x}$.
\end{proposition}
\begin{proof}
For each value $\tau, \pi, \pi', \sigma$,
where $0 \leq \tau < N$, $0 \leq \pi < \pi' \leq N$, and $\sigma \in \Sigma$, one has
\begin{equation*}
 \sym{\sigma}{\tau}{\pi} \land \cursor{\tau}{\pi'} \ra 
\sym{\sigma}{\tau+1}{\pi}
\end{equation*}
and 
\begin{equation*}
 \sym{\sigma}{\tau}{\pi'} \land \cursor{\tau}{\pi} \ra 
\sym{\sigma}{\tau+1}{\pi}
\end{equation*}
%To demonstrate the prior, consider an arbitrary $0\leq \tau < N$. Take $\sequ{x}(\tau)=(s, a, b)$ clearly by construction of $\sequint{I}{x}$, $\sequint{I}{x} \models  \cursor{\tau}{\pi'}$. Now, take some arbitrary $0 \leq \pi <\pi'$. $\sequ{x}(\tau)$ must have some character $\sigma \in \Sigma$ at position $\pi$ in $a$, i.e. $(\sequ{x}(\tau)=s, a_0 \dots a_{\pi-1} \sigma a_{\pi+1} \dots a_{\pi'-1} \gamma, b)$. Therefore, by construction one obtains 
%$\sequint{I}{x} \models  \sym{\sigma}{\tau}{\pi}$. This implies that the premiss of the inertia rule
%\begin{equation*}
% \sym{\sigma}{\tau}{\pi} \land \cursor{\tau}{\pi'} \ra \sym{\sigma}{\tau+1}{\pi}
%\end{equation*}
%is satisfied. Especially, due to the fact that for this particular $\tau, \pi$ and $\pi'$ this is the only inertia rule with satisfied indices. That is, a Turing machine can not contain two characters in a single cell. It remains to show that the  

Assume an arbitrary $\tau, \pi, \pi', \sigma$ such that $0 \leq \tau < N$, $0 \leq \pi < \pi' \leq N$, and $\sigma \in \Sigma$.
As the implication will always be satisfied, if the premise is false, assume that $\sequint{I}{x} \models   \sym{\sigma}{\tau}{\pi} $ and $\sequint{I}{x} \models   \cursor{\tau}{\pi'}$. However, this can only be the case
if $\sequ{x}(\tau)=(s, a, b) =(s, a_0 \dots a_{\pi-1} \sigma a_{\pi+1} \dots a_{\pi'-1} \gamma, b)$ for some state $s$ and $ab\in \Sigma^{\leq N} $. Moreover, as $\sequ{x}(\tau+1)= (s', a', b')$ is obtained by a transition $\delta=(s,\gamma,s',\gamma',d) \in \Delta$, it must be that $a'b'[\pi]=\sigma$ as the value of a cell can only change, if the head is positioned at this cell. However, $a'b'[\pi]=\sigma$ implies that $\sequint{I}{x} \models \sym{\sigma}{\tau+1}{\pi}$. Thus demonstrating the prior. 
The latter can be done in analogue. That is, 
 assume that $\sequint{I}{x} \models   \sym{\sigma}{\tau}{\pi'} $ and $\sequint{I}{x} \models   \cursor{\tau}{\pi}$. However, this can only be the case
if $\sequ{x}(\tau)=(s, a, b) =(s, a1 \dots a_{pi-1}\gamma, b_{pi+1} \dots b_{\pi'-1} \sigma b_{\pi'+1} b_n)$ for some state $s$, $ab \in \Sigma^{\leq N} $ and some $\pi'+1 \leq n \leq N$. Moreover, as $\sequ{x}(\tau+1)= (s', a', b')$ is obtained by a transition $\delta=(s,\gamma,s',\gamma',d) \in \Delta$, it must be that $a'b'[\pi']=\sigma$. However, since $a'b'[\pi']=\sigma$ implies that $\sequint{I}{x} \models \sym{\sigma}{\tau+1}{\pi'}$, the latter is demonstrated. 
Since, this is done for all inertia rules with their premise satisfied. It can be concluded that all inertia rules are satisfied under $\sequint{I}{x} $.
\end{proof}

The last remaining part, is to check that \emph{Acceptance} holds.

\begin{proposition}
\label{prop:r-accept}
Let $L \in \NP$, let $x \in L$ and let $T$ be a NTM deciding $L$ such that all computations require $N$ steps. Then \emph{Acceptance} in $R(x)$ is satisfied under $\sequint{I}{x}$.
\end{proposition}
\begin{proof}
$R(x)$ contains $\state{s_m}{N})$, since the last (and the $N^{th}$) element in $\sequ{x}$ is of the form $\sequ{x}(N)=(\yess, w,u)$, it follows from the construction of $\sequint{I}{x}$ that the atom $\state{s_m}{N}$ is satisfied under $\sequint{I}{x}$.
\end{proof}

Finally, allowing for the corollary
\begin{proposition}
Let $L \in \NP$, let $x \in L$ and let $T$ be a NTM deciding $L$ such that all computations require $N$ steps. Then $R(x)$ is satisfied under $\sequint{I}{x}$.
\end{proposition}
\begin{proof}
It follows directly from the Propositions \ref{prop:r-init}, \ref{prop:r-trans}, \ref{prop:r-uniqueness}, \ref{prop:r-inertia} and \ref{prop:r-accept}.
\end{proof}

\smallskip

Hence, it was demonstrated that, if $x \in L$ then we can construct an interpretation $\mathcal{I}$ that satisfies $R(x)$ and therefore one can conclude that $R(x) \in \SAT$.
\bigskip

$"\Longleftarrow"$

\bigskip


To show that $R(x) \in \SAT \Ra x \in L $, it will be assumed that $R(x) \in \SAT $.  Hence, there must exists an interpretation of $R(x)$. Using this interpretation, a valid sequence of configurations for the NTM $T$ that decides $L$ will be constructed, such that 
its first element will reflect the start configuration of $T$ on input $x$ and the last element will terminate in an accepting state.
   
 
Firstly, some auxiliary results ought to be established.



\begin{proposition}
\label{prop:unique}
Let $L\in \NP$ decided by some NTM $T$ where each computation requires exactly $N$ steps. Moreover, for the string $x \in \Sigma^{\leq N}$ such that $R(x)$  being the corresponding propositional formula as defined above is satisfiable.
Then it must be that for all interpretations $\mathcal{I} \models R(x)$ the uniqueness constraints ensure uniqueness.
\end{proposition}
\begin{proof}
As the poof is analogue for all types, i.e. state, cursor and symbol, of uniqueness constraints. The claim will only be demonstrated for state.
Take an arbitrary state predicate $\state{s_i}{\tau}$. If $\mathcal{I}\models \state{s_i}{\tau}$ assume there exists another $\state{s_j}{\tau}$ with $s_i \neq s_j$ such that $\mathcal{I} \models \state{s_j}{\tau}$. However, this would imply that 
\begin{equation*}
\begin{split}
\mathcal{I} \nmodels (\neg \state{s_0}{\tau} \wedge \dots \wedge \neg \state{s_{i-1}}{\tau} \wedge  \neg \state{s_{i+1}}{\tau} \wedge \dots \wedge \neg \state{s_{m}}{\tau})
\end{split}
\end{equation*}
and thus
\begin{equation*}
\begin{split}
\mathcal{I} \nmodels\state{s_i}{\tau}  \liff  (\neg \state{s_0}{\tau} \wedge \dots \wedge \neg \state{s_{i-1}}{\tau} \wedge  \neg \state{s_{i+1}}{\tau} \wedge \dots \wedge \neg \state{s_{m}}{\tau})
\end{split}
\end{equation*}
which is clearly a contradiction.
On the other hand, if $\mathcal{I}\nmodels \state{s_i}{\tau}$ then using the knowledge that $\mathcal{I} \models R(x)$, there must be a predicate $\state{s_j}{\tau}$ with $s_i \neq s_j$ such that $\mathcal{I} \models \state{s_j}{\tau}$. As otherwise,
\begin{equation*}
\begin{split}
\mathcal{I} \models (\neg \state{s_0}{\tau} \wedge \dots \wedge \neg \state{s_{i-1}}{\tau} \wedge  \neg \state{s_{i+1}}{\tau} \wedge \dots \wedge \neg \state{s_{m}}{\tau})
\end{split}
\end{equation*}
which is an impossibility.
\end{proof}


As a corollary one obtains the following

\begin{corollary}
\label{cor:unique-trans}
Let $L$ language decided by some NTM $T$ where each computation requires exactly $N$ steps. Moreover, for the string $x \in \Sigma^{\leq N}$ such that $R(x)$  being the corresponding propositional formula as defined above is satisfiable.
Then it must be that for all interpretations $\mathcal{I} \models R(x)$ and an arbitrary transition rule 
\begin{equation*}
\begin{split}
\state{s}{\tau}& \wedge \sym{\sigma}{\tau}{\pi} \wedge \cursor{\tau}{\pi} \to \\
& \Big(({\state{s'_1}{\tau+1}}\wedge \sym{\sigma'_1}{\tau+1}{\pi} \wedge {\cursor{\tau+1}{\pi+d_1}} ) \vee \dots \vee \\
& \; \; \,  ({\state{s'_k}{\tau+1}} \wedge \sym{\sigma'_k}{\tau+1}{\pi} \wedge {\cursor{\tau+1}{\pi+d_k}} ) \Big)
\end{split}
\end{equation*}
if $\mathcal{I} \models \state{s}{\tau} \wedge \sym{\sigma}{\tau}{\pi} \wedge \cursor{\tau}{\pi}$, then there exists exactly one $1 \leq i \leq k$ such that 
\begin{equation*}
\begin{split}
\mathcal{I} \models ({\state{s'_i}{\tau+1}}\wedge \sym{\sigma'_i}{\tau+1}{\pi} \wedge {\cursor{\tau+1}{\pi+d_i}} )
\end{split}
\end{equation*}
That is, exactly a single disjunct is satisfied.
\end{corollary}
\begin{proof}
By assumption it is known that $\mathcal{I} \models R(x)$,  as well as $\mathcal{I} \models \state{s}{\tau} \wedge \sym{\sigma}{\tau}{\pi} \wedge \cursor{\tau}{\pi}$. In particular this means that at least for one $1 \leq i \leq k$  
\begin{equation*}
\begin{split}
\mathcal{I} \models ({\state{s'_i}{\tau+1}}\wedge \sym{\sigma'_i}{\tau+1}{\pi} \wedge {\cursor{\tau+1}{\pi+d_i}} )
\end{split}
\end{equation*}
Now assume that there exists a  $1 \leq j \leq k$  with $i\neq j$ such that 
\begin{equation*}
\begin{split}
\mathcal{I} \models ({\state{s'_j}{\tau+1}}\wedge \sym{\sigma'_j}{\tau+1}{\pi} \wedge {\cursor{\tau+1}{\pi+d_j}} )
\end{split}
\end{equation*}
Notice that by construction of $R(x)$,  $i\neq j$ requires that  $s'_i \neq s'_j $ or $ \sigma'_i \neq \sigma'_j $  or $ d_i\neq d'_j $. 
Given the assumption it must be that $\mathcal{I} \models \state{s'_i}{\tau+1} \land  \state{s'_j}{\tau+1} $,  $\mathcal{I} \models  \sym{\sigma'_i}{\tau+1}{\pi} \land \sym{\sigma'_j}{\tau+1}{\pi} $  and 
$\mathcal{I} \models \cursor{\tau+1}{\pi+d_i} \land \cursor{\tau+1}{\pi+d_j}$. All of which would contradict Proposition \ref{prop:unique}. 
Thereby, $j$ can not exist thus demonstrating the validity of the 'only'-part of the claim.
%
%However,
%this implies that
%\begin{equation*}
%\begin{split}
%\mathcal{I} \nmodels\state{s'_i}{\tau} \leftrightarrow (\neg \state{s'_0}{\tau} \wedge \dots \wedge \neg \state{s'_{i-1}}{\tau} \wedge  \neg \state{s'_{i+1}}{\tau} \wedge \dots \wedge \neg \state{s'_{m}}{\tau})
%\end{split}
%\end{equation*}
%which considering the fact that $i\neq j$ contradict $\mathcal{I} \models R(x)$. The same can be done in analogue for $\sym{\sigma'_i}{\tau+1}{\pi}$ and $\sym{\sigma'_j}{\tau+1}{\pi}$, as well as for $\cursor{\tau+1}{\pi+d_i}$ and $\cursor{\tau+1}{\pi+d_ij}$.
\end{proof}

Moreover, an additional support is required.
\begin{proposition}
\label{prop:allways-trans}
Let $L$ language decided by some NTM $T$ where each computation requires exactly $N$ steps. Moreover, for the string $x \in \Sigma^{\leq N}$ such that $R(x)$  being the corresponding propositional formula as defined above is satisfiable.
Then it must be that for all interpretations $\mathcal{I} \models R(x)$ and for every $\tau$ with $0 \leq \tau < N$, there exists a transition rule where both its premise and its consequent are satisfied by $\mathcal{I}$.
\end{proposition}
\begin{proof}
Assume there exists a $0 \leq \tau< N$ where this is not the case. By construction, $R(x)$ contains a transition rule for every $s \in K$, for every $\sigma \in \Sigma$ and for every $0 \leq \pi \leq N$. Hence, there must exists premises of the form
\begin{equation*}
\state{s}{\tau} \wedge \sym{\sigma}{\tau}{\pi} \wedge \cursor{\tau}{\pi}
\end{equation*}
for the fixed $\tau$. However, the uniqueness constraints force that for every $\tau$ there is exactly one stata, one symbol and one cursor predicate satisfied. W.l.o.g assume that they are the ones stated above.
Hence, the only remaining possibility for the assumption to hold is that there does not exist a transition from this particular state 
in the set of all $\Delta$. However, by construction this would imply that the particular rule in question has an empty disjunction as consequent, i.e.
\begin{equation*}
\state{s}{\tau} \wedge \sym{\sigma}{\tau}{\pi} \wedge \cursor{\tau}{\pi} \to \bigvee_{i \in \emptyset}	\state{s'_i}{\tau+1} \wedge \sym{\sigma'_i}{\tau+1}{\pi} \wedge \cursor{\tau+1}{\pi+d_i}
\end{equation*}
Now with an empty disjunction being always false, the interpretation in question can not satisfy this particular rule. Thereby, causing the desired contradiction. 
\end{proof}


%
%Allowing further to conclude.
%
%\begin{corollary}
%\label{cor:unique-trans}
%Let $L$ language decided by some NTM $T$. Moreover, for $x \in L$ let $R(x)$  be the corresponding propositional formula as defined above.
%If $R(x)$ is satisfiable, then it must be that for all interpretations $\mathcal{I} \models R(x)$ there exists exactly one sequence of transitions
%\end{corollary}


Secondly, a sequence of configuration ought to be established from an interpretation $\mathcal{I}$ that satisfies $R(x)$. 
%However, it is only possible to talk about a sequence, because of Proposition \ref{prop:unique} it is known that there exists precisely one transition rule with its premise satisfied and due Corollary \ref{cor:unique-trans} is is known that precisely one of its disjuncts is satisfied.
%Thereby, creating a single sequence of transition rules. 
Using the knowledge obtained from Proposition \ref{prop:unique}, the following operator is defined.



\begin{definition}
\label{def:trans}
Let $L$ language decided by some NTM $T$ where each computation requires exactly $N$ steps. Moreover, for the string $x \in \Sigma^{\leq N}$ such that $R(x)$  being the corresponding propositional formula as defined above is satisfiable.  For any $\mathcal{I}$ such that $\mathcal{I} \models R(x)$, let $\trans{I}$ be defined as follows. For any $0 \leq \tau \leq N$
\begin{equation*}
\trans{I}(\tau):=(s,a,b)
\end{equation*}
Let $\pi$ being from the only\footnote{As ensured by Proposition \ref{prop:unique}} $\cursor{\tau}{\pi}$ such that $\mathcal{I}\models \cursor{\tau}{\pi}$, 
\begin{itemize}
\item $s$ is from the only$^1$ $\state{s}{\tau}$ such that $\mathcal{I}\models \state{s}{\tau}$;
\item $a = a_1 \dots a_{\pi}$ with $a_i=\sigma$ being from the only$^1$ $\sym{\sigma}{\tau}{i}$ such that $\mathcal{I}\models \sym{\sigma}{\tau}{i}$; 
\item $b = b_{\pi+1} \dots b_{N}$ with $b_i=\sigma$ being from the only$^1$  $\sym{\sigma}{\tau}{i}$ such that $\mathcal{I}\models \sym{\sigma}{\tau}{i}$.
\end{itemize}

%\begin{itemize}
%\item $\trans{I}(0):=(s_0, \ssym, x)$ where 
%\begin{itemize}
%\item $s_0$ is from $\state{s_0}{0}$;
%\item $\ssym$ is from $\sym{\ssym}{0}{0}$;
%\item for some $0 \leq n \leq N$ $x= x_1 \dots x_n$ such that for some $1  \leq \pi \leq n$, $x_{\pi}$ is some $\sigma$ taken from $\sym{\sigma}{0}{\pi}$.
%\end{itemize}
%\item For $\tau +1 \leq N $, $\trans{I}(\tau+1):= (s', a', b')$, where $(s', a', b')$ is constructed as follows.
%Let $\trans{I}(\tau):= (s, a, b)$. Moreover, from Proposition \ref{prop:unique} it is known that there exist exactly one consequence of a transition function that is satisfied by $\mathcal{I}$. 
%That is, $\mathcal{I} \models  \state{s''}{\tau} \wedge \sym{\sigma''}{\tau}{\pi} \wedge \cursor{\tau}{\pi}$. From this, let it be that 
%\begin{itemize}
%\item $s' := s''$;
%\item $a'= a'_1 \dots a'_{\pi}$ with $a'_i:=ab[i]$ for all $i \in \{1, \dots \pi-1\}$ and with $a_{\pi}:=\sigma''$;
%\item $b'= b'_{\pi+1} \dots b'_{|ab|}$ with $a'_i:=ab[i]$ for all $i \in \{\pi+1, \dots |ab|\}$.
%\end{itemize}
%\item For $\tau+1 \geq N$, $\trans{I}(\tau+1)=\trans{I}(\tau)$.
%\end{itemize}
%Take the one transition relation 
%\begin{equation*}
%\begin{split}
% \state{s}{\tau}& \wedge \sym{\sigma}{\tau}{\pi} \wedge \cursor{\tau}{\pi} \to \\
%& \Big(({\state{s'_1}{\tau+1}}\wedge \sym{\sigma'_1}{\tau+1}{\pi} \wedge {\cursor{\tau+1}{\pi+d_1}} ) \vee \dots \vee \\
%& \; \; \,  ({\state{s'_k}{\tau+1}} \wedge \sym{\sigma'_k}{\tau+1}{\pi} \wedge {\cursor{\tau+1}{\pi+d_k}} ) \Big)
%\end{split}
%\end{equation*}
%where $\mathcal{I} \models  \state{s}{\tau} \wedge \sym{\sigma}{\tau}{\pi} \wedge \cursor{\tau}{\pi} $ (as ensured by Proposition \ref{prop:unique}). We know by Corollary \ref{cor:unique-trans} at exactly one the disjuncts in the consequent is satisfied. W.l.o.g assume it is the $j^{\text{th}}$, i.e.
% $\mathcal{I} \models {\state{s'_j}{\tau+1}} \wedge \sym{\sigma'_j}{\tau+1}{\pi} \wedge \cursor{\tau+1}{\pi+d_j}$. Now consider  $\trans{I}(\pi):= (s'', a'', b'')=(s'', a_1'' \dots a_m, b_1'' \dots b_m'')$. Using this one obtains
% \begin{itemize}
% \item  if $d_j=0$ $s', a', b')$
% \end{itemize}
\end{definition}


 
Moreover, for convenience sake let consider the following definition.

 
 
\begin{definition}
Let $L$ language decided by some NTM $T$ where each computation requires exactly $N$ steps. Moreover, for the string $x \in \Sigma^{\leq N}$ such that $R(x)$  being the corresponding propositional formula as defined above is satisfiable.
Let $r$ be a transition rule in $R(x)$ of the form 
\begin{equation*}
\begin{split}
\state{s}{\tau}& \wedge \sym{\sigma}{\tau}{\pi} \wedge \cursor{\tau}{\pi} \to \\
& \Big(({\state{s'_1}{\tau+1}}\wedge \sym{\sigma'_1}{\tau+1}{\pi} \wedge {\cursor{\tau+1}{\pi+d_1}} ) \vee \dots \vee \\
& \; \; \,  ({\state{s'_k}{\tau+1}} \wedge \sym{\sigma'_k}{\tau+1}{\pi} \wedge {\cursor{\tau+1}{\pi+d_k}} ) \Big)
\end{split}
\end{equation*}
For any $\mathcal{I}$ such that $\mathcal{I} \models R(x)$, let $\delta_{\mathcal{I}}(r)$ be defined as follows.
If $\mathcal{I} \models \state{s}{\tau} \wedge \sym{\sigma}{\tau}{\pi} \wedge \cursor{\tau}{\pi}$ and if some $1 \leq i \leq k$
\begin{equation*}
\begin{split}
 \mathcal{I} \models \state{s'_i}{\tau+1} \wedge \sym{\sigma'_i}{\tau+1}{\pi} \wedge \cursor{\tau+1}{\pi+d_i}
\end{split}
\end{equation*}
then $\delta_{\mathcal{I}}(r):= (s,\sigma, s'_i,\sigma'_i, d_i)$ (recall that by Corollary \ref{cor:unique-trans} only a single such $i$ can exist).
In all other cases, $\delta_{\mathcal{I}}(r)$ is undefined.
\end{definition}

Clearly, if $\mathcal{I}$ satisfies the premise and the consequent of a transition rule $r$ the $\delta_{\mathcal{I}}(r) \in \Delta$ of the respective NTM

\bigskip

What remains to be shown is that $\trans{I}(0)\dots \trans{I}(N)$ is a valid sequence of configurations for the NTM $T$ given input $x$ on the basis of which $R(x)$ was constructed.

\begin{proposition}
\label{prop:valid-config}
Let $L$ language decided by some NTM $T$ where each computation requires exactly $N$ steps. Moreover, for the string $x \in \Sigma^{\leq N}$ such that $R(x)$  being the corresponding propositional formula as defined above is satisfiable.
Then the sequence $\trans{I}(0)\dots \trans{I}(N)$ is a valid sequence of configurations for $T$ on input $x$.
\end{proposition}
\begin{proof}
This claim shall be confirmed by means of induction.
\begin{itemize}
\item \textbf{IH:} For $ i \leq N$ the sequence$\trans{I}(0)\dots \trans{I}(i)$ is a valid sequence of configurations for $T$ on input $x$.

\item \textbf{IB:} $i=0$. In this case, $\trans{I}(i)=\trans{I}(0)=(s_0, \ssym, x)$ as it is characterised in the initialisation facts. That is,
\begin{itemize}
\item $s_0$ is from $\state{s_0}{0}$;
\item $\ssym$ is from $\sym{\ssym}{0}{0}$;
\item for some $0 \leq n \leq N$ $x= x_1 \dots x_n$ such that for some $1  \leq \pi \leq n$, $x_{\pi}$ is some $\sigma$ taken from $\sym{\sigma}{0}{\pi}$.
\end{itemize}
However, this is precisely the start configuration for $T$ on input $x$.

\item \textbf{IS:} $i=\tau+1$. Let $\trans{I}(\tau+1)=(s',a',b')$. By \textbf{IH} $\trans{I}(\tau)=(s,a,b)$ was obtained through a valid sequence of configuration. Hence, it remains to show that there exists a $\delta \in \Delta$ such that 
$(s,a,b) \to_{\delta}(s',a',b')$. Firstly, from Proposition \ref{prop:unique} \& \ref{prop:allways-trans} it is known that there exists exactly one transition rule $r$, where both its premises and its conclusions are satisfied by $\mathcal{I}$. Due to 
the Proposition \ref{prop:unique} and due the definition of $\trans{I}$,  it is ensured that the premises in this rule correspond to the sate $(s,a,b)$. That is, the premises of said rule is
\begin{equation*}
\state{s}{\tau} \wedge \sym{\sigma}{\tau}{\pi} \wedge \cursor{\tau}{\pi}
\end{equation*}
while $(s,a,b)=(s,a_1 \dots a_{\pi-1} \sigma,b_{\pi+1} \dots b_{N})$.
Moreover, again due to Proposition \ref{prop:unique} due to  the definition of $\trans{I}$, it must be that $(s',a',b')$ corresponds with the only satisfied disjunction of the consequent in $r$ in a similar fashion. That is, the only satisfied disjunction of the consequent in $r$ is
\begin{equation*}
\state{s'}{\tau+1} \wedge \sym{\sigma'}{\tau+1}{\pi} \wedge \cursor{\tau+1}{\pi+d}
\end{equation*}
while 
\begin{itemize}
\item $(s',a',b')=(s',a'_1 \dots a'_{\pi-1} \sigma',b'_{\pi+1} \dots b'_{N})$ if $d =0$;
\item  $(s',a',b')=(s',a'_1 \dots a'_{\pi-1} \sigma' b'_{\pi+1} , b'_{\pi+2}\dots b'_{N})$ if $d =1$;
\item  $(s',a',b')=(s',a'_1 \dots a'_{\pi-1} ,\sigma' b'_{\pi+1} \dots b'_{N})$ if $d =-1$.
\end{itemize}
with its premises satisfied by $(s,a,b)$. 
As of now it is known that $\trans{I}(\tau)$ satisfies the premise and $\trans{I}(\tau+1)$ satisfies the consequent of $r$. Hence, one obtains that $\delta_{\mathcal{I}}(r) \in \Delta$. Therefore, the last issue in question to establish that $\trans{I}{\tau} \to_{\delta_{\mathcal{I}}(r)} \trans{I}{\tau+1}$ is to demonstrate that for all $1 \leq i \leq N$ such that  $i \neq \pi$ $ab[i] =a'b'[i]$. However, this is precisely accomplished through the inertia rules. To be precise,  it is known that for arbitrary $i$ such that $1 \leq i < \pi  $ one has 
$\mathcal{I} \models \sym{\sigma}{\tau}{i}$ for $ab[i]=\sigma$ and thereby by construction one obtains $a'b'[i]=\sigma$. Similarly for $\pi  < i \leq N$, one has $\mathcal{I} \models \sym{\sigma}{\tau}{i}$ for $ab[i]=\sigma$ and thereby by definition one obtains $a'b'[i]=\sigma$.
Thus, allowing one to conclude that $\trans{I}{\tau} \to_{\delta_{\mathcal{I}}(r)} \trans{I}{\tau+1}$ is valid.
\end{itemize}
\end{proof}
Thus allowing the following simple and final corollary.


\begin{corollary}
\label{prop:fin-config}
Let $L$ language decided by some NTM $T$ where each computation requires exactly $N$ steps. Moreover, for the string $x \in \Sigma^{\leq N}$ such that $R(x)$  being the corresponding propositional formula as defined above is satisfiable.
Then the sequence $\trans{I}(0)\dots \trans{I}(N)$ is a valid sequence of configurations for $T$ on input $x$ ending it the accepting state $\yess$. 
\end{corollary}
\begin{proof}
From Proposition \ref{prop:valid-config}, we know that $\trans{I}(0)\dots \trans{I}(N)$ is a valid sequence of configuration for the NTM $T$. By construction, (due to the Acceptance-formula) $\trans{I}(N)=(\yess, a,b)$ for some strings $a$ and $b$. 
\end{proof}

Using this one can finally conclude that $x \in L$. And thus the validity of the reduction is established.
\newpage

\noindent
\begin{exercise}[5 credits]
{\em 
Recall the basic, polynomial-time decision procedure for HORNSAT (see cc04.pdf).
The correctness of this decision procedure 
relies on the following lemma:
}%em

\medskip

\noindent
{\bf Lemma.}
Let $\phi$ be a propositional Horn-formula. Let 
$Y$ denote the set of atoms which are obtained by 
initializing $Y$ to the set of facts in $\phi$ and by 
exhaustively applying the rules in $\phi$ to $Y$.
Then, for every atom $x$ in $\phi$, the following equivalence holds: 
%
$x \in Y$ $\LR$ 
$x$ is implied by the facts and rules in $\phi$.

\medskip
\noindent
{\em 
Give a rigorous proof of this lemma.
}%em

\medskip
\noindent
{\em 
Terminology. We say that a formula $\beta$ is implied by a set of formulas $\alpha$, if
$\alpha \models \beta$ holds, i.e., every model of (all formulas in) $\alpha$ is a 
model of $\beta$. In the above lemma, let $\psi$ denote the set consisting of the 
rules and facts (but not the goals) in $\phi$. The lemma thus claims that   
$x \in Y$ $\LR$ $\psi \models x$ holds.
}%em

\end{exercise}


\paragraph*{Solution}

Firstly, some notation.
\begin{definition}
For a given Horn-formula $\varphi$, let $\mathcal{C}(\varphi)$ denote the set of clauses in the formula $\varphi$. Moreover, let $\mathcal{C}^*(\varphi)$ be the set of clauses $\mathcal{C}(\varphi)$ reduced to facts and rules only.
\end{definition}

Secondly, a formal definition of $Y$ is required.
\begin{definition}
Let $\mathcal{C}(\varphi)$ denote the set of clauses in the formula $\varphi$. Using this consider the following construction
\begin{equation*}
\begin{split}
Y_0&:=\{p \mid \forall (p)\in \mathcal{C}(\varphi) \; p \mathit{\; prop. \, variable}\} \\
Y_i&:= \{p \mid \forall n \geq 0 \forall (q_1 \land \dots \land q_n \lto p) \in \mathcal{C}(\varphi) \; \{q_1, \dots, q_n\}  \subseteq Y_{i-1}\}
\end{split}
\end{equation*}
Furthermore, for $i>0$ let $Y_i^{\Delta}:= Y_i \setminus Y_{i-1}$ and for $i=0$ $Y_i^{\Delta}=Y_0$. Leading to $Y$ being defined as the fixpoint of this construction, i.e.$Y:= Y_i$ iff $Y_i^{\Delta}= \{\}$.
\end{definition}
That is, $Y_0$ is the set of facts in $\varphi$, $Y_i$ is the set of atoms obtained after $i$ direct inferences. Moreover, it is easy to see that $Y_0 \subseteq Y_i$.


 
\bigskip

To show  $\mathcal{C}^*(\varphi) \models x  \iff x \in Y$ holds for every atom $x$ in $\varphi$, the first step will be to demonstrate $x \in Y \implies  \mathcal{C}^*(\varphi) \models x $. However, fist consider the following proposition. 
\begin{proposition}
\label{prop:yi-induction}
Let $Y$ be constructed from the Horn-formula $\varphi$ as described above. Then it holds for all atoms $x$ in $\varphi$ that $x \in Y_i \implies \mathcal{C}^*(\varphi) \models x$.
\end{proposition}
\begin{proof}
Firstly, if $x$ is an atom in $\varphi$ such that $x \notin Y_i$ for an arbitrary $i$, then $x \in Y_i \implies \mathcal{C}^*(\varphi) \models x$ holds trivially. Hence, it suffices to focus on the claim that for any $x \in Y_i $ it must be that $\mathcal{C}^*(\varphi) \models x$.
This can be demonstrated by induction on $i$.
\begin{itemize}
\item \textbf{IH}: $\forall x \in Y_i \; \mathcal{C}^*(\varphi) \models x$.
\item \textbf{IB}: $i=0$ if this is the case $Y_i$ is a collection the facts in $\varpi$, all of which are contained as clauses in $\varphi$. That is, $x \in \mathcal{C}^*(\varphi)$. Hence, from the deduction theorem it follows that $\mathcal{C}^*(\varphi) \models x$.
\item \textbf{IS}: For $i=k>0$. We know that $Y_i=Y_{i-1} \cup Y_i^{\Delta}$. If $Y_i^{\Delta}=\{\}$ then $Y_i=Y_{i-1}$ and thus by \textbf{IH} it follows that for all $x \in Y_i $ we have $\mathcal{C}^*(\varphi) \models x$. Otherwise, there exist an $x \in Y_i^{\Delta}$. By definition, this implies that there exists a rule $(q_1 \land \dots \land q_n \lto x)$ for some $n>0$ in $\mathcal{C}^*(\varphi)$ such that $q_1 , \dots , q_n \in Y_{i-1}$. 
This implies that $\mathcal{C}^*(\varphi) \models q_1 \land \dots \land q_n \lto x$. Moreover, by \textbf{IH} one obtains 
for $1 \leq i \leq n$ that $\mathcal{C}^*(\varphi) \models q_i$. Using normal semantics one thus obtains $\mathcal{C}^*(\varphi) \models q_1 \land \dots \land q_n$.  Now having $\mathcal{C}^*(\varphi) \models q_1 \land \dots \land q_n \lto x$ and $\mathcal{C}^*(\varphi) \models q_1 \land \dots \land q_n$ it thereby follows that $\mathcal{C}^*(\varphi) \models x$.
\end{itemize} 
\end{proof}

From this proposition the desired statement follows directly as a corollary. 

\begin{corollary} 
\label{cor:forth}
Let $Y$ be constructed from the Horn-formula $\varphi$ as described above. Then it holds for all atoms $x$ in $\varphi$ that $x \in Y \implies \mathcal{C}^*(\varphi) \models x$.
\end{corollary}
\begin{proof}
Let $x$ be an atom in $\varphi$. To show that $x \in Y \implies \mathcal{C}^*(\varphi) \models x$, assume $x \in Y$. Hence, there must exists an $i$ such that for all $k\geq i$ one has $x \in Y_k$. By Proposition \ref{prop:yi-induction}, if follows that $\mathcal{C}^*(\varphi)\models x$.
\end{proof}

\bigskip
To show that $\mathcal{C}^*(\varphi) \models x  \implies x \in Y$, its contrapositive  $x  \notin Y  \implies \mathcal{C}^*(\varphi) \nmodels x$ will be demonstrated. That is, assuming $x  \notin Y$ it will be shown that there exist an interpretation $\mathcal{I}$ that on the one hand satisfies $ \mathcal{C}^*(\varphi)$, while on the other evaluates $x$ to $0$. To make this notion more precise consider the following proposition.
\begin{proposition}
\label{prop:yi-model}
For $Y$ constructed from the Horn-formula $\varphi$ as described above,
let $\mathcal{I}$ be an interpretation such that for all atoms $x$ in $\varphi$,  $\mathcal{I}(x)=1$ if and only if $x \in Y$.
Then $\forall \psi \in \mathcal{C}^*(\varphi) \; \mathcal{I} \models \psi$, i.e. $ \mathcal{I} \models \mathcal{C}^*(\varphi)$.
\end{proposition}
\begin{proof}
Consider an arbitrary $ \psi \in \mathcal{C}^*(\varphi)$. 
\begin{itemize}
\item Case 1: If $\psi = p$ where $p$ is a fact, i.e. a propositional atom, then $p \in Y$ per definition. From this one obtains by construction of $\mathcal{I}$ that $\mathcal{I}(p)=1$. 
\item Case 2: Otherwise, $\psi = q_1 \land \dots \land q_n \lto p$. Assume that $\mathcal{I} \nmodels \psi$. Hence, it must be that
$\mathcal{I}\models q_1 \land \dots \land q_n $  and $\mathcal{I}\nmodels p $. However, this would imply that $q_1 , \dots , q_n \in Y$, which by construction of $Y$ requires $p \in Y$, thus forcing $\mathcal{I} \models p$. Thereby, one obtains a contradiction, which in turn allows the conclusion of $\mathcal{I} \models \psi$.
\end{itemize}
As those are the only two cases, recall $\mathcal{C}^*(\varphi)$ contains only facts and rules, it was demonstrated that every $\psi \in  \mathcal{C}^*(\varphi)$ is modelled by $\mathcal{I}$. 
\end{proof}

Using this, the desired statement follows as a corollary. 

\begin{corollary} 
\label{cor:back}
Let $Y$ be constructed from the Horn-formula $\varphi$ as described above and let $x$ be an atom in $\varphi$.  Then it holds that  $\mathcal{C}^*(\varphi) \models x  \implies x \in Y$
\end{corollary}
\begin{proof}
Let $x$ be an atom in $\varphi$. To show that $x \notin Y \implies \mathcal{C}^*(\varphi) \nmodels x$, assume $x \notin Y$. 
From Proposition \ref{prop:yi-model}, it is known that $Y$ induces an interpretation $\mathcal{I}$ that models  $ \mathcal{C}^*(\varphi)$.
However, by construction of $\mathcal{I}$, $\mathcal{I}\nmodels x$ holds.  Thereby, not every interpretation that satisfies $ \mathcal{C}^*(\varphi)$
also satisfies $x$. Allowing, to conclude $\mathcal{C}^*(\varphi) \nmodels x$, which is equivalent to $\mathcal{C}^*(\varphi) \models x  \implies x \in Y$.
\end{proof}

Finally, from Corollary  \ref{cor:forth} \& \ref{cor:back} it directly follows that  $\mathcal{C}^*(\varphi) \models x  \iff x \in Y$ for every atom $x$ in $\varphi$.
\end{document}


