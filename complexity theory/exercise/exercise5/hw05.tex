   \documentclass [11pt]{article}
   \usepackage{latexsym}
   \usepackage{amssymb}
      \usepackage{amsmath}
      \usepackage{amsthm}
   \usepackage{url}

   \textwidth      15cm
   \textheight     23cm
   \oddsidemargin 0.5cm
   \topmargin    -0.5cm
   \evensidemargin\oddsidemargin

 \newcommand{\nop}[1]{}


   \pagestyle{plain}
   \bibliographystyle{plain}


%  \newtheorem{theorem}{Theorem}
%  \newtheorem{lemma}[theorem]{Lemma}
%  \newtheorem{exercise}{Exercise}

  \newcommand{\ra}{\rightarrow}
  \newcommand{\Ra}{\Rightarrow}
  \newcommand{\La}{\Leftarrow}
  \newcommand{\la}{\leftarrow}
  \newcommand{\LR}{\Leftrightarrow}

  \renewcommand{\phi}{\varphi}
  \renewcommand{\theta}{\vartheta}



\newcommand{\ccfont}[1]{\protect\mathsf{#1}}
\newcommand{\NP}{\ccfont{NP}}
\newcommand{\SAT}{\mbox{\bf SAT}}
\newcommand{\Ptime}{\ccfont{P}}
\newcommand{\phs}[1]{\Sigma_{#1}\Ptime}
\newcommand{\php}[1]{\Pi_{#1}\Ptime}
\newcommand{\QSAT}[1]{\ccfont{QSAT}_{#1}}
\newcommand{\MINSAT}{\mbox{\bf MINIMAL MODEL SAT}}
\newcommand{\True}{\mathbf{true}}
\newcommand{\False}{\mathbf{false}}


\renewcommand{\labelenumi}{(\alph{enumi})}

\newcommand{\ol}[1]{\overline{#1}}
\newcommand{\mc}[1]{\mathcal{#1}}

  \newcommand{\lto}{\rightarrow}
  \newcommand{\sto}{\Rightarrow}
  \newcommand{\snto}{\Leftarrow}
  \newcommand{\lnto}{\leftarrow}
  \newcommand{\siff}{\Leftrightarrow}
   \newcommand{\liff}{\leftrightarrow}
   \newcommand{\nmodels}{\not\models}

\newcommand{\clau}{\mathit{Clause}}
\newcommand{\var}{\mathit{Var}}
\newcommand{\lit}{\mathit{Lit}}
\newcommand{\dom}{\mathit{Dom}}



  \newtheorem{theorem}{Theorem}
  \newtheorem{lemma}[theorem]{Lemma}
  \newtheorem{corollary}[theorem]{Corollary}
    \newtheorem{observation}[theorem]{Observation}
  \newtheorem{proposition}[theorem]{Proposition}
  \newtheorem{conjecture}[theorem]{Conjecture}
  \newtheorem{definition}[theorem]{Definition}
  \newtheorem{example}[theorem]{Example}
  \newtheorem{remark}[theorem]{Remark}
  \newtheorem{exercise}[theorem]{Exercise}

%%%%%%%%%%%%%%%%%%%%%%%%%%%%%%%%%%%%%%%%%%%%%%%%%%%%%%%%%%%%%%%%%%%

\begin{document}

\noindent
\begin{exercise}[5 credits]
{\em 
Recall the $\phs{2}$-hardness proof of the Abduction Solvability problem by reduction from $\QSAT{2}$:
Let an arbitrary instance of the $\QSAT{2}$ problem be given by the formula 
$\phi =  (\exists X) (\forall Y) \psi(X,Y)$
with $X = \{x_1, \dots, x_k \}$ and $Y = \{y_1, \dots, y_l\}$. Moreover, let 
$X' = \{x'_1, \dots, x'_k \}$, $R = \{r_1, \dots, r_k \}$, and $t$ be fresh variables. 
Then we define an instance of Solvability as
$\mathcal{P}=\langle V,H,M,T \rangle$ with
%
\begin{eqnarray*}
V &=& X \cup Y  \cup X' \cup R \cup \{t\}\\
H &=& X \cup X' \\
M &=& R \cup \{t\}\\
T &=& \{ \psi(X,Y)\ra t \} 
\cup \{\neg x_i \vee \neg x'_i, x_i \ra r_i, x'_i \ra r_i \mid 1 \leq i \leq k\} 
\end{eqnarray*}
%

\smallskip
\noindent
Give a rigorous correctness proof of this problem reduction, i.e., 
$\phi \equiv \True$ $\LR$ $Sol(\mathcal{P}) \neq \emptyset$.

}%em
\end{exercise}

\noindent
{\bf Hint.} As usual, prove both directions of the equivalence separately. It is convenient to use the notation from the lecture: For 
$A \subseteq X$, let $A'$  denote
the set $\{x' \mid x \in A\}$.

\begin{itemize}
\item 
For the ``$\Ra$''-direction, you start off with a partial assignment $I$ on $X$.
Let $A = I^{-1}(\True)$. Then it can be shown that $S = A \cup (X \setminus A)'$ 
is a solution of $\mathcal{P}$. In order to show that $S$ is indeed a solution, you must prove carefully the two conditions that (1) $T \cup S$ is satisfiable
and (2) $T \cup S \models M$. 

\item For the ``$\La$''-direction, first show that a solution $S$ of $\mathcal{P}$
contains exactly one  of $\{x_i, x'_i\}$. Why?  Hence, $S$ must be of the form
$S = A \cup (X \setminus A)'$ for some $A \subseteq X$.
It remains to show that for the assignment $I$ on $X$ with $I^{-1}(\True) = A$, 
every extension $J$ of $I$ to the variables $Y$ satisfies the formula $\psi(X,Y)$.
\end{itemize}


\paragraph*{Solution}


\begin{lemma}
Let $\varphi := \exists X \forall Y \;  \psi(X,Y)$ be a $\mathrm{QBF}_{\exists,2}$-formula, with $X = \{x_1, \dots, x_k \}$ and $Y = \{y_1, \dots, y_l\}$. Moreover, let $\tau(\varphi)$ the abduction instance presented in the reduction.
Then  $\phi \equiv \True \;  \iff \; Sol(\mathcal{P}) \neq \emptyset$.
\end{lemma}
\begin{proof}
\begin{itemize}
\item[$\Rightarrow$] Assume $\phi \equiv \True $ then there exists a partial assignment $\mathcal{I}_{|X}$ for the variables in $X$ such that for any extension $\mathcal{I}_{|X \cup Y}$ by the variables in $Y$ one has $\mathcal{I}_{|X \cup Y} \models \psi(X,Y)$. Using the partial assignment $\mathcal{I}_{|X}$ one can build the set $A:=\{x \mid \mathcal{I}_{|X} \models x\}$. Now consider the set $S:= A \cup (X \setminus A)'$. As suggested, the claim to be demonstrated is that $S$ is a solution for $\tau(\varphi)$. To that end consider $T \cup S$, take an arbitrary model $\mathcal{J}$ satisfying $T \cup S$. Notice that currently it is not known that such a model actually exist. 
Firstly, since $\mathcal{J} \models S$ it must be that by construction one has $\forall x \in A\; \mathcal{J} \models x$ and $\forall x \in X\setminus A \; \mathcal{J}\models x'$. Since $x \in S$ if and only if $x' \notin S$, and the fact that  for some $i \in \{1, \dots k\}$ $\mathcal{J} \models \neg x_i \lor \neg x_i'$, one has $\mathcal{J} \models x$ if and only if $\mathcal{J} \nmodels x'$. Therefore, for every $i \in \{1, \dots  , k\}$ $\mathcal{J}$ models either $x_i$ or $x_i'$, and thus it follows that $\mathcal{J} \models r_i$. 
%Moreover, one has for some $i \in \{1, \dots k\}$ that $\mathcal{J} \models \neg x_i \lor \neg x_i' $, thus $\mathcal{J} \models x$ implies $\mathcal{J} \nmodels x'$ and vice versa.
Secondly, by virtue of $A$ being constructed from $\mathcal{I}_{|X}$ and by the fact that $\mathcal{J} \models A$, every $x_i$ satisfied by $\mathcal{I}_{|X}$ must be satisfied by $\mathcal{J}$. Moreover, if there would exist an $x_i$ such that $\mathcal{J} \models x_i$ but $\mathcal{I}_{|X} \nmodels x_i$, then $x_i' \in S$ and thus $\mathcal{J} \models x_i'$. Thereby, violating the fact that $\mathcal{J} \models \neg x_i \lor \neg x_i'$. Hence, it is known that $J$ agrees on $I_{|X}$ on the variables $X$. Therefore, it follows $\mathcal{J} \models \psi(X,Y)$ regardless of the truth values of the variables in $ Y$ under $\mathcal{J}$. Hence, it must be that $\mathcal{J} \models t$, which was the last piece required to establish that $\mathcal{J} \models M$. Thus one can conclude that $T \cup S \models M$. What remains is to verify that such a model actually exists, that is let $\mathcal{J}_e$ be the interpretation
\begin{itemize}
\item $\forall x \in A \; \mathcal{J}_e(x) := \True$;
\item $\forall x \in X \setminus A \; \mathcal{J}_e(x') := \True$;
\item $\forall r \in R \; \mathcal{J}_e(r) := \True$;
\item $\forall y \in Y \; \mathcal{J}_e(y) := \True$;
\item $\mathcal{J}_e(t) := \True$.
\end{itemize}
Since, $\mathcal{J}_e$ agrees with $\mathcal{I}_{|X}$ on all variables on $X$, the assignment of $y \in Y$ is irrelevant and thus it follows that $\mathcal{J}_e \models \psi(X,Y)$. With $\mathcal{J}_e \models \psi(X,Y) \land t$, it must be that $\mathcal{J}_e \models \psi(X,Y) \to t$. By construction, for an arbitrary $i \in \{1, \dots , k\}$, $\mathcal{J}_e \models x_i$ iff $\mathcal{J}_e \nmodels x_i'$ and thus $\mathcal{J}_e \models \neg x_i \lor \neg x_i'$. Lastly, since for any $i \in i \in \{1, \dots , k\}$ one has, $\mathcal{J}_e \models x_i \lor x_i'$ it follows  that $x_i \lto r$ and $x_i' \lto r$ are satisfied. Hence, $\mathcal{J}_e \models T \cup S$. Thereby, establishing that $S$ is indeed a solution. 


\item[$\Leftarrow$] Assume that there exists a solution $S$ for $\tau(\varphi)$. For $S$ to be a solution it must that $S \subseteq H$,  $T \cup S$ is satisfiable and that $T \cup S \models M$. Hence, it is known that there exists an interpretation $\mathcal{I}$ that $\mathcal{I} \models T \cup S$ and $\mathcal{I} \models M$. Now restrict $\mathcal{I}$ by removing the 
assignments concerning the variables in $Y$ to create $\mathcal{J}$. Furthermore, assume that $\mathcal{J}'$ is an extension of $\mathcal{J}$ by the variables in $Y$ such that $\mathcal{J}' \nmodels \psi(X,Y)$. Since $\mathcal{J}'$ agrees with $\mathcal{I}$ on all variables in $R, X$ and $X'$, $\mathcal{J}' \models S \cup \{\neg x_i \vee \neg x'_i, x_i \ra r_i, x'_i \ra r_i \mid 1 \leq i \leq k\} $. However, since by assumption, $\mathcal{J}' \nmodels \psi(X,Y)$, it must be that $\mathcal{J}' \models \psi(X,Y) \to t$ holds. Therefore, it follows that $\mathcal{J}' \models T \cup S$. Moreover, with $\psi(X,Y) \to t$ being vacuously true under $\mathcal{J}'$, there must exists an interpretation where $t$ is not satisfied. W.l.o.g. let $\mathcal{J}'$ be this interpretation. Therefore, $\mathcal{J}' \models T \cup S$ but $\mathcal{J}' \nmodels M$, which is clearly a contradiction. Hence, in every extension of $\mathcal{J}$ to the variables in $Y$, it must be that $\psi(X,Y)$ is satisfied. Now, by restricting $\mathcal{J}$ to the variables in $X$ one has found a partial assignments for the variables in $X$ where every extension by the variables $Y$ satisfies the formula $\psi(X,Y)$. Meaning that $\phi \equiv \True $.
%
%
%
%\item[$\Leftarrow$]  Assume that there exists a solution $S$ for $\tau(\varphi)$. For $S$ to be a solution it must that $S \subseteq H$,  $T \cup S$ is satisfiable and that $T \cup S \models M$. 
%Hence, there exists an interpretation $\mathcal{I}$ such that for any $i \in I$ one has $\mathcal{I} \models r_i$, thus $\mathcal{I} \models x_i \lor x_i'$. Together with $\mathcal{I} \models \neg x_i \lor \neg x_i'$ it follows that $\mathcal{I} \models x_i$ if and only if $\mathcal{I} \models x_i'$. Now consider $A:= S \cap X$, due to the previous observation, it must be that $S = A \cup (X \setminus A)'$. Consider the partial assignment $\mathcal{I}_{|X}$ such that $\forall x \in A  \; \mathcal{I}_{|X}(x):= \True$ and $\forall x \in X \setminus A  \; \mathcal{I}_{|X}(x):= \False$. By the fact that $S = A \cup (X \setminus A)'$ all $x$ are assigned a truth value by $\mathcal{I}_{|X}$. One knows that $T \cup S \models t$. 
%Therefore, any model of $T \cup S$ must model $\psi(X,Y)$. Especially, any model that agrees on all $x \in X$ with the constructed interpretation $\mathcal{I}_{|X}$. Moreover, considering the fact that the variables in $Y$ occur only in $\psi(X,Y)$ the there are no restrictions on their truth values. Hence, the truth values of $y \in Y$ is immaterial. And thus one can conclude that for any arbitrary extension $\mathcal{I}_{|X \cup Y}$ of $\mathcal{I}_{|X}$ it must be that $\mathcal{I}_{|X\cup Y} \models \psi(X,Y)$. Thus $\varphi \equiv \True$.
\end{itemize}

\end{proof}
\medskip

\newpage


\noindent
\begin{exercise}[5 credits]
{\em Recall the $\phs{2}$-hardness proof of 
the Abduction Relevance problem by reduction from 
the Solvability problem:
Let an arbitrary instance of the Solvability problem be given 
by the PAP  $\mathcal{P}=\langle V,H,M,T \rangle$. 
W.l.o.g., let $T$ consist of a single formula $\phi$ and
let $h, h',m'$ be fresh variables. Then we define an 
instance of the Relevance (resp.\ the Necessity) problem with the 
following PAP
 $\mathcal{P}'=\langle V', H',M',T' \rangle$:
%
\begin{eqnarray*}
V' &=& V\cup\{h,h',m'\} \\
H' &=& H\cup\{h,h'\} \\
M' &=& M\cup\{m'\}\\
T' &=& \{\neg h \vee \phi\} \cup \{h'\ra m \mid m\in M\} \cup \{\neg h \vee \neg h', h \ra m', h' \ra m'\}
\end{eqnarray*}
%
This reduction fulfills the following equivalences: 

\begin{quote}
$\mathcal{P}$ has at least one
solution iff $h$ is relevant in $\mathcal{P}'$ iff 
$h'$ is not necessary in $\mathcal{P}'$.%
\end{quote}

\smallskip
\noindent
Give a rigorous proof of these equivalences.
} % em
\end{exercise}


\noindent
{\bf Hint.} 
The second equivalence is easy to show. For Then first equivalence, 
show both directions separately:


\begin{itemize}
\item 
For the ``$\Ra$''-direction, you start off with a
solution $S$ of $\mathcal{P}$ and construct a solution $S'$ of $\mathcal{P}'$ with $h \in S'$. Prove carefully that $S'$ is indeed a solution of $\mathcal{P}'$, i.e. (1) $T' \cup S'$ is satisfiable
and (2) $T' \cup S' \models M'$. 

\item For the ``$\La$''-direction, you start off with a
solution $S'$ of $\mathcal{P}'$, s.t.\ $h \in S'$ and construct a solution $S$ of $\mathcal{P}$.
Prove carefully that $S$ is indeed a solution of $\mathcal{P}$, i.e. (1) $T \cup S$ is satisfiable
and (2) $T \cup S \models M$. 

\end{itemize}

\paragraph*{Solution}


\begin{observation}
\label{obs:2-h-distinct}
Take an arbitrary PAP  $\mathcal{P}=\langle V,H,M,T \rangle$. Let $\mathcal{P}_{\tau}$ be the PAP as constructed in the reduction. Then for any solution $S \in Sol(\mathcal{P}_{\tau})$ it must be the case that $h \in S$ if and only if $h' \notin S$.
\end{observation}
\begin{proof}
For any solution $S \in Sol(\mathcal{P}_{\tau})$ it must hold that $T \cup S \models m'$. However, this requires that either $h \in S$ or $h' \in S$. Assume that $h \in S$. Hence, any interpretation $\mathcal{I}$ satisfying $T \cup S$ must satisfy $h$. Moreover, it must also satisfy $\neg h \lor \neg h'$, which can only be the case if $\mathcal{I} \nmodels h'$. Now assume that $h' \in S$ if this is the case $\mathcal{I} \models h'$, which is impossible. That is, there can not be an interpretation modelling $S$. However, since $S$ is a solution, it must be that there exists at least one model of $S$. Thus the only possible conclusion is that  $h' \notin S$. Due to symmetry the other direction can be done in analogue.
\end{proof}


\begin{lemma}
\label{lemma:2-equiv2}
Take an arbitrary PAP  $\mathcal{P}=\langle V,H,M,T \rangle$. Let $\mathcal{P}_{\tau}$ be the PAP as constructed in the reduction.  Then $h$ is relevant in $\mathcal{P}_{\tau}$ iff 
$h'$ is not necessary in $\mathcal{P}_{\tau}$.
\end{lemma}
\begin{proof}
%\begin{itemize}
%\item[$\Rightarrow$]
%By definition, $h$ is relevant in $\mathcal{P}_{\tau}$ if there exists a solution $S \in Sol(\mathcal{P}_{\tau})$ such that $h\in S$. By Observation \ref{obs:2-h-distinct}, this implies that there exists a solution $S \in Sol(\mathcal{P}_{\tau})$ such that $h'\notin S$. By definition, this means that $h'$ is not necessary in $\mathcal{P}_{\tau}$.
%\item[$\Leftarrow$] 
%\end{itemize}
By definition, $h$ is relevant in $\mathcal{P}_{\tau}$ there exists a solution $S \in Sol(\mathcal{P}_{\tau})$ such that $h\in S$. By Observation \ref{obs:2-h-distinct}, this is equivalent to there exists a solution $S \in Sol(\mathcal{P}_{\tau})$ such that $h'\notin S$. By definition, this is equivalent to $h'$ is not necessary in $\mathcal{P}_{\tau}$.
\end{proof}



\begin{lemma}
\label{lemma:2-equiv1}
Take an arbitrary PAP  $\mathcal{P}=\langle V,H,M,T \rangle$. Let $\mathcal{P}_{\tau}$ be the PAP as constructed in the reduction.  Then $\mathcal{P}$ has at least one
solution iff 
$h$ is relevant in $\mathcal{P}_{\tau}$.
\end{lemma}
\begin{proof}
\begin{itemize}
\item[$\Rightarrow$] Let $S \in Sol(\mathcal{P})$. That is, $S \subseteq H$, $T \cup S \models M$ and that there exists an interpretation $\mathcal{I}$ such that $\mathcal{I} \models T \cup S$. Consider $S_{\tau}:=S \cup \{h\}$. The claim is that $S_{\tau}$ is a solution of $\mathcal{P}'$ where $h \in S$. Firstly, $S_{\tau} \subseteq H \cup\{h, h'\}$ and $h\in S$ by construction. Secondly, take some model $\mathcal{I}$ of $T \cup S$, extend it to the interpretation $\mathcal{I}_{\tau}$ by the variables $\{h,h',m'\}$ such that $\mathcal{I}_{\tau} \models h \land m'$ and $\mathcal{I}_{\tau} \nmodels h'$. The claim is that $\mathcal{I}_{\tau} \models T_{\tau} \cup S_{\tau}$.
Since $\mathcal{I} \models \varphi$ and $\mathcal{I}_{\tau}$ agrees with $\mathcal{I}$ on all variables in $\varphi$ one has $\mathcal{I}_{\tau} \models \neg h \lor \varphi$. Due to  $\mathcal{I}_{\tau}\nmodels h'$ the implications $h' \lto m$ for all $m \in M$ hold vacuously. 
By construction one has $\mathcal{I}_{\tau}\models \neg h \lor \neg h'$. From $\mathcal{I}_{\tau}\models m'$ it follows that $h \lto m'$ and $h' \lto m'$ hold. Since $\mathcal{I} \models S$ and $\mathcal{I}_{\tau}$ agrees with $\mathcal{I}$ on all variables in $S$ and since $\mathcal{I}_{\tau}\models h$ it follows that $\mathcal{I}_{\tau}\models S_{\tau}$. Hence, one can conclude that $\mathcal{I}_{\tau} \models T_{\tau} \cup S_{\tau}$.
Thirdly, consider an arbitrary model  $\mathcal{J}$ of  $T_{\tau} \cup S_{\tau}$. By construction, it is known that $h \in S$. Hence, $\mathcal{J} \models h$ and by extension $\mathcal{J} \models \varphi$. Moreover, since $T = \varphi$ and $S_{\tau}=S \cup  \{h\}$, it must be that $\mathcal{J} \models T \cup S$. However, it is known that $T \cup S \models M$. Hence, it follows that $\mathcal{J} \models M$. Furthermore, due to the fact that $\mathcal{J} \models h$ it must be the case that $\mathcal{J} \models m'$. Hence, $\mathcal{J} \models M_{\tau}$. 
Having checked all conditions, one can conclude that $S_{\tau}$ is a solution of $\mathcal{P}'$ with $h \in S_{\tau}$.


\item[$\Leftarrow$]  Let $S_{\tau} \in Sol(\mathcal{P}')$ such that $h \in S_{\tau}$. Moreover, it is known that $S_{\tau} \subseteq H_{\tau}$, $T_{\tau} \cup S_{\tau} \models M_{\tau}$ and that there exists an interpretation $\mathcal{I}_{\tau}$ such that $\mathcal{I}_{\tau} \models T_{\tau} \cup S_{\tau}$. Consider $S:=S_{\tau} \cap H$.  Restrict $\mathcal{I}_{\tau}$ by removing the assignments of the variables $h,h'$ and $m'$, thereby creating $\mathcal{I}$. Claim $\mathcal{I} \models T\cup S$. Since $S \subset S_{\tau}$ it follows that $\mathcal{I}_{\tau} \models S$, and thus by construction $\mathcal{I} \models S$. From the fact that $\mathcal{I}_{\tau} \models T_{\tau}$ and the fact that $h \in S_{\tau}$ it must be that $\mathcal{I}_{\tau} \models \varphi$. Again with $\mathcal{I}$ and $\mathcal{I}_{\tau}$ agreeing on all variables in $\varphi$ it follows that
$\mathcal{I} \models \varphi$. Hence, $\mathcal{I} \models T \cup S$. Now one has to check whether $T \cup S \models M$. To that end, take an arbitrary $\mathcal{J} \models T \cup S$. Extend it by the variables $h,h',m'$ to obtain $\mathcal{J}_{\tau}$ such that $\mathcal{J}_{\tau} \models h \land m'$ and  $\mathcal{J}_{\tau} \nmodels h'$. Clearly, $\mathcal{J}_{\tau} \models T_{\tau} \cup S_{\tau}$ (see above). However, this implies that $\mathcal{J}_{\tau} \models M_{\tau}$ and 
especially $\mathcal{J}_{\tau} \models M$. However, since $\mathcal{J}$ and $\mathcal{J}_{\tau}$ agree on all variables in $M$ it follows that $\mathcal{J} \models M$ and therefore, $T \cup S \models M$. Having checked all conditions, one can conclude that $S$ is a solution of $\mathcal{P}$.
\end{itemize}
\end{proof}
\end{document}


