\documentclass[11pt]{article}
\usepackage[a4paper]{geometry}
\usepackage{amsmath}
\usepackage{amssymb}
\usepackage{amsfonts}
\usepackage{enumitem}
\usepackage{amsthm}
\usepackage{import}
\usepackage{MnSymbol}
\usepackage{graphicx}
%\setlength{\parindent}{0pt}
\usepackage[utf8]{inputenc}
\usepackage{listings} [python]
\usepackage{bussproofs}
%\usepackage{MnSymbol}
\usepackage{stmaryrd}
\usepackage{adjustbox}

\usepackage{comment}

%\usepackage[square]{natbib}

\usepackage[utf8]{inputenc}


%\usepackage[backend=biber, style=authoryear]{biblatex}


%\bibliographystyle{unsrtnat}
%\addbibresource{irl_paper.bib}




\begin{document}




This essay argues that the design of the Bretton Woods System (BWS) created favourable structural condition for the shift in economic orthodoxy from Keynesian economic policy to neoliberal economic policy. Firstly, it will be demonstrated that structural flaws in the BWS's design promoted its demise. Secondly, the demise of the BWS, the oil shocks, the stagflation of the 1970s and the decline of Keynesian economic theory will be connected.
Finally, the relationship between the BWS and the formation of neoliberal think-tanks in Britain and the US will be discussed.
 \\



%Firstly, one has to establish that for the thesis argued in this essay it is inconsequential to root the design of the BWS in either structural or agentic factors. This is due to the fact that this and subsequent arguments, take the design of the BWS as a given. Hence, it is not within the scope of this argument to clarify, whether the agents John Maynard Keynes and Harry Dexter White or the structure the inhabited, were the most prominent force behind its design. Therefore, even if one concedes that the design of the BWS is best explained by agentic factors, the BWS itself remains a structure. Moreover, by rooting the subsequent line of arguments after its design, the distinction between the BWS being a structure that emerged from structure or being a structure that emerged from agency becomes negligible. This argument is of utmost importance, as it breaks the constant interdependencies between structural and agentic factors, thus providing a firm structural foundation upon which to build the argument.  \\

The first argument to be made is that internal contradictions within the structure of the  BWS promoted its demise. 
%That is, as argued in \cite{KlimiukZbigniew2016TPaO} and supported by \cite{bordo2014tales}, the design of the BWS itself produced a structure that conflicted with the world it sought to create.
That is, given the dominance of the dollar within the BWS the following can be observed.
On the one hand,  if the dollar supply drastically exceeds gold reserves, confidence in convertibility may erode. While on the other, international trade requires sufficient levels of liquidity, thus the US is compelled by the economy to increase the dollar supply. Hence, if it can be demonstrated that the collapse the BWS is rooted within this contradiction, a powerful structural explanation of how a structure promotes its own demise is obtained
%the elevated role of the dollar combined with its fixed convertibility to gold prohibits the US to accrue extensive dollar liabilities, i.e. if the dollar supply drastically excedes gold reserves, confidence in convertibility may erode. However, international trade requires sufficient levels of liquidity, thus a dominant position of the dollar would require the US to increase its dollar liabilities.  
(\cite{KlimiukZbigniew2016TPaO,bordo2014tales}).  
%Hence, to root the demise of the BWS within its design, it is required to show the structural and material condition that supported the dollar ascent and to identify the causes that tipped the balance between those contradictory pressures (\cite{KlimiukZbigniew2016TPaO, bordo2014tales}). 
First, the dollar's ascent to "Top Currency" status.
%structural and material variables such as the global monetary system and the unique political and economic position of the US seem to have strong explanatory power. That is,
After the second world war (WW2) the US domitated global output, had a massive trade surplus, served as the main creditor for capitalist countries and had virtually no competition within among these countries.
% GDP amounted to $39.5\%$ of global output, thus US exports increased drastically. This in fact was so dramatic that financial aid was required to maintain exports, which contributed to the US becoming the main creditor country of capitalist countries.
%Furthermore, with previously industrious countries in shambles and their currencies constantly fluxuating, there was virtually no competition within capitalist countries. 
Moreover, in contrast to pre-WW2 US, material, structural and ideational factors aligned and led to a more internationally minded US.
That is, it used its position to construct the BWS favouring heavily the US and its position in the world. The BWS pegging the dollar to gold and the US having $70 \%$ of global gold reserves, it the dollar was soon considered as good as gold. Hence, while it was the dollar that emerged from the structural condition after WW2 as the currency for international trade, its significance was virtually enshrined into the global economic structure by the BWS (\cite{KlimiukZbigniew2016TPaO,bordo2014tales,kindleberger1986world}; \cite[p.~107]{CohenBenjaminJ2004Tfom}; \cite[p.~229-230]{BeaudMichel1983Ahoc};  \cite[p.~34]{1993Arot}).
Therefore, after an initially dollar shortage, other countries started to build up dollar reserves. Due to the structure of the economy and the monetary system other countries simply preferred to hold dollar reserves. There was this idea that the dollar was as good as gold.
The US used this structural advantage, to build up dollar liabilities rather than adjusting domestic policies, thus exporting inflation to surplus counties. However, starting under John F. Kennedy and continuing under Lyndon B. Johnson, dollar liabilities increased drastically while US gold reserves continued to dwindle. Much to the discontent of Charles DeGaulle who tried fight against France's structural disadvantages in the BWS by promoting the systematic conversion of dollar liabilities. Leading to a further decrease of US gold reserves. 
%However, the position of the dollar within the global monetary system allowed for a tacit agreement among countries that dollar reserves would not be exchanged for gold, thus creating a de-facto dollar standard. 
%The US was aware of arrangement and used its position to build up dollar liabilities rather than adjusting domestic policies, thus exporting inflation to surplus counties. However, starting under John F. Kennedy and continuing under Lyndon B. Johnson, dollar liabilities increased drastically while US gold reserves continued to decrease, in order finance domestic issues such as tax reductions, the War on Poverty and the Great Society project or to finance military projects such as the Vietnam war. Much to the discontent of Charles De Gaulle who tried fight against France's structural disadvantages in the BWS by braking this tacit agreement and promoting the systematic conversion of dollar liabilities, thus further decreasing the US gold reserves. 
In the early 1970s there was such a massive disequilibrium between US gold reserves and foreign dollar reserves that confidence in convertibility was virtually non-existent. Therefore, Richard Nixon was forced by concrete prospects of a gold run, i.e. structural factors imposted by the global economy, to suspend dollar convertibility. This so called Nixon Shock may not have been the end of the BWS, however, it was the decisive blow leading to its demise a view years later 
(\cite{KlimiukZbigniew2016TPaO,bordo2014tales, }; \cite[p.~229-230]{BeaudMichel1983Ahoc};  \cite[p.~73]{1993Arot}; \cite[p.~197]{JonesDanielStedman2014Motu}). 
%Certainly agents, such as DeGaulle served as a catalyst for the BWS's demise.  
%It would be hubris to claim that the space for agency was restricted so severely, such that an arbitrary substitution of agents could be employed. 
As demonstrated the inability of the US to hold the delicate balance of dollar liabilities to gold reserves within a changing global economy. The force of these structural interdependencies becomes especially apparent if one considered the incredibly beneficial initial position of the US. Therefore, while agents were not insignificant, e.g. DeGaulle, it was the structure of the BWS, which due to its internal contradictions that promoted its own demise. \\


The second piece of the puzzle, requires one to establish a connection between the structural conditions created by the BWS and an decline of confidence in Keynesian economics. This is, the collapse of the BWS promoted a rise in oil prices, which then elevated the stagflation, i.e. inflation and unemployment, leading to a reduced confidence in Keynesian economics.
Firstly, due to its position within the global economic structure, most oil prices were denominated in dollar. Therefore, the conditions after the Nixon shock, i.e. a significant devaluation of the dollar and a rise in inflation, drastically reduced the profitability of oil. Furthermore, an incredible surge of commodity prices, including gold, further threatened profitability. Hence, to secure its material needs, i.e. to retain profitability, the Organization of Petroleum Exporting Countries was strongly compelled by the international market structure, to stabilise the falling real price of oil.
(\cite[p.~231]{BeaudMichel1983Ahoc}; \cite{hammes2005black})
Secondly, the stagflation in the 1970s may have been caused by expansionist monetary policy and inflexible labour markets or by the oil shocks described above (\cite{hunt2006oil,barsky2001we}; \cite[p.~192]{OlsonMancur1982Trad}). In the latter case the desired structural connection follows, from the argument that an increase in oil prices increases production cost, thus leading to inflation. In the prior case one can argue that, the inflationary expectations pressures labour to seek wage increases, thus reducing the willingness of companies to employ labour and capital and since, inflation and an overvalued dollar were common during the end of the BWS, the relation is back to the structure of the BWS is partly secured.
%the fall in real wages in the US can be  and the inflationary expectations, can partly be attributed to the oil shocks, to the devaluation of the dollar or inflation exported during the Bretton Woods Era for other countries. 
Moreover, since one of the objectives of the BWS was full employment, it allowed the formation of strong labour unions, which subsequently resisted the real wage reduction. Hence, we have shown for both explanations that the BWS elevated the stagflation of the 1970s. (\cite[p.~214]{JonesDanielStedman2014Motu}; \cite[p.~224]{BeaudMichel1983Ahoc}; \cite{KlimiukZbigniew2016TPaO}). 
Thirdly, in Keynesian economic theory stagflation was a theoretical impossibility. Therefore, the stagflation of the 1970s, which was encouraged by the structural and material conditions of the collapse of the BWS, led to a questioning of the Keynesian macro economic model. Moreover, the Philips curve, an indispensable part of Keynesian economics, increasingly amounted criticism during that time. 
%These two inadequacies promoted the radical shift towards a new-classical economic theory and monetarism promoted by prominent figures associated with the neoliberal shift.
%, e.g. Friedrich A. Hayek and Milton Friedman. 
Moreover, the policies based on Keynesian economics actually invigorated the crisis. Therefore, the failure of old ideas and structures created an ideational and policy vacuum in which political leader and academic were left craving for new ideas    
(\cite[p.~192]{OlsonMancur1982Trad}; \cite[p.~148-149]{TrevithickJamesAnthony1992Ium}
\cite[p.~215-218]{JonesDanielStedman2014Motu}). \\

The last step is to explain why the structural conditions created by the BWS allowed neoliberal ideas to gain prominence. 
%To do so the arguments are yet again focused around structural and material conditions imposed upon agents by the market. However, if contrasted with the previous arguments one can observe interdependence of powerful strutural and agentic variables. 
Hayek, a prominent figure of neoliberalsim, was dissatisfied with the emerging world order, e.g. he saw exchange rate control as tyranny of the state.
In the late 1940s he founded a network of think tanks, to combat ideas from the left within academia. Even though, this network started to grow, neoliberal thought was still in a minority position at the start of the 1970s. Only obtaining dominance at start of the 1980s.
(\cite{sally2000hayek}; \cite[p.~77-78/178-179]{JonesDanielStedman2014Motu}).
What changed? Let's backtrack.
The BWS, with its focus on maintaining full employment, invigorating labour and enabling drastic wage increases. 
%Initially his was matched with productivity gains, thus allowing for an increasing rate of profit until the mid of the 1960s. 
Decreasing productivity and demand and increasing inflation and competition, as well as increasing regulation together with the economic turmoil of the 1970s, resulted in a declining rate of profit across several countries. 
%In the US this was amplified by an overvalued dollar and an increase in regulation.
%Additionally, if one considers the economic turmoil of the 1970s it becomes apparent that capital saw an erosion in its ability to extract surplus value.
Slave to market forces and their demand for profit, businesses started to support neoliberal ideas. As observed by the the drastic increase of funding for neoliberal think tanks, thus increasing funding for literature promoting such ideas.
%towards neoliberal pThis provided funding for journalist and academics, which wanted to write articles about the benefits of the free market, thus slowly popularising neoliberal ideas. 
Particularly, in the stagflationary period the monetarist theory and policies of Milton Friedman, seriously challenged the failing Keynesian orthodoxy. The described desire for alternatives in a precerious economic situation, the sharp rise in influence of neoliberal think tanks with their connection to journalist and politicians culminated in implementation of the first neoliberal policies. Which then were later enshrined during the terms of Reagan and Thatcher, both of whom were active or strongly connected to the introduced network of think tanks.
(\cite[p.~223-228]{BeaudMichel1983Ahoc}; \cite[p.~63-69/77-80]{KotzDavidMDavidMichael2015Traf}; \cite[p.~ix]{1977TEot}; \cite[p.~173-174/210-211/254]{JonesDanielStedman2014Motu})
Here a beautiful interplay between structural, agentic, ideational and material variables 
can be observed. Hayek, the initial figurehead of neoliberal thought, dissatisfied with the initial structure used his agency and his conviction in the power of ideas to create a structure with the explicit aim to change the ideational landscape. However, this alone was insufficient, it required the structure generated by the BWS and its collapse to constrain material interests to such an extend that the attempts for their preservation served as a vital catalyst for the ideas festering in obscurity to emerge, thereby creating a new structural reality. \\

To conclude, the structure of the BWS required the US to maintain a balance between dollar supply and gold reserves, which given increasing international trade and a limited gold supply became increasingly impossible. Moreover, the BWS invigorated labour and promoted a falling rate of profit, leading to an increased funding in neoliberal think tanks, by business interest groups. Finally, after its collapse and in conjunction with market forces a precarious economic situation arose, which was claimed to impossible according to economic orthodoxy. Therefore, opening up space for neolibaral though, which was gestating within these think tanks. Hence, structural, material and to some extent ideational variables were vital in promoting the shift towards neoliberalism.

\bibliographystyle{plain}
\bibliography{irl_paper}

\end{document}
