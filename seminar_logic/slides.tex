\documentclass[8pt]{beamer}
\usepackage{amsmath}
\usepackage{amssymb}
\usepackage{amsfonts}
\usepackage{enumitem}
\usepackage{amsthm}
\usepackage{import}
\usepackage{MnSymbol}
\usepackage{graphicx}
\setlength{\parindent}{0pt}
\usepackage[utf8]{inputenc}
\usepackage{listings} [python]
\usepackage{bussproofs}
%\usepackage{MnSymbol}
\usepackage{stmaryrd}
\usepackage{adjustbox}
\usepackage{fancybox}
\usepackage{tikz}

\usepackage{comment}

%\usepackage[square]{natbib}

\usepackage[utf8]{inputenc}


\usepackage[backend=biber, style=authoryear]{biblatex}

\newtheorem{thm}{Theorem}[section]
\newtheorem{cor}[thm]{Corollary}
\newtheorem{lem}[thm]{Lemma}
\newtheorem{mydef}[thm]{Definition}

\newcommand\altfootnote[1]{%
  \tikz[remember picture,overlay]
  \draw (current page.south west) +(1in + \oddsidemargin,0.5em)
  node[anchor=south west,inner sep=0pt]{\parbox{\textwidth}{%
      \rlap{\rule{10em}{0.4pt}}\raggedright\scriptsize#1}};}

\newcommand{\myfootnote}[1]{
    \renewcommand{\thefootnote}{}
    \footnotetext{\hspace{-4pt}\scriptsize#1}
    \renewcommand{\thefootnote}{\arabic{footnote}}
}

\newcommand*{\skepcon}{\ensuremath{\mathrel{\medvert\mskip-5.7mu\clipbox{1 0 0 0}{$\sim$}}}}

\usetheme[progressbar=frametitle]{metropolis}
\usepackage{appendixnumberbeamer}

\usepackage{booktabs}
\usepackage[scale=2]{ccicons}

\usepackage{pgfplots}
\usepgfplotslibrary{dateplot}

\usepackage{xspace}
\newcommand{\themename}{\textbf{\textsc{metropolis}}\xspace}

\title{Seminar in Logic}
\subtitle{Non-Monotonic Reasoning}
% \date{\today}
\date{}
\author{Konstantin Kueffner}
% \titlegraphic{\hfill\includegraphics[height=1.5cm]{logo.pdf}}



%\bibliographystyle{unsrtnat}
\addbibresource{seminar_logic.bib}


\begin{document}



\maketitle

\begin{frame}{Table of contents}
  \setbeamertemplate{section in toc}[sections numbered]
  \tableofcontents[subsubsectionstyle=hide]
\end{frame}


\section{Introduction}
\subsection{Monotonic Reasoning}

\begin{frame}{Monotonic Reasoning}
\metroset{block=fill}
\begin{block}{Monotonicity (Intuition)}
Additional information can not invalidate previous conclusions.
\end{block}

\begin{block}{Global Monotonicity}   
A logic is called globally monotonic if for $A$ and $B$ being theories, such that $A \subseteq B$ it follows that $Th(A) \subseteq Th(B)$, with $Th(S):=\{p \mid S \models p \}$ (or syntactically $Th(S)=\{p \mid S \vdash p \}$)
\end{block}
For example: $\Gamma \models \varphi$ implies $\Gamma \cup \{\psi\} \models \varphi$ or \emph{weak-l} in Sequent Calculus.

\begin{block}{Local Monotonicity}    
A logic is called locally monotonic, if it allows for "Strengthening the Antecedent".                    
\end{block} 
For example: In propositional logic it is a tautology $(A \to C) \to (A \wedge B \to C)$.
\myfootnote{\cite{bochman2005explanatory,mcdermott1980non, mccarthy1981circumscription}s}
\end{frame}

\subsection{Motivating Problems}


\subsubsection{Tweety-Problem}

\begin{frame}{Motivating Problems: Tweety Problem}

\begin{block}{Stage 1: Naive Model}
\begin{equation*}
\begin{split}
\mathcal{T}_1:= \{ \forall x \; Person(x) \to Innocent(x), Person(Tweety) \} 
\end{split}
\end{equation*}
\end{block}
\begin{block}{Stage 2: Expansion 1}
\begin{equation*}
\begin{split}
\mathcal{T}_2:= \mathcal{T}_1   \cup \{& Murderer(Tweety),\\
&\forall x \; Murderer(x) \to Person(x), \\
&\forall x \; Murderer(x) \to \neg Innocent(x)\}
\end{split}
\end{equation*}
\end{block}
\begin{block}{Stage 3: Naive Solution}
\begin{equation*}
\begin{split}
\mathcal{T}_3:= \mathcal{T}_2 &\setminus \{ \forall x \; Person(x) \to Innocent(x)\} \\
& \cup \{ \forall x \; Person(x) \wedge \neg Murderer(x) \to Innocent(x)\} \\
\end{split}
\end{equation*}
\end{block} 

\end{frame}

\begin{frame}{Motivating Problems: Tweety Problem 2}
\begin{block}{Stage 4: Expansion 2}
\begin{equation*}
\begin{split}
\mathcal{T}_4:= \mathcal{T}_3   \cup \{& Person(Polly)\}\\
\end{split}
\end{equation*}
\end{block}

But now 
\begin{equation*}
\begin{split}
&\mathcal{T}_4 \nmodels Innocent(Polly)\\
&\mathcal{T}_4 \cup \{\neg Murderer(Polly)\} \models Innocent(Polly)\\
\end{split}
\end{equation*}

Hence,
\begin{center}
\emph{Innocent when proven to be not a murderer?}
\end{center}

\textbf{Problems:}
\begin{itemize}
\item - listing all exceptions (not yet classified crimes)
\item - expansion may lead to inconsistency
\item - checking all exceptions (absence of complete information)


\end{itemize}

\myfootnote{\cite{reiter1980logic,mccarthy1981circumscription}}
\end{frame}

\subsubsection{General Frame-Problem}

\begin{frame}{Motivating Problems: Frame Problem}

\begin{block}{Temporal Projection Problem}
\begin{itemize}
\item - Frame / Persistence Problem $\to$ "What does not change?"
\item - Ramification Problem $\to$ "What changes implicitly?"
\item - Qualification Problem $\to$ "When is an action possible?"
\end{itemize}
\end{block} 
\myfootnote{\cite{ginsberg1987reasoning,BOCHMAN2007557,nonmonoton_stanford2018,stanford2016frame,ginsberg1987reasoning}}
\end{frame}

\begin{comment}
\begin{block}{Naive Model}
\begin{equation*}
\begin{split}
&\forall x \; (PAINT(x,c) \to COLOUR_{n+1}(x,c)) \\
&\forall x \; (MOVE(x,p) \to POSITION_{n+1}(x,p)) \\
\end{split}
\end{equation*}
\end{block} 

\begin{block}{Inertia}
\begin{equation*}
\begin{split}
&\forall x \; ((MOVE(x,p) \wedge COLOUR_{n}(x,c)) \to COLOUR_{n+1}(x,c)) \\
&\forall x \; ((PAINT(x,p) \wedge POSITION_{n}(x,p)) \to POSITION_{n+1}(x,p)) \\
\end{split}
\end{equation*}
\end{block} 
\end{comment}


\begin{frame}{Why Non-Monotonic Reasoning?}

\begin{block}{What is desired?}
Capture statements such as:
\begin{itemize}
\item - Normally X holds
\item - Typically X is the case
\item - Assume X as a default
\end{itemize}
\end{block}


In order to 
\begin{itemize}
\item - modelling normality and abnormality
\item - modelling a reasoned use of assumptions
\item - distinguish between information
\item - allow for safe reasoning in a dynamic environment
\end{itemize}

\myfootnote{\cite{nonmonoton_stanford2018,BOCHMAN2007557}}
\end{frame}


\section{Approaches to Non-Monotonic Reasoning}
\begin{frame}{General Approaches to Non-Monotonic Reasoning }


\begin{block}{Preferential Non-Monotonic Reasoning}    
\begin{itemize}
\item Type: locally non-monotonic (possibly globally monotonic)
\item Core Concept: Cumulative / Preferential Models
\item Formalisms: Circumscription, Closed World Assumption, Conditional Logic
\item (Strongly connected to Meta-Theoretic Approach)
\end{itemize}
\end{block} 


\begin{block}{Explanatory Non-Monotonic Reasoning}    
\begin{itemize}
\item Type: globally non-monotonic (possibly locally monotonic)
\item Core Concept: Explanatory Closure / Stable Extensions
\item Formalisms: Default Logic, Modal Non-Monotonic Logic 
\end{itemize}
\end{block} 

\begin{block}{Other methods of characterisation}    
\begin{itemize}
\item logic formulas vs. inference rules
\item credulous vs. sceptical inference
\item cumulative vs. not-cumulative
\end{itemize}
\end{block} 
\myfootnote{\cite{bochman2005explanatory,BOCHMAN2007557,brewka1997nonmonotonic}}
\end{frame}



\subsection{Preferential Non-Monotonic Reasoning}
\subsubsection{Core Concept}
\begin{frame}{Preferential Non-Monotonic Reasoning: Core Concept}
\metroset{block=fill}
\begin{block}{Cumulative Models}
$\langle S,l, \prec \rangle$ where 
\begin{itemize}
\item  - $S$ states
\item  - $l:S \to \powerset(\mathcal{U})$ with $\mathcal{U}$ set of all interpretations
\item  - $\prec$ is a strict partial order on $S$ satisfying the smoothness condition.
\end{itemize} 
\end{block}

\begin{block}{Preferential Models}
$\langle S,l, \prec \rangle$ cumulative where $|l(s)|=1$
\end{block}
\begin{block}{Model Preference Logics}
Let $W:=\langle S,l, \prec \rangle$ be a preferential model, such that 
\begin{itemize}
\item - $S$ is a subset of all interpretations, i.e. $S \subseteq \mathcal{U}$ 
\item - $l$ is the identity
\item - $\prec$ is well-founded
\end{itemize}
\end{block}
\myfootnote{\cite{kraus1990nonmonotonic,brewka1997nonmonotonic}}
\end{frame}
\subsection{Explanatory Non-Monotonic Reasoning}

\subsubsection{Core Concept}


\begin{frame}{Explanatory Non-Monotonic Reasoning: Core Concept}
\metroset{block=fill}

\begin{block}{Explanation Closure (Intuition)}
Any fact holding in a model should be explained by the rules of the domain.
\end{block}

\begin{block}{Stable Extensions (Intuition)}
A \emph{Stable Extension} is explanatory closed set of formulas representing one possible set of consistent beliefs. 
\end{block}
Commonly:
\begin{itemize}
\item - multiple stable extensions exist
\item - defined by a fixed-point operation
\begin{itemize}
\item For operator $T$,  $S$ is a fixed-point iff $T(S)=S$
\end{itemize}
\end{itemize}

\myfootnote{\cite{bochman2005explanatory,BOCHMAN2007557,brewka1997nonmonotonic}}

\end{frame}


\begin{comment}

\subsubsection{Meta Theoretic Approach}
\begin{frame}{Preferential Non-Monotonic Reasoning: Meta Theoretic Approach}
Important meta-theoretical properties:
\begin{prooftree}
\AxiomC{$\varphi \skepcon \psi$}
\AxiomC{$\varphi \skepcon \chi$}
\RightLabel{\scriptsize(Cautious Monotonicity)}
\BinaryInfC{$\varphi \wedge \psi \skepcon \chi$}
\end{prooftree}

allows for the formalisation of \emph{Lemmas} and 

\begin{prooftree}
\AxiomC{$\varphi \skepcon \chi$}
\AxiomC{$\psi \skepcon \chi$}
\RightLabel{\scriptsize(Or)}
\BinaryInfC{$\varphi \vee \psi \skepcon \chi$}
\end{prooftree}

enables the reasoning about cases. \\
\metroset{block=fill}

\begin{block}{Cumulative Models and Preferential Models}
Cumulative Models and Preferential Models captured on meta-theoretic level (System \textbf{C} and \textbf{P} = \textbf{C} + Or)
\end{block}
\begin{block}{Cumulative Models and Preferential Models}
Every model preference logic has a preferential inference relation.  
\end{block}
\myfootnote{\cite{kraus1990nonmonotonic, brewka1997nonmonotonic}}
\end{frame}

\end{comment}

\section{Specific Non-Monotonic Formalisms}
\subsection{Circumscription}

\begin{frame}{Circumscription}
\metroset{block=fill}
\begin{block}{Circumscription}
Let $S$ FOL-sentence containing $P(x_1, \dots, x_n)$, short $P(\overline{x})$. Let $S(\Phi)$ replaces $P$ with $\Phi$. \\
Then the schema
\begin{equation*}
\begin{split}
\forall \Phi \; ((S(\Phi) \wedge  \forall \overline{x} \; (\Phi(\overline{x})\to P(\overline{x}))) \to \forall \overline{x} \; (P(\overline{x}) \to \Phi(\overline{x})))
\end{split}
\end{equation*}
is called the circumscription of $P$.
\end{block}

\begin{block}{Model Theoretic Notion}
Let $\langle S,l, \prec \rangle$ be a model preference logic, s.t. 
\begin{equation*}
\begin{split}
s_1 \prec s_2  \textit{  iff  } &l(s_1) \models P(\overline{x}) \textit{ implies } l(s_2) \models P(\overline{x}) \textit{ and} \\
& \textit{not } l(s_2) \models P(\overline{x}) \textit{ implies } l(s_1) \models P(\overline{x})
\end{split}
\end{equation*}
with $P$ being circumscribed.
\end{block}
Note: Expresses normality based on \emph{Abnormality Theory}
\myfootnote{\cite{mccarthy1981circumscription,brewka1997nonmonotonic}}
\end{frame}

\begin{frame}{Circumscription: Example 1}
\metroset{block=fill}
Let $\mathcal{I}_x:=(D_{\mathcal{I}_x}, I_{\mathcal{I}_x})$ s.t. 
\begin{itemize}

\item $I_{\mathcal{I}_0}(Guilty):=\{\}$
\item $I_{\mathcal{I}_1}(Guilty):=\{\delta\}$
\item  $I_{\mathcal{I}_2}(Guilty):=\{\delta, \sigma, \eta\}$
\item $I_{\mathcal{I}_3}(Guilty):=\{\delta, \sigma, \eta, \gamma\}$
\end{itemize}

with $D_{\mathcal{I}_x}:=\{\delta, \sigma, \eta, \gamma\}$ for all x.
The following preference can be established
\begin{equation*}
\begin{split}
\mathcal{I}_0 \prec \mathcal{I}_1 \prec \mathcal{I}_2 \prec \mathcal{I}_3 
\end{split}
\end{equation*}

Given
\begin{equation*}
\begin{split}
&S = Guilty(A) \wedge Guilty(B) \wedge Guilty(C) \\
\end{split}
\end{equation*}
$\mathcal{I}_1$ is chosen.
\myfootnote{\cite{brewka1997nonmonotonic}}
\end{frame}




\begin{frame}{Circumscription: Example 2}
\metroset{block=fill}
Recall
\begin{equation*}
\begin{split}
\forall \Phi \; (S(\Phi) \wedge  \forall \overline{x} \; (\Phi(\overline{x})\to P(\overline{x}))) \to \forall \overline{x} \; (P(\overline{x}) \to \Phi(\overline{x}))
\end{split}
\end{equation*} 
Given
\begin{equation*}
\begin{split}
&S = Guilty(A) \wedge Guilty(B) \wedge Guilty(C) \\
&\Phi(x) = (x=A \vee x=B \vee x=C)
\end{split}
\end{equation*}

\begin{equation*}
\begin{split}
\Phi(A) \wedge \Phi(B) \wedge \Phi(C) \wedge \forall \overline{x} \; (\Phi(\overline{x})\to Guilty(\overline{x}))) \to \forall \overline{x} \; (Guilty(\overline{x}) \to \Phi(\overline{x}))
\end{split}
\end{equation*}

\begin{equation*}
\begin{split}
&(A=A \vee A=B \vee A=C)\wedge (B=A \vee B=B \vee B=C) \wedge (C=A \vee C=B \vee C=C) \\
& \wedge \forall \overline{x} \; ((\overline{x}=A \vee \overline{x}=B \vee \overline{x}=C)\to Guilty(\overline{x})))  \\
& \to  \forall \overline{x} \; (Guilty(\overline{x})   \to  (\overline{x}=A \vee \overline{x}=B \vee \overline{x}=C))
\end{split}
\end{equation*} 
\myfootnote{\cite{mccarthy1981circumscription}}

\end{frame}


\begin{frame}{Circumscription: Example 3}
\metroset{block=fill}


\begin{equation*}
\begin{split}
&S = Guilty(A) \wedge Guilty(B) \wedge Guilty(C) \wedge Guilty(D) \\
&\Phi(x) = (x=A \vee x=B \vee x=C)
\end{split}
\end{equation*}

\begin{equation*}
\begin{split}
&(A=A \vee A=B \vee A=C)\wedge (B=A \vee B=B \vee B=C) \\
& \wedge (C=A \vee C=B \vee C=C) \wedge (D=A \vee D=B \vee D=C) \\
& \wedge \forall \overline{x} \; ((\overline{x}=A \vee \overline{x}=B \vee \overline{x}=C)\to Guilty(\overline{x})))  \\
& \to  \forall \overline{x} \; (Guilty(\overline{x})   \to  (\overline{x}=A \vee \overline{x}=B \vee \overline{x}=C ))
\end{split}
\end{equation*}

\myfootnote{\cite{mccarthy1981circumscription}}

\end{frame}



\subsection{Default Logic}
\begin{frame}{Default Logic}
\metroset{block=fill}
\begin{block}{Default Logic}
A default $\delta$ has the form
\begin{prooftree}
\AxiomC{$\varphi: \psi_1, \dots, \psi_n$}
\UnaryInfC{$\chi$}
\end{prooftree}
with $\varphi, \chi, \psi_1, \dots, \psi_n$ being closed propositional formulas for $n>0$.
\end{block}

\begin{block}{Default Theory}
$\Delta = (D,W)$ is a default theory. With $W$ a set of predicate formulas and $D$ a set of defaults. For any $S \subseteq \mathcal{L}$, let $\Gamma(S)$ be the smallest set satisfying

\begin{itemize}
\item D1: $W \subseteq \Gamma(S)$
\item D2: $Th_{\mathcal{L}}(\Gamma(S))=\Gamma(S)$
\item D3: if $(\varphi: \psi_1, \dots, \psi_n / \chi) \in D$ and $\varphi \in  \Gamma(S)$ and $\neg \psi_1, \dots, \neg \psi_n \nin S$  then $\chi \in \Gamma(S)$.
\end{itemize}
\end{block}
\myfootnote{\cite{ANTONIOU2007517,reiter1980logic}}
\end{frame}



\begin{frame}{Default Logic: Example}
\metroset{block=fill}
Given $\Delta:=(W,D)$
\begin{equation*}
\begin{split}
 W:= \{&Murderer(Tweety), Person(Polly), \\
&  Murderer(x) \to \neg Innocent(x), Murderer(x) \to  Person(x) \}\\
 D:=\{&Person(x):Innocent(x)/Innocent(x)\}
\end{split}
\end{equation*}
Possible sets:
\begin{equation*}
\begin{split}
&E_1:= W \cup \{Innocent(Polly), Person(Tweety), \neg Innocent(Tweety)\} \\
&E_2:= W \cup \{Innocent(Polly), Person(Tweety), \neg Innocent(Tweety), Innocent(Tweety)\}
\end{split}
\end{equation*}
$E_1$ is a stable extension, $E_2$ is not. 
\myfootnote{\cite{ANTONIOU2007517}}
\end{frame}



\begin{frame}{Default Logic: Example - Credulous and Sceptical Reasoing}
\metroset{block=fill}
Given $\Delta:=(W,D)$
\begin{equation*}
\begin{split}
 W:= \{&Murderer(Tweety), Person(Polly),  Murderer(x) \to  Person(x) \}\\
 D:=\{&Person(x):Innocent(x)/Innocent(x), \\
 & Murderer(x): \neg Innocent(x)/\neg Innocent(x)\}
\end{split}
\end{equation*}

Possible sets:
\begin{equation*}
\begin{split}
&E_1:= W \cup \{Innocent(Polly), Person(Tweety), Innocent(Tweety)\}\\
&E_2:= W \cup \{Innocent(Polly), Person(Tweety), \neg Innocent(Tweety)\}\}
\end{split}
\end{equation*}
$E_1$ and $E_2$  are stable extensions. 
 
\myfootnote{\cite{nonmonoton_stanford2018,brewka1997nonmonotonic}}
\end{frame}

\section{Applications}
\subsection{Law}
\begin{frame}{Non-Monotonic Reasoning in Law}

\textbf{Areas of legal reasoning}
\begin{itemize}
\item - reasoning with laws - axiomatic view
\item - reasoning about laws - interpretation of legal rules
\item - reasoning about facts - burden of proof
\item - reasoning about interactions - dynamic disputes between agents
\item - reasoning about legal action - legality of future actions
\end{itemize}

\textbf{Argumentation Theory}
\begin{itemize}
\item central to law
\item abstract argumentation theory
\begin{itemize}
\item - non-monotonic
\item - notion of argument, attack and defeat
\item - semantics for Default Logic or logic programs
\end{itemize}
\end{itemize}

\myfootnote{\cite{prakken2015law, Prakken2017,BOCHMAN2007557}}

\end{frame}


\begin{frame}{Non-Monotonic Reasoning in Law}


\textbf{Rules with exceptions}
\begin{itemize}
\item Exceptions and Context
\begin{itemize}
\item - higher force
\item - self-defence
\end{itemize}
\end{itemize}

\textbf{Rules with conflicting conclusions}
\begin{itemize}
\item dynamic hierarchies of preference
\begin{itemize}
\item - specificity, authority and recency
\end{itemize}
\item Possible application: preference default logic
\end{itemize}

\myfootnote{\cite{prakken2015law, Prakken2017,ANTONIOU2007517}  }
\end{frame}


\subsection{Other Applications}
\begin{frame}{Other Applications of Non-Monotonic Reasoning}
\textbf{Computer Science}
\begin{itemize}
\item - Truth Maintenance Systems
\item - Normal Logic Programs
\item - Database Theory
\end{itemize}

\textbf{Cognitive Sciences}
\begin{itemize}
\item - closer approximation of human reasoning
\item - possible bridge between symbolic and connectionist approaches
\end{itemize}
\textbf{Biology}
\begin{itemize}
\item - exception hierarchies
\end{itemize}
\myfootnote{\cite{isaac2014logic,evans2002logic,BOCHMAN2007557,ANTONIOU2007517}}
\end{frame}

\printbibliography

\begin{frame}{Smoothness Condition}
\metroset{block=fill}
\begin{block}{Smoothness Condition}
$\langle S, l \prec \rangle$ satisfies smoothness condition if $\forall \alpha \in \mathcal{L}$ the set $ \{s \in S \mid \forall m \in l(s)  \; m \models \alpha\}$ is smooth, i.e. every state is either minimal or is in relation to a minimal state.
\end{block}
\myfootnote{\cite{kraus1990nonmonotonic}}
\end{frame}








\end{document}
