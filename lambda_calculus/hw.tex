\documentclass{article}
\usepackage{amsmath}
\usepackage{amssymb}
\usepackage{amsfonts}
\usepackage{enumitem}
\usepackage{amsthm}
\usepackage{import}
\usepackage{MnSymbol}
\usepackage{graphicx}
\setlength{\parindent}{0pt}
\usepackage[utf8]{inputenc}
\usepackage{listings} [python]
\usepackage{bussproofs}
\usepackage{verbatim}
\usepackage{tikz}
\usepackage[utf8]{inputenc}
\usetikzlibrary{arrows, automata}
\usepackage{booktabs}
\usepackage{rotating}
\usepackage[style=ieee]{biblatex}
\usepackage{rotating}

\newtheorem{thm}{Theorem}[section]
\newtheorem{cor}[thm]{Corollary}
\newtheorem{lem}[thm]{Lemma}
\newtheorem{mydef}[thm]{Definition}



\makeatletter
\def\section{\@ifstar\unnumberedsection\numberedsection}
\def\numberedsection{\@ifnextchar[%]
  \numberedsectionwithtwoarguments\numberedsectionwithoneargument}
\def\unnumberedsection{\@ifnextchar[%]
  \unnumberedsectionwithtwoarguments\unnumberedsectionwithoneargument}
\def\numberedsectionwithoneargument#1{\numberedsectionwithtwoarguments[#1]{#1}}
\def\unnumberedsectionwithoneargument#1{\unnumberedsectionwithtwoarguments[#1]{#1}}
\def\numberedsectionwithtwoarguments[#1]#2{%
  \ifhmode\par\fi
  \removelastskip
  \vskip 3ex\goodbreak
  \refstepcounter{section}%
  \hbox to \hsize{%
    \vtop{\parindent=0pt \leavevmode\Large\bfseries\raggedright #2\quad\thesection\par}%
    }
  \vskip 2ex\nobreak\noindent%
  \addcontentsline{toc}{section}{%
    \protect\numberline{\thesection}%
    #1}%
  \ignorespaces}
\def\unnumberedsectionwithtwoarguments[#1]#2{%
  \ifhmode\par\fi
  \removelastskip
  \vskip 3ex\goodbreak
  \hbox to \hsize{%
    \vtop{\parindent=0pt\leavevmode\Large\bfseries\raggedright #2\par}%
    }
  \vskip 2ex\nobreak\noindent%
  \addcontentsline{toc}{section}{%
    #1}%
  \ignorespaces}
\makeatother


\makeatletter
\def\subsection{\@ifstar\unnumberedsubsection\numberedsubsection}
\def\numberedsubsection{\@ifnextchar[%]
  \numberedsubsectionwithtwoarguments\numberedsubsectionwithoneargument}
\def\unnumberedsubsection{\@ifnextchar[%]
  \unnumberedsubsectionwithtwoarguments\unnumberedsubsectionwithoneargument}
\def\numberedsubsectionwithoneargument#1{\numberedsubsectionwithtwoarguments[#1]{#1}}
\def\unnumberedsubsectionwithoneargument#1{\unnumberedsubsectionwithtwoarguments[#1]{#1}}
\def\numberedsubsectionwithtwoarguments[#1]#2{%
  \ifhmode\par\fi
  \removelastskip
  \vskip 3ex\goodbreak
  \refstepcounter{subsection}%
  \hbox to \hsize{%
    \vtop{\parindent=0pt \leavevmode\Large\bfseries\raggedright #2\quad\thesubsection\par}%
    }
  \vskip 2ex\nobreak\noindent%
  \addcontentsline{toc}{subsection}{%
    \protect\numberline{\thesubsection}%
    #1}%
  \ignorespaces}
\def\unnumberedsubsectionwithtwoarguments[#1]#2{%
  \ifhmode\par\fi
  \removelastskip
  \vskip 3ex\goodbreak
  \hbox to \hsize{%
    \vtop{\parindent=0pt\leavevmode\Large\bfseries\raggedright #2\par}%
    }
  \vskip 2ex\nobreak\noindent%
  \addcontentsline{toc}{subsection}{%
    #1}%
  \ignorespaces}
\makeatother







\begin{document}

%\maketitle
\section{Homework}
\subsection{Exercise}
Define $l:=\lambda x \lambda y \, (x \; y)$, $r:=\lambda x \lambda y \, (y \; x)$. Then write a term $\textit{if-then-else}$ such that for all $u$, $v$
\begin{equation*}
\begin{split}
(\textit{if-then-else})\; l \; u \; v \mapsto u \\
(\textit{if-then-else})\; r \; u \; v \mapsto v \\
\end{split}
\end{equation*}
\paragraph{Solution:} One possible definition for $(\textit{if-then-else})$ is 
\begin{equation*}
\begin{split}
(\textit{if-then-else}):= \lambda o \lambda u \lambda v \, (o \; False \; True \; True \; u \; v)
\end{split}
\end{equation*}
with $True:= \lambda x \lambda y \; x$ and $False:= \lambda x \lambda y \;y$ as defined in the lecture.

We can observe for arbitrary $\lambda$-terms $u$, $v$ that 
\begin{equation*}
\begin{split}
&(\textit{if-then-else})\; l \; u \; v = (\lambda o \lambda u \lambda v \,(o \; False \; True \; True \; u \; v)) \; l \; u \; v \\
&\mapsto^* l \; False \; True \; True \; u \; v = (\lambda x \lambda y \;(x  \;  y))\;False \; True \; True \; u \; v \\
&\mapsto^* False \; True \; True \; u \; v = (\lambda x \lambda y \; y) \;True \; True \; u \; v \\
&\mapsto^* True \; u \; v = (\lambda x \lambda y \; x)\; u \; v \\
&\mapsto^*  u \\
\end{split}
\end{equation*}
and
\begin{equation*}
\begin{split}
&(\textit{if-then-else})\; r \; u \; v = (\lambda o \lambda u \lambda v \,(o \; False \; True \; True \; u \; v)) \; r \; u \; v \\
&\mapsto^* r \; False \; True \; True \; u \; v = (\lambda x \lambda y \;(y  \;  x))\;False \; True \; True \; u \; v \\
&\mapsto^* True \; False \; True \; u \; v = (\lambda x \lambda y \; x) \;False \; True \; u \; v \\
&\mapsto^* False \; u \; v = (\lambda x \lambda y \; y)\; u \; v \\
&\mapsto^*  v \\
\end{split}
\end{equation*}
Hence, the term $(\textit{if-then-else})$ exhibits the desired behaviour.

\subsection{Exercise}
Write a $\lambda$-term $\textit{Smaller}$ such that 
\begin{equation*}
\begin{split}
\textit{Smaller}\; \overline{n} \; \overline{m} &\mapsto True \quad \textit{ if } n < m \\
\textit{Smaller}\; \overline{n} \; \overline{m} &\mapsto False \quad \textit{otw.} \\
\end{split}
\end{equation*}
\paragraph{Solution:} Firstly, housekeeping. It is assumed that, as presented in the lecture, $\overline{n}$ and $\overline{m}$ refer to Church-numerals. Moreover, it is assumed that "$<$" refers to the smaller relation in $\mathbb{N}$.

Secondly, the general idea of the presented approach will be discussed. We shall construct the smaller relation as follows. 
\begin{equation*}
\textit{not}(\textit{isZero}(\dotminus (x,y)))
\end{equation*}
with  $\textit{not}$ being defined as 
\begin{equation*}
\textit{not}(x):=
\begin{cases}
0 \quad &\textit{if }x=1 \\
1 \quad &\textit{if }x=0 \\
undef. \quad &\textit{otw.} \\
\end{cases}
\end{equation*}
with $\textit{isZero}$ being defined as 
\begin{equation*}
\textit{isZero}(x):=
\begin{cases}
1 \quad &\textit{if }x=0 \\
0 \quad &\textit{otw.}
\end{cases}
\end{equation*}
and with $\dotminus$ being defined as 
\begin{equation*}
\dotminus(x,y):=
\begin{cases}
x-y \quad &\textit{if }x>y \\
0 \quad &\textit{otw.}
\end{cases}
\end{equation*}
Hence, the following behaviour can be observed. 

If $x \leq y$ the term evaluates to 
\begin{equation*}
\textit{not}(\textit{isZero}(\dotminus (x,y)))= \textit{not}(\textit{isZero}(0)) =  \textit{not}(1) =  0  
\end{equation*}
and if $x>y$ we have $x-y>0$ resulting in
\begin{equation*}
\textit{not}(\textit{isZero}(\dotminus (x,y)))= \textit{not}(\textit{isZero}(x-y)) =  \textit{not}(0) = 1  
\end{equation*}

Thirdly, we have to translate this intuition into $\lambda$-terms. Before we do so, however, a small remark regarding notation. That is, a $\lambda$-term of the shape $f^n \; x$ represents $\underbrace{f(f \cdots(f \; }_{\text{n-times}}x) \cdots)$. 

We start the translation with $\dotminus$. To capture the behaviour
\begin{equation*}
\begin{split}
\dotminus \; \overline{x} \; \overline{y} &\mapsto \overline{x-y} \quad \textit{  if } x>y \\
\dotminus \; \overline{x} \; \overline{y} &\mapsto \overline{0} \quad \quad \quad \textit{otw }\\
\end{split}
\end{equation*}
we define $\dotminus := \lambda x \lambda y \,(y \; \textit{Pred} \; x)$, with $\textit{Pred}$ as defined in the lecture and thus obtain for the arbitrary Church-numerals $\overline{n}$ and $\overline{m}$ 

\begin{equation*}
\begin{split}
\dotminus \; \overline{n} \; \overline{m} &= (\lambda x \lambda y \,(y \; \textit{Pred} \; x)) \; \overline{n} \; \overline{m} \\
& \mapsto^* \overline{m} \; \textit{Pred} \; \overline{n} = (\lambda f \lambda x \, (f^m \; x)) \; \textit{Pred} \; \overline{n} \\
& \mapsto^* \textit{Pred}^m \; \overline{n} =  \textit{Pred}^m \; (\lambda f \lambda x \, (f^n \; x)) \\
\end{split}
\end{equation*}
From there we have two cases to account for.
if $n>m$, then
\begin{equation*}
\mapsto^* \textit{Pred}^{m-m} \; (\lambda f \lambda x \, (f^{n-m} \; x) = \lambda f \lambda x \, (f^{n-m} \; x) = \overline{n-m}
\end{equation*}
and if $n \leq m$, then
\begin{equation*}
\begin{split}
&\mapsto^* \textit{Pred}^{m-n} \;( \lambda f \lambda x \, (f^{n-n} \; x)) = \textit{Pred}^{m-n} \; (\lambda f \lambda x \, x)
=  \textit{Pred}^{m-n} \; \overline{0} \mapsto^* \overline{0}\\
 \end{split}
\end{equation*}
because $Pred \; \overline{0} \mapsto^* \overline{0}$.

Now, moving on towards $\textit{isZero}$. Here we simply use the definition provided in the lecture. That is,
\begin{equation*}
\begin{split}
\textit{isZero} \; \overline{0}  &\mapsto True\\
\textit{isZero} \; \overline{n+1} &\mapsto False \\
\end{split}
\end{equation*}
with 
\begin{equation*}
\textit{isZero}:= \lambda n \, n \;(\lambda z \, False)\; True
\end{equation*}
Moreover, for $\textit{not}$ we define 

\begin{equation*}
\begin{split}
\textit{not} \; True &\mapsto False\\
\textit{not} \; False  &\mapsto True \\
\end{split}
\end{equation*}
with 

\begin{equation*}
\textit{not}:= \lambda o \lambda u \lambda v \, (o \; v \; u)
\end{equation*}
Allowing us to observe

\begin{equation*}
\begin{split}
\textit{not} \; True &= (\lambda o \lambda u \lambda v \, (o \; v \; u))\; True \mapsto  \lambda u \lambda v \, (True \; v \; u) \\
 &= \lambda u \lambda v \, ((\lambda x \lambda y \, x) \; v \; u) \mapsto^* \lambda u \lambda v \,  v = False
 \end{split}
\end{equation*}

and


\begin{equation*}
\begin{split}
\textit{not} \; False &= (\lambda o \lambda u \lambda v \, (o \; v \; u))\; False \mapsto  \lambda u \lambda v \, (False \; v \; u) \\
 &= \lambda u \lambda v \, ((\lambda x \lambda y \, y) \; v \; u) \mapsto^* \lambda u \lambda v \,  u = True
 \end{split}
\end{equation*}

Lastly, we have to combine these terms to create 
\begin{equation*}
\textit{Smaller}:= \lambda n \lambda m \, (\textit{not}\; (\textit{isZero}\; (\dotminus \; m \; n)))
\end{equation*}
Hence, for any Church-numerals $\overline{n}$ and $\overline{m}$ we have

\begin{equation*}
\begin{split}
&\textit{Smaller} \; \overline{n} \; \overline{m}= (\lambda n \lambda m \, (\textit{not}\; (\textit{isZero}\; (\dotminus \; m \; n)))) \; \overline{n} \; \overline{m} \\
&\mapsto^* \textit{not}\; (\textit{isZero}\; (\dotminus \; \overline{m}\; \overline{n})) =  \textit{not}\; (\textit{isZero}\; ((\lambda x \lambda y \, y \; \textit{Pred} \; x) \; \overline{m}\; \overline{n})) \\
&\mapsto^* \textit{not}\; (\textit{isZero}\; (\overline{n} \; \textit{Pred} \; \overline{m})) = \textit{not}\; (\textit{isZero}\; ((\lambda f \lambda x \, (f^n \; x)) \; \textit{Pred} \; \overline{m})) \\
&\mapsto^* \textit{not}\; (\textit{isZero}\; (\textit{Pred}^n \; \overline{m})) \mapsto^* \textit{not}\;(\textit{isZero}\; \overline{m \dotminus n}) \\
 \end{split}
\end{equation*}
Now we are again confronted with two cases. That is, in the case $m \leq n$ it follows that $m-n\leq 0$ and thus $\overline{m \dotminus n}=\overline{0}$ resulting in

\begin{equation*}
\begin{split}
= \textit{not}\;(\textit{isZero}\; \overline{0}) &\mapsto^*  not\; True = (\lambda o \lambda u \lambda v \, (o \; v \; u))\; True \\
&\mapsto  \lambda u \lambda v \, (True \; v \; u) \mapsto \lambda u \lambda v \, v = False
 \end{split}
\end{equation*}

and in the case $n<m$ we have $\overline{m \dotminus n}=\overline{m - n} \neq \overline{0}$ resulting in
\\
\begin{equation*}
\begin{split}
= \textit{not}\;(\textit{isZero}\; \overline{m - n}) &\mapsto^*  not\; False = (\lambda o \lambda u \lambda v \, (o \; v \; u))\; False \\
&\mapsto  \lambda u \lambda v \, (False \; v \; u) \mapsto \lambda u \lambda v \, u = True
 \end{split}
\end{equation*}
which is exactly the desired behaviour of $\textit{Smaller}$.


\subsection{Exercise}
Define for every $n\in \mathbb{N}$
\begin{equation*}
\begin{split}
\underline{0}&:= \lambda x \lambda f \, x \\
\underline{n+1}&:= \lambda x \lambda f \, (f\; \underline{n}\; (f \;  \underline{n-1}\;(\cdots (f\; \underline{0}\; x) \cdots)))
 \end{split}
\end{equation*}

Examples:
\begin{equation*}
\begin{split}
\underline{1}&:= \lambda x \lambda f \, (f \; \underline{0} \; x) \\
\underline{2}&:= \lambda x \lambda f \, (f \; \underline{1} \; (f \; \underline{0} \; x) )\\
\underline{3}&:= \lambda x \lambda f \, (f \; \underline{2} \; (f \; \underline{1} \; (f \; \underline{0} \; x) ))\\
\end{split}
\end{equation*}
\begin{enumerate}
\item Write a $\lambda$-term $t$ such that for all $n\in \mathbb{N}$
\begin{equation*}
\begin{split}
t \; \underline{0} &\mapsto \underline{0} \\
t \; \underline{n+1} &\mapsto \underline{n} \\
\end{split}
\end{equation*}
\item Write a $\lambda$-term $t$ such that for all $n\in \mathbb{N}$
\begin{equation*}
\begin{split}
t \; \underline{n} \mapsto \underline{n+1} \\
\end{split}
\end{equation*}
\item Write a $\lambda$-term $t$ such that for all $n\in \mathbb{N}$
\begin{equation*}
\begin{split}
t \; \underline{n}\; \underline{m} \mapsto \underline{n+m} \\
\end{split}
\end{equation*}
\end{enumerate}
\paragraph{Solution:}

\subsubsection*{Term 1}
Let 
\begin{equation*}
t = P := \lambda n \, n \;\underline{0}\; True
\end{equation*}

For the case $\underline{n}=\underline{0}$ we have, 
\begin{equation*}
\begin{split}
&P\; \underline{0} = (\lambda n \, n \;\underline{0}\; True) \; \underline{0} \\
&\mapsto^* \underline{0} \; \underline{0} \; True = (\lambda x \lambda f \, x) \; \underline{0} \; True \mapsto^* \underline{0}
\end{split}
\end{equation*}
and otherwise for $\underline{n+1}$ we have 
\begin{equation*}
\begin{split}
&P\; \underline{n+1} = (\lambda n \, n \;\underline{0}\; True) \; \underline{n+1} 
\mapsto^* \underline{n+1} \; \underline{0} \; True\\
&= (\lambda x \lambda f \, (f\; \underline{n}\; (\cdots \;  x \; \cdots))) \; \underline{0} \; True  \\
&\mapsto^* True\; \underline{n}\; (\cdots \; \underline{0}\; \cdots) \\
&=  (\lambda x \lambda y \, x)\; True\; \underline{n}\; (\cdots \; \underline{0}\; \cdots) \mapsto^* \underline{n}
\end{split}
\end{equation*}


\subsubsection*{Term 2}
Let 
\begin{equation*}
t = S := \lambda n \lambda y \lambda g \,(g \; n \; ( n \; y \; g))
\end{equation*}

We separate the proof into two cases.

For the case $\underline{n} =\underline{0}$ we have, 
\begin{equation*}
\begin{split}
&S\; \underline{0} =  (\lambda n \lambda y \lambda g \,(g \; n \; ( n \; y \; g))) \; \underline{0} \mapsto  \lambda y \lambda g \,(g \;  \underline{0} \; (  \underline{0} \; y \; g)) \\
&= \lambda y \lambda g \,(g \;  \underline{0} \; (  (\lambda x \lambda f \, x)  \; y \; g)) 
\mapsto^*  \lambda y \lambda g \,(g \;  \underline{0} \; y) = \underline{0}
\end{split}
\end{equation*}

and otherwise for $\underline{n+1}$ we have 

\begin{equation*}
\begin{split}
&S\; \underline{n+1} =  (\lambda n \lambda y \lambda g \,(g \; n \; ( n \; y \; g))) \; \underline{n+1} \mapsto  \lambda y \lambda g \,(g \;  \underline{n+1} \; (  \underline{n+1} \; y \; g)) \\
&= \lambda y \lambda g \,(g \;  \underline{n+1} \; ( (\lambda x \lambda f \, (f\; \underline{n}\; (\cdots \; x\; \cdots ))) \; y \; g)) \\
&\mapsto^* \lambda y \lambda g \, (g \;  \underline{n+1} \; (g \;  \underline{n} \; (\cdots \; y \; \cdots ))) = \underline{n+2}
\end{split}
\end{equation*}


\subsubsection*{Term 3}
Let 
\begin{equation*}
t=A:= \lambda n \lambda m \, n \; m \; ( \lambda x \lambda y \, S \; y)
\end{equation*}
Let $\underline{m}$ be an arbitrary numeral, then we have for $\underline{n}=\underline{0}$
\begin{equation*}
\begin{split}
A \; \underline{n} \; \underline{m} &= (\lambda n \lambda m \, n \; m \; ( \lambda x \lambda y \, S \; y)) \; \underline{n} \; \underline{m} \mapsto^*  \underline{n} \; \underline{m} \; ( \lambda x \lambda y \, S \; y) \\
&= (\lambda x \lambda f \, x)\; \underline{m} \; ( \lambda x \lambda y \, S \; y) 
\mapsto^* \underline{m}
\end{split}
\end{equation*}


and otherwise for $\underline{n+1}$ we have
\begin{equation*}
\begin{split}
A \; \underline{n+1} \; \underline{m} &= (\lambda n \lambda m \, n \; m \; ( \lambda x \lambda y \, S \; y)) \; \underline{n+1} \; \underline{m} \mapsto^*  \underline{n+1} \; \underline{m} \; ( \lambda x \lambda y \, S \; y) \\
&= (\lambda x \lambda f \, (f\; \underline{n+1}\; (\cdots \; x \; \cdots))) \; \underline{m} \; ( \lambda x \lambda y \, S \; y) \\
&\mapsto^* ( \lambda x \lambda y \, S \; y) \; \underline{n+1}\; (\cdots \; \underline{m}  \; \cdots) \\
&\mapsto^*  S \; (\cdots \;  \underline{m}  \; \cdots) \mapsto^* S^{n+1} \; \underline{m} \\
&\mapsto^* S^{n} \; \underline{m+1} \mapsto^* S \; \underline{m+n} \mapsto^* \underline{m+(n+1)}
\end{split}
\end{equation*}







\section{Homework}
\subsection{Exercise}

Formally prove by induction on $u$ that 
\begin{equation*}
u[v/y][t/x]=u[t/x][v[t/x]/y]
\end{equation*}
provided $x \neq y$ and $y$ does not occur in $t$.

\paragraph{Solution:} By induction on $u$
\begin{itemize}
\item For $u=y$ we have
\begin{equation*}
\begin{split}
u[v/y][t/x]&=y[v/y][t/x]=v[t/x]=y[v[t/x]/y] \\
&\stackrel{x\neq y}{=}y[t/x][v[t/x]/y]=u[t/x][v[t/x]/y]
\end{split}
\end{equation*}

\item For $u=x$ we have
\begin{equation*}
\begin{split}
u[v/y][t/x]&=x[v/y][t/x]\stackrel{x\neq y}{=}x[t/x]=t\\
&\stackrel{y\textit{ n. in }t}{=} t[v[t/x]/y]=x[t/x][v[t/x]/y]=u[t/x][v[t/x]/y]
\end{split}
\end{equation*}

\item For $u=z$ with $x \neq z$ and $y \neq z$ we have
\begin{equation*}
\begin{split}
u[v/y][t/x]&=z[v/y][t/x]\stackrel{y\neq z}{=}z[t/x]\stackrel{x\neq z}{=}z \\
&\stackrel{y\neq z}{=}z[v[t/x]/y]\stackrel{x\neq z}{=}z[t/x][v[t/x]/y]=u[t/x][v[t/x]/y]
\end{split}
\end{equation*}



\item For $u=\lambda z \, w$ we have
\begin{equation*}
\begin{split}
u[v/y][t/x]&=(\lambda z \, w)[v/y][t/x] \stackrel{Def.}{=} \lambda z \, w[v/y][t/x] \stackrel{IH}{=}
 \lambda z \, w[t/x][v[t/x]/y] \\
 &\stackrel{Def.}{=}  (\lambda z \, w)[t/x][v[t/x]/y]
=u[t/x][v[t/x]/y]
\end{split}
\end{equation*}




\item For $u=w_1\; w_2$ we have
\begin{equation*}
\begin{split}
u[v/y][t/x]&=(w_1\; w_2)[v/y][t/x] \stackrel{Def.}{=} w_1[v/y][t/x]\; w_2[v/y][t/x] \\
&\stackrel{IH}{=}
 w_1[t/x][v[t/x]/y] \; w_2[t/x][v[t/x]/y] \\
 &\stackrel{Def.}{=}  (w_1\; w_2)[t/x][v[t/x]/y]
=u[t/x][v[t/x]/y]
\end{split}
\end{equation*}

\end{itemize}



\subsection{Exercise}
Prove or disprove:
\begin{quote}
If $w$ is elementary and $w \mapsto w'$, then $w'$ is elementary.
\end{quote}

\paragraph{Solution:} We shall disprove this claim by constructing an elementary $\lambda$-term which does not reduce (within one step) to an elementary $\lambda$-term. \\
Firstly, we know that 
\begin{itemize}
\item a $\lambda$-term $u\; t$ is elementary, if $t\in SN$, $u \in SN$ but $t\; u \nin SN$ and that
\item a $\lambda$-term $t$ is strongly normalisable, i.e. $t \in SN$, if there is no infinite reduction of $t$, i.e. $h(t)$ is finite.
\end{itemize}
Secondly, let $w$ be the $\lambda$-term 
\begin{equation*}
w:= t \; u = (\lambda x (z(x\;x)))(\lambda x\,x\;x)
\end{equation*}
Thirdly, we need to show that $w$ is elementary. 
\begin{enumerate}
\item We can observe that $t$ and $u$ contain no redex, thus can no longer be reduced and are therefore in normal form. Hence, allowing us to conclude $t \in SN$ and $u \in SN$.

\item To show that $w \nin SN$ we will reduce the term once. In this case the only possibility to do so is to apply $t$ to $u$, i.e. 
\begin{equation*}
w = t \; u = (\lambda x (z(x\;x)))(\lambda x\,x\;x)\mapsto  z((\lambda x\,x\;x)\;(\lambda x\,x\;x)) = w'
\end{equation*}
Given $w'$ we can observe that $w'=t'u'$ with $h(u')$ being infinite. That is, as shown in the lecture the term $u'=((\lambda x\,x\;x)\;(\lambda x\,x\;x))$ has an infinite reduction. Therefore, $h(w')$ is infinite. Moreover, since $w'$ has not the form $(\lambda x \, v)\; t \; t_1 \ldots \,t_n$ with $u,t,t_1,\ldots,t_n \in SN$ it can not be elementary. 
\end{enumerate}
Hence, the claim is disproven.


\subsection{Exercise}
Prove or disprove:
\begin{quote}
There are terms $u$ and $t$ such that $x$ does not occur in $u$ and $(\lambda x \, u)\; t$ is elementary.
\end{quote}

\paragraph{Solution:} Assume that the $\lambda$-terms $u$ and $t$ exists. That is, we let $u$ and $t$ be $\lambda$-terms, such that 
\begin{equation*}
w = v \; t = (\lambda x \, u) \; t
\end{equation*}
with $x$ not in $u$, is elementary. 

Since, $w$ is elementary we know that $(\lambda x \, u) \in SN$, $t \in SN$ and $w \nin SN$. 
Moreover, by Prop. 12 we know that 
\begin{center}
$(\lambda x \, u)\; t$ is elementary implies $u[t/x] \nin SN$
\end{center}
However, since $x$ not in $u$ we obtain 
\begin{equation*}
u[t/x]=u \nin SN
\end{equation*}
Thus we obtain $u \nin SN$ and $(\lambda x \, u) \in SN$, which is a contradiction. That is, with $h(u)$ being infinite, it follows that $h((\lambda x \, u))$ is also infinite, as we can always find a redex within $u$ to contract.
To conclude, assuming the existence of such terms induces a contradiction. 






\section{Homework}
\subsection{Exercise}
Let $t:A$ be a term that does not contain free variables and is in normal form. Prove that 
\begin{itemize}
\item if $A=B \wedge C$ then $t=\langle u,v \rangle$ and 
\item if $A=B\to C$ then $t= \lambda x^B\, u$
\end{itemize}

\paragraph{Solution:} Proof by induction on the type derivation $\mathcal{D}$ of $t$, with the $IH$: 
\\

For a $\lambda$-term $t:A$, which does not contain free variables and is in normal form, its type derivation $\mathcal{D}$ must be of the shape.


\begin{prooftree}
          \noLine
          \AxiomC{$\mathcal{D}'$}
          \UnaryInfC{$u:C$}
           \LeftLabel{$\mathcal{D}=$ }
           \RightLabel{\quad \textit{if} $A=B\to C$ }
          \UnaryInfC{$\lambda x^B \, u: B\to C $}       
\end{prooftree}
and 
\begin{prooftree}
          \noLine
          \AxiomC{$\mathcal{D}'$}
          \UnaryInfC{$u:B$}
          \noLine
          \AxiomC{$\mathcal{\mathcal{E}}'$}
          \UnaryInfC{$v:C$}
           \LeftLabel{$\mathcal{D}=$ }
           \RightLabel{\quad \textit{if} $A=B \wedge C$ }
          \BinaryInfC{$\langle u,v \rangle: B \wedge C$}       
\end{prooftree}

Now we are going to distinguish by cases according to the last rule of the type derivation $\mathcal{D}$ of $t$.

\begin{enumerate}
\item If $\mathcal{D}=$
          \AxiomC{}
          \UnaryInfC{$t:A$}       
\DisplayProof
we know that $t$ is a variable. Hence, $t$ contains itself as a free variable. However, we know that $t$ can not contain free variables. Therefore, this rule can not be the last rule of the type derivation $\mathcal{D}$.

\item If $\mathcal{D}=$
          \noLine
          \AxiomC{$\mathcal{D}'$}
          \UnaryInfC{$u:C$}
          \UnaryInfC{$\lambda x^B \, u: B\to C $} 
  \DisplayProof 
  we have our thesis. That is, $A=B\to C$ and $t$ was derived by the desired rule.  
  

\item If $\mathcal{D}=$
          \noLine
          \AxiomC{$\mathcal{D}'$}
          \UnaryInfC{$u:B$}
          \noLine
          \AxiomC{$\mathcal{\mathcal{E}}'$}
          \UnaryInfC{$v:C$}
          \BinaryInfC{$\langle u,v \rangle: B \wedge C$}    
  \DisplayProof 
  we have our thesis. That is, $A=B\wedge C$ and $t$ was derived by the desired rule.   
    
    
\item If $\mathcal{D}=$
          \noLine
          \AxiomC{$\mathcal{D}'$}
          \UnaryInfC{$u:E \wedge F$} 
          \UnaryInfC{$u\;\pi_0:E $} 
  \DisplayProof 
  with $t=u\;\pi_0$. We can observe that if $u$ contains free variables, then $u\;\pi_0$ must contain the same free variables. Because the applied rule does not bin free variables in $u$. Therefore, it follows from $t=u\;\pi_0$ being free variable free, that $u$ does not contain free variables. Moreover, if $u$ contains a redex or the subterm $\langle t_0, t_1 \rangle \; \pi_i$ with $i \in \{0,1\}$. Then the same must be contained in $u\;\pi_0$
 as $u$ was not modified by this rule. Therefore, if $t=u\;\pi_0$ is in normal form, then $u$ must also be in normal form.
 
Hence, we conclude that $u$ must be in normal form and can not contain free variables, allowing us to apply the $IH$ for $u:E\wedge F$. That is, we obtain
  \begin{prooftree}
           \noLine
          \AxiomC{$\mathcal{D}''$}
          \UnaryInfC{$r:E$}
          \noLine
          \AxiomC{$\mathcal{E}''$}
          \UnaryInfC{$o:F$}
          \LeftLabel{$\mathcal{D}=$ }
          \BinaryInfC{$\langle r,o \rangle: E \wedge F$} 
          \UnaryInfC{$\langle r,o \rangle \pi_0:E $}      
\end{prooftree}
However, as $\langle t_0, t_1 \rangle \; \pi_i \mapsto t_i$ with $i \in \{0,1\}$ it follows that $t=\langle r,o \rangle \;\pi_0$ is not in normal form. Hence, the type derivation of $t$ can not be of this shape.


\item If $\mathcal{D}=$
          \noLine
          \AxiomC{$\mathcal{D}'$}
          \UnaryInfC{$u:E \wedge F$} 
          \UnaryInfC{$u\;\pi_1:F $} 
  \DisplayProof 
we can discard this case in analogue to the case 4.

\item If $\mathcal{D}=$
          \noLine
          \AxiomC{$\mathcal{D}'$}
          \UnaryInfC{$u:E \to F$}
          \noLine
          \AxiomC{$\mathcal{\mathcal{E}}'$}
          \UnaryInfC{$v:E$}
          \BinaryInfC{$u\;v: F$}    
\DisplayProof 
with $t=u\; v$. Here we argue in a similar fashion as in case 4. 
That is, we will argue that $u$ must be free variable free and in normal form.

As this rule does not bind free variables in $u$, those variables must remain free in $u\; v$. Furthermore, if $u$ contains a redex or  $\langle t_0, t_1 \rangle \pi_i$ with $i \in \{0,1\}$, those must also be present in $u\; v$, as $u$ is not modified by this rule. Hence, it follows that $u$ must be in normal form and can not contain free variables. Furthermore, with $u: E\to F$ it follows by $IH$ that

\begin{prooftree}
          \noLine
          \AxiomC{$\mathcal{D}''$}
          \UnaryInfC{$r: F$}
          \LeftLabel{$\mathcal{D}=$}
          \UnaryInfC{$\lambda x^E\, r: E \to F$}
          \noLine
          \AxiomC{$\mathcal{\mathcal{E}}'$}
          \UnaryInfC{$v:E$}
          \BinaryInfC{$(\lambda x^E\, r)\;v: F$} 
\end{prooftree}
that is $t=(\lambda x^E\, r)\;v$ and thus $t$ contains a redex and is therefore not in normal form. Hence, the type derivation of $t$ can not be of this shape.
\end{enumerate}
Given this result our thesis follows immediately. 




\section{Homework}
\subsection{Exercise}
Define $\neg A := A \to \bot$, where $\bot$ is a type variable. Write a natural deduction of 
\begin{equation*}
\neg\neg (A \wedge B) \to \neg \neg A
\end{equation*}

\paragraph{Solution:} Firstly, we substitute
\begin{equation*}
\begin{split}
\neg\neg (A \wedge B) \to \neg \neg A &= \neg ((A\wedge B) \to \bot) \to  \neg(A\to\bot) \\
 &= ((A\wedge B) \to \bot) \to  \bot) \to ((A\to\bot) \to \bot) 
\end{split}
\end{equation*}
Secondly, the derivation. 

Two derivations will be presented. The first one is merely the natural deduction, while the second one depicts the natural deduction together with its corresponding $\lambda$-terms.


\begin{prooftree}
\AxiomC{$[((A\wedge B) \to \bot) \to  \bot]^{(1)}$ }
\AxiomC{$[A\to  \bot]^{(2)}$ }
\AxiomC{$[(A\wedge B) ]^{(3)}$ }
\UnaryInfC{$A$}
\BinaryInfC{$ \bot$}
\RightLabel{\scriptsize$(3)$}
\UnaryInfC{$(A\wedge B) \to \bot$}
\BinaryInfC{$ \bot$}
\RightLabel{\scriptsize$(2)$}
\UnaryInfC{$(A\to\bot) \to \bot $}
\RightLabel{\scriptsize$(1)$}
\UnaryInfC{$(((A\wedge B) \to \bot) \to  \bot) \to ((A\to\bot) \to \bot )$}
\end{prooftree}



And now with the corresponding $\lambda$-terms


\begin{prooftree}
\small
\AxiomC{$ z^{((A\wedge B) \to \bot) \to  \bot} : [((A\wedge B) \to \bot) \to  \bot]^{(1)}$ }
\AxiomC{$y^{A\to\bot}:[A\to  \bot]^{(2)}$ }
\AxiomC{$x^{A\wedge B}: [(A\wedge B) ]^{(3)}$ }
\UnaryInfC{$ x^{A\wedge B}\, \pi_0:A$}
\BinaryInfC{$y^{A\to\bot}\, ( x^{A\wedge B}\, \pi_0):  \bot$}
\RightLabel{\scriptsize$(3)$}
\UnaryInfC{$\lambda x^{A\wedge B}(y^{A\to\bot}\, (x\, \pi_0)): (A\wedge B) \to \bot$}
\BinaryInfC{$ z^{((A\wedge B) \to \bot) \to  \bot} \lambda x^{A\wedge B}(y^{A\to\bot}\, (x\, \pi_0)): \bot$}
\RightLabel{\scriptsize$(2)$}
\UnaryInfC{$\lambda y^{A\to\bot}(z^{((A\wedge B) \to \bot) \to  \bot} \lambda x^{A\wedge B}(y\, (x\, \pi_0))):(A\to\bot) \to \bot $}
\RightLabel{\scriptsize$(1)$}
\UnaryInfC{$\lambda z^{((A\wedge B) \to \bot) \to  \bot} \lambda y^{A\to\bot}(z\, \lambda x^{A\wedge B}(y\, (x\, \pi_0))):
(((A\wedge B) \to \bot) \to  \bot) \to ((A\to\bot) \to \bot )$}
\end{prooftree}



\section{Homework}
\subsection{Exercise}
Show that the $\lambda$-term 
\begin{equation*}
\lambda y \, y\; y \; (\lambda x \, x \; y \; x)  
\end{equation*}
is typable in $D\Omega$ with a type that does not contain $\top$.

\paragraph{Solution:} 
Given the bracketing convention we have
\begin{equation*}
t:= \lambda y \, y\; y \; (\lambda x \, x \; y \; x)  = \lambda y \,( (y\; y) \; (\lambda x \, ((x \; y) \; x)) )
\end{equation*}


for which we shall provide a type derivation.

\begin{prooftree}
\AxiomC{$  y : b_0$}
\UnaryInfC{$ y :b_2$}
\AxiomC{$  y : b_0$}
\UnaryInfC{$ y :b_3$}
\BinaryInfC{$y\; y: c_3$}
\AxiomC{$  x : a_0$}
\UnaryInfC{$ x :a_1$}
\AxiomC{$  y : b_0$}
\UnaryInfC{$ y :b_1$}
\BinaryInfC{$ x \; y: c_1$}
\AxiomC{$  x : a_0$}
\UnaryInfC{$ x :a_2$}
\BinaryInfC{$ (x \; y) \; x : c_2$}
\UnaryInfC{$ \lambda x \, ((x \; y) \; x) :c_4$}
\BinaryInfC{$ (y\; y) \; (\lambda x \, ((x \; y) \; x)) : c_5$}
\UnaryInfC{$ \lambda y \,( (y\; y) \; (\lambda x \, ((x \; y) \; x)) ):c_6$}
\end{prooftree}



Based on this derivation we start by constructing the type schema.

\begin{equation*}
\begin{split}
a_0&= a_1 \wedge a_2 = (b_1 \to (a_2 \to c_2)) \wedge a_2 \\
a_1&= b_1 \to c_1 =  (b_1 \to (a_2 \to c_2)) \\
a_2&= ?\\   
b_0&= b_1 \wedge b_2 \wedge b_3 = b_1 \wedge (b_3 \to ((( (b_1 \to (a_2 \to c_2)) \wedge a_2) \to c_2)\to c_5)) \wedge b_3\\    
b_1&= ?\\   
b_2&= b_3 \to c_3 = b_3 \to (a_0 \to c_2)\to c_5= b_3 \to ((( (b_1 \to (a_2 \to c_2)) \wedge a_2) \to c_2)\to c_5)\\
b_3&= ?\\ 
c_1&=a_2 \to c_2 \\
c_2&= ?\\
c_3&=c_4 \to c_5 = (((b_1 \to (a_2 \to c_2)) \wedge a_2) \to c_2) \to c_5\\
c_4&= a_0 \to c_2 = ((b_1 \to (a_2 \to c_2)) \wedge a_2) \to c_2\\ 
c_5&= ?\\
c_6&= b_0 \to c_5= (b_1 \wedge (b_3 \to ((( (b_1 \to (a_2 \to c_2)) \wedge a_2) \to c_2)\to c_5)) \wedge b_3) \to c_5
\end{split}
\end{equation*}

This is the general schema from which we are going to build our types.

We start by setting the following types.

\begin{equation*}
\begin{split}
a_2= A; \quad b_1= B; \quad b_3 = B; \quad c_2= C; \quad c_5= D\\ 
\end{split}
\end{equation*}
and substitute to obtain

\begin{equation*}
\begin{split}
a_0&= (B \to (A \to C)) \wedge A \\
a_1&= (B \to (A \to C)) \\
a_2&= A\\   
b_0&= B \wedge (B \to ((( (B \to (A \to C)) \wedge A) \to C)\to D))\\    
b_1&= B\\   
b_2&=  B \to ((( (B \to (A \to C)) \wedge A) \to C)\to D)\\
b_3&= B\\ 
c_1&=A \to C \\
c_2&= C\\
c_3&= (((B \to (A \to C)) \wedge A) \to C) \to D\\
c_4&= ((B \to (A \to C)) \wedge A) \to C\\ 
c_5&= D\\
c_6&= (B \wedge (B \to ((( (B \to (A \to C)) \wedge A) \to C)\to D)) \wedge B) \to D
\end{split}
\end{equation*}

resulting in \\


\begin{sidewaysfigure}

\begin{prooftree}
\AxiomC{$  y : B \wedge (B \to ((( (B \to (A \to C)) \wedge A) \to C)\to D))^{(*)}$}
\UnaryInfC{$ y :B \to ((( (B \to (A \to C)) \wedge A) \to C)\to D)$}
\AxiomC{$  y : (*)$}
\UnaryInfC{$ y :B$}
\BinaryInfC{$y\; y: (((B \to (A \to C)) \wedge A) \to C)\to D$}
\AxiomC{$  x : (B \to (A \to C)) \wedge A$}
\UnaryInfC{$ x :B \to (A \to C) $}
\AxiomC{$  y : (*)$}
\UnaryInfC{$ y :B$}
\BinaryInfC{$ x \; y: A \to C$}
\AxiomC{$  x : (B \to (A \to C)) \wedge A$}
\UnaryInfC{$ x :A$}
\BinaryInfC{$ (x \; y) \; x : C$}
\UnaryInfC{$ \lambda x \, ((x \; y) \; x) : ((B \to (A \to C)) \wedge A) \to C$}
\BinaryInfC{$ (y\; y) \; (\lambda x \, ((x \; y) \; x)) : D$}
\UnaryInfC{$ \lambda y \,( (y\; y) \; (\lambda x \, ((x \; y) \; x)) ): (B \wedge (B \to ((( (B \to (A \to C)) \wedge A) \to C)\to D)))\to D$}
\end{prooftree}
\end{sidewaysfigure}






















\end{document}